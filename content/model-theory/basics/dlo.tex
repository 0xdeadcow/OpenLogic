% Part: first-order-logic
% Chapter: model-theory
% Section: partial-iso

\documentclass[../../../include/open-logic-section]{subfiles}

\begin{document}

\olfileid{mod}{bas}{dlo}
\section{Dense Linear Orders}

\begin{defn}
  A \emph{dense linear ordering without endpoints} is !!a{structure}
  $\Struct{M}$ for the !!{language} containg a single 2-place
  !!{predicate}~$<$ satisfying the following sentences:
  \begin{enumerate}
  \item $\lforall[x] \ x < x$;
  \item $\lforall[x][\lforall[y][\lforall[z][(x < y \lif (y < z \lif x
    <z ))]]]$;
  \item $\lforall[x][\lforall[y][(x< y \lor \eq[x][y] \lor y < x)]]$;
  \item $\lforall[x][\lexists[y][x < y]]$;
  \item $\lforall[x][\lexists[y][y < x]]$;
  \item $\lforall[x][\lforall[y][(x < y \lif \lexists[z][(x < z \land
        z < y))]]]$.
 \end{enumerate}
\end{defn}

\begin{thm}\ollabel{thm:cantorQ}
  Any two !!{enumerable} dense linear orderings without
  endpoints are isomorphic.
\end{thm}

\begin{proof}
  Let $\Struct{M_1}$ and $\Struct{M_2}$ be !!{enumerable} dense linear
  orderings without endpoints, with ${<_1} = \Assign{<}{M_1}$ and ${<_2} =
  \Assign{<}{M_2}$, and let $\PIso{I}$ be the set of all partial
  isomorphisms between them. $\PIso{I}$ is not empty since at least
  $\emptyset \in \PIso{I}$. We show that $\PIso{I}$ satisfies the
  Back-and-Forth property.  Then $\Struct{M_1} \iso[p] \Struct{M_2}$,
  and the theorem follows by \olref[pis]{thm:p-isom1}.

  To show $\PIso{I}$ satisifes the Forth property, let $p \in
  \PIso{I}$ and let $p(a_i) = b_i$ for $i = 1$, \dots,~$n$, and
  without loss of generality suppose $a_1 <_1 a_2 <_1 \cdots <_1
  a_n$. Given $a \in \Domain{M_1}$, find $b \in \Domain{M_2}$ as
  follows:
  \begin{enumerate}
  \item if $a <_2 a_1$ let $b \in \Domain{M_2}$ be such that $b <_2
    b_1$;
  \item if $a_n <_1 a$ let $b \in \Domain{M_2}$ be such that $b_n <_2 b$;
 \item if $a_i <_1 a <_1 a_{i+1}$ for some $i$, then let $b \in
   \Domain{M_2}$ be such that $b_i <_2 b <_2 b_{i+1}$.
  \end{enumerate}
  It is always possible to find a $b$ with the desired property since
  $\Struct{M_2}$ is a dense linear ordering without endpoints. Define
  $q = p \cup \{ \langle a, b \rangle \}$ so that $q \in \PIso{I}$ is
  the desired extension of $p$. This establishes the Forth
  property. The Back property is similar. So $\Struct{M_1} \iso[p]
  \Struct{M_2}$; by \olref[pis]{thm:p-isom1}, $\Struct{M_1} \iso
  \Struct{M_2}$.
\end{proof}

\begin{prob}
  Complete the proof of \olref[mod][bas][dlo]{thm:cantorQ} by
  verifying that $\PIso{I}$ satisfies the Back property.
\end{prob}

\begin{rem}
  Let $\Struct{S}$ be any !!{enumerable} dense linear ordering without
  endpoints. Then (by \olref{thm:cantorQ}) $\Struct{S} \iso
  \Struct{Q}$, where $\Struct{Q} = (\Rat, <)$ is the !!{enumerable}
  dense linear ordering having the set $\Rat$ of the rational numbers
  as its domain. Now consider again the !!{structure} $\Struct{R} =
  (\Real, <)$ from \olref[thm]{remark:R}. We saw that there is
  !!a{enumerable} !!{structure} $\Struct{S}$ such that $\Struct{R}
  \elemequiv \Struct{S}$. But $\Struct{S}$ is !!a{enumerable} dense
  linear ordering without endpoints, and so it is isomorphic (and
  hence elementarily equivalent) to the !!{structure}~$\Struct{Q}$. By
  transitivity of elementary equivalence, $\Struct{R} \elemequiv
  \Struct{Q}$. (We could have shown this directly by establishing
  $\Struct{R} \iso[p] \Struct{Q}$ by the same back-and-forth
  argument.)
\end{rem}
\end{document}
