% Part: first-order-logic
% Chapter: model-theory
% Section: isomorphism

\documentclass[../../../include/open-logic-section]{subfiles}

\begin{document}

\olfileid{mod}{bas}{iso}
\olsection{Isomorphic Structures}

First-order !!{structure}s can be alike in one of two ways. One way in
which the can be alike is that they make the same !!{sentence}s
true. We call such !!{structure}s \emph{elementarily equivalent}. But
structures can be very different and still make the same !!{sentence}s
true---for instance, one can be !!{enumerable} and the other not.
This is because there are lots of features of !!a{structure} that
cannot be expressed in first-order languages, either because the
language is not rich enough, or because of fundamental limitations of
first-order logic such as the L\"owenheim-Skolem theorem. So another,
stricter, aspect in which !!{structure}s can be alike is if they are
fundamentally the same, in the sense that they only differ in the
objects that make them up, but not in their structural features. A way
of making this precise is by the notion of an \emph{isomorphism}.

\begin{defn}
\ollabel{defn:elem-equiv} 
Given two !!{structure}s $\Struct{M}$ and $\Struct M'$ for the same
!!{language}~$\Lang{L}$, we say that $\Struct{M}$ is \emph{elementarily
  equivalent to} $\Struct M'$, written $\Struct{M} \equiv \Struct M'$,
if and only if for every !!{sentence}~$!A$ of~$\Lang{L}$,
$\Sat{M}{!A}$ iff $\Sat{M'}{!A}$.
\end{defn}

\begin{defn}
\ollabel{defn:isomorphism} Given two !!{structure}s $\Struct{M}$ and
$\Struct M'$ for the same !!{language}~$\Lang L$, we say that
$\Struct{M}$ is \emph{isomorphic to}~$\Struct M'$, written $\Struct{M}
\simeq \Struct M'$, if and only if there is a function $h \colon
\Domain{M} \to \Domain{M'}$ such that:
\begin{enumerate}
\item $h$ is !!{injective}: if $h(x) =
  h(y)$ then $x = y$; 
\item $h$ is !!{surjective}: for every $y \in \Domain{M'}$ there
  is $x \in \Domain{M}$ such that $h(x) = y$;
\item \ollabel{defn:iso-const}for every !!{constant} $c$:
  $h(\Assign{c}{M}) = \Assign{c}{M'}$;
\item \ollabel{defn:iso-pred}for every $n$-place !!{predicate}~$P$:
  \[
  \tuple{a_1, \dots, a_n}\in \Assign{P}{M} \quad\text{iff}\quad
  \tuple{h(a_1), \dots, h(a_n)} \in \Assign{P}{M'};
  \]
\item \ollabel{defn:iso-func}for every $n$-place !!{function} $f$:
  \[
  h(\Assign{f}{M}(a_1, \dots, a_n)) =
  \Assign{f}{M'}(h(a_1), \dots, h(a_n)).
  \]
\end{enumerate}
\end{defn}

\begin{thm}
\ollabel{thm:isom}
If $\Struct{M} \iso \Struct M'$ then $\Struct{M} \elemequiv
\Struct{M'}$.
\end{thm}

\begin{proof}
Let $h$ be an isomorphism of $\Struct{M}$ onto $\Struct M'$. For any
assignment~$s$, $h \circ s$ is the composition of $h$ and $s$, i.e.,
the assignment in $\Struct{M'}$ such that $(h \circ s)(x) = h(s(x))$.
By induction on $t$ and $!A$ one can prove the stronger claims:
\begin{enumerate}
  \item[a.] $h(\Value{t}{M}[s]) = \Value{t}{M'}[h\circ s]$.
  \item[b.] $\Sat{M}{!A}[s]$ iff $\Sat{M'}{!A}[h \circ s]$.
\end{enumerate}
The first is proved by induction on the complexity of~$t$.
\begin{enumerate}
\item If $t \ident c$, then $\Value{c}{M}[s] = \Assign{c}{M}$ and
  $\Value{c}{M'}[h \circ s] = \Assign{c}{M'}$. Thus,
  $h(\Value{t}{M}[s]) = h(\Assign{c}{M}) = \Assign{c}{M'}$ (by
  \olref{defn:iso-const} of \olref{defn:isomorphism}) $=
  \Value{t}{M'}[h \circ s]$.
\item If $t \ident x$, then $\Value{x}{M}[s] = s(x)$ and
  $\Value{x}{M'}[h \circ s] = h(s(x))$. Thus, $h(\Value{x}{M}[s]) =
  h(s(x)) = \Value{x}{M'}[h \circ s]$.
\item If $t \ident f(t_1, \dots, t_n)$, then
  \begin{align*}
    \Value{t}{M}[s] & = \Assign{f}{M}(\Value{t_1}{M}[s], \dots, \Value{t_n}{M}[s]) \quad\text{and}\\
  \Value{t}{M'}[h \circ s] & = \Assign{f}{M}(\Value{t_1}{M'}[h \circ
    s], \dots, \Value{t_n}{M'}[h \circ s]).
  \end{align*}
  The induction hypothesis is that for each $i$, $h(\Value{t_i}{M}[s])
  = \Value{t_i}{M'}[h\circ s]$. So,
  \begin{align}
    h(\Value{t}{M}[s]) 
    & = h(\Assign{f}{M}(\Value{t_1}{M}[s], \dots, \Value{t_n}{M}[s]) \notag\\
    & = h(\Assign{f}{M}(\Value{t_1}{M'}[h \circ s], \dots,
    \Value{t_n}{M'}[h \circ s]) \ollabel{iso-1}\\
    & = \Assign{f}{M'}(\Value{t_1}{M'}[h \circ s], \dots,
    \Value{t_n}{M'}[h \circ s]) \ollabel{iso-2}\\
    & = \Value{t}{M'}[h\circ s] \notag
  \end{align}
  Here, \olref{iso-1} follows by induction hypothesis and \olref{iso-2} by
  \olref{defn:iso-func} of \olref{defn:isomorphism}.
\end{enumerate}
Part (2) is left as an exercise.

If $!A$ is a sentence, the assignments~$s$ and $h \circ s$ are
irrelevant, and we have $\Sat{M}{!A}$ iff $\Sat{M'}{!A}$.
\end{proof}

\begin{prob}
Carry out the proof of (b) of \olref[mod][bas][iso]{thm:isom} in
detail. Make sure to note where each of the five properties
characterizing isomorphisms of \olref[mod][bas][iso]{defn:isomorphism}
is used.
\end{prob}

\begin{defn}
An \emph{automorphism} of a structure $\mathfrak{M}$ is an isomorphism
of $\mathfrak{M}$ onto itself.
\end{defn}

\begin{prob}
Show that for any structure $\Struct{M}$, if $X$ is a definable subset
of $\Struct{M}$, and $h$ is an automorphism of $\Struct{M}$, then $X
= \Setabs{h(x)}{x \in X}$ (i.e., $X$ is fixed under $h$).
\end{prob}


\end{document}
