% Part: model-theory
% Chapter: models-of-arithmetic
% Section: computable-models

\documentclass[../../../include/open-logic-section]{subfiles}

\begin{document}

\olfileid{mod}{mar}{cmp}
\section{Computable Models of Arithmetic}

\begin{explain}
The standard model~$\Struct{N}$ has two nice features. Its domain is
the natural numbers~$\Nat$, i.e., its elements are just the kinds of
things we want to talk about using the language of arithmetic, and the
standard numeral~$\num{n}$ actually picks out~$n$. The other nice
feature is that the interpretations of the non-logical symbols
of~$\Lang{L_A}$ are all \emph{computable}. The successor, addition,
and multiplication functions which serve as $\Assign{\prime}{N}$,
$\Assign{+}{N}$, and $\Assign{\times}{N}$ are computable functions of
numbers. (Computable by Turing machines, or definable by primitive
recursion, say.) And the less-than relation on~$\Struct{N}$, i.e.,
$\Assign{<}{N}$, is decidable.

Non-standard models of arithmetical theories such as $\Th{Q}$ and
$\Th{PA}$ must contain non-standard elements. Thus their domains
typically include !!{element}s in addition to~$\Nat$. However, any
countable !!{structure} can be built on any !!{denumerable} set,
including~$\Nat$. So there are also non-standard models with
domain~$\Nat$. In such models~$\Struct{M}$, of course, at least some
numbers cannot play the roles they usually play, since some~$k$ must
be different from~$\Value{\num{n}}{M}$ for all~$n \in \Nat$.
\end{explain}

\begin{defn}
!!^a{structure}~$\Struct{M}$ for $\Lang{L_A}$ is \emph{computable} iff
  $\Domain{M} = \Nat$ and $\Assign{\prime}{M}$, $\Assign{+}{M}$,
  $\Assign{\times}{M}$ are computable functions and $\Assign{<}{M}$ is
  a decidable relation.
\end{defn}

\begin{ex}
Recall the structure $\Struct{K}$ from \olref[mdq]{ex:model-K-of-Q}
Its domain was $\Domain{K} = \Nat
\cup \{a\}$ and interpretations
\begin{align*}
  \Assign{\Obj{0}}{K} & = 0\\
  \Assign{\prime}{K}(x) & =
  \begin{cases}
    x+1 & \text{if $x\in \Nat$}\\
    a & \text{if $x = a$}
  \end{cases}\\
  \Assign{+}{K}(x, y) & =
  \begin{cases}
    x+y & \text{if $x$, $y \in\Nat$}\\
    a & \text{otherwise}
  \end{cases}\\
  \Assign{\times}{K}(x, y) & =
  \begin{cases}
    xy & \text{if $x$, $y \in\Nat$}\\
    a & \text{otherwise}\\
  \end{cases}\\
  \Assign{<}{K} & =
  \Setabs{\tuple{x,y}}{x, y \in \Nat \text{ and } x<y} \cup
  \Setabs{\tuple{x,a}}{n \in \Domain{K}}
\end{align*}
But $\Domain{K}$ is !!{denumerable} and so is equinumerous
with~$\Nat$. For instance, $g\colon \Nat \to \Domain{K}$ with $g(0) =
a$ and $g(n) = n+1$ for $n>0$ is !!a{bijection}.  We can turn it into
an isomorphism between a new model~$\Struct{K'}$ of~$\Th{Q}$ and
$\Struct{K}$.  In $\Struct{K'}$, we have to assign different functions
and relations to the symbols of~$\Lang{L_A}$, since different
!!{element}s of~$\Nat$ play the roles of standard and non-standard
numbers.

Specifically, $0$ now plays the role of~$a$, not of the smallest
standard number. The smallest standard number is now~$1$. So we assign
$\Assign{\Obj{0}}{K'} = 1$. The successor function is also different
now: given a standard number, i.e., an $n > 0$, it still returns
$n+1$. But $0$ now plays the role of~$a$, which is its own
successor. So $\Assign{\prime}{K'}(0) = 0$.  For addition and
multiplication we likewise have
\begin{align*}
\Assign{+}{K'}(x, y) & =
  \begin{cases}
    x+y & \text{if $x$, $y >0$}\\
    0 & \text{otherwise}
  \end{cases}\\
  \Assign{\times}{K'}(x, y) & =
  \begin{cases}
    xy & \text{if $x$, $y > 0$}\\
    0 & \text{otherwise}\\
  \end{cases}
\end{align*}
And we have $\tuple{x, y} \in \Assign{<}{K'}$ iff $x < y$ and $x > 0$
and $y > 0$, or if $y = 0$.

All of these functions are computable functions of natural numbers and
$\Assign{<}{K'}$ is a decidable relation on~$\Nat$---but they are not
the same functions as successor, addition, and multiplication
on~$\Nat$, and $\Assign{<}{K'}$ is not the same relation as~$<$
on~$\Nat$.
\end{ex}

\begin{prob}
Give !!a{structure}~$\Struct{L'}$ with $\Domain{L'} = \Nat$ isomorphic
to~$\Struct{L}$ of \olref[mod][mar][mdq]{ex:model-L-of-Q}.
\end{prob}

\begin{explain}
This example shows that $\Th{Q}$ has computable non-standard models
with domain~$\Nat$.  However, the following result shows that this is
not true for models of~$\Th{PA}$ (and thus also for models
of~$\Th{TA}$).
\end{explain}

\begin{thm}[Tennenbaum's Theorem]
$\Struct{N}$ is the only computable model of~$\Th{PA}$.
\end{thm}
  
\end{document}
