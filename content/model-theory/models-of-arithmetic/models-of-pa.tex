% Part: first-order-logic
% Chapter: models-of-arithmetic
% Section: models-of-pa

\documentclass[../../../include/open-logic-section]{subfiles}

\begin{document}

\olfileid{mod}{mar}{mpa}
\section{Models of $\Th{PA}$}

\begin{explain}
Any non-standard model of~$\Th{TA}$ is also one of~$\Th{PA}$. We know
that non-standard models of~$\Th{TA}$ and hence of~$\Th{PA}$ exist. We
also know that such non-standard models contain non-standard
``numbers,'' i.e., !!{element}s of the domain that are ``beyond'' all
the standard ``numbers.''  But how are they arranged? How many are
there? We've seen that models of the weaker theory~$\Th{Q}$ can
contain as few as a single non-standard number. But these simple
!!{structure}s are not models of $\Th{PA}$ or $\Th{TA}$.

The key to understanding the structure of models of $\Th{PA}$ or
$\Th{TA}$ is to see what facts are !!{derivable} in these theories.
For instance, already $\Th{PA}$ proves that $\lforall[x][\eq/[x][x']]$
and $\lforall[x][\lforall[y][\eq[(x+y)][(y+x)]]]$, so this rules out
simple structures (in which these !!{sentence}s are false) as models
of~$\Th{PA}$.

Suppose~$\Struct{M}$ is a model of~$\Th{PA}$. Then if $\Th{PA} \Proves
!A$, $\Sat{M}{!A}$.  Let's again use $\nszero$ for
$\Assign{\Obj{0}}{M}$, $\nssucc$ for $\Assign{\prime}{M}$, $\nsplus$
for $\Assign{+}{M}$, $\nstimes$ for $\Assign{\times}{M}$, and
$\nsless$ for $\Assign{<}{M}$. Any !!{sentence}~$!A$ then states some
condition about $\nszero$, $\nssucc$, $\nsplus$, $\nstimes$, and
$\nsless$, and if $\Sat{M}{!A}$ that condition must be satisfied. For
instance, if $\Sat{M}{!Q_1}$, i.e.,
$\Sat{M}{\lforall[x][\lforall[y][(\eq[x'][y'] \lif \eq[x][y])]]}$,
then $\nssucc$ must be !!{injective}.
\end{explain}

\begin{prop}
In $\Struct{M}$, $\nsless$ is a linear strict order, i.e., it
satisfies:
\begin{enumerate}
\item Not $x \nsless x$ for any~$x \in \Domain{M}$.
\item If $x \nsless y$ and $y \nsless z$ then $x \nsless z$.
\item For any $x \neq y$, $x \nsless y$ or $y \nsless x$
\end{enumerate}
\end{prop}

\begin{proof}
$\Th{PA}$ proves:
\begin{enumerate}
\item $\lforall[x][\lnot x < x]$
\item $\lforall[x][\lforall[y][\lforall[z][((x < y \land y < z) \lif x < z)]]]$
\item $\lforall[x][\lforall[y][((x < y \lor y < x) \lor \eq[x][y]))]]$
\end{enumerate}
\end{proof}

\begin{prop}
\ollabel{prop:M-discrete} $\nszero$ is the least !!{element}
of~$\Domain{M}$ in the $\nsless$-ordering. For any $x$, $x \nsless
x^\nssucc$, and $x^\nssucc$ is the $\nsless$-least !!{element} with
that property.  For any $x$, there is a unique $y$ such that
$y^\nssucc = x$. (We call $y$ the ``predecessor'' of~$x$
in~$\Struct{M}$, and denote it by~$^\nssucc x$.)
\end{prop}

\begin{proof}
Exercise.  
\end{proof}

\begin{prob}
Find !!{sentence}s in~$\Lang{L_A}$ !!{derivable} in~$\Th{PA}$ (and
hence true in~$\Struct{N}$) which guarantee the properties of
$\nszero$, $\nssucc$, and $\nsless$ in
\olref[mod][mar][mpa]{prop:M-discrete}
\end{prob}

\begin{prop}
All standard !!{element}s of~$\Struct{M}$ are less than (according
to~$\nsless$) than all non-standard !!{element}s.
\end{prop}

\begin{proof}
We'll use $n$ as short for $\Value{\num{n}}{M}$, a standard
!!{element} of~$\Struct{M}$.  Already $\Th{Q}$ proves that, for any~$n
\in \Nat$, $\lforall[x][(x < \num{n}' \lif (\eq[x][\num{0}] \lor
  \eq[x][\num{1}] \lor \dots \lor \eq[x][\num{n}]))]$. There are no
!!{element}s that are $\nsless \nszero$. So if $n$ is standard and $x$
is non-standard, we cannot have $x \nsless n$. By definition, a
non-standard element is one that isn't $\Value{\num{n}}{M}$ for any~$n
\in \Nat$, so $x \neq n$ as well. Since $\nsless$ is a linear order,
we must have $n \nsless x$.
\end{proof}

\begin{prop}
Every nonstandard !!{element}~$x$ of~$\Domain{M}$ is an element of the subset
\[
\dots ^{\nssucc\nssucc\nssucc}x \nsless ^{\nssucc\nssucc}x \nsless
^{\nssucc}x \nsless x \nsless x^{\nssucc} \nsless x^{\nssucc\nssucc}
\nsless x^{\nssucc\nssucc\nssucc} \nsless \dots
\]
We call this subset the \emph{block of~$x$} and write it as $[x]$. It
has no least and no greatest !!{element}. It can be characterized as
the set of those $y \in \Domain{M}$ such that, for some standard~$n$,
$x \nsplus n = y$ or $y \nsplus n = x$.
\end{prop}

\begin{proof}
Clearly, such a set~$[x]$ always exists since every !!{element}~$y$ of
$\Domain{M}$ has a unique successor~$y^\nssucc$ and unique
predecessor~$^\nssucc y$. For successive !!{element}s $y$, $y^\nssucc$
we have $y \nsless y^\nssucc$ and $y^\nssucc$ is the $\nsless$-least
!!{element} of~$\Domain{M}$ such that $y$ is $\nsless$-less than it.
Since always $^\nssucc y \nsless y$ and $y \nsless y^\nssucc$, $[x]$
has no least or greatest !!{element}. If $y \in [x]$ then $x \in [y]$,
for then either $y^{\nssucc\dots\nssucc} = x$ or
$x^{\nssucc\dots\nssucc} = y$.  If $y^{\nssucc\dots\nssucc} = x$ (with
$n$ $\nssucc$'s), then $y \nsplus n = x$ and conversely, since
$\Th{PA} \Proves \lforall[x][\eq[x^{\prime\dots\prime}][(x +
    \num{n})]]$ (if $n$ is the number of~$\prime$'s).
\end{proof}

\begin{prop}
If $[x] \neq [y]$ and $x \nsless y$, then for any $u \in [x]$ and any
$v \in [y]$, $u \nsless v$.
\end{prop}

\begin{proof}
Note that $\Th{PA} \Proves \lforall[x][\lforall[y][(x < y \lif (x' < y
    \lor x' = y))]]$. Thus, if $u \nsless v$, we also have $u \nsplus
n^\nssucc \nsless v$ for any~$n$ if $[u] \neq [v]$.

Any $u \in [x]$ is $\nsless y$: $x \nsless y$ by assumption. If $u
\nsless x$, $u \nsless y$ by transitivity. And if $x \nsless u$ but $u
\in [x]$, we have $u = x\nsplus n^\nssucc$ for some~$n$, and so $u
\nsless y$ by the fact just proved.
  
Now suppose that $v \in [y]$ is $\nsless y$, i.e., $v \nsplus
m^\nssucc = y$ for some standard~$m$. This rules out $v \nsless x$,
otherwise $y = v \nsplus m^\nssucc \nsless x$. Clearly also, $x \neq
v$, otherwise $x \nsplus m^\nssucc = v \nsplus m^\nssucc = y$ and we
would have $[x] = [y]$. So, $x \nsless v$. But then also $x \nsplus
n^\nssucc \nsless v$ for any~$n$. Hence, if $x \nsless u$ and $u \in
[x]$, we have $u \nsless v$. If $u \nsless x$ then $u \nsless v$ by
transitivity.

Lastly, if $y \nsless v$, $u \nsless v$ since, as we've shown, $u
\nsless y$ and $y \nsless v$.
\end{proof}

\begin{cor}
If $[x] \neq [y]$, $[x] \cap [y] = \emptyset$.
\end{cor}

\begin{proof}
Suppose $z \in [x]$ and $x \nsless y$. Then $z \nsless u$ for all $u
\in [y]$. If $z \in [y]$, we would have $z \nsless z$. Similarly if $y
\nsless x$.
\end{proof}

\begin{explain}
This means that the blocks themselves can be ordered in a way that
respects $\nsless$: $[x] \nsless [y]$ iff $x \nsless y$, or,
equivalently, if $u \nsless v$ for any $u \in [x]$ and $v \in [y]$.
Clearly, the standard block $[0]$ is the least block. It intersects
with no non-standard block, and no two non-standard blocks intersect
either. Specifically, you cannot ``reach'' a different block by taking
repeated successors or predecessors.
\end{explain}

\begin{prop}
If $x$ and $y$ are non-standard, then $x \nsless x \nsplus y$ and $x
\nsplus y \notin [x]$.
\end{prop}

\begin{proof}
If $y$ is nonstandard, then $y \neq \nszero$. $\Th{PA} \Proves
\lforall[x][(y \neq \Obj{0} \lif x < (x+y))]$.  Now suppose $x
\nsplus y \in [x]$. Since $x \nsless x \nsplus y$, we would have $x
\nsplus n^\nssucc = x \nsplus y$. But $\Th{PA} \Proves
\lforall[x][\lforall[y][\lforall[z][(\eq[(x+y)][(x+z)] \lif y = z)]]]$
(the cancellation law for addition). This would mean $y = n^\nssucc$
for some standard~$n$; but $y$ is assumed to be non-standard.
\end{proof}

\begin{prop}
There is no least non-standard block.
\end{prop}

\begin{proof}
$\Th{PA} \Proves \lforall[x][\lexists[y][(\eq[(y+y)][x] \lor
      \eq[(y+y)'][x])]]$, i.e., that every $x$ is divisible by~$2$
  (possibly with remainder~$1$. If $x$ is non-standard, so is~$y$. By
  the preceding proposition, $y \nsless y \nsplus y$ and $y \nsplus y
  \notin [y]$. Then also $y \nsless (y \nsplus y)^\nssucc$ and $(y
  \nsplus y)^\nssucc \notin [y]$. But $x = y \nsplus y$ or $x = (y
  \nsplus y)^\nssucc$, so $y \nsless x$ and $y \notin [x]$.
\end{proof}

\begin{prop}
There is no largest block.
\end{prop}

\begin{proof}
Exercise.
\end{proof}

\begin{prob}
Show that in a non-standard model of~$\Th{PA}$, there is no largest
block.
\end{prob}

\begin{prop}
\ollabel{prop:blocks-dense}
The ordering of the blocks is dense. That is, if $x \nsless y$ and
$[x] \neq [y]$, then there is a block $[z]$ distinct from both that is
between them.
\end{prop}

\begin{proof}
Suppose $x \nsless y$. As before, $x \nsplus y$ is divisible by two
(possibly with remainder): there is a $z \in \Domain{M}$ such that
either $x \oplus y = z \oplus z$ or $x \oplus y = (z \oplus
z)^\nssucc$. The element $z$ is the ``average'' of $x$ and~$y$, and $x
\nsless z$ and $z \nsless y$.
\end{proof}

\begin{prob}
Write out a detailed proof of
\olref[mod][mar][mpa]{prop:blocks-dense}. Which !!{sentence} must
$\Th{PA}$ !!{derive} in order to guarantee the existence of~$z$? Why
is $x \nsless z$ and $z \nsless y$, and why is $[x] \neq [z]$ and $[z]
\neq [y]$?
\end{prob}

\begin{explain}
The non-standard blocks are therefore ordered like the rationals: they
form !!a{denumerable} linear ordering without endpoints.  One can show
that any two such !!{denumerable} orderings are isomorphic. It follows
that for any two !!{enumerable} non-standard models $\Struct{M}_1$ and
$\Struct{M_2}$ of true arithmetic, their reducts to the language
containing $<$ and $=$ only are isomorphic. Indeed, an isomorphism $h$
can be defined as follows: the standard parts of $\Struct{M_1}$ and
$\Struct{M_2}$ are isomorphic to the standard model $\Struct{N}$ and
hence to each other. The blocks making up the non-standard part are
themselves ordered like the rationals and therefore isomorphic; an
isomorphism of the blocks can be extended to an isomorphism
\emph{within} the blocks by matching up arbitrary elements in each,
and then taking the image of the successor of $x$ in $\Struct{M_1}$ to
be the successor of the image of $x$ in $\Struct{M_2}$. Note that it
does \emph{not} follow that $\mathfrak{M}_1$ and $\mathfrak{M}_2$ are
isomorphic in the full language of arithmetic (indeed, isomorphism is
always relative to !!a{language}), as there are non-isomorphic ways to
define addition and multiplication over $\Domain{M_1}$ and
$\Domain{M_2}$. (This also follows from a famous theorem due to Vaught
that the number of countable models of a complete theory cannot be~2.)
\end{explain}
\end{document}
