% Part: methods
% Chapter: proofs
% Section: using-definitions

\documentclass[../../../include/open-logic-section]{subfiles}

\begin{document}

\olfileid{mth}{prf}{def}

\olsection{Using Definitions}

We mentioned that you must be familiar with all definitions that may
be used in the proof, and that you can properly apply them. This is a
really important point, and it is worth looking at in a bit more
detail. Definitions are used to abbreviate properties and relations so
we can talk about them more succinctly. The introduced abbreviation is
called the \emph{definiendum}, and what it abbreviates is the
\emph{definiens}.  In proofs, we often have to go back to how the
definiendum was introduced, because we have to exploit the logical
structure of the definiens (the long version of which the defined term
is the abbreviation) to get through our proof.  By unpacking
definitions, you're ensuring that you're getting to the heart of where
the logical action is.

We'll start with an example. Suppose you want to prove the following:

\begin{prop}
For any sets $X$ and $Y$, $X \cup Y = Y \cup X$.
\end{prop}

In order to even start the proof, we need to know what it means for
two sets to be identical; i.e., we need to know what the ``$=$'' in
that equation means for sets.  Sets are defined to be identical
whenever they have the same !!{element}s.  So the definition we have
to unpack is:

\begin{defn}
Sets $X$ and $Y$ are \emph{identical}, $X = Y$, if every !!{element}
of~$X$ is !!a{element} of~$Y$, and vice versa.
\end{defn}

This definition uses $X$ and~$Y$ as placeholders for arbitrary
sets. What it defines---the \emph{definiendum}---is the expression
``$X = Y$'' by giving the condition under which $X = Y$ is true.  This
condition---``every !!{element} of~$X$ is !!a{element} of~$Y$, and
vice versa''---is the \emph{definiens}.\footnote{In this particular
  case---and very confusingly!---when $X = Y$, the sets $X$ and $Y$
  are just one and the same set, even though we use different letters
  for it on the left and the right side.  But the ways in which that
  set is picked out may be different, and that makes the definition
  non-trivial.}

When you apply the definition, you have to match the $X$ and $Y$ in
the definition to the case you're dealing with.  In our case, it means
that in order for $X \cup Y = Y \cup X$ to be true, each $z \in X \cup
Y$ must also be in $Y \cup X$, and vice versa.  The expression $X \cup
Y$ in the proposition plays the role of~$X$ in the definition, and $Y
\cup X$ that of~$Y$. Since $X$ and $Y$ are used both in the definition
and in the statement of the proposition we're proving, but in
different uses, you have to be careful to make sure you don't mix up
the two.  For instance, it would be a mistake to think that you could
prove the proposition by showing that every !!{element} of~$X$ is
!!a{element} of~$Y$, and vice versa---that would show that $X = Y$,
not that $X \cup Y = Y \cup X$. (Also, since $X$ and $Y$ may be any
two sets, you won't get very far, because if nothing is assumed about
$X$ and~$Y$ they may well be different sets.)

Within the proof we are dealing with set-theoretic notions such as
union, and so we must also know the meanings of the symbol $\cup$ in
order to understand how the proof should proceed. And sometimes,
unpacking the definition gives rise to further definitions to
unpack. For instance, $X \cup Y$ is defined as $\Setabs{z}{z \in X
  \text{ or } z \in Y}$. So if you want to prove that $x \in X \cup
Y$, unpacking the definition of $\cup$ tells you that you have to
prove $x \in \Setabs{z}{z \in X \text{ or } z \in Y}$.  Now you also
have to remember that $x \in \Setabs{z}{\dots z\dots}$ iff $\dots
x\dots$.  So, further unpacking the definition of the
$\Setabs{z}{\dots z \dots}$ notation, what you have to show is: $x \in
X$ or $x \in Y$. So, ``every !!{element} of $X \cup Y$ is also
!!a{element} of $Y \cup X$'' really means: ``for every $x$, if $x \in
X$ or $x \in Y$, then $x \in Y$ or $x \in X$.''  If we fully
unpack the definitions in the proposition, we see that what we have to
show is this:

\begin{prop}
For any sets $X$ and $Y$: (a) for every $x$, if $x \in X$ or $x \in
Y$, then $x \in Y$ or $x \in X$, and (b) for every $x$, if $x \in Y$
or $x \in X$, then $x \in X$ or $x \in Y$.
\end{prop}

What's important is that unpacking definitions is a necessary part of
constructing a proof. Properly doing it is sometimes difficult: you
must be careful to distinguish and match the variables in the
definition and the terms in the claim you're proving.  In order to be
successful, you must know what the question is asking and what all the
terms used in the question mean---you will often need to unpack more
than one definition.  In simple proofs such as the ones below, the
solution follows almost immediately from the definitions
themselves. Of course, it won't always be this simple.

\begin{prob}
Suppose you are asked to prove that $X \cap Y \neq \emptyset$. Unpack
all the definitions occuring here, i.e., restate this in a way that
does not mention ``$\cap$'', ``='', or ``$\emptyset$.
\end{prob}

\end{document}
