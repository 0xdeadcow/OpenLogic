% Part: methods
% Chapter: proofs
% Section: reading-proofs

\documentclass[../../../include/open-logic-section]{subfiles}

\begin{document}

\olfileid{mth}{prf}{rea}
\olsection{Reading Proofs}

Proofs you find in textbooks and articles very seldom give all the
details we have so far included in our examples. Authors often do not
draw attention to when they distinguish cases, when they give an
indirect proof, or don't mention that they use a definition.  So when
you read a proof in a textbook, you will often have to fill in those
details for yourself in order to understand the proof. Doing this is
also good practice to get the hang of the various moves you have to
make in a proof. Let's look at an example.

\begin{prop}[Absorption]
For all sets $A$, $B$,
\[
A \cap (A \cup B) = A
\]
\end{prop}

\begin{proof}
If $z \in A \cap (A \cup B)$, then $z \in A$, so $A \cap (A \cup B)
\subseteq A$. Now suppose $z \in A$. Then also $z \in A \cup B$, and
therefore also $z \in A \cap (A \cup B)$.
\end{proof}

The preceding proof of the absorption law is very condensed. There is
no mention of any definitions used, no ``we have to prove that''
before we prove it, etc. Let's unpack it.  The proposition proved is a
general claim about any sets $A$ and $B$, and when the proof mentions
$A$ or $B$, these are variables for arbitrary sets.  The general
claims the proof establishes is what's required to prove identity of
sets, i.e., that every !!{element} of the left side of the identity is
!!a{element} of the right and vice versa.  

\begin{quote}
``If $z \in A \cap (A \cup B)$, then $z \in A$, so $A \cap (A \cup B)
  \subseteq A$.''
\end{quote}

This is the first half of the proof of the identity: it establishes
that if an arbitrary~$z$ is !!a{element} of the left side, it is also
!!a{element} of the right, i.e., $A \cap (A \cup B) \subseteq A$.
Assume that $z \in A \cap (A \cup B)$. Since $z$ is an !!{element} of
the intersection of two sets iff it is an !!{element} of both sets, we
can conclude that $z \in A$ and also $z \in A \cup B$. In particular,
$z \in A$, which is what we wanted to show.  Since that's all that has
to be done for the first half, we know that the rest of the proof must
be a proof of the second half, i.e., a proof that $A \subseteq A \cap
(A \cup B)$.

\begin{quote}
``Now suppose $z \in A$. Then also $z \in A \cup B$, and
therefore also $z \in A \cap (A \cup B)$.''
\end{quote}

We start by assuming that $z \in A$, since we are showing that, for
any~$z$, if $z \in A$ then $z \in A \cap (A \cup B)$.  To show that $z
\in A \cap (A \cup B)$, we have to show (by definition of ``$\cap$'')
that (i) $z \in A$ and also (ii) $z \in A \cup B$. Here (i) is just
our assumption, so there is nothing further to prove, and that's why
the proof does not mention it again. For (ii), recall that $z$ is
!!a{element} of a union of sets iff it is an !!{element} of at least
one of those sets. Since $z \in A$, and $A \cup B$ is the union of $A$
and $B$, this is the case here. So $z \in A \cup B$. We've shown both
(i) $z \in A$ and (ii) $z \in A \cup B$, hence, by definition of
``$\cap$,'' $z \in A \cap (A \cup B)$.  The proof doesn't mention
those definitions; it's assumed the reader has already internalized
them.  If you haven't, you'll have to go back and remind yourself what
they are. Then you'll also have to recognize why it follows from $z
\in A$ that $z \in A \cup B$, and from $z \in A$ and $z \in A \cup B$
that $z \in A \cap (A \cup B)$.

Here's another version of the proof above, with everything made
explicit:
\begin{proof}{}
[By definition of $=$ for sets, $A \cap (A \cup B) = A$ we have to
  show (a) $A \cap (A \cup B) \subseteq A$ and (b) $A \cap (A \cup B)
  \subseteq A$. (a): By definition of $\subseteq$, we have to show
  that if $z \in A \cap (A \cup B)$, then $z \in A$.]  If $z \in A
\cap (A \cup B)$, then $z \in A$ [since by definition of $\cap$, $z
  \in A \cap (A \cup B)$ iff $z \in A$ and $z \in A \cup B$], so $A
\cap (A \cup B) \subseteq A$. [(b): By definition of $\subseteq$, we
  have to show that if $z \in A$, then $z \in A \cap (A \cup B)$.] Now
suppose [(1)] $z \in A$. Then also [(2)] $z \in A \cup B$ [since by
  (1) $z \in A$ or $z \in B$, which by definition of $\cup$ means $z
  \in A \cup B$], and therefore also $z \in A \cap (A \cup B)$ [since
  the definition of $\cap$ requires that $z \in A$, i.e., (1), and $z
  \in A \cup B)$, i.e., (2)].
\end{proof}

\begin{prob}
Expand the following proof of $A \cup (A \cap B) = A$, where you
mention all the inference patterns used, why each step follows from
assumptions or claims established before it, and where we have to
appeal to which definitions.
\begin{proof}
  If $z \in A \cup (A \cap B)$ then $z \in A$ or $z \in A \cap B$. If
  $z \in A \cap B$, $z \in A$. Any $z \in A$ is also $\in A \cup (A
  \cap B)$.
\end{proof}
\end{prob}

\end{document}
