% Part: methods
% Chapter: induction
% Section: relations

\documentclass[../../../include/open-logic-section]{subfiles}

\begin{document}

\olfileid{mth}{ind}{rel}

\olsection{Relations and Functions}

When we have defined a set of objects (such as the natural numbers or
the nice terms) inductively, we can also define \emph{relations on}
these objects by induction.  For instance, consider the following
idea: a nice term~$t_1$ is a subterm of a nice term~$t_2$ if it occurs as
a part of it.  Let's use a symbol for it: $t_1 \sqsubseteq t_2$. Every
nice term is a subterm of itself, of course: $t \sqsubseteq t$. We can
give an inductive definition of this relation as follows:

\begin{defn}
  The relation of a nice term~$t_1$ being a subterm of~$t_2$, $t_1
  \sqsubseteq t_2$, is defined by induction on~$t_2$ as follows:
  \begin{enumerate}
  \item If $t_2$ is a letter, then $t_1 \sqsubseteq t_2$ iff $t_1 = t_2$.
  \item If $t_2$ is $[s_1 \circ s_2]$, then $t_1 \sqsubseteq t_2$ iff $t =
    t_2$, $t_1 \sqsubseteq s_1$, or $t_1 \sqsubseteq s_2$.
  \end{enumerate}
\end{defn}

This definition, for instance, will tell us that $\mathrm{a}
\sqsubseteq [\mathrm{b} \circ \mathrm{a}]$. For (2) says that
$\mathrm{a} \sqsubseteq [\mathrm{b} \circ \mathrm{a}]$ iff $\mathrm{a}
= [\mathrm{b} \circ \mathrm{a}]$, or $\mathrm{a} \sqsubseteq b$, or
$\mathrm{a} \sqsubseteq \mathrm{a}$. The first two are false:
$\mathrm{a}$ clearly isn't identical to $[\mathrm{b} \circ
  \mathrm{a}]$, and by~(1), $\mathrm{a} \sqsubseteq \mathrm{b}$ iff
$\mathrm{a} = \mathrm{b}$, which is also false. However, also by~(1),
$\mathrm{a} \sqsubseteq \mathrm{a}$ iff $\mathrm{a} = \mathrm{a}$,
which is true.

It's important to note that the success of this definition depends on
a fact that we haven't proved yet: every nice term~$t$ is either a
letter by itself, or there are \emph{uniquely determined} nice terms
$s_1$ and $s_2$ such that $t = [s_1 \circ s_2]$.  ``Uniquely
determined'' here means that if $t = [s_1 \circ s_2]$ it isn't
\emph{also} $= [r_1 \circ r_2]$ with $s_1 \neq r_1$ or $s_2 \neq r_2$.
If this were the case, then clause~(2) may come in conflict with
itself: reading $t_2$ as $[s_1 \circ s_2]$ we might get $t_1
\sqsubseteq t_2$, but if we read $t_2$ as $[r_1 \circ r_2]$ we might
get not $t_1 \sqsubseteq t_2$.  Before we prove that this can't
happen, let's look at an example where it \emph{can} happen.

\begin{defn}
  Define \emph{bracketless terms} inductively by
  \begin{enumerate}
  \item Every letter is a bracketless term.
    \item If $s_1$ and $s_2$ are bracketless terms, then $s_1 \circ s_2$ is
      a bracketless term.
    \item Nothing else is a bracketless term.
  \end{enumerate}
\end{defn}

Bracketless terms are, e.g., $\mathrm{a}$, $\mathrm{b} \circ
\mathrm{d}$, $\mathrm{b} \circ \mathrm{a} \circ \mathrm{b}$. Now if we
defined ``subterm'' for bracketless terms the way we did above, the
second clause would read
\begin{center}
  If $t_2 = s_1 \circ s_2$, then $t_1 \sqsubseteq t_2$ iff $t_1 = t_2$, $t_1
  \sqsubseteq s_1$, or $t_1 \sqsubseteq s_2$.
\end{center}

Now $\mathrm{b} \circ \mathrm{a} \circ \mathrm{b}$ is of the form $s_1
\circ s_2$ with 
\begin{align*}
  s_1 & = \mathrm{b} \text{ and} & s_2 & = \mathrm{a} \circ \mathrm{b}.
\intertext{It is also of the form $r_1 \circ r_2$ with}
  r_1 & = \mathrm{b} \circ \mathrm{a} \text{ and} & r_2 &= \mathrm{b}.
\end{align*}
Now is $\mathrm{a} \circ \mathrm{b}$ a subterm of $\mathrm{b} \circ
\mathrm{a} \circ \mathrm{b}$?  The answer is yes if we go by the first
reading, and no if we go by the second.

The property that the way a nice term is built up from other nice
terms is unique is called \emph{unique readability}. Since inductive
definitions of relations for such inductively defined objects are
important, we have to prove that it holds.

\begin{prop}
  Suppose $t$ is a nice term. Then either $t$ is a letter by itself,
  or there are uniquely determined nice terms $s_1$, $s_2$ such that~$t =
  [s_1 \circ s_2]$.
\end{prop}

\begin{proof}
  If $t$ is a letter by itself, the condition is satisfied. So assume
  $t$ isn't a letter by itself. We can tell from the inductive
  definition that then $t$ must be of the form $[s_1 \circ s_2]$ for some
  nice terms $s_1$ and~$s_2$. It remains to show that these are uniquely
  determined, i.e., if $t = [r_1 \circ r_2]$, then $s_1 = r_1$ and $s_2 = r_2$.

  So suppose $t = [s_1 \circ s_2]$ and also $t = [r_1 \circ r_2]$ for nice terms
  $s_1$, $s_2$, $r_1$, $r_2$. We have to show that $s_1 = r_1$ and $s_2 = r_2$.
  First, $s_1$ and $r_1$ must be identical, for otherwise one is a proper
  initial segment of the other. But by \olref[sti]{prop:initial}, that
  is impossible if $s_1$ and~$r_1$ are both nice terms.  But if $s_1 = r_1$,
  then clearly also $s_2 = r_2$.
\end{proof}

We can also define functions inductively: e.g., we can define the
function~$f$ that maps any nice term to the maximum depth of nested
$[\dots]$ in it as follows:

\begin{defn}
  \ollabel{defn:depth} The \emph{depth} of a nice term, $f(t)$, is
  defined inductively as follows:
  \[
  f(t) = \begin{cases}
    0 & \text{ if $t$ is a letter}\\
    \max(f(s), f(s')) + 1 & \text{ if $t = [s_1 \circ s_2]$.}
  \end{cases}
  \]
\end{defn}

For instance
\begin{align*}
  f([\mathrm{a} \circ \mathrm{b}]) & = 
    \max(f(\mathrm{a}),f(\mathrm{b})) + 1 = \\ 
  &= \max(0, 0) + 1 = 1, \text{ and}\\
  f([[\mathrm{a} \circ \mathrm{b}] \circ \mathrm{c}]) & = 
    \max(f([\mathrm{a} \circ \mathrm{b}]), f(\mathrm{c})) + 1 = \\ 
  & = \max(1,0) + 1 = 2.
\end{align*}

Here, of course, we assume that $s_1$ an $s_2$ are nice terms, and make
use of the fact that every nice term is either a letter or of the form
$[s_1 \circ s_2]$. It is again important that it can be of this form in
only one way. To see why, consider again the bracketless terms we
defined earlier. The corresponding ``definition'' would be:
\[
  g(t) = 
  \begin{cases}
    0 & \text{ if $t$ is a letter}\\
   \max(g(s), g(s')) + 1 & \text{ if $t = [s_1 \circ s_2]$.}
  \end{cases}
\]
Now consider the bracketless term $\mathrm{a} \circ \mathrm{b} \circ
\mathrm{c} \circ \mathrm{d}$. It can be read in more than one way,
e.g., as $s_1 \circ s_2$ with 
\begin{align*}
  s_1 & = \mathrm{a} \text{ and} & 
  s_2 & = \mathrm{b} \circ \mathrm{c} \circ \mathrm{d},
\intertext{or as $r_1 \circ r_2$ with}
  r_1 & = \mathrm{a} \circ b \text{ and} & 
  r_2 &= \mathrm{c} \circ \mathrm{d}.
\end{align*}
Calculating $g$ according to the first way of reading it would give
\begin{align*}
  g(s_1 \circ s_2) & =
  \max(g(\mathrm{a}), g(\mathrm{b} \circ \mathrm{c} \circ \mathrm{d})) + 1 =\\ & =
  \max(0,2) + 1 = 3
  \intertext{while according to the other reading we get}
  g(r_1 \circ r_2) & =
  \max(g(\mathrm{a} \circ \mathrm{b}), g(\mathrm{c} \circ \mathrm{d})) + 1=\\ &
  = \max(1,1) + 1 = 2
\end{align*}
But a function must always yield a unique value; so our ``definition''
of~$g$ doesn't define a function at all.

\begin{prob}
  Give an inductive definition of the function $l$, where $l(t)$ is
  the number of symbols in the nice term~$t$.
\end{prob}

\begin{prob}
  Prove by structural induction on nice terms $t$ that $f(t) < l(t)$ (where
  $l(t)$ is the number of symbols in~$t$ and $f(t)$ is the depth of
  $t$ as defined in \olref[mth][ind][rel]{defn:depth}).
\end{prob}

\end{document}
