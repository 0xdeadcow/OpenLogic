% Part: methods
% Chapter: induction
% Section: introduction

\documentclass[../../../include/open-logic-section]{subfiles}

\begin{document}

\olfileid{mth}{ind}{int}

\olsection{Introduction}

Induction is an important proof technique which is used, in different
forms, in almost all areas of logic, theoretical computer science, and
mathematics.  It is needed to prove many of the results in logic.

Induction is often contrasted with deduction, and characterized as the
inference from the particular to the general.  For instance, if we
observe many green emeralds, and nothing that we would call an emerald
that's not green, we might conclude that all emeralds are green. This
is an inductive inference, in that it proceeds from many particlar
cases (this emerald is green, that emerald is green, etc.) to a
general claim (all emeralds are green).  \emph{Mathematical} induction
is also an inference that concludes a general claim, but it is of a
very different kind that this ``simple induction.''

Very roughly, and inductive proof in mathematics concludes that all
mathematical objects of a certain sort have a certain property.  In
the simplest case, the mathematical objects an inductive proof is
concerned with are natural numbers.  In that case an inductive proof
is used to establish that all natural numbers have some property, and
it does this by showing that (1) $0$ has the property, and (2)
whenever a number~$n$ has the property, so does~$n+1$.  Induction on
natural numbers can then also often be used to prove general about
mathematical objects that can be assigned numbers. For instance,
finite sets each have a finite number~$n$ of elements, and if we can
use induction to show that every number~$n$ has the property ``all
finite sets of size~$n$ are \dots'' then we will have shown something
about all finite sets.

Induction can also be generalized to mathematical objects that are
\emph{inductively defined}.  For instance, expressions of a formal
language suchh as those of first-order logic are defined inductively.
\emph{Structural induction} is a way to prove results about all such
expressions.  Structural induction, in particular, is very
useful---and widely used---in logic.

\end{document}
