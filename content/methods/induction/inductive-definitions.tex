% Part: methods
% Chapter: induction
% Section: inductive-definitions

\documentclass[../../../include/open-logic-section]{subfiles}

\begin{document}

\olfileid{mth}{ind}{idf}

\olsection{Inductive Definitions}

In logic we very often define kinds of objects \emph{inductively},
i.e., by specifying rules for what counts as an object of the kind to
be defined which explain how to get new objects of that kind from old
objects of that kind.  For instance, we often define special kinds of
sequences of symbols, such as the terms and !!{formula}s of a
language, by induction.  For a simpler example, consider strings of
parentheses, such as ``(()('' or ``()(())''.  In the second string,
the parentheses ``balance,'' in the first one, they don't.  The
shortest such expression is ``()''. Actually, the very shortest string
of parentheses in which every opening parenthesis has a matching
closing parenthesis is ``'', i.e., the empty sequence~$\emptyseq$.  If
we already have a parenthesis expression~$p$, then putting matching
parentheses around it makes another balanced parenthesis
expression. And if $p$ and $p'$ are two balanced parentheses
expressions, writing one after the other, ``$pp'$'' is also a balanced
parenthesis expression.  In fact, any sequence of balanced parentheses
can be generated in this way, and we might use these operations to
\emph{define} the set of such expressions.  This is an \emph{inductive
  definition}.

\begin{defn}
  The set of \emph{parexpressions} is inductively defined as follows:
  \begin{enumerate}
  \item $\emptyseq$ is a parexpression.
  \item If $p$ is a parexpression, then so is $(p)$.
  \item If $p$ and $p'$ are parexpressions $\neq \emptyseq$, then so
    is $pp'$.
  \item Nothing else is a parexpression.
  \end{enumerate}
\end{defn}

(Note that we have not yet proved that every balanced parenthesis
expression is a parexpression, although it is quite clear that every
parexpression is a balanced parenthesis expression.)

The key feature of inductive definitions is that if you want to prove
something about all parexpressions, the definition tells you which
cases you must consider.  For instance, if you are told that $q$ is a
parexpression, the inductive definition tells you what $q$ can look
like: $q$ can be $\emptyseq$, it can be $(p)$ for some other
parexpression~$p$, or it can be $pp'$ for two parexpressions~$p$ and
$p' \neq \emptyseq$. Because of clause (4), those are all the
possibilities.

When proving claims about all of an inductively defined set, the
strong form of induction becomes particularly important. For instance,
suppose we want to prove that for every parexpression of length~$n$,
the number of $($ in it is~$n/2$.  This can be seen as a claim about
all~$n$: for every $n$, the number of $($ in any parexpression of
length~$n$ is $n/2$.

\begin{prop}
  For any $n$, the number of $($ in a parexpression of length~$n$ is
  $n/2$.
\end{prop}

\begin{proof}
To prove this result by (strong) induction, we have to show that the
following conditional claim is true:
\begin{quote}
  If for every $k < n$, any parexpression of length $k$ has $k/2$
  $($'s, then any parexpression of length~$n$ has $n/2$ $($'s.
\end{quote}
To show this conditional, assume that its antecedent is true, i.e.,
assume that for any $k<n$, parexpressions of length $k$ contain $k$
$($'s.  We call this assumption the inductive hypothesis. We want to
show the same is true for parexpressions of length~$n$.

So suppose $q$ is a parexpression of length~$n$.  Because
parexpressions are inductively defined, we have three cases: (1) $q$
is $\emptyseq$, (2) $q$ is $(p)$ for some parexpression~$p$, or (3)
$q$ is $pp'$ for some parexpressions $p$ and $p' \neq \emptyseq$.
\begin{enumerate}
\item $q$ is $\emptyseq$.  Then $n = 0$, and the number of $($ in
$q$ is also $0$. Since $0 = 0/2$, the claim holds.
\item $q$ is $(p)$ for some parexpression~$p$.  Since $q$ contains two
  more symbols than $p$, $\len{p} = n - 2$, in particular, $\len{p} <
  n$, so the inductive hypothesis applies: the number of $($ in $p$ is
  $\len{p}/2$. The number of $($ in $q$ is $1 + {}$ the number of $($
  in $p$, so $= 1 + \len{p}/2$, and since $\len{p} = n - 2$, this
  gives $1 + (n-2)/2 = n/2$.
\item $q$ is $pp'$ for some parexpression~$p$ and $p' \neq \emptyseq$.
  Since neither $p$ nor $p' =\emptyseq$, both $\len{p}$ and $\len{p'}
  < n$. Thus the inductive hypothesis applies in each case: The number
  of $($ in $p$ is $\len{p}/2$, and the number of $($ in $p'$ is
  $\len{p'}/2$.  On the other hand, the number of $($ in $q$ is
  obviously the sum of the numbers of $($ in $p$ and $p'$, since $q =
  pp'$.  Hence, the number of $($ in $q$ is $\len{p}/2 + \len{p'}/2 =
  (\len{p}+\len{p'})/2 = \len{pp'}/2 = n/2$.
\end{enumerate}
In each case, we've shown that teh number of $($ in $q$ is $n/2$ (on
the basis of the inductive hypothesis). By strong induction, the
proposition follows.
\end{proof}

\end{document}
