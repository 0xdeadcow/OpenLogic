% Part: methods
% Chapter: induction
% Section: inductive-definitions

\documentclass[../../../include/open-logic-section]{subfiles}

\begin{document}

\olfileid{mth}{ind}{idf}

\olsection{Inductive Definitions}

In logic we very often define kinds of objects \emph{inductively},
i.e., by specifying rules for what counts as an object of the kind to
be defined which explain how to get new objects of that kind from old
objects of that kind.  For instance, we often define special kinds of
sequences of symbols, such as the terms and !!{formula}s of a
language, by induction.  For a simple example, consider strings of
consisting of letters $\mathrm{a}$, $\mathrm{b}$, $\mathrm{c}$,
$\mathrm{d}$, the symbol~$\circ$, and brackets $[$ and $]$, such as
``$[[\mathrm{c} \circ \mathrm{d}][$'', ``$[\mathrm{a}[]\circ]$'',
``$\mathrm{a}$'' or ``$[[\mathrm{a} \circ \mathrm{b}]\circ
\mathrm{d}]$''.  You probably feel that there's something
``wrong'' with the first two strings: the brackets don't
``balance'' at all in the first, and you might feel that the
``$\circ$'' should ``connect'' expressions that themselves make
sense. The third and fourth string look better: for every ``$[$''
there's a closing ``$]$'' (if there are any at all), and for any
$\circ$ we can find ``nice'' expressions on either side,
surrounded by a pair of parenteses.

We would like to precisely specify what counts as a ``nice term.''
First of all, every letter by itself is nice.  Anything that's not
just a letter by itself should be of the form ``$[t \circ s]$'' where
$s$ and $t$ are themselves nice. Conversely, if $t$ and $s$ are nice,
then we can form a new nice term by putting a $\circ$ between them and
surround them by a pair of brackets.  We might use these operations
to \emph{define} the set of nice terms.  This is an \emph{inductive
  definition}.

\begin{defn}[Nice terms]
  The set of \emph{nice terms} is inductively defined as follows:
  \begin{enumerate}
  \item Any letter $\mathrm{a}$, $\mathrm{b}$, $\mathrm{c}$,
    $\mathrm{d}$ is a nice term.
  \item If $s$ and $s'$ are nice terms, then so
    is $[s \circ s']$.
  \item Nothing else is a nice term.
  \end{enumerate}
\end{defn}

This definition tells us that something counts as a nice term iff it
can be constructed according to the two conditions (1) and~(2) in some
finite number of steps. In the first step, we construct all nice terms
just consisting of letters by themselves, i.e.,
\[
\mathrm{a}, \mathrm{b}, \mathrm{c}, \mathrm{d}
\]
In the second step, we apply (2) to the terms we've constructed. We'll get
\[
[\mathrm{a} \circ \mathrm{a}], [\mathrm{a} \circ \mathrm{b}],
[\mathrm{b} \circ \mathrm{a}], \dots, [\mathrm{d} \circ \mathrm{d}]
\]
for all combinations of two letters. In the third step, we apply (2)
again, to any two nice terms we've constructed so far. We get new nice
term such as $[\mathrm{a} \circ [\mathrm{a} \circ
    \mathrm{a}]]$---where $t$ is $\mathrm{a}$ from step~1 and $s$ is
$[\mathrm{a} \circ \mathrm{a}]$ from step~2---and $[[\mathrm{b} \circ
    \mathrm{c}] \circ [\mathrm{d} \circ \mathrm{b}]]$ constructed out
of the two terms $[\mathrm{b} \circ \mathrm{c}]$ and $[\mathrm{d}
  \circ \mathrm{b}]$ from step~2. And so on.  Clause (3) rules out
that anything not constructed in this way sneaks into the set of nice
terms.

Note that we have not yet proved that every sequence of symbols that
``feels'' nice is nice according to this definition. However, it
should be clear that everything we can construct does in fact ``feel
nice:'' brackets are balanced, and $\circ$ connects parts that are
themselves nice.

The key feature of inductive definitions is that if you want to prove
something about all nice terms, the definition tells you which cases
you must consider.  For instance, if you are told that $t$ is a nice
term, the inductive definition tells you what $t$ can look like: $t$
can be a letter, or it can be $[r \circ s]$ for some other pair of
nice terms $r$ and~$s$. Because of clause (3), those are the only
possibilities.

When proving claims about all of an inductively defined set, the
strong form of induction becomes particularly important. For instance,
suppose we want to prove that for every nice term of length~$n$, the
number of $[$ in it is~$< n/2$.  This can be seen as a claim about
all~$n$: for every $n$, the number of $[$ in any nice term of
length~$n$ is $< n/2$.

\begin{prop}
  For any $n$, the number of $[$ in a nice term of length~$n$ is
  $< n/2$.
\end{prop}

\begin{proof}
To prove this result by (strong) induction, we have to show that the
following conditional claim is true:
\begin{quote}
  If for every $k < n$, any nice term of length $k$ has $k/2$
  $[$'s, then any nice term of length~$n$ has $n/2$ $[$'s.
\end{quote}
To show this conditional, assume that its antecedent is true, i.e.,
assume that for any $k<n$, nice terms of length $k$ contain $< k/2$
$[$'s.  We call this assumption the inductive hypothesis. We want to
show the same is true for nice terms of length~$n$.

So suppose $t$ is a nice term of length~$n$.  Because nice terms are
inductively defined, we have two cases: (1)~$t$ is a letter by
itself, or (2)~$t$ is $[r \circ s]$ for some nice terms $r$ and~$s$.
\begin{enumerate}
\item $t$ is a letter.  Then $n = 1$, and the number of $[$ in $t$ is
  $0$. Since $0 < 1/2$, the claim holds.
\item $t$ is $[r \circ s]$ for some nice terms $r$ and~$s$.  Let's
  let $k$ be the length of~$r$ and $l$ be the length of~$s$.  Then
  the length~$n$ of $t$ is $k+l+3$ (the lengths of $r$ and $s$ plus
  three symbols $[$, $\circ$, $]$). Since $k+l+3$ is always greater
  than $k$, $k < n$. Similarly, $l < n$. That means that the
  induction hypothesis applies to the terms $r$ and~$s$: the
  number~$m$ of $[$ in $r$ is $< k/2$, and the number~$o$ of $[$ in~$s$ is
  $< k'/2$.

  The number of $[$ in $t$ is the number of $[$ in $r$, plus the
  number of $[$ in $s$, plus~$1$, i.e., it is $m + o + 1$. Since $m
  < k/2$ and $o < l/2$ we have:
  \[
  m + o + 1 < \frac{k}{2} + \frac{l}{2} + 1 = \frac{k+l+2}{2} < \frac{k+l+3}{2} = n/2.
  \]
\end{enumerate}
In each case, we've shown that the number of $[$ in $t$ is $< n/2$ (on
the basis of the inductive hypothesis). By strong induction, the
proposition follows.
\end{proof}

\begin{prob}
  Define the set of supernice terms by
  \begin{enumerate}
  \item Any letter $\mathrm{a}$, $\mathrm{b}$, $\mathrm{c}$,
    $\mathrm{d}$ is a supernice term.
  \item If $s$ is a supernice term, then so is $[s]$.
  \item If $t$ and $s$ are supernice terms, then so
    is $[t \circ s]$.
  \item Nothing else is a supernice term.
  \end{enumerate}
  Show that the number of $[$ in a supernice term~$s$ of length~$n$ is
    $\le n/2 +1$.
\end{prob}

\end{document}
