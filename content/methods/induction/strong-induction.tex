% Part: methods
% Chapter: induction
% Section: strong-induction

\documentclass[../../../include/open-logic-section]{subfiles}

\begin{document}

\olfileid{mth}{ind}{str}

\olsection{Strong Induction}

In the principle of induction discussed above, we prove $P(0)$ and
also if $P(n)$, then $P(n+1)$.  In the second part, we assume that
$P(n)$ is true and use this assumption to prove $P(n+1)$.
Equivalently, of course, we could assume $P(n-1)$ and use it to prove
$P(n)$---the important part is that we be able to carry out the
inference from any number to its successor; that we can prove the
claim in question for any number under the assumption it holds for its
predecessor.

There is a variant of the principle of induction in which we don't
just assume that the claim holds for the predecessor $n-1$ of $n$, but
for all numbers smaller than~$n$, and use this assumption to establish
the claim for~$n$. This also gives us the claim $P(k)$ for all~$k \in
\Nat$.  For once we have established $P(0)$, we have thereby
established that $P$ holds for all numbers less than~$1$.  And if we
know that if $P(l)$ for all $l<n$ then $P(n)$, we know this in
particular for $n=1$.  So we can conclude $P(2)$.  With this we have
proved $P(0)$, $P(1)$, $P(2)$, i.e., $P(l)$ for all $l<3$, and since
we have also the conditional, if $P(l)$ for all $l<3$, then $P(3)$, we
can conclude~$P(3)$, and so on.

In fact, if we can establish the general conditional ``for all $n$, if
$P(l)$ for all $l<n$, then $P(n)$,'' we do not have to establish
$P(0)$ anymore, since it follows from it.  For remember that a general
claim like ``for all $l<n$, $P(l)$'' is true if there are no $l<n$.
This is a case of vacuous quantification: ``all $A$s are $B$s'' is
true if there are no $A$s, $\lforall[x][(!A(x) \lif !B(x))]$ is true
if no $x$ satisfies~$!A(x)$. In this case, the formalized version
would be ``$\lforall[l][(l < n \lif P(l))]$''---and that is true if
there are no $l < n$.  And if $n=0$ that's exactly the case: no $l<0$,
hence ``for all $l<0$, $P(0)$'' is true, whatever $P$ is.  A proof of
``if $P(l)$ for all $l<n$, then $P(n)$'' thus automatically
establishes~$P(0)$.

This variant is useful if establishing the claim for~$n$ can't be made
to just rely on the claim for $n-1$ but may require the assumption
that it is true for one or more $l<n$.  

\end{document}
