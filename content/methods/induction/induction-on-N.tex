% Part: methods
% Chapter: induction
% Section: induction-on-N

\documentclass[../../../include/open-logic-section]{subfiles}

\begin{document}

\olfileid{mth}{ind}{inN}

\olsection{Induction on~$\Nat$}

In its simplest form, induction is a technique used to prove results
for all natural numbers. It uses the fact that by starting from $0$
and repeatedly adding~$1$ we eventually reach every natural number.  So
to prove that something is true for every number, we can (1) establish
that it is true for $0$ and (2) show that whenever it is true for a number~$n$,
it is also true for the next number~$n+1$.  If we abbreviate ``number~$n$ has
property $P$'' by $P(n)$, then a proof by induction that $P(n)$ for
all $n \in \Nat$ consists of:
\begin{enumerate}
\item a proof of $P(0)$, and
\item a proof that, for any~$n$, if $P(n)$ then $P(n+1)$.
\end{enumerate}
To make this crystal clear, suppose we have both (1) and~(2).  Then
(1) tells us that $P(0)$ is true.  If we also have (2), we know in
particular that if $P(0)$ then $P(0+1)$, i.e., $P(1)$. (This follows
from the general statement ``for any~$n$, if $P(n)$ then $P(n+1)$'' by
putting~$0$ for~$n$.  So by modus ponens, we have that~$P(1)$.  From
(2) again, now taking $1$ for~$n$, we have: if $P(1)$ then~$P(2)$.
Since we've just established~$P(1)$, by modus ponens, we
have~$P(2)$. And so on.  For any number $k$, after doing this $k$
steps, we eventually arrive at~$P(k)$.  So (1) and (2) together
establish~$P(k)$ for any $k \in \Nat$.

Let's look at an example.  Suppose we want to find out how many
different sums we can throw with $n$ dice.  Although it might seem
silly, let's start with $0$ dice.  If you have no dice there's only one
possible sum you can ``throw'': no dots at all, which sums to~$0$. So
the number of different possible throws is~$1$. If you have only one
die, i.e., $n=1$, there are six possible values, $1$ through~$6$. With
two dice, we can throw any sum from $2$ through $12$, that's $11$
possibilities.  With three dice, we can throw any number from $3$ to
$18$, i.e., $16$ different possibilities.  $1$, $6$, $11$, $16$: looks
like a pattern: maybe the answer is $5n+1$?  Of course, $5n+1$ is the
maximum possible, because there are only $5n+1$ numbers between $n$,
the lowest value you can throw with $n$ dice (all $1$'s) and $6n$, the
highest you can throw (all $6$'s).

\begin{thm}
  With $n$ dice one can throw all $5n+1$ possible values between $n$
  and $6n$.
\end{thm}

\begin{proof}
Let $P(n)$ be the claim: ``It is possible to throw any number between
$n$ and~$6n$ using $n$~dice.''  To use induction, we prove:
\begin{enumerate}
\item The \emph{induction basis} $P(1)$, i.e., with just one die,
  you can throw any number between $1$ and $6$.
\item The \emph{induction step}, for all $k$, if $P(k)$ then~$P(k+1)$.
\end{enumerate}

(1) Is proved by inspecting a $6$-sided die. It has all 6 sides, and
every number between $1$ and~$6$ shows up one on of the sides. So it
is possible to throw any number between $1$ and~$6$ using a single
die.

To prove (2), we assume the antecedent of the conditional, i.e.,
$P(k)$. This assumption is called the \emph{inductive hypothesis}.  We
use it to prove~$P(k+1)$. The hard part is to find a way of thinking
about the possible values of a throw of $k+1$ dice in terms of the
possible values of throws of $k$~dice plus of throws of the extra
$k+1$-st die---this is what we have to do, though, if we want to use
the inductive hypothesis.

The inductive hypothesis says we can get any number between $k$
and~$6k$ using $k$~dice.  If we throw a~$1$ with our $(k+1)$-st die,
this adds $1$ to the total. So we can throw any value between $k+1$
and $6k+1$ by throwing $5$~dice and then rolling a~$1$ with the
$(k+1)$-st die.  What's left?  The values $6k+2$ through $6k+6$.  We
can get these by rolling $k$ $6$s and then a number between $2$ and
$6$ with our $(k+1)$-st die. Together, this means that with $k+1$ dice
we can throw any of the numbers between $k+1$ and $6(k+1)$, i.e.,
we've proved~$P(k+1)$ using the assumption~$P(k)$, the inductive
hypothesis.
\end{proof}

Very often we use induction when we want to prove something about a
series of objects (numbers, sets, etc.) that is itself defined
``inductively,'' i.e., by defining the $(n+1)$-st object in terms of
the $n$-th.  For instance, we can define the sum~$s_n$ of the natural
numbers up to~$n$ by
\begin{align*}
  s_0 & = 0\\
  s_{n+1} & = s_n + (n+1)
\end{align*}
This definition gives:
\begin{align*}
  s_0 & = 0,\\
  s_1 & = s_0 + 1 && = 1,\\
  s_2 & = s_1 + 2 && = 1 + 2 = 3\\
  s_3 & = s_2 + 3 && = 1 + 2 + 3 = 6, \text{ etc.}
\end{align*}
Now we can prove, by induction, that $s_n = n(n+1)/2$.

\begin{prop}
  $s_n = n(n+1)/2$.
\end{prop}

\begin{proof}
  We have to prove (1) that $s_0 = 0\cdot(0 + 1)/2$ and (2) if $s_n =
  n(n+1)/2$ then $s_{n+1} = (n+1)(n+2)/2$.  (1) is obvious. To prove
  (2), we assume the inductive hypothesis: $s_n = n(n+1)/2$. Using it,
  we have to show that $s_{n+1} = (n+1)(n+2)/2$.

  What is $s_{n+1}$?  By the definition, $s_{n+1} = s_n + (n+1)$.  By
  inductive hypothesis, $s_n = n(n+1)/2$. We can substitute this into
  the previous equation, and then just need a bit of arithmetic of
  fractions:
  \begin{align*}
    s_{n+1} & = \frac{n(n+1)}{2} + (n+1) = {}\\
    & = \frac{n(n+1)}{2} + \frac{2(n+1)}{2} = {}\\
    & = \frac{n(n+1) + 2(n+1)}{2} = {}\\
    & = \frac{(n+2)(n+1)}{2}.
  \end{align*}
\end{proof}

The important lesson here is that if you're proving something about
some inductively defined sequence $a_n$, induction is the obvious way
to go. And even if it isn't (as in the case of the possibilities of
dice throws), you can use induction if you can somehow relate the case
for~$n+1$ to the case for~$n$.

\end{document}
