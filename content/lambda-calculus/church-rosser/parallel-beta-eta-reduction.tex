% Part: lambda-calculus
% Chapter: church-rosser
% Section: parallel-beta-eta-reduction

\documentclass[../../../include/open-logic-section]{subfiles}

\begin{document}

\olfileid{lam}{cr}{pbe}

\olsection{Parallel $\beta\eta$-reduction}

In this section we prove the Church-Rosser property for parallel
$\beta\eta$-reduction, the parallel reduction notion corresponding to
$\beta\eta$-reduction. 

\begin{defn}[Parallel $\beta\eta$-reduction, $\beredpar$] \ollabel{defn:beredpar}
  \emph{Parallel $\beta\eta$-reduction} ($\beredpar$) on terms is
  inductively defined as follows:
  \begin{enumerate}
    \item \ollabel{defn:beredpar1} $x \beredpar x$.
    \item \ollabel{defn:beredpar2} If $N \xrightarrow{\beta} N'$ then $\lambd[x][N] \beredpar
      \lambd[x][N']$.
    \item \ollabel{defn:beredpar3} If $P \beredpar P'$ and $Q \beredpar Q'$ then $PQ \beredpar
      P'Q'$.
    \item \ollabel{defn:beredpar4} If $N \beredpar N'$ and $Q \beredpar Q'$ then
      $(\lambd[x][N])Q \beredpar \Subst{N'}{Q'}{x}$.
    \item \ollabel{defn:beredpar5} If $N \beredpar N'$ then $\lambd[x][Nx]
      \beredpar N'$, provided $x \notin FV(N)$.
  \end{enumerate}
\end{defn}

\begin{thm}\ollabel{thm:refl}
  $M \beredpar M$.
\end{thm}

\begin{proof}
  Exercise.
\end{proof}

\begin{prob}
  Prove \olref[lam][cr][pbe]{thm:refl}.
\end{prob}

\begin{defn}[$\beta\eta$-complete development]\ollabel{defn:becd}
  The \emph{$\beta\eta$-complete development}~$\becd{M}$ of~$M$ is defined
  as follows:
  \begin{align}
    \becd{x} &= x \ollabel{defn:becd1} \\
    \becd{(\lambd[x][N])} &= \lambd[x][\becd{N}] \ollabel{defn:becd2}\\
    \becd{(PQ)} &= \becd{P}\becd{Q} && \text{if $P$ is not a $\lambd$-abstract} 
    \ollabel{defn:becd3} \\
    \becd{((\lambd[x][N])Q)} &= \Subst{\becd{N}}{\becd{Q}}{x}
                             \ollabel{defn:becd4} \\
    \becd{(\lambd[x][Nx])} &= \becd{N} \ollabel{defn:becd5} & \text{if $x
                                                      \notin FV(N)$}
  \end{align}
\end{defn}

\begin{lem}\ollabel{lem:comp}
  If $M \beredpar M'$ and $R \beredpar R'$, then $\Subst{M}{R}{y}
  \beredpar \Subst{M'}{R'}{y}$.
\end{lem}

\begin{proof}
  By induction on the derivation of $M \beredpar M'$.

  The first four cases are exactly like those in \olref[pb]{lem:comp}.
  If the last rule is \olref{defn:beredpar5}, then $M$ is
  $\lambd[x][Nx]$, $M'$ is $N'$ for some $x$ and~$N'$ where $x \notin
  FV(N)$, and $N \beredpar N'$. We want to show that
  $\Subst{(\lambd[x][Nx])}{R}{y} \beredpar \Subst{N'}{R'}{y}$, i.e.,
  $\lambd[x][\Subst{N}{R}{y} x] \beredpar \Subst{N'}{R'}{y}$. It
  follows by \olref{defn:beredpar}\olref{defn:beredpar5} and the
  induction hypothesis.
\end{proof}

\begin{lem}\ollabel{lem:cont}
  If $M \beredpar M'$ then $M' \beredpar \becd{M}$.
\end{lem}

\begin{proof}
  By induction on the derivation of $M \beredpar M'$.

  The first four cases are like those in \olref[pb]{lem:cont}. If the
  last rule is \olref{defn:beredpar5}, then $M$ is $\lambd[x][Nx]$ and
  $M'$ is $N'$ for some $x$, $N$, $N'$ where $x \notin FV(N)$ and $N
  \beredpar N'$. We want to show that $N' \beredpar
  \becd{(\lambd[x][Nx])}$, i.e., $N' \beredpar \becd{N}$, which is
  immediate by induction hypothesis.
\end{proof}

\begin{thm}\ollabel{thm:cr}
  $\beredpar$ has the Church-Rosser property.
\end{thm}

\begin{proof}
  Immediate from \olref{lem:cont}.
\end{proof}

\end{document}