% Part: lambda-calculus
% Chapter: church-rosser
% Section: definitions-and-properties

\documentclass[../../../include/open-logic-section]{subfiles}

\begin{document}

\olfileid{lam}{cr}{dap}

\olsection{Definition and Properties}

In this chapter we introduce the concept of Church-Rosser property and
some common properties of this property.

\begin{defn}[Church-Rosser property, CR] A relation $\xredone$ on
  terms is said to satisfy the \emph{Church-Rosser property} iff,
  whenever $M \xredone P$ and $M \xredone Q$, then there exists
  some~$N$ such that $P \xredone N$ and $Q \xredone N$.
\end{defn}

We can view the lambda calculus as a model of computation in which
terms in normal form are ``values'' and a reducibility relation on
terms are the ``calculation rules.'' The Church-Rosser property states
is that when there is more than one way to proceed with a calculation,
there is still only a single value of the expression.

To take an example from elementary algebra, there's more than one way
to calculate $4 \times (1+2) + 3$. It can either be reduced to $4
\times 3+3$ (if we first reduce $1+2$ to~$3$) or to $4 \times 1+4
\times 2+3$ (if we first reduce $4 \times (1+2)$ using
distributivity). Both of these, however, can be further reduced to
$12+3$.

If we take $\xredone$ to be $\beta$-reduction, we easily see that a
consequence of the Church-Rosser property is that if a term has a
normal form, then it is unique. For suppose $M$ can be reduced to $P$
and $Q$, both of which are normal forms. By Church-Rosser property,
there exists some $N$ such that both $P$ and $Q$ reduce to it. Since
by assumption $P$ and $Q$ are normal forms, the reduction of $P$ and
$Q$ to $N$ can only be the trivial reduction, i.e., $P$, $Q$, and $N$
are identical. This justifies our speaking of \emph{the} normal form
of a term.

In viewing the lambda calculus as a model of computation, then, the
normal form of a term can be thought of as the ``final result'' of the
computation starting with that term. The above corollary means there's
only one, if any, final result of a computation, just like there is
only one result of computing $4 \times (1+2)+3$, namely~$15$.

\begin{thm} \ollabel{thm:str}
  If a relation $\xredone$ satisfies the Church-Rosser property, and $\xred$ is the
  smallest transitive relation containing $\xredone$, then $\xred$ satisfies
  the Church-Rosser property too.
\end{thm}

\begin{proof}
  Suppose 
  \begin{align*}
    M & \xredone P_1 \xredone \dots \xredone P_m \text{ and}\\
    M & \xredone Q_1 \xredone \dots \xredone Q_n.
  \end{align*}
  We will prove the theorem by constructing a grid $N$ of terms of
  height is $m + 1$ and width $n + 1$. We use $N_{i,j}$ to denote the
  term in the $i$-th row and $j$-th column.
  
  We construct $N$ in such a way that $N_{i,j} \xredone N_{i+1,j}$ and
  $N_{i,j} \xredone N_{i,j+1}$. It is defined as follows:
  \begin{align*}
    N_{0,0} &= M \\
    N_{i,0} &= P_i && \text{if } 1 \le i \le m \\
    N_{0,j} &= Q_j && \text{if } 1 \le j \le n \\
  \intertext{and otherwise:}
    N_{i,j} &= R
  \end{align*}
  where $R$ is a term such that $N_{i-1,j} \xredone R$ and $N_{i,j-1}
  \xredone R$. By the Church-Rosser property of $\xredone$, such a
  term always exists.

  Now we have $N_{m,0} \xredone \dots \xredone N_{m,n}$ and $N_{0,n}
  \xredone \dots \xredone N_{m,n}$. Note $N_{m,0}$ is $P$ and $N_{0,n}$
  is $Q$. By definition of $\xred$ the theorem follows.
\end{proof}

\end{document}
