% Part: lambda-calculus
% Chapter: lambda-definability
% Section: pairs

\documentclass[../../../include/open-logic-section]{subfiles}

\begin{document}

\olfileid{lam}{ldf}{pai}
\olsection{Pairs and Predecessor}

\begin{defn}
The pair of $M$ and $N$ (written $\tuple{M,N}$) is defined as follows:
\[
\tuple{M,N} \ident \lambd[f][fMN].
\]
\end{defn}
  
Intuitively it is a function that accepts a function, and applies that
function to the two elements of the pair. Following this idea we have
this constructor, which takes two terms and returns the pair containing
them:
\[
  \fn{Pair} \ident \lambd[mn][\lambd[f][fmn]]
\]
Given a pair, we also want to recover its elements.
For this we need two access functions, which accept a pair as argument and
return the first or second elements in it:
\begin{align*}
  \fn{Fst} & \ident \lambd[p][p(\lambd[mn][m])]\\
  \fn{Snd} & \ident \lambd[p][p(\lambd[mn][n])]
\end{align*}

\begin{prob}
  Explain why the access functions $\fn{Fst}$ and $\fn{Snd}$ work.
\end{prob}

Now with pairs we can !!{lambda define} the predecessor function:
\[
  \fn{Pred} \ident \lambd[n][\fn{Fst}(n (\lambd[p][\tuple{\fn{Snd}\, {p}, \fn{Succ}(\fn{Snd}\, {p})}]) \tuple{\num 0, \num 0})]
\]
Remember that $\num n\, f x$ reduces to $f^{n}(x)$; in this
case $f$ is a function that accepts a pair $p$ and returns a new
pair containing the second component of $p$ and the successor of the
second component; $x$ is the pair $\tuple{0,0}$. Thus, the
result is $\tuple{0,0}$ for $n=0$, and $\tuple{\num{n-1}, \num n}$
otherwise. $\fn{Pred}$ then returns the first component of the result.

Subtraction can be defined as $\fn{Pred}$ applied to $a$, $b$ times:
\[
\fn{Sub} \ident \lambd[ab][b \fn{Pred}\, a].
\]
    
\end{document}
