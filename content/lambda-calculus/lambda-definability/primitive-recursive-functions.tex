% Part: lambda-calculus
% Chapter: lambda-definability
% Section: primitive-recursive-functions

\documentclass[../../../include/open-logic-section]{subfiles}

\begin{document}

\olfileid{lam}{ldf}{prf}
\olsection{Primitive Recursive Functions are \usetoken{S}{lambda definable}} 

Recall that the primitive recursive functions are those that can be
defined from the basic functions $\Zero$, $\Succ$, and $\Proj{n}{i}$
by composition and primitive recursion.

\begin{lem}
  \ollabel{lem:basic}
  The basic primitive recursive functions $\Zero$, $\Succ$, and
  projections~$\Proj{n}{i}$ are !!{lambda definable}.
\end{lem}

\begin{proof}
They are !!{lambda define}d by the following terms:
\begin{align*}
  \fn{Zero} & = \lambd[a][\lambd[fx][x]]\\
  \fn{Succ} & = \lambd[a][\lambd[fx][f (a f x)]] \\
  \fn{Proj}^n_i & = \lambd[x_0\dots x_{n-1}][x_i] \qedhere
\end{align*}
\end{proof}

\begin{lem}
  \ollabel{lem:comp} Suppose the $k$-ary function $f$, and $n$-ary
  functions $g_0, \dots, g_{k-1}$, are !!{lambda definable} by terms
  $F$, $G_0$, \dots, $G_k$, and $h$ is defined from them by composition.
  Then $H$ is !!{lambda definable}
\end{lem}

\begin{proof}
  $h$ can be !!{lambda define}d by the term
  \[
  H = \lambd[x_0 \dots x_{n-1}][F\, (G_0 x_0 \dots
    x_{n-1}) \dots (G_{k-1} x_0 \dots x_{n-1})]
  \]
  We leave verification of this fact as an exercise.
\end{proof}

\begin{prob}
  Complete the proof of \olref[lam][ldf][prf]{lem:comp} by showing
  that $H\num{n_0}\dots\num{n_{n-1}} \red \num{h(n_0, \dots,
    n_{n-1})}$.
\end{prob}

Note that \olref{lem:comp} did not require that $f$ and $g_0$,
\dots,~$g_{k-1}$ are primitive recursive; it is only required that
they are total and !!{lambda definable}.

\begin{lem}\ollabel{lem:prim}
  Suppose $f$ is an $n$-ary function and~$g$ is an $n+2$-ary function,
  they are !!{lambda definable} by terms $F$ and~$G$, and the
  function~$h$ is defined from $f$ and $g$ by primitive
  recursion. Then $h$ is also !!{lambda definable}.
\end{lem}

\begin{proof}
  Recall that $h$ is defined by
  \begin{align*}
    h(x_1, \dots, x_n, 0) &= f(x_1, \dots, x_n)\\
    h(x_1, \dots, x_n, y+1) & = h(x_1, \dots, x_n, y, h(x_1, \dots, x_n, y)).
  \end{align*}
  Informally speaking, the primitive recursive definition iterates the
  application of the function $h$ $y$ times and applies it to $f(x_1,
  \dots, x_n)$. This is reminiscent of the definition of Church
  numerals, which is also defined as a iterator.

  For simplicity, we give the definition and proof for a single
  additional argument~$x$. The function $h$ is !!{lambda define}d by:
  \begin{align*}
    H = & \lambd[x][\lambd[y][\fn{Snd} (y
        D \tuple{\num{0}, F x})]]
    \intertext{where }
    D = & (\lambd[p].\tuple{\fn{Succ} (\fn{Fst}\, p), (G x
          (\fn{Fst}\, p) (\fn{Snd}\, p))})
  \end{align*}
  The iteration state we maintain is a pair, the first of which is the
  current $y$ and the second is the correspondng value of~$h$. For
  every step of iteration we create a pair of new values of $y$ and
  $h$; after the iteration is done we return the second part of the
  pair and that's the final $h$ value. We now prove this is indeed a
  representation of primitive recursion.

  We want to prove that for any $n$ and $m$, $H\,\num{n}\,\num{m} =
  \num{h(n,m)}$. To do this we first show that if $D_n =
  \Subst{D}{\num{n}}{x}$, then $D_n^m \tuple{\num{0}, F\, \num{n}} \red
  \tuple{\num{m}, \num{h(n, m)}}$ We proceed by induction on~$m$.

  If $m=0$, we want $D_n^0 \tuple{\num{0}, F\, \num{n}} \red
  \tuple{\num{0}, \num{h(n, 0)}}$. But $D_n^0 \tuple{\num{0}, F\,
    \num{n}}$ just is $\tuple{\num{0}, F\, \num{n}}$. Since $F$
  !!{lambda define}s, this reduces to $\tuple{\num{0}, \num{f(n)}}$, and
  $f(n) = h(n, 0)$.

  Now suppose that $D_n^m \tuple{\num{0}, F\, \num{n}} \red
  \tuple{\num{m}, \num{h(n, m)}}$. We want to
  show that $D_n^{m+1} \tuple{\num{0}, F\, \num{n}} \red
  \tuple{\num{m+1}, \num{h(n, m+1)}}$.
  \begin{align*}
    D_n^{m+1} \tuple{\num{0}, F\, \num{n}}
    & =     D_n(D_n^{m} \tuple{\num{0}, F\, \num{n}})\\
    & \red D_n(\tuple{\num{m}, \num{h(n, m)}}) \text{ (by IH)}\\
    & = (\lambd[p].\tuple{\fn{Succ} (\fn{Fst}\, p), (G \, \num{n}
      (\fn{Fst}\, p) (\fn{Snd}\, p))}) \tuple{\num{m}, \num{h(n,m)}})\\
    & \redone
    \tuple{\fn{Succ} (\fn{Fst}\, \tuple{\num{m}, \num{h(n,m)}}), (G \, \num{n}
      (\fn{Fst}\, \tuple{\num{m}, \num{h(n,m)}}) (\fn{Snd}\, \tuple{\num{m}, \num{h(n,m)}}))}))\\
    & \red 
    \tuple{\fn{Succ}\, \num{m}, (G \, \num{n} \,\num{m} \, \num{h(n,m)})}\\
    & \red \tuple{\num{m+1}, \num{g(n, m, h(n, m))}}
  \end{align*}
  Since $g(n, m, h(n, m)) = h(n, m+1)$, we are done.

  Finally, consider
  \begin{align*}
    H\,\num n\,\num m &= \lambd[x][\lambd[y][\fn{Snd} (y
        (\lambd[p].\tuple{\fn{Succ} (\fn{Fst}\, p), (G\, x\,
          (\fn{Fst}\, p)\, (\fn{Snd}\, p))}) \tuple{\num{0}, F x})]] \,\num n\, \num m\\
    & \red \fn{Snd} (\num m \,
    \underbrace{(\lambd[p].\tuple{\fn{Succ} (\fn{Fst}\, p), (G \,\num n\,
      (\fn{Fst}\, p) (\fn{Snd}\, p))})}_{D_n} \tuple{\num{0}, F \num n})\\
    & = \fn{Snd} (\num m \, D_n\, \tuple{\num{0}, F \num n})\\
    & \red \fn{Snd} \,D_n^m  \tuple{\num{0}, F \num n})\\
    & \red \fn{Snd} \tuple{\num{m}, \num{h(n, m)}} \\
    & \red \num{h(n,m)}. \qedhere
  \end{align*}  
\end{proof}


\begin{prop}
  Every primitive recursive function is !!{lambda definable}.
\end{prop}

\begin{proof}
  By \olref{lem:basic}, all basic functions are !!{lambda definable},
  and by \olref{lem:comp} and \olref{lem:prim}, the !!{lambda definable}
  functions are closed under composition and primitive recursion.
\end{proof}

\end{document}

