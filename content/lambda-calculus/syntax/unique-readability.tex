% Part: lambda-calculus
% Chapter: introduction
% Section: unique-readability

\documentclass[../../../include/open-logic-section]{subfiles}

\begin{document}

\olfileid{lam}{syn}{unq}
\olsection{Unique Readability}

We may wonder if for each term there is a unique way of forming it,
and there is.  For each lambda term there is only one way to construct
and interpret it.  In the following discussion, a \emph{formation} is
the procedure of constructing a term using the formation rules (one or
several times) of \olref[trm]{defn:term}.

\begin{lem}\ollabel{lem:term-start}
A term starts with either a variable or a parenthesis.
\end{lem}

\begin{proof}
Something counts as a term only if it is constructed according to
\olref[trm]{defn:term}. If it is the result of
\olref[trm]{defn:term-var}, it must be a variable. If it is the result
of \olref[trm]{defn:term-abs} or \olref[trm]{defn:term-app}, it starts
with a parenthesis.
\end{proof}

\begin{lem}\ollabel{lem:app-start}
The result of an application starts with either two parentheses or a
parenthesis and a variable.
\end{lem}

\begin{proof}
If $M$ is the result of an application, it is of the form $(PQ)$, so
it begins with a parenthesis. Since $P$ is a term, by
\olref{lem:term-start}, it begins either with a parenthesis or a
variable.
\end{proof}

\begin{lem}\ollabel{lem:initial}
No proper initial part of a term is itself a term.
\end{lem}

\begin{prob}
Prove \olref[lam][syn][unq]{lem:initial} by induction on the length
of terms.
\end{prob}

\begin{prop}[Unique Readability] \ollabel{prop:unq}
There is a unique formation for each term. In other words, if a term
$M$ is formed by a formation, then it is the only formation that can
form this term.
\end{prop}

\begin{proof}
  We prove this by induction on the formation of terms. 
  \begin{enumerate}
    \item $M$ is of the form
      $x$, where $x$ is some variable. Since the results of abstractions and applications always start with parentheses, they cannot have been used to
      construct $M$; Thus, the formation of $M$ must be a single step
      of \olref[trm]{defn:term}\olref[trm]{defn:term-var}.
    \item $M$ is of the form $(\lambd[x][N])$, where $x$ is some
      variable and $N$ is a term. It could not have been constructed
      according to \olref[trm]{defn:term}\olref[trm]{defn:term-var},
      because it is not a single variable.  It is not the result of an
      application, by \olref{lem:app-start}. Thus $M$ can only be the
      result of an abstraction on~$N$.  By inductive hypothesis we
      know that formation of $N$ is itself unique.
    \item $M$ is of the form $(PQ)$, where $P$ and $Q$ are terms.
      Since it starts with a parentheses, it cannot also be
      constructed by
      \olref[trm]{defn:term}\olref[trm]{defn:term-var}.  By
      \olref{lem:term-start}, $P$ cannot begin with $\lambd$, so
      $(PQ)$ cannot be the result of an abstraction.  Now suppose
      there were another way of constructing $M$ by application, e.g.,
      it is also of the form $(P'Q')$. Then $P$ is a proper initial
      segment of $P'$ (or vice versa), and this is impossible by
      \olref{lem:initial}. So $P$ and $Q$ are uniquely
      determined, and by inductive hypothesis we know that formations
      of $P$ and $Q$ is unique.
  \end{enumerate}
\end{proof}

A more readable paraphrase of the above proposition is as follows:
\begin{prop}
  A term $M$ can only be one of the following forms:
  \begin{enumerate}
    \item $x$, where $x$ is a variable uniquely determined by $M$.
    \item $(\lambd[x][N])$, where $x$ is a variable and $N$ is
      another term, both of which is uniquely determined by $M$.
    \item $(PQ)$, where $P$ and $Q$ are two terms uniquely
      determined by $M$.
  \end{enumerate}
\end{prop}

\end{document}
