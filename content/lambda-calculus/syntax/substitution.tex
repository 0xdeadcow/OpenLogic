% Part: lambda-calculus
% Chapter: syntax
% Section: substitution

\documentclass[../../../include/open-logic-section]{subfiles}

\begin{document}

\olfileid{lam}{syn}{sub}
\olsection{Substitution}

\begin{explain}
Free variables are references to environment variables, thus it makes
sense to actually use a specific value in the place of a free
variable. For example, we may want to replace $f$ in $\lambd[x][f x]$
with a specific term, like the identity function $\lambd[y][y]$. This
results in $\lambd[x][(\lambd[y][y]) x]$. The process of replacing
free variables with lambda terms is called substitution.
\end{explain}

\begin{defn}[Substitution] \ollabel{defn:substitution}
  The \emph{substitution} of a term $N$ for a variable $x$ in a term
  $M$, $\Subst{M}{N}{x}$, is defined inductively by:
  \begin{enumerate}
    \item $\Subst{x}{N}{x} = N$. \ollabel{defn:substitution-1} 
    \item $\Subst{y}{N}{x}  = y$ if $x \neq y$. \ollabel{defn:substitution-2} 
    \item $\Subst{PQ}{N}{x}  = (\Subst{P}{N}{x}) (\Subst{Q}{N}{x})$.
      \ollabel{def:substitution-3}
    \item $\Subst{(\lambd[y][P])}{N}{x} = \lambd[y][\Subst{P}{N}{x}]$,
      if $x \neq y$ and $y \notin \FV{N}$, otherwise undefined.
      \ollabel{defn:substitution-4}
  \end{enumerate}
\end{defn}

\begin{explain}
In \olref{defn:substitution}\olref{defn:substitution-4}, we require $x
\neq y$ because we don't want to replace \emph{bound} occurrences of
the variable~$x$ in~$M$ by~$N$.  For example, if we compute the
substitution $\Subst{\lambd[x][x]}{y}{x}$, the result should not be
$\lambd[x][y]$ but simply $\lambd[x][x]$.

When substituting $N$ for~$x$ in $\lambd[y][P]$, we also require that
$y \notin \FV{N}$.  For example, we cannot substitute $y$ for $x$ in
$\lambd[y][x]$, i.e., $\Subst{\lambd[y][x]}{y}{x}$, because it would
result in $\lambd[y][y]$, a term that stands for the function that
accepts an argument and returns it directly. But the term
$\lambd[y][x]$ stands for a function that always returns the term~$x$
(or whatever $x$ refers to). So the result we actually want is a
function that accepts an argument, drop it, and returns the
environment variable~$y$. To do this properly, we would first have to
``rename'' the bound variable~$y$.
\end{explain}

\begin{prob}
  What is the result of the following substitutions?
  \begin{enumerate}
  \item $\Subst{\lambd[y][x(\lambd[w][vwx])]}{(uv)}{x}$
  \item $\Subst{\lambd[y][x(\lambd[x][x])]}{(\lambd[y][xy])}{x}$
  \item $\Subst{y(\lambd[v][xv])}{(\lambd[y][vy])}{x}$
  \end{enumerate}
\end{prob}

\begin{thm} \ollabel{thm:notinfv}
  If $x \notin \FV{M}$, then $\FV{\Subst{M}{N}{x}} = \FV{M}$, if
  the left-hand side is defined.
\end{thm}

\begin{proof}
  By induction on the formation of $M$.
  \begin{enumerate}
  \item $M$ is a variable: exercise.
  \item $M$ is of the form $(PQ)$: exercise.
  \item $M$ is of the form $\lambd[y][P]$, and since
    $\Subst{\lambd[y][P]}{N}{x}$ is defined, it has to be
    $\lambd[y][\Subst{P}{N}{x}]$. Then $\Subst{P}{N}{x}$ has to be
    defined; also, $x \neq y$ and $x \notin \FV{Q}$. Then:
    \begin{multline*}
      \FV{\Subst{\lambd[y][P]}{N}{x}} = \\
    \begin{aligned}
      & = \FV{\lambd[y][\Subst{P}{N}{x}]} &&
      \text{by \olref{defn:substitution-4}}\\
      & = \FV{\Subst{P}{N}{x}} \setminus \{y\} &&
      \text{by \olref[fv]{def:fv}\olref[fv]{def:fv2}}\\
      & = \FV{P} \setminus \{y\} && \text{by inductive hypothesis} \\
      & = \FV{\lambd[y][P]} && \text{by \olref[fv]{def:fv}\olref[fv]{def:fv2}}
    \end{aligned}
    \end{multline*}
  \end{enumerate}
\end{proof}

\begin{prob}
  Complete the proof of \olref[lam][syn][sub]{thm:notinfv}.
\end{prob}

\begin{thm} \ollabel{thm:infv}
  If $x \in \FV{M})$, then $\FV{\Subst{M}{N}{x}} = (\FV{M} \setminus
  \{x\}) \cup \FV{N}$, provided the left hand is defined.
\end{thm}

\begin{proof}
  By induction on the formation of $M$.
  \begin{enumerate}
  \item $M$ is a variable: exercise.
  \item $M$ is of the form $PQ$: Since
    $\Subst{(PQ)}{N}{y}$ is defined, it has to be
    $(\Subst{P}{N}{x})(\Subst{Q}{N}{x})$ with both substitution
    defined. Also, since $x \in \FV{PQ}$, either $x \in \FV{P}$ or
    $x \in \FV{Q}$ or both. The rest is left as an exercise.
  \item $M$ is of the form $\lambd[y][P]$. Since
    $\Subst{\lambd[y][P]}{N}{x}$ is defined, it has to be
    $\lambd[y][\Subst{P}{N}{x}]$, with $\Subst{P}{N}{x}$
    defined, $x \neq y$ and $y \notin \FV{N}$; also, since $y \in
    \FV{\lambd[x][P]}$, we have $y \in \FV{P}$ too. Now:
    \begin{multline*}
      \FV{\Subst{(\lambd[y][P])}{N}{x}} = \\
      \begin{aligned}
      & = \FV{\lambd[y][\Subst{P}{N}{x}]} \\
      & = \FV{\Subst{P}{N}{x}} \setminus \{y\} \\
      & = ((\FV{P} \setminus \{y\}) \cup (\FV{N} \setminus \{x\})
       && \text{by inductive hypothesis} \\
      & = (\FV{P} \setminus \{x, y\}) \cup \FV{N}
       && x \notin \FV{N} \\
      & = (\FV{\lambd[y][P]} \setminus \{x\}) \cup \FV{N}
      \end{aligned}
      \end{multline*}
  \end{enumerate}
\end{proof}

\begin{prob}
  Complete the proof of \olref[lam][syn][sub]{thm:infv}.
\end{prob}

\begin{thm}\ollabel{thm:clr}
  $x \notin \FV{\Subst{M}{N}{x}}$, if the right-hand side is
  defined and $x \notin \FV{N}$.
\end{thm}

\begin{proof}
  Exercise.
\end{proof}

\begin{prob}
  Prove \olref[lam][syn][sub]{thm:clr}.
\end{prob}


\begin{thm}\ollabel{thm:inv}
  If $\Subst{M}{y}{x}$ is defined and $y \notin \FV{M}$, then
  $\Subst{\Subst{M}{y}{x}}{x}{y} = M$.
\end{thm}

\begin{proof}
  By induction on the formation of $M$.
  \begin{enumerate}
  \item $M$ is a variable $z$: Exercise.
  \item $M$ is of the form $(PQ)$. Then:
    \begin{align*}
      \Subst{\Subst{(PQ)}{y}{x}}{x}{y}
      &=\Subst{((\Subst{P}{y}{x})(\Subst{Q}{y}{x}))}{x}{y} \\
      &= (\Subst{\Subst{P}{y}{x}}{x}{y})(\Subst{\Subst{Q}{y}{x}}{x}{y}) \\
      &= (PQ) \text{ by inductive hypothesis}
    \end{align*}
  \item $M$ is of the form $\lambd[z][N]$. Because
    $\Subst{\lambd[z][N]}{y}{x}$ is defined, we know
    that $z \neq y$. So:
    \begin{align*}
      \Subst{\Subst{(\lambd[z][N])}{y}{x}}{x}{y}\\
      & = \Subst{(\lambd[z][\Subst{N}{y}{x}])}{x}{y} \\
      & = \lambd[z][\Subst{\Subst{N}{y}{x}}{x}{y}] \\
      &= \lambd[z][N] \text{ by inductive hypothesis} 
    \end{align*}
  \end{enumerate}
\end{proof}

\begin{prob}
  Complete the proof of \olref[lam][syn][sub]{thm:inv}.
\end{prob}

\end{document}
