% Part: lambda-calculus
% Chapter: introduction
% Section: syntax

\documentclass[../../../include/open-logic-section]{subfiles}

\begin{document}

\olfileid{lam}{int}{syn}
\olsection{The Syntax of the Lambda Calculus}

One starts with a sequence of variables $x$, $y$, $z$,~\dots and some
constant symbols $a$, $b$, $c$,~\dots. The set of terms is defined
inductively, as follows:
\begin{enumerate}
\item Each variable is a term.
\item Each constant is a term.
\item If $M$ and $N$ are terms, so is $(MN)$.
\item If $M$ is a term and $x$ is a variable, then $(\lambd[x][M])$ is a
  term.
\end{enumerate}
The system without any constants at all is called the \emph{pure}
lambda calculus. We'll mainly be working in the pure $\lambd$-calculus,
so all lowerce letters will stand for variables. We use uppercase
letters ($M$, $N$, etc.) to stand for terms of the $\lambd$-calculus.

We will follow a few notational conventions:

\begin{conv}
\begin{enumerate}
\item When parentheses are left out, application takes place from left
  to right. For example, if $M$, $N$, $P$, and $Q$ are terms, then
  $MNPQ$ abbreviates $(((MN)P)Q)$.
\item Again, when parentheses are left out, lambda abstraction is to
  be given the widest scope possible. From example, $\lambd[x][MNP]$ is
  read $(\lambd[x][((MN)P)])$.
\item A lambda can be used to abstract multiple variables. For
  example, $\lambd[xyz][M]$ is short for
  $\lambd[x][\lambd[y][\lambd[z][M]]]$.
\end{enumerate}
\end{conv}

For example,
\[
\lambd[xy][xxyx \lambd[z][xz]]
\]
abbreviates
\[
\lambd[x][\lambd[y][((((xx)y)x)(\lambd[z][(xz)]))]].
\]
You should memorize these conventions. They will drive you crazy at
first, but you will get used to them, and after a while they will
drive you less crazy than having to deal with a morass of parentheses.

Two terms that differ only in the names of the bound variables are
called $\alpha$-equivalent; for example, $\lambd[x][x]$ and
$\lambd[y][y]$. It will be convenient to think of these as being the
``same'' term; in other words, when we say that $M$ and $N$ are the
same, we also mean ``up to renamings of the bound variables.''
Variables that are in the scope of a $\lambd$ are called ``bound'',
while others are called ``free.'' There are no free variables in the
previous example; but in
\[
(\lambd[z][yz])x
\]
$y$ and $x$ are free, and $z$ is bound.

\end{document}
