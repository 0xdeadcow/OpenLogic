% Part: first-order-logic
% Chapter: sequent-calculus
% Section: derivations

\documentclass[../../../include/open-logic-section]{subfiles}

\begin{document}

\iftag{FOL}
      {\olfileid{fol}{seq}{der}}
      {\olfileid{pl}{seq}{der}}

\olsection{\usetoken{P}{derivation}}

\begin{explain}
We've said what an initial sequent looks like, and we've given the
rules of inference.  !!^{derivation}s in the sequent calculus are
inductively generated from these: each !!{derivation} either is an
initial sequent on its own, or consists of one or two !!{derivation}s
followed by an inference.
\end{explain}

\begin{defn}[$\Log{LK}$ !!{derivation}]
An \emph{$\Log{LK}$-!!{derivation}} of a sequent~$S$ is a finite tree
of sequents satisfying the following conditions:
\begin{enumerate}
\item The topmost sequents of the tree are initial sequents.
\item The bottommost sequent of the tree is~$S$.
\item Every sequent in the tree except $S$ is a premise of a correct
  application of an inference rule whose conclusion stands directly
  below that sequent in the tree.
\end{enumerate}
We then say that $S$ is the \emph{end-sequent} of the !!{derivation} and
that $S$ is \emph{!!{derivable} in $\Log{LK}$} (or $\Log{LK}$-!!{derivable}).
\end{defn}

\begin{ex}
Every initial sequent, e.g., $!C \Sequent !C$ is !!a{derivation}. We
can obtain a new !!{derivation} from this by applying, say, the
$\LeftR{\Weakening}$ rule,
\begin{prooftree}
\Axiom$ \Gamma \fCenter \Delta $
\RightLabel{\LeftR{\Weakening}}
\UnaryInf$ !A, \Gamma \fCenter \Delta$
\end{prooftree}
The rule, however, is meant to be general: we can replace the $!A$ in
the rule with any !!{sentence}, e.g., also with~$!D$. If the premise
matches our initial sequent $!C \Sequent !C$, that means that both
$\Gamma$ and $\Delta$ are just~$!C$, and the conclusion would then be
$!D, !C \Sequent !C$. So, the following is !!a{derivation}:
\begin{prooftree}
\Axiom$ !C \fCenter !C $
\RightLabel{\LeftR{\Weakening}}
\UnaryInf$ !D, !C \fCenter !C$
\end{prooftree}
We can now apply another rule, say $\LeftR{\Exchange}$, which allows
us to switch two !!{sentence}s on the left. So, the following is also
a correct !!{derivation}:
\begin{prooftree}
\Axiom$ !C \fCenter !C $
\RightLabel{\LeftR{\Weakening}}
\UnaryInf$ !D, !C \fCenter !C$
\RightLabel{\LeftR{\Exchange}}
\UnaryInf$ !C, !D \fCenter !C$
\end{prooftree}
In this application of the rule, which was given as
\begin{prooftree}
\Axiom$ \Gamma, !A, !B, \Pi \fCenter \Delta $
\RightLabel{\LeftR{\Exchange}}
\UnaryInf$ \Gamma, !B, !A, \Pi \fCenter \Delta,$
\end{prooftree}
both $\Gamma$ and $\Pi$ were empty, $\Delta$ is $!C$, and the roles of
$!A$ and $!B$ are played by $!D$ and~$!C$, respectively. In much the
same way, we also see that
\begin{prooftree}
\Axiom$ !D \fCenter !D $
\RightLabel{\LeftR{\Weakening}}
\UnaryInf$ !C, !D \fCenter !D$
\end{prooftree}
is !!a{derivation}. Now we can take these two derivations, and combine
them using $\RightR{\land}$. That rule was
\begin{prooftree}
\Axiom$\Gamma \fCenter \Delta, !A$
\Axiom$ \Gamma \fCenter \Delta, !B$
\RightLabel{\RightR{\land}}
\BinaryInf$ \Gamma \fCenter \Delta, !A \land !B $
\end{prooftree}
In our case, the premises must match the last sequents of the
!!{derivation}s ending in the premises. That means that $\Gamma$ is
$!C, !D$, $\Delta$ is empty, $!A$ is $!C$ and $!B$ is $!D$. So the
conclusion, if the inference should be correct, is $!C, !D \Sequent !C
\land !D$.
\begin{prooftree}
\Axiom$ !C \fCenter !C $
\RightLabel{\LeftR{\Weakening}}
\UnaryInf$ !D, !C \fCenter !C$
\RightLabel{\LeftR{\Exchange}}
\UnaryInf$ !C, !D \fCenter !C$
\Axiom$ !D \fCenter !D $
\RightLabel{\LeftR{\Weakening}}
\UnaryInf$ !C, !D \fCenter !D$
\RightLabel{\RightR{\land}}
\BinaryInf$ !C, !D \fCenter !C \land !D $
\end{prooftree}
Of course, we can also reverse the premises, then $!A$
would be $!D$ and $!B$ would be~$!C$. 
\begin{prooftree}
\Axiom$ !D \fCenter !D $
\RightLabel{\LeftR{\Weakening}}
\UnaryInf$ !C, !D \fCenter !D$
\Axiom$ !C \fCenter !C $
\RightLabel{\LeftR{\Weakening}}
\UnaryInf$ !D, !C \fCenter !C$
\RightLabel{\LeftR{\Exchange}}
\UnaryInf$ !C, !D \fCenter !C$
\RightLabel{\RightR{\land}}
\BinaryInf$ !C, !D \fCenter !D \land !C $
\end{prooftree}
\end{ex}

\end{document}
