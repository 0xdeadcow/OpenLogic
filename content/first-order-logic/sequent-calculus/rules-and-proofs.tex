% Part: first-order-logic
% Chapter: sequent-calculus
% Section: rules-and-proofs

\documentclass[../../../include/open-logic-section]{subfiles}

\begin{document}

\olfileid{fol}{seq}{rul}

\olsection{Rules and \usetoken{P}{derivation}}

\begin{editorial}
  This section collects all the rules propositional connectives and
  quantifiers, but not for identity.  It is planned to divide this
  into separate sections on connectives and quantifiers so that proofs
  for propositional logic can be treated separately
  (\gitissue{77}).
\end{editorial}

Let $\Lang L$ be a first-order language with the usual constants,
!!{variable}s, logical symbols, and auxiliary symbols (parentheses
and the comma).

\begin{defn}[sequent]
A \emph{sequent} is an expression of the form
\[ \Gamma \Sequent \Delta \]
where $\Gamma$ and $\Delta$ are finite (possibly empty) sets of
!!{sentence}s of the language $\Lang L$. The !!{formula}s in $\Gamma$
are the \emph{antecedent !!{formula}s}, while the formulae in $\Delta$ are
the \emph{succedent !!{formula}s}.

The intuitive idea behind a sequent is: if all of the antecedent
!!{formula}s hold, then at least one of the succedent !!{formula}s
holds. That is, if $\Gamma = \{ \Gamma_1, \dots, \Gamma_m\}$ and
$\Delta = \{ \Delta_1, \dots, \Delta_n\}$, then $\Gamma \Sequent
\Delta$ holds iff
\[
(\Gamma_1 \land \cdots \land \Gamma_m) \lif (\Delta_1 \lor \cdots \lor
\Delta_n) \] holds.

When $m=0$, $\quad \Sequent \Delta$ holds iff $\Delta_1 \lor \dots
\lor \Delta_n$ holds. When $n=0$, $\Gamma \Sequent \quad$ holds iff
$\Gamma_1 \land \dots \land \Gamma_m$ does not. An empty succedent is
sometimes filled with the $\lfalse$ symbol. The empty sequent
$\quad \Sequent \quad$ canonically represents a
contradiction.
\end{defn}

We write $\Gamma, !A$ (or $!A, \Gamma$) for $\Gamma \cup \{!A\}$, and
$\Gamma, \Delta$ for $\Gamma \cup \Delta$.

\begin{defn}[Inference]
An \emph{inference} is an expression of the form
\[
\AxiomC{$S_1$}
\UnaryInfC{$S$}
\DisplayProof
\quad
\textrm{  or  }
\quad
\AxiomC{$S_1$}
\AxiomC{$S_2$}
\BinaryInfC{$S$}
\DisplayProof
\]
where $S, S_1$, and $S_2$ are sequents. $S_1$ and $S_2$ are called the
\emph{upper sequents} and $S$ the \emph{lower sequent} of the
inference.
\end{defn}

In sequent calculus !!{derivation}s, a correct inference yields valid
a valid sequent, provided the upper sequents are valid.

For the following, let $\Gamma, \Delta, \Pi, \Lambda$ represent finite
sets of !!{sentence}s.

The rules for $\Log{LK}$ are divided into two main types:
\emph{structural} rules and \emph{logical} rules. The logical rules
are further divided into \emph{propositional} rules (quantifier-free)
and \emph{quantifier} rules.

\paragraph{Structural rules:}

Weakening:
\[
\Axiom$ \Gamma \fCenter \Delta $
\RightLabel{\LeftR{\Weakening}}
\UnaryInf$ !A, \Gamma \fCenter \Delta$
\DisplayProof
\quad
\textrm{  and  }
\quad
\Axiom$ \Gamma \fCenter \Delta$
\RightLabel{\RightR{\Weakening}}
\UnaryInf$ \Gamma \fCenter \Delta, !A$
\DisplayProof
\]
where $!A$ is called the \emph{weakening !!{formula}}.

A series of weakening inferences will often be indicated by double
inference lines.

Cut:
\[
\Axiom$ \Gamma \fCenter \Delta, !A$
\Axiom$ !A, \Pi \fCenter \Lambda $
\BinaryInf$ \Gamma, \Pi \fCenter \Delta, \Lambda$
\DisplayProof
\]

\paragraph{Logical rules:}

The rules are named by the !!{main operator} of the \emph{principal
  !!{formula}} of the inference (the formula containing $!A$ and/or
$!B$ in the lower sequent). The designations ``left'' and ``right''
indicate whether the logical symbol has been introduced in an
antecedent formula or a succedent formula (to the left or to the right
of the sequent symbol).

\emph{Propositional Rules:}
\[
\Axiom$ \Gamma \fCenter \Delta, !A $
\RightLabel{\LeftR{\lnot}}
\UnaryInf$ \lnot !A, \Gamma \fCenter \Delta$
\DisplayProof
\quad
\Axiom$!A, \Gamma \fCenter \Delta$
\RightLabel{\RightR{\lnot}}
\UnaryInf$ \Gamma \fCenter \Delta, \lnot !A $
\DisplayProof
\]

\[
\Axiom$ !A, \Gamma \fCenter \Delta$
\RightLabel{\LeftR{\land}}
\UnaryInf$ !A \land !B, \Gamma \fCenter \Delta$
\DisplayProof
\quad
\Axiom$!B, \Gamma \fCenter \Delta$
\RightLabel{\LeftR{\land}}
\UnaryInf$!A \land !B, \Gamma \fCenter \Delta$
\DisplayProof
\quad
\Axiom$\Gamma \fCenter \Delta, !A$
\Axiom$ \Gamma \fCenter \Delta, !B$
\RightLabel{\RightR{\land}}
\BinaryInf$ \Gamma \fCenter \Delta, !A \land !B $
\DisplayProof
\]

\[
\Axiom$!A,\Gamma\fCenter \Delta$
\Axiom$ !B, \Gamma \fCenter \Delta$
\RightLabel{\LeftR{\lor}}
\BinaryInf$ !A \lor !B, \Gamma \fCenter \Delta$
\DisplayProof
\quad
\Axiom$\Gamma \fCenter \Delta, !A$
\RightLabel{\RightR{\lor}}
\UnaryInf$ \Gamma \fCenter \Delta, !A \lor !B$
\DisplayProof
\quad
\Axiom$ \Gamma \fCenter \Delta, !B$
\RightLabel{\RightR{\lor}}
\UnaryInf$ \Gamma \fCenter \Delta, !A \lor !B$
\DisplayProof
\]

\[
\Axiom$ \Gamma \fCenter \Delta, !A$
\Axiom$ !B, \Pi \fCenter \Lambda$
\RightLabel{\LeftR{\lif}}
\BinaryInf$ !A \lif !B, \Gamma, \Pi \fCenter \Delta, \Lambda$
\DisplayProof
\quad
\Axiom$ !A, \Gamma \fCenter \Delta, !B$
\RightLabel{\RightR{\lif}}
\UnaryInf$ \Gamma \fCenter \Delta, !A \lif !B $
\DisplayProof
\]

\emph{Quantifier Rules:}

\[
\Axiom$ !A(t), \Gamma \fCenter \Delta$
\RightLabel{\LeftR{\lforall}}
\UnaryInf$ \lforall[x][!A(x)],\Gamma \fCenter \Delta$
\DisplayProof
\quad
\Axiom$ \Gamma \fCenter \Delta, !A(a) $
\RightLabel{\RightR{\lforall}}
\UnaryInf$ \Gamma \fCenter \Delta, \lforall[x][!A(x)]$
\DisplayProof
\]
where $t$ is a ground term (i.e., one without variables), and $a$~is a
constant which does not occur anywhere in the lower sequent of the
\RightR{\lforall} rule. We call $a$ the \emph{eigenvariable} of the
\RightR{\forall} inference.

\[
\Axiom$ !A(a), \Gamma \fCenter \Delta $
\RightLabel{\LeftR{\lexists}}
\UnaryInf$ \lexists[x][!A(x)], \Gamma \fCenter \Delta$
\DisplayProof
\quad
\Axiom$ \Gamma \fCenter \Delta, !A(t) $
\RightLabel{\RightR{\lexists}}
\UnaryInf$ \Gamma \fCenter \Delta, \lexists[x][!A(x)]$
\DisplayProof
\]
where $t$ is a ground term, and $a$ is a constant which does not occur
in the lower sequent of the \LeftR{\lexists} rule. We call $a$
the \emph{eigenvariable} of the \LeftR{\lexists} inference.

The condition that an eigenvariable not occur in the upper sequent of
the \RightR{\lforall} or \LeftR{\lexists} inference is called the
\emph{eigenvariable condition}.

\begin{explain}
We use the term ``eigenvariable'' even though $a$ in the above rules
is a constant. This has historical reasons.

In \RightR{\lexists} and \LeftR{\lforall} there are no restrictions, and
the term~$t$ can be anything, so we do not have to worry about any
conditions. However, because the $t$ may appear elsewhere in the
sequent, the values of~$t$ for which the sequent is satisfied are
constrained. On the other hand, in the \LeftR{\lexists} and $\lforall$
right rules, the eigenvariable condition requires that $a$ does not
occur anywhere else in the sequent. Thus, if the upper sequent is
valid, the truth values of the formulas other than $!A(a)$ are
independent of~$a$.
\end{explain}

\begin{defn}[Initial Sequent]
An \emph{initial sequent} is a sequent of the form $!A \Sequent !A$
for any !!{sentence} $!A$ in the language.
\end{defn}

\begin{defn}[LK !!{derivation}]
An \emph{$\Log{LK}$-!!{derivation}} of a sequent $S$ is a tree of sequents
satisfying the following conditions:
\begin{enumerate}
\item The topmost sequents of the tree are initial sequents.
\item Every sequent in the tree (except $S$) is an upper sequent of an
  inference whose lower sequent stands directly below that sequent in
  the tree.
\end{enumerate}
We then say that $S$ is the \emph{end-sequent} of the !!{derivation} and
that $S$ is \emph{!!{derivable} in $\Log{LK}$} (or $\Log{LK}$-!!{derivable}).
\end{defn}

\begin{defn}[LK theorem]
!!^a{sentence} $!A$ is a \emph{theorem} of $\Log{LK}$ if the sequent
$\quad \Sequent !A$ is $\Log{LK}$-!!{derivable}.
\end{defn}

\end{document}
