% Part: first-order-logic
% Chapter: sequent-calculus
% Section: soundness.tex

\documentclass[../../../include/open-logic-section]{subfiles}

\begin{document}

\olfileid{fol}{seq}{sou}
\olsection{Soundness}

\begin{explain}
A !!{derivation} system, such as the sequent calculus, is \emph{sound}
if it cannot !!{derive} things that do not actually hold.  Soundness is
thus a kind of guaranteed safety property for !!{derivation} systems.
Depending on which proof theoretic property is in question, we would
like to know for instance, that
\begin{enumerate}
\item every !!{derivable} !!{sentence} is valid;
\item if a !!{sentence} is !!{derivable} from some others, it is also a
  consequence of them;
\item if a set of !!{sentence}s is inconsistent, it is unsatisfiable.
\end{enumerate}
These are important properties of a !!{derivation} system.  If any of them do
not hold, the !!{derivation} system is deficient---it would !!{derive} too much.
Consequently, establishing the soundness of a !!{derivation} system is of the
utmost importance.

Because all these proof-theoretic properties are
defined via !!{derivability} in the sequent calculus of certain sequents,
proving (1)--(3) above requires proving something about the semantic
properties of !!{derivable} sequents.  We will first define what it means
for a sequent to be \emph{valid}, and then show that every !!{derivable}
sequent is valid.  (1)--(3) then follow as corollaries from this
result.
\end{explain}

\begin{defn}
A !!{structure}~$\Struct M$ \emph{satisfies} a sequent $\Gamma
\Sequent \Delta$ iff either $\Sat/{M}{!E}$ for some $!E
\in \Gamma$ or $\Sat{M}{!E}$ for some $!E \in \Delta$.

A sequent is \emph{valid} iff every !!{structure}~$\Struct M$
satisfies it.
\end{defn}

\begin{thm}[Soundness]
\ollabel{sequent-soundness} If $\Log{LK}$ !!{derive}s $\Gamma \Sequent
\Delta$, then $\Gamma \Sequent \Delta$ is valid.
\end{thm}

\begin{proof}
Let $\Pi$ be a !!{derivation} of $\Gamma \Sequent \Delta$. We proceed by
induction on the number of inferences in~$\Pi$.

If the number of inferences is~0, then $\Pi$ consists only of an
initial sequent. Every initial sequent $!A \Sequent !A$ is obviously
valid, since for every $\Struct M$, either $\Sat/{M}{!A}$ or
$\Sat{M}{!A}$.

If the number of inferences is greater than~0, we distinguish cases
according to the type of the lowermost inference. By induction
hypothesis, we can assume that the premises of that inference are
valid.

First, we consider the possible inferences with only one premise
$\Gamma' \Sequent \Delta'$.
\begin{enumerate}
\item The last inference is a weakening.  Then $\Gamma' \subseteq
  \Gamma$ and $\Delta = \Delta'$ if it's a weakening on the left, or
  $\Gamma = \Gamma'$ and $\Delta' \subseteq \Delta$ if it's a weaking
  on the right.  In either case, $\Delta' \subseteq \Delta$ and
  $\Gamma' \subseteq \Gamma$.  If $\Sat/{M}{!E}$ for some $!E \in
  \Gamma'$, then, since $\Gamma' \subseteq \Gamma$, $!E \in \Gamma$ as
  well, and so $\Sat/{M}{!E}$ for the same $!E \in \Gamma$.  Similarly,
  if $\Sat{M}{!E}$ for some $!E \in \Delta'$, as $!E \in \Delta$,
  $\Sat{M}{!E}$ for some $!E \in \Delta$.  Since $\Gamma' \Sequent
  \Delta'$ is valid, one of these cases obtains for every~$\Struct
  M$. Consequently, $\Gamma \Sequent \Delta$ is valid.
\item The last inference is $\lnot$~left: Then for some $!A \in
  \Delta'$, $\lnot !A \in \Gamma$.  Also, $\Gamma' \subseteq \Gamma$,
  and $\Delta' \setminus \{!A\} \subseteq \Delta$.

  If $\Sat{M}{!A}$, then $\Sat/{M}{\lnot !A}$, and since $\lnot !A
  \in \Gamma$, $\Struct M$ satisfies $\Gamma \Sequent \Delta$.  Since
  $\Gamma' \Sequent \Delta'$ is valid, if $\Sat/{M}{!A}$, then either
  $\Sat/{M}{!E}$ for some $!E \in \Gamma'$ or $\Sat{M}{!E}$ for some
  $!E \in \Delta'$ different from~$!A$.  Consequently, $\Sat/{M}{!E}$
  for some $!E \in \Gamma$ (since $\Gamma' \subseteq \Gamma$) or
  $\Sat{M}{!E}$ for some $!E \in \Delta'$ different from~$!A$ (since
  $\Delta' \setminus \{!A\} \subseteq \Delta$).
\item The last inference is $\lnot$~right: Exercise.
\item The last inference is $\land$~left: There are two variants: $!A
  \land !B$ may be inferred on the left from $!A$ or from $!B$ on the
  left side of the premise.  In the first case, $!A \in \Gamma'$.
  Consider a !!{structure}~$\Struct M$.  Since $\Gamma' \Sequent
  \Delta'$ is valid, (a) $\Sat/{M}{!A}$, (b) $\Sat/{M}{!E}$ for some
  $!E \in \Gamma' \setminus \{!A\}$, or (c)~$\Sat{M}{!E}$ for some $!E
  \in \Delta'$.  In case (a), $\Sat/{M}{!A \land !B}$.  In case (b),
  there is an $!E \in \Gamma \setminus \{!A \land !B\}$ such that
  $\Sat/{M}{!E}$, since $\Gamma' \setminus \{!A\} \subseteq \Gamma
  \setminus \{!A \land !B\}$.  In case (c), there is a $!E \in \Delta$
  such that $\Sat{M}{!E}$, as $\Delta = \Delta'$.  So in each case,
  $\Struct M$ satisfies $!A \land !B, \Gamma \Sequent \Delta$.  Since
  $\Struct M$ was arbitrary, $\Gamma \Sequent \Delta$ is valid.  The
  case where $!A \land !B$ is inferred from $!B$ is handled the same,
  changing $!A$ to $!B$.
\item The last inference is $\lor$~right: There are two variants: $!A
  \lor !B$ may be inferred on the right from $!A$ or from $!B$ on the
  right side of the premise.  In the first case, $!A \in \Delta'$.
  Consider a !!{structure}~$\Struct M$.  Since $\Gamma' \Sequent
  \Delta'$ is valid, (a) $\Sat{M}{!A}$, (b) $\Sat/{M}{!E}$ for some
  $!E \in \Gamma'$, or (c)~$\Sat{M}{!E}$ for some $!E \in \Delta'
  \setminus \{!A\}$.  In case (a), $\Sat{M}{!A \lor !B}$.  In case (b),
  there is $!E \in \Gamma$ such that $\Sat/{M}{!E}$, as $\Gamma =
  \Gamma'$.  In case (c), there is an $!E \in \Delta$ such that
  $\Sat{M}{!E}$, since $\Delta' \setminus \{!A\} \subseteq \Delta$.  So
  in each case, $\Struct M$ satisfies $!A \land !B, \Gamma \Sequent
  \Delta$.  Since $\Struct M$ was arbitrary, $\Gamma \Sequent \Delta$
  is valid.  The case where $!A \lor !B$ is inferred from $!B$ is
  handled the same, changing $!A$ to $!B$.
\item The last inference is $\lif$~right: Then $!A \in \Gamma'$, $!B
  \in \Delta'$, $\Gamma' \setminus \{!A\} \subseteq \Gamma$ and
  $\Delta' \setminus \{!B\} \subseteq \Delta$.  Since $\Gamma'
  \Sequent \Delta'$ is valid, for any !!{structure}~$\Struct M$, (a)
  $\Sat/{M}{!A}$, (b) $\Sat{M}{!B}$, (c) $\Sat/{M}{!E}$ for some~$~!E
  \in \Gamma' \setminus \{!A\}$, or $\Sat{M}{!E}$ for some $!E \in
  \Delta' \setminus \{!B\}$.  In cases (a) and (b), $\Sat{M}{!A \lif
    !B}$.  In case (c), for some $!E \in \Gamma$, $\Sat/{M}{!E}$. In
  case (d), for some $!E \in \Delta$, $\Sat{M}{!E}$.  In each case,
  $\Struct M$ satisfies $\Gamma \Sequent \Delta$.  Since $\Struct M$
  was arbitrary, $\Gamma \Sequent \Delta$ is valid.
\item The last inference is $\forall$~left: Then there is a
  !!{formula}~$!A(x)$ and a ground term~$t$ such that $!A(t) \in
  \Gamma'$, $\lforall[x][!A(x)] \in \Gamma$, and $\Gamma' \setminus
  \{!A(t)\} \subseteq \Gamma$.  Consider a !!{structure}~$\Struct
  M$.  Since $\Gamma' \Sequent \Delta'$ is valid, (a)
  $\Sat/{M}{!A(t)}$, (b) $\Sat/{M}{!E}$ for some $!E \in \Gamma'
  \setminus \{!A(t)\}$, or (c)~$\Sat{M}{!E}$ for some $!E \in
  \Delta'$.  In case (a), $\Sat/{M}{\lforall[x][!A(x)]}$. In case (b),
  there is an $!E \in \Gamma \setminus \{!A(t)\}$ such that
  $\Sat/{M}{!E}$.  In case (c), there is a $!E \in \Delta$ such that
  $\Sat{M}{!E}$, as $\Delta = \Delta'$.  So in each case, $\Struct M$
  satisfies $\Gamma \Sequent \Delta$.  Since $\Struct M$ was arbitrary,
  $\Gamma \Sequent \Delta$ is valid.
\item The last inference is $\lexists$~right: Exercise.
\item The last inference is $\forall$~right: Then there is a
  !!{formula}~$!A(x)$ and a !!{constant}~$a$ such that $!A(a) \in
  \Delta'$, $\lforall[x][!A(x)] \in \Delta$, and $\Delta' \setminus
  \{!A(a)\} \subseteq \Delta$.  Furthermore, $a \notin \Gamma \cup
  \Delta$.  Consider a !!{structure}~$\Struct M$.  Since $\Gamma'
  \Sequent \Delta'$ is valid, (a) $\Sat{M}{!A(a)}$, (b) $\Sat/{M}{!E}$
  for some $!E \in \Gamma'$, or (c)~$\Sat{M}{!E}$ for some $!E \in
  \Delta' \setminus \{!A(a)\}$.

  First, suppose (a) is the case but neither (b) nor (c), i.e.,
  $\Sat{M}{!E}$ for all $!E \in \Gamma'$ and $\Sat/{M}{!E}$ for all
  $!E \in \Delta'\setminus \{!A(a)\}$.  In other words, assume
  $\Sat{M}{!A(a)}$ and that $\Struct M$ does not satisfy $\Gamma'
  \Sequent \Delta' \setminus \{!A(a)\}$.  Since $a \notin \Gamma \cup
  \Delta$, also $a \notin \Gamma' \cup (\Delta' \setminus
  \{!A(a)\})$.  Thus, if $\Struct{M'}$ is like $\Struct M$ except that
  $\Assign{a}{M} \neq \Assign{a}{M'}$, $\Struct{M'}$ also does not
  satisfy $\Gamma' \Sequent \Delta' \setminus \{!A(a)\}$ by
  extensionality.  But since $\Gamma' \Sequent \Delta'$ is valid, we
  must have $\Sat{M'}{!A(a)}$.

  We now show that $\Sat{M}{\lforall[x][!A(x)]}$.  To do this, we have
  to show that for every variable assignment~$s$,
  $\Sat{M}{\lforall[x][!A(x)]}[s]$.  This in turn means that for every
  $x$-variant $s'$ of $s$, we must have $\Sat{M}{!A(x)}[s']$.  So
  consider any variable assignment~$s$ and let $s'$ be an $x$-variant
  of~$s$.  Since $\Gamma'$ and $\Delta'$ consist entirely of sentences,
  $\Sat{M}{!E}[s]$ iff $\Sat{M}{!E}[s']$ iff $\Sat{M}{!E}$ for all
  $!E \in \Gamma' \cup \Delta'$.  Let $\Struct M'$ be like $\Struct M$
  except that $\Assign{a}{M'} = s'(x)$.  Then $\Sat{M}{!A(x)}[s']$ iff
  $\Sat{M'}{!A(a)}$ (as $!A(x)$ does not contain~$a$).  Since we've
  already established that $\Sat{M'}{!A(a)}$ for all $\Struct M'$
  which differ from $\Struct M$ at most in what they assign to~$a$,
  this means that $\Sat{M}{!A(x)}[s']$.  Thus weve shown that
  $\Sat{M}{\lforall[x][!A(x)]}[s]$.  Since $s$ is an arbitrary variable
  assignment and $\lforall[x][!A(x)]$ is a sentence, then
  $\Sat{M}{\lforall[x][!A(x)]}$.

  If (b) is the case, there is a $!E \in \Gamma$ such that
  $\Sat/{M}{!E}$, as $\Gamma = \Gamma'$.  If (c) is the case, there is
  an $!E \in \Delta' \setminus \{!A(a)\}$ such that $\Sat{M}{!E}$.  So
  in each case, $\Struct M$ satisfies $\Gamma \Sequent \Delta$.  Since
  $\Struct M$ was arbitrary, $\Gamma \Sequent \Delta$ is valid.
\item The last inference is $\lexists$~left: Exercise.
\end{enumerate}
Now let's consider the possible inferences with two premises: cut,
$\lor$~left, $\land$~right, and $\lif$~left.
\begin{enumerate}
\item The last inference is a cut: Suppose the premises are $\Gamma'
  \Sequent \Delta'$ and $\Pi' \Sequent \Lambda'$ and the cut formula
  $!A$ is in both $\Delta'$ and $\Pi'$.  Since each is valid, every
  !!{structure}~$\Struct M$ satisfies both premises.  We distinguish
  two cases: (a) $\Sat/{M}{!A}$ and (b) $\Sat{M}{!A}$.  In case (a), in
  order for $\Struct M$ to satisfy the left premise, it must satisfy
  $\Gamma' \Sequent \Delta' \setminus \{!A\}$.  But $\Gamma' \subseteq
  \Gamma$ and $\Delta' \setminus \{!A\} \subseteq \Delta$, so $\Struct
  M$ also satisfies $\Gamma \Sequent \Delta$.  In case (b), in order
  for $\Struct M$ to satisfy the right premise, it must satisfy $\Pi'
  \setminus \{!A\} \Sequent \Lambda'$.  But $\Pi' \setminus \{!A\}
  \subseteq \Gamma$ and $\Lambda' \subseteq \Delta$, so $\Struct M$
  also satisfies $\Gamma \Sequent \Delta$.
\item The last inference is $\land$~right.  The premises are $\Gamma
  \Sequent \Delta'$ and $\Gamma \Sequent \Delta''$, where $!A \in
  \Delta'$ an $!B \in \Delta''$.  By induction hypothesis, both are
  valid.  Consider a !!{structure}~$\Struct M$.  We have two cases:
  (a) $\Sat/{M}{!A \land !B}$ or (b) $\Sat{M}{!A \land !B}$. In case
  (a), either $\Sat/{M}{!A}$ or $\Sat/{M}{!B}$.  In the former case,
  in order for $\Struct M$ to satisfy $\Gamma \Sequent \Delta'$, it
  must already satisfy $\Gamma \Sequent \Delta' \setminus \{!A\}$.  In
  the latter case, it must satisfy $\Gamma \Sequent \Delta'' \setminus
  \{!B\}$.  But since both $\Delta' \setminus \{!A\} \subseteq \Delta$
  and $\Delta'' \setminus \{!B\} \subseteq \Delta$, that means
  $\Struct M$ satisfies $\Gamma \Sequent \Delta$.  In case (b),
  $\Struct M$ satisfies $\Gamma \Sequent \Delta$ since $!A \land !B
  \in \Delta$.
\item The last inference is $\lor$~left: Exercise.
\item The last inference is $\lif$~left.  The premises are $\Gamma
  \Sequent \Delta'$ and $\Gamma' \Sequent \Delta$, where $!A \in
  \Delta'$ and $!B \in \Gamma'$.  By induction hypothesis, both are
  valid.  Consider a !!{structure}~$\Struct M$.  We have two cases:
  (a) $\Sat{M}{!A \lif !B}$ or (b) $\Sat/{M}{!A \lif !B}$.  In case
  (a), either $\Sat/{M}{!A}$ or $\Sat{M}{!B}$.  In the former case, in
  order for $\Struct M$ to satisfy $\Gamma \Sequent \Delta'$, it must
  already satisfy $\Gamma \Sequent \Delta' \setminus \{!A\}$.  In the
  latter case, it must satisfy $\Gamma' \setminus \{!B\} \Sequent
  \Delta$.  But since both $\Delta' \setminus \{!A\} \subseteq \Delta$
  and $\Gamma' \setminus \{!B\} \subseteq \Gamma$, that means $\Struct
  M$ satisfies $\Gamma \Sequent \Delta$.  In case (b), $\Struct M$
  satisfies $\Gamma \Sequent \Delta$ since $!A \lif !B \in \Gamma$.
\end{enumerate}
\end{proof}

\begin{prob}
Complete the proof of \olref[fol][seq][sou]{sequent-soundness}.
\end{prob}

\begin{cor}
\ollabel{weak-soundness}
If $\Proves !A$ then $!A$ is valid.
\end{cor}

\begin{cor}
\ollabel{entailment-soundness}
If $\Gamma \Proves !A$ then $\Gamma \Entails !A$.
\end{cor}

\begin{proof}
If $\Gamma \Proves !A$ then for some finite subset $\Gamma_0 \subseteq
\Gamma$, there is !!a{derivation} of $\Gamma_0 \Sequent !A$.  By
\olref{sequent-soundness}, every !!{structure} $\Struct M$ either
makes some $!B \in \Gamma_0$ false or makes $!A$ true.  Hence, if
$\Sat{M}{\Gamma_0}$ then also $\Sat{M}{!A}$.
\end{proof}

\begin{cor}
\ollabel{consistency-soundness}
If $\Gamma$ is satisfiable, then it is consistent.
\end{cor}

\begin{proof}
We prove the contrapositive.  Suppose that $\Gamma$ is not
consistent.  Then $\Gamma \Proves \lfalse$, i.e., there is a finite
$\Gamma_0 \subseteq \Gamma$ and !!a{derivation} of $\Gamma_0 \Sequent
\lfalse$.  By \olref{sequent-soundness}, $\Gamma_0 \Sequent \lfalse$ is
valid.  Since $\Sat/{M}{\lfalse}$ for every !!{structure}~$\Struct M$,
for $\Struct M$ to satisfy $\Gamma_0 \Sequent \lfalse$ there must be
an $!E \in \Gamma_0$ so that $\Sat/{M}{!E}$, and since $\Gamma_0
\subseteq \Gamma$, that $!E$ is also in~$\Gamma$.  In other words, no
$\Struct M$ satisfies $\Gamma$, i.e., $\Gamma$ is not satisfiable.
\end{proof}

\end{document}
