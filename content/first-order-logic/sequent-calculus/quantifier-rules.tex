% Part: first-order-logic
% Chapter: sequent-calculus
% Section: quantifier-rules

\documentclass[../../../include/open-logic-section]{subfiles}

\begin{document}

\olfileid{fol}{seq}{qrl}

\olsection{Quantifier Rules}

\subsection{Rules for $\lforall$}

\begin{defish}
\Axiom$ !A(t), \Gamma \fCenter \Delta$
\RightLabel{\LeftR{\lforall}}
\UnaryInf$ \lforall[x][!A(x)],\Gamma \fCenter \Delta$
\DisplayProof
\hfill
\Axiom$ \Gamma \fCenter \Delta, !A(a) $
\RightLabel{\RightR{\lforall}}
\UnaryInf$ \Gamma \fCenter \Delta, \lforall[x][!A(x)]$
\DisplayProof
\end{defish}

In \LeftR{\lforall}, $t$ is a closed term (i.e., one without
variables). In \RightR{\lforall}, $a$~is !!a{constant} which must
not occur anywhere in the lower sequent of the \RightR{\lforall}
rule. We call $a$ the \emph{eigenvariable} of the \RightR{\forall}
inference.

\subsection{Rules for $\lexists$}

\begin{defish}
\Axiom$ !A(a), \Gamma \fCenter \Delta $
\RightLabel{\LeftR{\lexists}}
\UnaryInf$ \lexists[x][!A(x)], \Gamma \fCenter \Delta$
\DisplayProof
\hfill
\Axiom$ \Gamma \fCenter \Delta, !A(t) $
\RightLabel{\RightR{\lexists}}
\UnaryInf$ \Gamma \fCenter \Delta, \lexists[x][!A(x)]$
\DisplayProof
\end{defish}

Again, $t$~is a closed term, and $a$~is !!a{constant} which does
not occur in the lower sequent of the \LeftR{\lexists} rule. We call
$a$ the \emph{eigenvariable} of the \LeftR{\lexists} inference.

The condition that an eigenvariable not occur in the lower sequent of
the \RightR{\lforall} or \LeftR{\lexists} inference is called the
\emph{eigenvariable condition}.

\begin{explain}
We use the term ``eigenvariable'' even though $a$ in the above rules
is !!a{constant}. This has historical reasons.

In \RightR{\lexists} and \LeftR{\lforall} there are no restrictions on
the term~$t$. On the other hand, in the \LeftR{\lexists} and
\RightR{\lforall} rules, the eigenvariable condition requires that the
!!{constant}~$a$ does not occur anywhere outside of~$!A(a)$ in the
upper sequent. It is necessary to ensure that the system is sound,
i.e., only !!{derive}s sequents that are valid. Without this
condition, the following would be allowed:
\begin{prooftree}
  \Axiom$!A(a) \fCenter !A(a)$
  \RightLabel{*\LeftR{\lexists}}
  \UnaryInf$\lexists[x][!A(x)] \fCenter !A(a)$
  \RightLabel{\RightR{\lforall}}
  \UnaryInf$\lexists[x][!A(x)] \fCenter \lforall[x][!A(x)]$
  \DisplayProof
  \qquad
  \Axiom$!A(a) \fCenter !A(a)$
  \RightLabel{*\RightR{\lforall}}
  \UnaryInf$!A(a) \fCenter \lforall[x][!A(x)]$
  \RightLabel{\LeftR{\lexists}}
  \UnaryInf$\lexists[x][!A(x)] \fCenter \lforall[x][!A(x)]$
\end{prooftree}
However, $\lexists[x][!A(x)] \Sequent \lforall[x][!A(x)]$ is not valid.
\end{explain}

\end{document}
