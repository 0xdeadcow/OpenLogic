% Part: first-order-logic
% Chapter: sequent-calculus
% Section: proving-things-quant

\documentclass[../../../include/open-logic-section]{subfiles}

\begin{document}

\olfileid{fol}{seq}{prq}

\olsection{\usetoken{P}{derivation} with Quantifiers}

\begin{ex}
Give an $\Log{LK}$-derivation of the sequent $\lexists[x][\lnot !A(x)]
\Sequent \lnot \lforall[x][!A(x)]$.

When dealing with quantifiers, we have to make sure not to violate the
eigenvariable condition, and sometimes this requires us to play around
with the order of carrying out certain inferences. In general, it
helps to try and take care of rules subject to the eigenvariable
condition first (they will be lower down in the finished proof). Also,
it is a good idea to try and look ahead and try to guess what the
initial sequent might look like. In our case, it will have to be
something like $!A(a) \Sequent !A(a)$. That means that when we are
``reversing'' the quantifier rules, we will have to pick the same
term---what we will call $a$---for both the $\lforall$ and the
$\lexists$ rule. If we picked different terms for each rule, we would
end up with something like $!A(a) \Sequent !A(b)$, which, of course,
is not derivable.

Starting as usual, we write
\begin{prooftree}
\AxiomC{}
\UnaryInf$\lexists[x][\lnot !A(x)] \fCenter \lnot \lforall[x][!A(x)]$
\end{prooftree}
We could either carry out the \LeftR{\exists} rule or the \RightR{\lnot}
rule. Since the \LeftR{\exists} rule is subject to the eigenvariable
condition, it's a good idea to take care of it sooner rather than
later, so we'll do that one first.
\begin{prooftree}
\AxiomC{}
\UnaryInf$ \lnot !A(a) \fCenter \lnot \lforall[x][!A(x)]$
\RightLabel{\LeftR{\lexists}}
\UnaryInf$ \lexists[x][\lnot !A(x)] \fCenter \lnot \lforall[x][!A(x)]$
\end{prooftree}
Applying the \LeftR{\lnot} and \RightR{\lnot} rules backwards, we get
\begin{prooftree}
\AxiomC{}
\UnaryInf$\lforall[x][!A(x)] \fCenter !A(a)$
\RightLabel{\LeftR{\lnot}}
\UnaryInf$\lnot !A(a), \lforall[x][!A(x)] \fCenter $
\RightLabel{\LeftR{\Exchange}}
\UnaryInf$\lforall[x][!A(x)], \lnot !A(a) \fCenter $
\RightLabel{\RightR{\lnot}}
\UnaryInf$ \lnot !A(a) \fCenter \lnot \lforall[x] !A(x)$
\RightLabel{\LeftR{\lexists}}
\UnaryInf$ \lexists[x] \lnot !A(x) \fCenter \lnot \lforall[x] !A(x)$
\end{prooftree}
At this point, our only option is to carry out the \LeftR{\forall}
rule. Since this rule is not subject to the eigenvariable restriction,
we're in the clear. Remember, we want to try and obtain an initial
sequent (of the form $!A(a) \Sequent !A(a)$), so we should choose $a$
as our argument for $!A$ when we apply the rule.
\begin{prooftree}
\Axiom$!A(a) \fCenter !A(a)$
\RightLabel{\LeftR{\lforall}}
\UnaryInf$\lforall[x][!A(x)] \fCenter !A(a)$
\RightLabel{\LeftR{\lnot}}
\UnaryInf$\lnot !A(a), \lforall[x][!A(x)] \fCenter $
\RightLabel{\LeftR{\Exchange}}
\UnaryInf$\lforall[x][!A(x)], \lnot !A(a) \fCenter $
\RightLabel{\RightR{\lnot}}
\UnaryInf$ \lnot !A(a) \fCenter \lnot \lforall[x][!A(x)]$
\RightLabel{\LeftR{\lexists}}
\UnaryInf$ \lexists[x][ \lnot !A(x)] \fCenter \lnot \lforall[x][!A(x)]$
\end{prooftree}
It is important, especially when dealing with quantifiers, to double
check at this point that the eigenvariable condition has not been
violated. Since the only rule we applied that is subject to the
eigenvariable condition was \LeftR{\exists}, and the eigenvariable~$a$
does not occur in its lower sequent (the end-sequent), this is a
correct derivation.
\end{ex}

\begin{prob}
Give !!{derivation}s of the following sequents:
\begin{enumerate}
\item $\lforall[x][(!A(x) \lif !B)] \Sequent (\lexists[y][!A(y)] \lif !B)$
\item $\lexists[x][(!A(x) \lif \lforall[y][!A(y)])]$
\end{enumerate}
\end{prob}

\end{document}
