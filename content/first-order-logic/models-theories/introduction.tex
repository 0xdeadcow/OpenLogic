% Part: first-order-logic
% Chapter: models-theories
% Section: introduction

\documentclass[../../../include/open-logic-section]{subfiles}

\begin{document}

\olfileid{fol}{mat}{int}
\olsection{Introduction}

\begin{explain}
The development of the axiomatic method is a significant achievement
in the history of science, and is of special importnace in the history
of mathematics.  An axiomatic development of a field involves the
clarification of many questions: What is the field about?  What are the
most fundamental concepts?  How are they related?  Can all the
concepts of the field be defined in terms of these fundamental
concepts?  What laws do, and must, these concepts obey?

The axiomatic method and logic were made for each other.  Formal logic
provides the tools for formulating axiomatic theories, for proving
theorems from the axioms of the theory in a precisely specified way,
for studying the properties of all systems satisfying the axioms in a
systematic way.
\end{explain}

\begin{defn}
A set of !!{sentence}s~$\Gamma$ is \emph{closed} iff, whenever
$\Gamma \Entails !A$ then $!A \in \Gamma$.  The \emph{closure} of a set
of !!{sentence}s~$\Gamma$ is $\Setabs{!A}{\Gamma \Entails !A}$.

We say that~$\Gamma$ is \emph{axiomatized by} a set of
sentences~$\Delta$ if $\Gamma$ is the closure of~$\Delta$
\end{defn}

\begin{explain}
We can think of an axiomatic theory as the set of sentences that is
axiomatized by its set of axioms~$\Delta$.  In other words, when we
have a first-order language which contains non-logical symbols for the
primitives of the axiomatically developed science we wish to study,
together with a set of !!{sentence}s that express the fundamental laws
of the science, we can think of the theory as represented by all the
!!{sentence}s in this language that are entailed by the axioms.  This
ranges from simple examples with only a single primitive and
simple axioms, such as the theory of partial orders, to complex
theories such as Newtonian mechanics.

The important logical facts that make this formal approach to the
axiomatic method so important are the following.  Suppose $\Gamma$ is
an axiom system for a theory, i.e., a set of sentences.
\begin{enumerate}
\item We can state precisely when an axiom system captures an intended
  class of !!{structure}s.  That is, if we are interested in a certain
  class of !!{structure}s, we will successfully capture that class by
  an axiom system~$\Gamma$ iff the !!{structure}s are exactly
  those~$\Struct M$ such that $\Sat{M}{\Gamma}$.
\item We may fail in this respect because there are $\Struct M$ such
  that $\Sat{M}{\Gamma}$, but $\Struct M$ is not one of the
  !!{structure}s we intend.  This may lead us to add axioms which are
  not true in~$\Struct M$.
\item If we are successful at least in the respect that $\Gamma$ is
  true in all the intended !!{structure}s, then a sentence~$!A$ is true in
  all intended !!{structure}s whenever $\Gamma \Entails !A$.  Thus we can
  use logical tools (such as proof methods) to show that sentences are
  true in all intended !!{structure}s simply by showing that they are
  entailed by the axioms.
\item Sometimes we don't have intended !!{structure}s in mind, but instead
  start from the axioms themselves: we begin with some primitives that
  we want to satisfy certain laws which we codify in an axiom system.
  One thing that we would like to verify right away is that the axioms
  do not contradict each other: if they do, there can be no concepts
  that obey these laws, and we have tried to set up an incoherent
  theory.  We can verify that this doesn't happen by finding a model
  of~$\Gamma$.  And if there are models of our theory, we can use
  logical methods to investigate them, and we can also use logical
  methods to construct models.
\item The independence of the axioms is likewise an important
  question.  It may happen that one of the axioms is actually a
  consequence of the others, and so is redundant.  We can prove that
  an axiom $!A$ in $\Gamma$ is redundant by proving $\Gamma \setminus
  \{!A\} \Entails !A$.  We can also prove that an axiom is not
  redundant by showing that $(\Gamma \setminus \{!A\}) \cup \{\lnot
  !A\}$ is satisfiable.  For instance, this is how it was shown that the
  parallel postulate is independent of the other axioms of geometry.
\item Another important question is that of definability of concepts
  in a theory: The choice of the language determines what the models
  of a theory consists of.  But not every aspect of a theory must be
  represented separately in its models.  For instance, every ordering
  $\le$ determines a corresponding strict ordering~$<$---given one, we
  can define the other.  So it is not necessary that a model of a
  theory involving such an order must \emph{also} contain the
  corresponding strict ordering.  When is it the case, in general,
  that one relation can be defined in terms of others?  When is it
  impossible to define a relation in terms of other (and hence must
  add it to the primitives of the language)?
\end{enumerate}
\end{explain}


\end{document}
