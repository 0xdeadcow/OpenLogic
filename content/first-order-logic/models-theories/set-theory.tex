% Part: first-order-logic
% Chapter: models-theories
% Section: set-theory

\documentclass[../../../include/open-logic-section]{subfiles}

\begin{document}

\olfileid{fol}{mat}{set}

\olsection{The Theory of Sets}

Almost all of mathematics can be developed in the theory of sets.
Developing mathematics in this theory involves a number of things.
First, it requires a set of axioms for the relation~$\in$.  A number
of different axiom systems have been developed, sometimes with
conflicting properties of~$\in$.  The axiom system known as
$\Log{ZFC}$, Zermelo-Fraenkel set theory with the axiom of choice
stands out: it is by far the most widely used and studied, because it
turns out that its axioms suffice to prove almost all the things
mathematicians expect to be able to prove.  But before that can be
established, it first is necessary to make clear how we can even
\emph{express} all the things mathematicians would like to express.
For starters, the language contains no !!{constant}s or !!{function}s,
so it seems at first glance unclear that we can talk about particular
sets (such as $\emptyset$ or $\Nat$), can talk about operations on
sets (such as $X \cup Y$ and $\Pow{X}$), let alone other
constructions which involve things other than sets, such as relations
and functions.

To begin with, ``is an element of'' is not the only relation we are
interested in: ``is a subset of'' seems almost as important.  But we
can \emph{define} ``is a subset of'' in terms of ``is an element of.''
To do this, we have to find a !!{formula}~$!A(x, y)$ in
the language of set theory which is satisfied by a pair of
sets~$\tuple{X, Y}$ iff $X \subseteq Y$.  But $X$ is a subset of $Y$
just in case all !!{element}s of~$X$ are also !!{element}s of~$Y$.  So
we can define $\subseteq$ by the formula
\[
\lforall[z][(z \in x \lif z \in y)]
\]
Now, whenever we want to use the relation~$\subseteq$ in a formula, we
could instead use that formula (with $x$ and $y$ suitably replaced,
and the bound variable~$z$ renamed if necessary).  For instance,
extensionality of sets means that if any sets~$x$ and $y$ are
contained in each other, then $x$ and $y$ must be the same set. This
can be expressed by $\lforall[x][\lforall[y][((x \subseteq y \land y
    \subseteq x) \lif x = y)]]$, or, if we replace $\subseteq$ by the
above definition, by
\[
\lforall[x][\lforall[y][((\lforall[z][(z \in x \lif z \in y)] \land
    \lforall[z][(z \in y \lif z \in x)]) \lif x = y)]].
\]
This is in fact one of the axioms of $\Log{ZFC}$, the ``axiom of
extensionality.''

There is no !!{constant} for $\emptyset$, but we can express ``$x$ is
empty'' by $\lnot \lexists[y][y \in x]$.  Then ``$\emptyset$ exists''
becomes the !!{sentence}~$\lexists[x][\lnot \lexists[y][y \in
    x]]$. This is another axiom of~$\Log{ZFC}$.  (Note that the axiom
of extensionality implies that there is only one empty set.)  Whenever
we want to talk about $\emptyset$ in the language of set theory, we
would write this as ``there is a set that's empty and \dots'' As an
example, to express the fact that $\emptyset$ is a subset of every
set, we could write
\[
\lexists[x][(\lnot\lexists[y][y \in x] \land \lforall[z][x \subseteq
    z])]
\]
where, of course, $x \subseteq z$ would in turn have to be replaced by
its definition.

To talk about operations on sets, such has $X \cup Y$ and $\Pow{X}$, we
have to use a similar trick.  There are no function symbols in the
language of set theory, but we can express the functional relations $X
\cup Y = Z$ and $\Pow{X} = Y$ by
\begin{align*}
& \lforall[u][((u \in x \lor u \in y) \liff u \in z)]\\
& \lforall[u][(u \subseteq x \liff u \in y )]
\end{align*}
since the !!{element}s of $X \cup Y$ are exactly the sets that are
either !!{element}s of~$X$ or !!{element}s of~$Y$, and the
!!{element}s of $\Pow{X}$ are exactly the subsets of~$X$.  However,
this doesn't allow us to use $x \cup y$ or $\Pow{x}$ as if they were
terms: we can only use the entire !!{formula}s that define the
relations $X \cup Y = Z$ and $\Pow{X} = Y$. In fact, we do not know
that these relations are ever satisfied, i.e., we do not know that
unions and power sets always exist. For instance, the !!{sentence}
$\lforall[x][\lexists[y][\Pow{x} = y]]$ is another axiom
of~$\Log{ZFC}$ (the power set axiom).

Now what about talk of ordered pairs or functions?  Here we have to
explain how we can think of ordered pairs and functions as special
kinds of sets.  One way to define the ordered pair $\tuple{x, y}$ is
as the set $\{\{x\}, \{x, y\}\}$.  But like before, we cannot
introduce a !!{function} that names this set; we can only define the
relation $\tuple{x, y} = z$, i.e., $\{\{x\}, \{x, y\}\} = z$:
\[
\lforall[u][(u \in z \liff (\lforall[v][(v \in u \liff v = x)] \lor
  \lforall[v][(v \in u \liff (v = x \lor v = y))]))]
\]
This says that the !!{element}s~$u$ of~$z$ are exactly those sets which
either have $x$ as its only !!{element} or have $x$ and~$y$ as its
only !!{element}s (in other words, those sets that are either identical
to $\{x\}$ or identical to $\{x, y\}$).  Once we have this, we can say
further things, e.g., that $X \times Y = Z$:
\[
\lforall[z][(z \in Z \liff \lexists[x][\lexists[y][(x \in
        X \land y \in Y \land \tuple{x, y} = z)]])]
\]

A function $f \colon X \to Y$ can be thought of as the relation $f(x)
= y$, i.e., as the set of pairs~$\Setabs{\tuple{x,y}}{f(x) = y}$. We
can then say that a set~$f$ is a function from $X$ to $Y$ if (a) it is
a relation $\subseteq X \times Y$, (b) it is total, i.e., for all $x
\in X$ there is some $y \in Y$ such that $\tuple{x, y} \in f$ and (c)
it is functional, i.e., whenever $\tuple{x, y}, \tuple{x, y'} \in f$,
$y = y'$ (because values of functions must be unique). So ``$f$ is a
function from $X$ to $Y$'' can be written as:
\begin{align*}
 \lforall[u][(u \in f \lif {}] & \lexists[x][\lexists[y][(x \in X \land y \in
      Y \land \tuple{x, y} = u)]]) \land {}\\
 \lforall[x][(x \in X \lif {}] &
      (\lexists[y][(y \in Y \land \mathrm{maps}(f, x, y))] \land {}\\
& (\lforall[y][\lforall[y'][((\mathrm{maps}(f, x, y) \land
    \mathrm{maps}(f, x, y')) \lif y = y')]]))
\end{align*}
where $\mathrm{maps}(f, x, y)$ abbreviates $\lexists[v][(v \in f \land
  \tuple{x, y} = v)]$ (this !!{formula} expresses ``$f(x) = y$'').

It is now also not hard to express that $f\colon X \to Y$ is
!!{injective}, for instance:
\begin{multline*}
 f \colon X \to Y \land \lforall[x][\lforall[x'][((x \in X \land x' \in
     X \land {}]] \\
 \lexists[y][(\mathrm{maps}(f, x, y) \land \mathrm{maps}(f,
        x', y))]) \lif x = x')
\end{multline*}
A function~$f\colon X \to Y$ is !!{injective} iff, whenever $f$ maps $x, x'
\in X$ to a single~$y$, $x = x'$.  If we abbreviate this formula as
$\mathrm{inj}(f, X, Y)$, we're already in a position to state in the
language of set theory something as non-trivial as Cantor's theorem:
there is no !!{injective} function from $\Pow{X}$ to $X$:
\[
\lforall[X][\lforall[Y][(\Pow{X} = Y \lif
    \lnot\lexists[f][\mathrm{inj}(f, Y, X)])]]
\]


\end{document}
