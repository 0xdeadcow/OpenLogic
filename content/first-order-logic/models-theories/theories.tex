% Part: first-order-logic
% Chapter: models-theories
% Section: theories

\documentclass[../../../include/open-logic-section]{subfiles}

\begin{document}

\olfileid{fol}{mat}{the}

\olsection{Examples of First-Order Theories}

\begin{ex}
The theory of strict linear orders in the language~$\Lang L_<$ is
axiomatized by the set
\begin{align*}
\{\quad & \lforall[x][\lnot x < x], \\
& \lforall[x][\lforall[y][((x < y \lor y <
    x) \lor x = y)]], \\
& \lforall[x][\lforall[y][\lforall[z][((x < y
      \land y < z) \lif x < z)]]] \quad \}
\end{align*}
It completely captures the intended !!{structure}s: every strict
linear order is a model of this axiom system, and vice versa, if $R$
is a linear order on a set $X$, then the structure $\Struct M$ with
$\Domain M = X$ and $\Assign{<}{M} = R$ is a model of this theory.
\end{ex}

\begin{ex}
The theory of groups in the language $\Obj 1$ (!!{constant}), $\cdot$
(two-place !!{function}) is axiomatized by
\begin{align*}
& \lforall[x][\eq[(x \cdot \Obj 1)][x]]\\
& \lforall[x][\lforall[y][\lforall[z][\eq[(x \cdot (y \cdot z))][((x
          \cdot y) \cdot z)]]]]\\
& \lforall[x][\lexists[y][\eq[(x \cdot y)][\Obj 1]]]
\end{align*}
\end{ex}

\begin{ex}
The theory of Peano arithmetic is axiomatized by the following
sentences in the language of arithmetic~$\Lang L_A$.
\begin{align*}
& \lforall[x][\lforall[y][(\eq[x'][y'] \lif \eq[x][y])]]\\
& \lforall[x][\eq/[\Obj 0][x']]\\
& \lforall[x][\eq[(x + \Obj 0)][x]]\\
& \lforall[x][\lforall[y][\eq[(x + y')][(x + y)']]]\\
& \lforall[x][\eq[(x \times \Obj 0)][\Obj 0]]\\
& \lforall[x][\lforall[y][\eq[(x \times y')][((x \times y) + x)]]]\\
& \lforall[x][\lforall[y][(x < y \liff \lexists[z][\eq[(z' + x)][y])]]]\\
\intertext{plus all sentences of the form}
& (!A(\Obj 0) \land \lforall[x][(!A(x) \lif !A(x'))]) \lif \lforall[x][!A(x)]
\end{align*}
Since there are infinitely many sentences of the latter form, this
axiom system is infinite.  The latter form is called the
\emph{induction schema}. (Actually, the induction schema is a bit more
complicated than we let on here.)

The last axiom is an \emph{explicit definition} of~$<$.
\end{ex}

\begin{ex}
The theory of pure sets plays an important role in the foundations
(and in the philosophy) of mathematics.  A set is pure if all its
!!{element}s are also pure sets.  The empty set counts therefore as
pure, but a set that has something as !!a{element} that is not a set
would not be pure.  So the pure sets are those that are formed just
from the empty set and no ``urelements,'' i.e., objects that are not
themselves sets.

The following might be considered as an axiom system for a theory of
pure sets:
\begin{align*}
& \lexists[x][\lnot \lexists[y][y \in x]]\\
& \lforall[x][\lforall[y][(\lforall[z](z \in x \liff z \in y) \lif 
\eq[x][y])]]\\
& \lforall[x][\lforall[y][\lexists[z][\lforall[u][(u \in z \liff
(\eq[u][x] \lor \eq[u][y]))]]]]\\
& \lforall[x][\lexists[y][\lforall[z][(z \in y \liff \lexists[u][(z \in
        u \land u \in x)])]]]\\
\intertext{plus all sentences of the form} &
\lexists[x][\lforall[y][(y \in x \liff !A(y))]]
\end{align*}
The first axiom says that there is a set with no !!{element}s (i.e.,
$\emptyset$ exists); the second says that sets are extensional; the
third that for any sets $X$ and $Y$, the set $\{X, Y\}$ exists; the
fourth that for any set $X$, the set $\cup X$ exists, where $\cup X$ is the 
union of all the elements of $X$.

The !!{sentence}s mentioned last are collectively called the
\emph{naive comprehension scheme}.  It essentially says that for every
$!A(x)$, the set $\Setabs{x}{!A(x)}$ exists---so at first glance a
true, useful, and perhaps even necessary axiom.  It is called ``naive''
because, as it turns out, it makes this theory unsatisfiable: if you
take $!A(y)$ to be $\lnot y \in y$, you get the !!{sentence}
\[
\lexists[x][\lforall[y][(y \in x \liff \lnot y \in y)]]
\]
and this !!{sentence} is not satisfied in any !!{structure}.
\end{ex}

\begin{ex}
In the area of \emph{mereology}, the relation of \emph{parthood} is a
fundamental relation.  Just like theories of sets, there are theories
of parthood that axiomatize various conceptions (sometimes
conflicting) of this relation.

The language of mereology contains a single two-place predicate
symbol~$\Obj P$, and $\Atom{\Obj P}{x, y}$ ``means'' that $x$ is a
part of~$y$.  When we have this interpretation in mind, !!a{structure}
for this language is called a \emph{parthood structure}.  Of course,
not every structure for a single two-place predicate will really
deserve this name.  To have a chance of capturing ``parthood,''
$\Assign{\Obj P}{M}$ must satisfy some conditions, which we can lay
down as axioms for a theory of parthood.  For instance, parthood is a
partial order on objects: every object is a part (albeit an
\emph{improper} part) of itself; no two different objects can be parts
of each other; a part of a part of an object is itself part of that
object.  Note that in this sense ``is a part of'' resembles ``is a
subset of,'' but does not resemble ``is an element of'' which is
neither reflexive nor transitive.
\begin{align*}
& \lforall[x][\Atom{\Obj P}{x,x}] \\
& \lforall[x][\lforall[y][((\Part{x}{y} \land \Part{y}{x})
      \lif \eq[x][y])]] \\
& \lforall[x][\lforall[y][\lforall[z][((\Part{x}{y} \land
        \Part{y}{z}) \lif \Part{x}{z})]]]\\
\intertext{Moreover, any two objects have a mereological sum (an object that has
  these two objects as parts, and is minimal in this respect).}  &
\lforall[x][\lforall[y][\lexists[z][\lforall[u][(\Part{z}{u} \liff
        (\Part{x}{u} \land \Part{y}{u}))]]]]
\end{align*}
These are only some of the basic principles of parthood considered by
metaphysicians.  Further principles, however, quickly become hard to
formulate or write down without first introducing some defined
relations.  For instance, most metaphysicians interested in mereology
also view the following as a valid principle: whenever an
object~$x$ has a proper part~$y$, it also has a part~$z$ that has no
parts in common with~$y$, and so that the fusion of $y$ and $z$ is
$x$.
\end{ex}

\end{document}
