% Part: first-order-logic
% Chapter: natural-deduction
% Section: provability-quantifiers

% verification of properties of provability needed for maximally
% consistent sets in the completeness chapter.

\documentclass[../../../include/open-logic-section]{subfiles}

\begin{document}

\olfileid{fol}{axd}{qpr}

\olsection{\usetoken{S}{derivability} and the Quantifiers}

\begin{thm}
\ollabel{thm:strong-generalization} If $c$ is a constant not occurring
in $\Gamma$ or $!A(x)$ and $\Gamma \Proves !A(c)$, then $\Gamma
\Proves \lforall[x][!A(x)]$.
\end{thm}

\begin{proof}
By the deduction theorem, $\Gamma \Proves \ltrue \lif !A(c)$. Since
$c$ does not occur in $\Gamma$ or $\top$, we get $\Gamma \Proves
\ltrue \lif !A(c)$. By the deduction theorem again, $\Gamma \Proves
\lforall[x][!A(x)]$.
\end{proof}

\begin{prop}
\ollabel{prop:provability-quantifiers}
\begin{tagenumerate}{prvEx,prvAll}
\tagitem{prvEx}{$!A(t) \Proves \lexists[x][!A(x)]$.}{}

\tagitem{prvAll}{$\lforall[x][!A(x)] \Proves !A(t)$.}{}
\end{tagenumerate}
\end{prop}

\begin{proof}
\begin{tagenumerate}{prvEx,prvAll}
\tagitem{prvEx}{By \olref[qua]{ax:q2} and the deduction theorem.}{}
\tagitem{prvAll}{By \olref[qua]{ax:q1} and the deduction theorem.}{}
\end{tagenumerate}
\end{proof}

\end{document}
