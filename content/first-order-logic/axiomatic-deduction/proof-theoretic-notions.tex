% Part: first-order-logic
% Chapter: axiomatic-deduction
% Section: proof-theoretic-notions

\documentclass[../../../include/open-logic-section]{subfiles}

\begin{document}

\olfileid{fol}{axd}{ptn}
\olsection{Proof-Theoretic Notions}

\begin{explain}
Just as we've defined a number of important semantic notions
(validity, entailment, satisfiabilty), we now define corresponding
\emph{proof-theoretic notions}.  These are not defined by appeal to
satisfaction of !!{sentence}s in !!{structure}s, but by appeal to the
!!{derivability} or !!{nonderivability} of certain formulas.  It was
an important discovery, due to G\"odel, that these notions coincide.
That they do is the content of the \emph{completeness theorem}.

The proof-theoretic notions for propositional logic are similar.
\end{explain}

\begin{defn}[!!^{derivability}]
!!^a{formula} $!A$ is \emph{!!{derivable}} from $\Gamma$, written
$\Gamma \Proves !A$, if there is !!a{derivation} from~$\Gamma$ ending
in~$!A$.
\end{defn}

\begin{defn}[Theorems]
!!^a{formula}~$!A$ is a \emph{theorem} if there is !!a{derivation} of
$!A$ from the empty set.  We write $\Proves !A$ if $!A$ is a theorem
and $\Proves/ !A$ if it is not.
\end{defn}

\begin{defn}[Consistency]
A set $\Gamma$ of !!{formula}s is \emph{consistent} if and only if
$\Gamma\Proves/ \lfalse$; it is \emph{inconsistent} otherwise.
\end{defn}

\begin{prop}[Reflexivity]
\ollabel{prop:reflexivity}
If $!A \in \Gamma$, then $\Gamma \Proves !A$.
\end{prop}

\begin{proof}
  The !!{formula}~$!A$ by itself is !!a{derivation} of~$!A$ from~$\Gamma$.
\end{proof}

\begin{prop}[Monotony]
\ollabel{prop:monotony}
If $\Gamma \subseteq \Delta$ and $\Gamma \Proves !A$, then $\Delta
\Proves !A$.
\end{prop}

\begin{proof}
Any !!{derivation} of $!A$ from $\Gamma$ is also !!a{derivation} of
$!A$ from~$\Delta$.
\end{proof}

\begin{prop}[Transitivity]
\ollabel{prop:transitivity}
If $\Gamma \Proves !A$ for every $!A \in \Delta$ and $\Delta \Proves
!B$, then $\Gamma \Proves !B$.
\end{prop}

\begin{proof}
  Suppose $\Delta \Proves !B$. Then there is !!a{derivation} $!B_1$, \dots, $!B_l = !B$
  from~$\Delta$. Some of the steps in that derivation will be correct
  because of a rule which refers to a prior line~$!B_i$ that includes
  !!a{formula}~$!B_i \in \Delta$. Let $!A_1$, \dots, $!A_n$ be all the
  !!{formula}s in~$\Delta$ so referenced. For each one of them, by
  hypothesis, there is !!a{derivation} of~$!A_i$ from~$\Gamma$, i.e.,
  !!a{derivation}~$!A_i^1$, \dots, $!A_i^{k_i} = !A_i$ where every $!A_i^j$
  is an axiom, !!a{element} of~$\Gamma$, or correct by a rule of
  inference. Now consider the sequence
  \[
  !A_1^1, \dots, !A_1^{k_1}, \dots, !A_n^1, \dots, !A_n^{k_n}, !B_1, \dots, !B_l = !B.
  \]
  This is a correct !!{derivation} of~$!B$ from $\Gamma$ since each
  $!B_i \in \Delta$ needed to justify the inferences in the part
  $!B_1$, \dots, $!B_l$ is now justified itself.
\end{proof}
  
\begin{prop}
\ollabel{prop:incons}
$\Gamma$ is inconsistent iff $\Gamma \Proves !A$ for every
  sentence~$!A$.
\end{prop}

\begin{proof}
Exercise.
\end{proof}

\begin{prob}
Prove \olref[fol][axd][ptn]{prop:incons}
\end{prob}

\begin{prop}[Compactness]
\ollabel{prop:proves-compact}
  \begin{enumerate}
  \item If $\Gamma \Proves !A$ then there is a finite subset $\Gamma_0
    \subseteq \Gamma$ such that $\Gamma_0 \Proves !A$.
  \item If every finite subset of~$\Gamma$ is
    consistent, then $\Gamma$ is consistent.
  \end{enumerate}
\end{prop}

\begin{proof}
  \begin{enumerate}
    \item If $\Gamma \Proves !A$, then there is a finite sequence of
      !!{formula}s $!A_1$, \dots,~$!A_n$ so that $!A \ident !A_n$ and
      each $!A_i$ is either a logical axiom, !!a{element} of~$\Gamma$
      or follows from previous !!{formula}s by modus ponens.  Take
      $\Gamma_0$ to be those $!A_i$ which are in~$\Gamma$.  Then the
      !!{derivation} is likewise a !!{derivation} from~$\Gamma_0$, and
      so $\Gamma_0 \Proves !A$.
    \item This is the contrapositive of~(1) for the special case $!A
      \ident \lfalse$.
\end{enumerate}
\end{proof}

\end{document}
