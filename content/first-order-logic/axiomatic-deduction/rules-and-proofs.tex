% Part: first-order-logic 
% Chapter: axiomatic-deduction 
% Section: rules-and-proofs

\documentclass[../../include/open-logic-section]{subfiles}

\begin{document}

\olfileid{fol}{axd}{rul}

\olsection{Rules and \usetoken{P}{derivation}}

\begin{explain}
  Axiomatic !!{derivation}s are perhaps the simplest proof system for
  logic. A !!{derivation} is just a sequence of !!{formula}s.  To
  count as !!a{derivation}, every !!{formula} in the sequence must
  either be an instance of an axiom, or must follow from one or more
  !!{formula}s that precede it in the sequence by a rule of inference.
  A !!{derivation} !!{derive}s its last !!{formula}.
\end{explain}

\begin{defn}[!!^{derivability}]
If $\Gamma$ is a set of !!{formula}s of $\Lang L$ and $!A$ a
!!{formula}, then a \emph{!!{derivation}} of $!A$ from $\Gamma$ is a
finite sequence $!A_1$, \dots,~$!A_n$ of !!{formula}s such that $!A_n
= !A$ and for each $i \le n$ one of the following holds:
\begin{enumerate}
\item $!A_i \in \Gamma$; or
\item $!A_i$ is an axiom; or
\item $!A_i$ follows from some $!A_j$ (and $!A_k$) with $j < i$ (and $k < i$)
  by a rule of inference.
\end{enumerate}
We write $\Gamma \Proves !A$ (``$\Gamma$ !!{derive}s $!A$,'' or ``$!A$
is !!{derivable} from $\Gamma$'') to mean that there is a
!!{derivation} of $!A$ from $\Gamma$. When $\Gamma$ is empty, we write
$\Proves !A$ to mean $\emptyset \Proves !A$.
\end{defn}

\end{document}
