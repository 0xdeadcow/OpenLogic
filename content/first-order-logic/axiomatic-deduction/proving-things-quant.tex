% Part: first-order-logic 
% Chapter: axiomatic-deduction 
% Section: proving-things-quant

\documentclass[../../../include/open-logic-section]{subfiles}

\begin{document}

\olfileid{fol}{axd}{prq}

\olsection{\usetoken{P}{derivation} with Quantifiers}

\begin{ex}
Let us give a derivation of $(\lforall[x][!A(x)] \land
\lforall[y][!B(y)]) \lif \lforall[x][(!A(x) \land !B(x))]$.

First, note that
\begin{align*}
  (\lforall[x][!A(x)] \land \lforall[y][!B(y)]) & \lif \lforall[x][!A(x)]\\
  \intertext{is an instance of \olref[prp]{ax:land1}, and}
  \lforall[x][!A(x)] & \lif !A(a) \\
  \intertext{of \olref[qua]{ax:q1}. So, by \olref[pro]{prop:chain}, we know that}
  (\lforall[x][!A(x)] \land \lforall[y][!B(y)]) & \lif !A(a) \\
  \intertext{is !!{derivable}. Likewise, since}
  (\lforall[x][!A(x)] \land \lforall[y][!B(y)]) & \lif \lforall[y][!B(y)] \qquad\text{and}\\
    \lforall[y][!B(y)] & \lif !B(a)\\
    \intertext{are instances of \olref[prp]{ax:land2} and \olref[qua]{ax:q1}, respectively,}
    (\lforall[x][!A(x)] \land \lforall[y][!B(y)]) & \lif !B(a)\\
    \intertext{is derivable by \olref[pro]{prop:chain}. Using an appropriate instance of \olref[prp]{ax:land3} and two applications of~\MP, we see that}
    (\lforall[x][!A(x)] \land \lforall[y][!B(y)]) & \lif (!A(a) \land !B(a))\\
    \intertext{is derivable. We can now apply \QR{} to obtain}
(\lforall[x][!A(x)] \land \lforall[y][!B(y)]) & \lif \lforall[x][(!A(x) \land !B(x))].
\end{align*}
\end{ex}

\end{document}
