% Part: first-order-logic
% Chapter: introduction
% Section: sentences

\documentclass[../../../include/open-logic-section]{subfiles}

\begin{document}

\olfileid{fol}{int}{snt}

\olsection{\usetoken{P}{sentence}}

Ok, now we have a (sketch of a) definition of satisfaction (``true
in'') for !!{structure}s and !!{formula}s. But it needs this
additional bit---!!a{variable} assignment---and what we wanted is a
definition of !!{sentence}s. How do we get rid of assignments, and
what are !!{sentence}s?

You probably remember a discussion in your first introduction to
formal logic about the relation between !!{variable}s and quantifiers.
A quantifier is always followed by !!a{variable}, and then in the part
of the !!{sentence} to which that quantifier applies (its ``scope''),
we understand that the !!{variable} is ``bound'' by that quantifier.
In !!{formula}s it was not required that every !!{variable} has a
matching quantifier, and !!{variable}s without matching quantifiers
are ``free'' or ``unbound.''  We will take !!{sentence}s to be all
those !!{formula}s that have no free !!{variable}s.

Again, the intuitive idea of when an occurrence of !!a{variable} in
!!a{formula}~$!A$ is bound, which quantifier binds it, and when it is
free, is not difficult to get. You may have learned a method for
testing this, perhaps involving counting parentheses.  We have to
insist on a precise definition---and because we have defined
!!{formula}s by induction, we can give a definition of the free and
bound occurrences of !!a{variable}~$x$ in !!a{formula}~$!A$ also by
induction.  E.g., it might look like this for our simplified language:
\begin{enumerate}
  \item If $!A$ is atomic, all occurrences of $x$ in it are free (that
  is, the occurrence of $x$ in~$\Atom{\Obj P}{x}$ is free).
  \item If $!A$ is of the form $\lnot !B$, then an occurrence of~$x$
  in~$\lnot !B$ is free iff the corresponding occurrence of~$x$ is
  free in~$!B$ (that is, the free occurrences of variables in~$
  !B$ are exactly the corresponding occurrences in~$\lnot !B$).
  \item If $!A$ is of the form $(!B \land !C)$, then an occurrence of~$x$
  in~$(!B \land !C)$ is free iff the corresponding occurrence of~$x$ is
  free in~$!B$ or in~$!C$.
  \item If $!A$ is of the form $\lexists[x][!B]$, then no occurrence
  of $x$ in~$!A$ is free; if it is of the form $\lexists[y][!B]$ where
  $y$ is a different !!{variable} than~$x$, then an occurrence of~$x$
  in $\lexists[y][!B]$ is free iff the corresponding occurrence of~$x$
  is free in~$!B$.
\end{enumerate}

Once we have a precise definition of free and bound occurrences of
variables, we can simply say: !!a{sentence} is any !!{formula} without
free occurrences of !!{variable}s.

\end{document}
