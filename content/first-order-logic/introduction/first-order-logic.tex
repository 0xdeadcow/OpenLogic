% Part: first-order-logic
% Chapter: introduction
% Section: first-order-logic

\documentclass[../../../include/open-logic-section]{subfiles}

\begin{document}

\olfileid{fol}{int}{fol}

\olsection{First-Order Logic}

You are probably familiar with first-order logic from your first
introduction to formal logic.\footnote{In fact, we more or less assume
you are!{} If you're not, you could review a more elementary textbook,
such as \emph{forall x} \citep{Magnus2021}.} You may know it as
``quantificational logic'' or ``predicate logic.''  First-order
logic, first of all, is a formal language.  That means, it has a
certain vocabulary, and its expressions are strings from this
vocabulary.  But not every string is permitted.  There are different
kinds of permitted expressions: terms, !!{formula}s, and
!!{sentence}s.  We are mainly interested in !!{sentence}s of
first-order logic: they provide us with a formal analogue of sentences
of English, and about them we can ask the questions a logician
typically is interested in. For instance: 
\begin{itemize}
    \item Does $!B$ follow from~$!A$ logically?
    \item Is $!A$ logically true, logically false, or
contingent?
    \item Are $!A$ and $!B$ equivalent?
\end{itemize}

These questions are primarily questions about the ``meaning'' of
!!{sentence}s of first-order logic.  For instance, a philosopher would
analyze the question of whether $!B$ follows logically from~$!A$ as
asking: is there a case where $!A$ is true but~$!B$ is false ($!B$
doesn't follow from~$!A$), or does every case that makes $!A$ true
also make~$!B$ true ($!B$ does follow from~$!A$)?  But we haven't been
told yet what a ``case'' is---that is the job of \emph{semantics}.  The
semantics of first-order logic provides a mathematically precise model
of the philosopher's intuitive idea of ``case,'' and also---and this
is important---of what it is for !!a{sentence}~$!A$ to be \emph{true
in} a case. We call the mathematically precise model that we will
develop !!a{structure}. The relation which makes ``true in'' precise,
is called the relation of \emph{satisfaction}.  So what we will define
is ``$!A$ is satisfied in~$\Struct{M}$'' (in symbols: $\Sat{M}{!A}$)
for !!{sentence}s~$!A$ and !!{structure}s~$\Struct{M}$. Once this is
done, we can also give precise definitions of the other semantical
terms such as ``follows from'' or ``is logically true.'' These
definitions will make it possible to settle, again with mathematical
precision, whether, e.g., $\lforall[x][(!A(x) \lif !B(x)),
\lexists[x][!A(x)] \Entails \lexists[x][!B(x)]]$. The answer will, of
course, be ``yes.'' If you've already been trained to symbolize
sentences of English in first-order logic, you will recognize this as,
e.g., the symbolizations of, say, ``All ants are insects, there are
ants, therefore there are insects.'' That is obviously a valid
argument, and so our mathematical model of ``follows from'' for our
formal language should give the same answer.

Another topic you probably remember from your first introduction to
formal logic is that there are \emph{!!{derivation}s}.  If you have
taken a first formal logic course, your instructor will have made you
practice finding such !!{derivation}s, perhaps even !!a{derivation}
that shows that the above entailment holds.  There are many different
ways to give !!{derivation}s: you may have done something called
``natural deduction'' or ``truth trees,'' but there are many others.
The purpose of !!{derivation} systems is to provide tools using which the
logicians' questions above can be answered: e.g., a natural deduction
!!{derivation} in which $\lforall[x][(!A(x) \lif !B(x))$ and
$\lexists[x][!A(x)]$ are premises and $\lexists[x][!B(x)]]$ is the
conclusion (last line) \emph{verifies} that $\lexists[x][!B(x)]$
logically follows from $\lforall[x][(!A(x) \lif !B(x))]$ and
$\lexists[x][!A(x)]$.  

But why is that?  On the face of it, !!{derivation} systems have nothing to do
with semantics: giving a formal !!{derivation} merely involves arranging symbols
in certain rule-governed ways; they don't mention ``cases'' or ``true
in'' at all.  The connection between !!{derivation} systems and semantics has
to be established by a meta-logical investigation. What's needed is a
mathematical proof, e.g., that a formal !!{derivation} of $\lexists[x][!B(x)]$
from premises $\lforall[x][(!A(x) \lif !B(x))]$ and
$\lexists[x][!A(x)]$ is possible, if, and only if, $\lforall[x][(!A(x)
\lif !B(x))$ and $\lexists[x][!A(x)]$ together
entails~$\lexists[x][!B(x)]]$.  Before this can be done, however, a
lot of painstaking work has to be carried out to get the definitions
of syntax and semantics correct.

\end{document}