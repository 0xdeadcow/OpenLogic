% Part: first-order-logic
% Chapter: natural-deduction
% Section: identity

\documentclass[../../../include/open-logic-section]{subfiles}

\begin{document}

\olfileid{fol}{ntd}{ide}

\olsection{\usetoken{P}{derivation} with \usetoken{S}{identity}}

!!^{derivation}s with the !!{identity} require additional inference rules.

\paragraph{Rules for $\eq$:}

\begin{gather*}
\AxiomC{}
\RightLabel{\Intro{\eq}}
\UnaryInfC{$\eq[t][t]$}
\DisplayProof
\\
\AxiomC{$\eq[t_1][t_2]$}
\AxiomC{$!A(t_1)$}
\RightLabel{\Elim{\eq}}
\BinaryInfC{$!A(t_2)$}
\DisplayProof
\quad
\textrm{  and  }
\quad
\AxiomC{$\eq[t_1][t_2]$}
\AxiomC{$!A(t_2)$}
\RightLabel{\Elim{\eq}}
\BinaryInfC{$!A(t_1)$}
\DisplayProof
\end{gather*}
where $t_1$ and $t_2$ are closed terms. The \Intro{\eq} rule allows us
to !!{derive} any identity statement of teh form $\eq[t][t]$
outright.

\begin{ex}
If $s$ and $t$ are closed terms, then $!A(s), \eq[s][t] \Proves !A(t)$:
\[
\AxiomC{$!A(s)$}
\AxiomC{$\eq[s][t]$}
\RightLabel{$\Elim{\eq}$}
\BinaryInfC{$!A(t)$}
\DisplayProof
\]
This may be familiar as the ``principle of substitutability of
identicals,'' or Leibniz' Law.
\end{ex}

\begin{prob}
Prove that $=$ is both symmetric and transitive, i.e., give
!!{derivation}s of $\lforall[x][\lforall[y][(\eq[x][y] \lif
    \eq[y][x])]]$ and $\lforall[x][\lforall[y][\lforall[z]((\eq[x][y]
    \land \eq[y][z]) \lif \eq[x][z])]]$
\end{prob}

\begin{prob}
Give !!{derivation}s of the following !!{formula}s:
\begin{enumerate}
\item $\lforall[x][\lforall[y][((\eq[x][y] \land !A(x)) \lif !A(y))]]$
\item $\lexists[x][!A(x)] \land \lforall[y][\lforall[z][((!A(y) \land
    !A(z)) \lif \eq[y][z])]] \lif \lexists[x][(!A(x) \land
  \lforall[y][(!A(y) \lif \eq[y][x])])]$
\end{enumerate}
\end{prob}

\begin{prop}
Natural deduction with rules for identity is sound.
\end{prop}

\begin{proof}
Any !!{formula} of the form $\eq[t][t]$ is valid, since
for every !!{structure}~$\Struct M$, $\Sat{M}{\eq[t][t]}$. (Note that
we assume the term $t$ to be ground, i.e., it contains no variables,
so variable assignments are irrelevant).

Suppose the last inference in a !!{derivation} is \Elim{\eq}. Then the
premises are $\eq[t_1][t_2]$ and $!A(t_1)$; they are !!{derive}d from
!!{undischarged} assumptions~$\Gamma$ and $\Delta$, respectively.  We
want to show that $!A(s)$ follows from $\Gamma \cup \Delta$.  Consider
a !!{structure}~$\Struct{M}$ with $\Sat{M}{\Gamma \cup \Delta}$. By induction
hypothesis, $\Struct{M}$ satisfies the two premises by induction
hypothesis. So, $\Sat{M}{\eq[t_1][t_2]}$. Therefore, $\Value{t_1}{M} =
\Value{t_2}{M}$. Let $s$ be any variable assignment, and $s'$ be the
$x$-variant given by $s'(x) = \Value{t_1}{M} = \Value {t_2}{M}$. By
\olref[fol][syn][ext]{prop:ext-formulas}, $\Sat{M}{!A(t_2)}[s]$ iff
$\Sat{M}{!A(x)}[s']$ iff $\Sat{M}{!A(t_1)}[s]$. Since $\Sat{M}
      {!A(t_1)}$ therefore $\Sat{M}{!A(t_2)}$.
\end{proof}

\end{document}
