% Part: first-order-logic
% Chapter: natural-deduction
% Section: proving-things-quant

\documentclass[../../../include/open-logic-section]{subfiles}

\begin{document}

\olfileid{fol}{ntd}{prq}

\olsection{\usetoken{P}{derivation} with Quantifiers}

\begin{ex}
When dealing with quantifiers, we have to make sure not to violate the
eigenvariable condition, and sometimes this requires us to play around
with the order of carrying out certain inferences. In general, it
helps to try and take care of rules subject to the eigenvariable
condition first (they will be lower down in the finished proof).

Let's see how we'd give a !!{derivation} of the !!{formula}
$\lexists[x][\lnot !A(x)] \lif \lnot \lforall[x][!A(x)]$.
Starting as usual, we write
\begin{prooftree}
\AxiomC{}
\UnaryInfC{$\lexists[x][\lnot !A(x)]\lif \lnot \lforall[x][!A(x)]$}
\end{prooftree}
We start by writing down what it would take to justify that last step
using the \Intro{\lif} rule.
\begin{prooftree}
\AxiomC{$\Discharge{\lexists[x][\lnot !A(x)]}{1}$}
\DeduceC{$\lnot \lforall[x][!A(x)]$}
\DischargeRule{\Intro{\lif}}{1}
\UnaryInfC{$\lexists[x][\lnot !A(x)]\lif \lnot \lforall[x][!A(x)]$}
\end{prooftree}
Since there is no obvious rule to apply to $\lnot \lforall[x][!A(x)]$,
we will proceed by setting up the !!{derivation} so we can use the
\Elim{\lexists} rule. Here we must pay attention to the eigenvariable
condition, and choose a constant that does not appear in
$\lexists[x][!A(x)]$ or any assumptions that it depends on.
(Since no !!{constant}s appear, however, any choice will do fine.)
\begin{prooftree}
\AxiomC{$\Discharge{\lexists[x][\lnot !A(x)]}{1}$}
\AxiomC{$\Discharge{\lnot !A(a)}{2}$}
\DeduceC{$\lnot \lforall[x][!A(x)]$}
\DischargeRule{\Elim{\lexists}}{2}
\BinaryInfC{$\lnot \lforall[x][!A(x)]$}
\DischargeRule{\Intro{\lif}}{1}
\UnaryInfC{$\lexists[x][\lnot !A(x)] \lif \lnot \lforall[x][!A(x)]$}
\end{prooftree}
In order to derive $\lnot \lforall[x][!A(x)]$, we will attempt to use
the \Intro{\lnot} rule: this requires that we derive a contradiction,
possibly using $\lforall[x][!A(x)]$ as an additional assumption. Of
course, this contradiction may involve the assumption $\lnot !A(a)$
which will be discharged by the \Intro{\lif} inference. We can set it
up as follows:
\begin{prooftree}
\AxiomC{$\Discharge{\lexists[x][\lnot !A(x)]}{1}$}
\AxiomC{$\Discharge{\lnot !A(a)}{2}, \Discharge{\lforall[x][!A(x)]}{3}$}
\DeduceC{$\lfalse$}
\DischargeRule{\Intro{\lnot}}{3}
\UnaryInfC{$\lnot \lforall[x][!A(x)]$}
\DischargeRule{\Elim{\lexists}}{2}
\BinaryInfC{$\lnot \lforall[x][!A(x)]$}
\DischargeRule{\Intro{\lif}}{1}
\UnaryInfC{$\lexists[x][\lnot !A(x)]\lif \lnot \lforall[x][!A(x)]$}
\end{prooftree}
It looks like we are close to getting a contradiction. The easiest
rule to apply is the \Elim{\lforall}, which has no eigenvariable
conditions. Since we can use any term we want to replace the
universally quantified~$x$, it makes the most sense to continue
using~$a$ so we can reach a contradiction.
\begin{prooftree}
\AxiomC{$\Discharge{\lexists[x][\lnot !A(x)]}{1}$}
\AxiomC{$\Discharge{\lnot !A(a)}{2}$}
\AxiomC{$\Discharge{\lforall[x][!A(x)]}{3}$}
\RightLabel{\Elim{\lforall}}
\UnaryInfC{$!A(a)$}
\RightLabel{\Elim{\lnot}}
\BinaryInfC{$\lfalse$}
\DischargeRule{\Intro{\lnot}}{3}
\UnaryInfC{$\lnot \lforall[x][!A(x)]$}
\DischargeRule{\Elim{\lexists}}{2}
\BinaryInfC{$\lnot \lforall[x][!A(x)]$}
\DischargeRule{\Intro{\lif}}{1}
\UnaryInfC{$\lexists[x][\lnot !A(x)]\lif \lnot \lforall[x][!A(x)]$}
\end{prooftree}

It is important, especially when dealing with quantifiers, to double
check at this point that the eigenvariable condition has not been
violated. Since the only rule we applied that is subject to the
eigenvariable condition was \Elim{\exists}, and the eigenvariable~$a$
does not occur in any assumptions it depends on, this is a
correct derivation.
\end{ex}

\begin{ex}
Sometimes we may derive a !!{formula} from other !!{formula}s.
In these cases, we may have undischarged assumptions. It is 
important to keep track of our assumptions as well
as the end goal.

Let's see how we'd give a !!{derivation} of the !!{formula}
$\lexists[x][!C(x,b)]$ from the assumptions $\lexists[x][(!A(x) 
\land !B(x))]$ and $\lforall[x][(!B(x) \lif !C(x,b))]$.
Starting as usual, we write the conclusion at the
bottom.
\begin{prooftree}
\AxiomC{}
\UnaryInfC{$\lexists[x][!C(x,b)]$}
\end{prooftree}

We have two premises to work with. To use the first, i.e., try to find
a !!{derivation} of $\lexists[x][!C(x, b)]$ from $\lexists[x][(!A(x)
  \land !B(x))]$ we would use the \Elim{\lexists} rule. Since it has
an eigenvariable condition, we will apply that rule first. We get the
following:
\begin{prooftree}
\AxiomC{$\lexists[x][(!A(x) \land !B(x))]$}
\AxiomC{$\Discharge{!A(a) \land !B(a)}{1}$}
\DeduceC{$\lexists[x][!C(x,b)]$}
\DischargeRule{\Elim{\lexists}}{1}
\BinaryInfC{$\lexists[x][!C(x,b)]$}
\end{prooftree}
The two assumptions we are working with share~$!B$.  It may be useful
at this point to apply \Elim{\land} to separate out~$!B(a)$.
\begin{prooftree}
\AxiomC{$\lexists[x][(!A(x) \land !B(x)])$}
\AxiomC{$\Discharge{!A(a) 
\land !B(a)}{1}$}
\RightLabel{\Elim{\land}}
\UnaryInfC{$!B(a)$}
\DeduceC{$\lexists[x][!C(x,b)]$}
\DischargeRule{\Elim{\lexists}}{1}
\BinaryInfC{$\lexists[x][!C(x,b)]$}
\end{prooftree}

The second assumption we have to work with is~$\lforall[x][(!B(x) \lif
  !C(x,b))]$. Since there is no eigenvariable condition we can
instantiate $x$ with the !!{constant}~$a$ using \Elim{\lforall} to get
$!B(a) \lif !C(a, b)$.  We now have both $!B(a) \lif !C(a,b)$ and
$!B(a)$. Our next move should be a straightforward application of the
\Elim{\lif} rule.
\begin{prooftree}
\AxiomC{$\lexists[x][(!A(x) \land !B(x))]$}
\AxiomC{$\lforall[x][(!B(x) \lif !C(x,b))]$}
\RightLabel{\Elim{\lforall}}
\UnaryInfC{$!B(a) \lif !C(a,b)$}
\AxiomC{$\Discharge{!A(a) 
\land !B(a)}{1}$}
\RightLabel{\Elim{\land}}
\UnaryInfC{$!B(a)$}
\RightLabel{\Elim{\lif}}
\BinaryInfC{$!C(a,b)$}
\DeduceC{$\lexists[x][!C(x,b)]$}
\DischargeRule{\Elim{\lexists}}{1}
\insertBetweenHyps{\hspace{-5em}}
\BinaryInfC{$\lexists[x][!C(x,b)]$}
\end{prooftree}
We are so close!{} One application of \Intro{\lexists} and we
have reached our goal.
\begin{prooftree}
\AxiomC{$\lexists[x][(!A(x) \land !B(x))]$}
\AxiomC{$\lforall[x][(!B(x) \lif !C(x,b))]$}
\RightLabel{\Elim{\lforall}}
\UnaryInfC{$!B(a) \lif !C(a,b)$}
\AxiomC{$\Discharge{!A(a) 
\land !B(a)}{1}$}
\RightLabel{\Elim{\land}}
\UnaryInfC{$!B(a)$}
\RightLabel{\Elim{\lif}}
\BinaryInfC{$!C(a,b)$}
\RightLabel{\Intro{\lexists}}
\UnaryInfC{$\lexists[x][!C(x,b)]$}
\DischargeRule{\Elim{\lexists}}{1}
\insertBetweenHyps{\hspace{-5em}}
\BinaryInfC{$\lexists[x][!C(x,b)]$}
\end{prooftree}
Since we ensured at each step that the eigenvariable
conditions were not violated, we can be confident that this
is a correct derivation.
\end{ex}

\begin{ex}
Give a !!{derivation} of the !!{formula}
$\lnot\lforall[x][!A(x)]$ from the assumptions $\lforall[x][!A(x)] 
\lif \lexists[y][!B(y)]$ and $\lnot\lexists[y][!B(y)]$.
Starting as usual, we write the target !!{formula} at the bottom.
\begin{prooftree}
\AxiomC{}
\UnaryInfC{$\lnot\lforall[x][!A(x)]$}
\end{prooftree}
The last line of the !!{derivation} is a negation, so let's try using
\Intro{\lnot}. This will require that we figure out how to !!{derive}
a contradiction.
\begin{prooftree}
\AxiomC{$\Discharge{\lforall[x][!A(x)]}{1}$}
\DeduceC{$\lfalse$}
\DischargeRule{\Intro{\lnot}}{1}
\UnaryInfC{$\lnot\lforall[x][!A(x)]$}
\end{prooftree}
So far so good. We can use \Elim{\lforall} but it's not obvious
if that will help us get to our goal. Instead, let's use one of our 
assumptions. $\lforall[x][!A(x)] \lif \lexists[y][!B(y)]$ together
with $\lforall[x][!A(x)]$ will allow us to use the \Elim{\lif} rule.
\begin{prooftree}
\AxiomC{$\lforall[x][!A(x)] \lif \lexists[y][!B(y)]$}
\AxiomC{$\Discharge{\lforall[x][!A(x)]}{1}$}
\RightLabel{\Elim{\lif}}
\BinaryInfC{$\lexists[y][!B(y)]$}
\DeduceC{$\lfalse$}
\DischargeRule{\Intro{\lnot}}{1}
\UnaryInfC{$\lnot\lforall[x][!A(x)]$}
\end{prooftree}
We now have one final assumption to work with,
and it looks like this will help us reach a contradiction
by using \Elim{\lnot}.
\begin{prooftree}
\AxiomC{$\lnot\lexists[y][!B(y)]$}
\AxiomC{$\lforall[x][!A(x)] \lif \lexists[y][!B(y)]$}
\AxiomC{$\Discharge{\lforall[x][!A(x)]}{1}$}
\RightLabel{\Elim{\lif}}
\BinaryInfC{$\lexists[y][!B(y)]$}
\RightLabel{\Elim{\lnot}}
\BinaryInfC{$\lfalse$}
\DischargeRule{\Intro{\lnot}}{1}
\UnaryInfC{$\lnot\lforall[x][!A(x)]$}
\end{prooftree}
\end{ex}

\begin{prob}
Give !!{derivation}s of the following:
\begin{enumerate}
\item $\lexists[y][!A(y)] \lif !B$ from the assumption
  $\lforall[x][(!A(x) \lif !B)]$
\item $\lexists[x][(!A(x) \lif \lforall[y][!A(y)])]$
\end{enumerate}
\end{prob}



\end{document}
