% Part: first-order-logic
% Chapter: natural-deduction
% Section: soundness-identity

\documentclass[../../../include/open-logic-section]{subfiles}

\begin{document}

\olfileid{fol}{ntd}{sid}

\olsection{Soundness of \usetoken{S}{identity} Rules}

\begin{prop}
Natural deduction with rules for identity is sound.
\end{prop}

\begin{proof}
Any !!{formula} of the form $\eq[t][t]$ is valid, since
for every !!{structure}~$\Struct M$, $\Sat{M}{\eq[t][t]}$. (Note that
we assume the term $t$ to be ground, i.e., it contains no variables,
so variable assignments are irrelevant).

Suppose the last inference in a !!{derivation} is \Elim{\eq}. Then the
premises are $\eq[t_1][t_2]$ and $!A(t_1)$; they are !!{derive}d from
!!{undischarged} assumptions~$\Gamma$ and $\Delta$, respectively.  We
want to show that $!A(s)$ follows from $\Gamma \cup \Delta$.  Consider
a !!{structure}~$\Struct{M}$ with $\Sat{M}{\Gamma \cup \Delta}$. By induction
hypothesis, $\Struct{M}$ satisfies the two premises by induction
hypothesis. So, $\Sat{M}{\eq[t_1][t_2]}$. Therefore, $\Value{t_1}{M} =
\Value{t_2}{M}$. Let $s$ be any variable assignment, and $s'$ be the
$x$-variant given by $s'(x) = \Value{t_1}{M} = \Value {t_2}{M}$. By
\olref[fol][syn][ext]{prop:ext-formulas}, $\Sat{M}{!A(t_2)}[s]$ iff
$\Sat{M}{!A(x)}[s']$ iff $\Sat{M}{!A(t_1)}[s]$. Since $\Sat{M}
      {!A(t_1)}$ therefore $\Sat{M}{!A(t_2)}$.
\end{proof}

\end{document}
