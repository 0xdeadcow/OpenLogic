% Part: first-order-logic
% Chapter: sequent-calculus
% Section: proof-theoretic-notions

\documentclass[../../../include/open-logic-section]{subfiles}

\begin{document}

\olfileid{fol}{ntd}{ptn}
\olsection{Proof-Theoretic Notions}

\begin{editorial}
This section collects the definitions the provability relation
and consistency for LK.
\end{editorial}

\begin{explain}
Just as we've defined a number of important semantic notions
(validity, entailment, satisfiabilty), we now define corresponding
\emph{proof-theoretic notions}.  These are not defined by appeal to
satisfaction of sentences in !!{structure}s, but by appeal to the
!!{derivability} or !!{nonderivability} of certain !!{formula}s from
others.  It was an important discovery, due to G\"odel, that these
notions coincide.  That they do is the content of the
\emph{completeness theorem}.
\end{explain}


\begin{defn}[Theorems]
A !!{formula}~$!A$ is a \emph{theorem} if there is a !!{derivation}
of~$!A$ in natural deduction in which all assumptions are
!!{discharged}.  We write $\Proves !A$ if $!A$ is a theorem and
$\Proves/ !A$ if it is not.
\end{defn}

\begin{defn}[!!^{derivability}]
!!^a{sentence} $!A$ is \emph{!!{derivable} from} a set of
!!{sentence}s~$\Gamma$, $\Gamma \Proves !A$, if there is a
!!{derivation} with end-!!{formula}~$!A$ and in which every assumption
is either !!{discharged} or is in~$\Gamma$. If $!A$ is not
!!{derivable} from $\Gamma$ we write $\Gamma \Proves/ !A$.
\end{defn}

\begin{defn}[Consistency]
A set of sentences~$\Gamma$ is \emph{inconsistent} iff $\Gamma
\Proves \lfalse$.  If $\Gamma$ is not inconsistent, i.e., if
$\Gamma \Proves/ \lfalse$, we say it is \emph{consistent}.
\end{defn}

\begin{prop}
\ollabel{prop:prov-incons}
$\Gamma \Proves !A$ iff $\Gamma \cup \{\lnot !A\}$ is inconsistent.
\end{prop}

\begin{proof}
First suppose $\Gamma \Proves !A$, i.e., there is
!!a{derivation}~$\delta_0$ of~$!A$ from !!{undischarged}
assumptions~$\Gamma$. We obtain !!a{derivation} of $\lfalse$ from
$\Gamma \cup \{\lnot !A\}$ as follows:
\begin{prooftree}
  \AxiomC{$\lnot !A$}
  \AxiomC{$\Gamma$}
  \RightLabel{$\delta_0$}
  \DeduceC{$!A$}
  \RightLabel{\Elim{\lnot}}
  \BinaryInfC{$\lfalse$}
\end{prooftree}

Now assume $\Gamma \cup \{\lnot !A\}$ is inconsistent, and let
$\delta_1$ be the corresponding derivation of $\lfalse$ from
!!{undischarged} assumptions in~$\Gamma \cup \{\lnot !A\}$. We obtain
!!a{derivation} of~$!A$ from~$\Gamma$ alone by using $\FalseCl$:
\begin{prooftree}
  \AxiomC{$\Gamma, \Discharge{\lnot !A}{1}$}
  \RightLabel{$\delta_1$}
  \DeduceC{$\lfalse$}
  \RightLabel{\FalseCl}
  \UnaryInfC{$!A$}
\end{prooftree}
\end{proof}

\begin{prop}
\ollabel{prop:incons}
$\Gamma$ is inconsistent iff $\Gamma \Proves {!A}$ for every
  sentence~$!A$.
\end{prop}

\begin{proof}
Exercise.
\end{proof}

\begin{prob}
Prove \olref[fol][ntd][ptn]{prop:incons}
\end{prob}

\begin{prop}[Reflexivity]
\ollabel{prop:reflexivity}
If $!A \in \Gamma$, then $\Gamma \Proves !A$.
\end{prop}

\begin{proof}
The assumption $!A$ by itself is !!a{derivation} of~$!A$ where every
!!{undischarged} assumption (i.e., $!A$) is in~$\Gamma$.
\end{proof}
  

\begin{prop}[Monotony]
\ollabel{prop:monotony}
If $\Gamma \subseteq \Delta$ and $\Gamma \Proves !A$, then $\Delta
\Proves !A$.
\end{prop}

\begin{proof}
Any !!{derivation} of $!A$ from $\Gamma$ is also !!a{derivation} of
$!A$ from~$\Delta$.
\end{proof}

\begin{prop}[Transitivity]
\ollabel{prop:transitivity}
If $\Gamma \Proves !A$ for every $!A \in \Delta$ and $\Delta \Proves
!B$, then $\Gamma \Proves !B$.
\end{prop}

\begin{proof}
If $\Delta \Proves !B$, then there is !!a{derivation}~$\delta_0$
of~$!B$ with all !!{undischarged} assumptions in~$\Delta$.  We show
that $\Gamma \Proves !B$ by induction on the number~$n$ of
!!{undischarged} assumptions in~$\delta_0$.

If $n=0$, then $\delta_0$ has no !!{undischarged} assumptions, and so
also counts as !!a{derivation} of~$!B$ from~$\Gamma$.

Otherwise, pick !!a{undischarged} assumption~$!A$ in~$\delta_0$ and
let $\Delta_1$ be the remaining !!{undischarged} assumptions. We
obtain the !!{derivation}~$\delta_1$:
\begin{prooftree}
  \AxiomC{$\Delta_1, \Discharge{!A}{1}$}
  \RightLabel{$\delta_0$}
  \DeduceC{$!B$}
  \RightLabel{\Intro{\lif}}
  \UnaryInfC{$!A \lif !B$}
\end{prooftree}
Since the number of !!{undischarged} assumptions in $\delta_1$ is $n-1$, the
inductive hypothesis applies: there is a !!{derivation}~$\delta_2$ of
$!A \lif !B$ from~$\Gamma$. Since $\Gamma \Proves !A$ there is also
!!a{derivation}~$\delta_3$ of $!A$ from~$\Gamma$. Now consider:
\begin{prooftree}
  \AxiomC{$\Gamma$}
  \RightLabel{$\delta_2$}
  \DeduceC{$!A \lif !B$}
  \AxiomC{$\Gamma$}
  \RightLabel{$\delta_3$}
  \DeduceC{$!A$}
  \RightLabel{\Elim{\lif}}
  \BinaryInfC{$!B$}
\end{prooftree}
This shows $\Gamma \Proves !B$.
\end{proof}

\begin{prop}[Compactness]
  \ollabel{prop:proves-compact}
  \begin{enumerate}
  \item If $\Gamma \Proves !A$ then there is a finite subset $\Gamma_0
    \subseteq \Gamma$ such that $\Gamma_0 \Proves !A$.
  \item If every finite subset of~$\Gamma$ is
    consistent, then $\Gamma$ is consistent.
  \end{enumerate}
\end{prop}

\begin{proof}
  \begin{enumerate}
    \item If $\Gamma \Proves !A$, then there is !!a{derivation}
      of~$!A$ from~$\Gamma$. But any !!{derivation} is finite, so can
      only contain finitely many !!{undischarged} assumptions. Let
      $\Gamma_0$ be the set of !!{undischarged} assumptions of the
      !!{derivation}; it is !!a{derivation} of~$!A$ from a
      finite~$\Gamma_0 \subseteq \Gamma$.
    \item This is the contrapositive of (1) for the special case $!A
      \ident \lfalse$.
  \end{enumerate}
\end{proof}

\end{document}
