% Part: first-order-logic
% Chapter: tableaux
% Section: quantifier-rules

\documentclass[../../../include/open-logic-section]{subfiles}

\begin{document}

\olfileid{fol}{tab}{qrl}

\olsection{Quantifier Rules}

\subsection{Rules for $\lforall$}

\begin{defish}
\AxiomC{\sFmla{\True}{\lforall[x][!A(x)]}}
\RightLabel{\TRule{\True}{\forall}}
\UnaryInfC{\sFmla{\True}{!A(t)}}
\DisplayProof
\hfill
\AxiomC{\sFmla{\False}{\lforall[x][!A(x)]}}
\RightLabel{\TRule{\False}{\lforall}}
\UnaryInfC{\sFmla{\False}{!A(a)}}
\DisplayProof
\end{defish}

In \TRule{\True}{\lforall}, $t$ is a closed term (i.e., one without
variables). In \TRule{\False}{\lforall}, $a$~is !!a{constant} which
must not occur anywhere in the branch above \TRule{\False}{\lforall}
rule. We call $a$ the \emph{eigenvariable} of the
\TRule{\False}{\forall} inference.\footnote{We use the term
``eigenvariable'' even though $a$ in the above rule is !!a{constant}.
This has historical reasons.}

\subsection{Rules for $\lexists$}

\begin{defish}
\AxiomC{\sFmla{\True}{\lexists[x][!A(x)]}}
\RightLabel{\TRule{\True}{\lexists}}
\UnaryInfC{\sFmla{\True}{!A(a)}}
\DisplayProof
\hfill
\AxiomC{\sFmla{\False}{\lexists[x][!A(x)]}}
\RightLabel{\TRule{\False}{\lexists}}
\UnaryInfC{\sFmla{\False}{!A(t)}}
\DisplayProof
\end{defish}

Again, $t$~is a closed term, and $a$~is !!a{constant} which does not
occur in the branch above the~\TRule{\True}{\lexists} rule. We call
$a$ the \emph{eigenvariable} of the \TRule{\True}{\lexists} inference.

The condition that an eigenvariable not occur in the branch above the
\TRule{\False}{\lforall} or \TRule{\True}{\lexists} inference is
called the \emph{eigenvariable condition}.

\begin{explain}
Recall the convention that when $!A$ is !!a{formula} with the
!!{variable}~$x$ free, we indicate this by writing~$!A(x)$. In the
same context, $!A(t)$ then is short for~$\Subst{!A}{t}{x}$. So we
could also write the \TRule{\False}{\lexists} rule as:
\begin{prooftree}
  \AxiomC{\sFmla{\False}{\lexists[x][!A]}}
  \RightLabel{\TRule{\False}{\lexists}}
  \UnaryInfC{\sFmla{\False}{\Subst{!A}{t}{x}}}
\end{prooftree}
Note that $t$ may already occur in~$!A$, e.g., $!A$~might
be~$\Atom{\Obj P}{t,x}$. Thus, inferring $\sFmla{\False}{\Atom{\Obj
P}{t,t}}$ from~$ \sFmla{\False}{\lexists[x][\Atom{\Obj P}{t,x}]}$ is
a correct application of~\TRule{\False}{\lexists}. However, the
eigenvariable conditions in \TRule{\False}{\lforall}
and~\TRule{\True}{\lexists} require that the !!{constant}~$a$ does
not occur in~$!A$. So, you cannot correctly infer
$\sFmla{\False}{\Atom{\Obj P}{a,a}}$ from
$\sFmla{\False}{\lforall[x][\Atom{\Obj P}{a,x}]}$
using~$\TRule{\False}{\lforall}$.
\end{explain}

\begin{explain}
In \TRule{\True}{\lforall} and \TRule{\False}{\lexists} there are no
restrictions on the term~$t$. On the other hand, in the
\TRule{\True}{\lexists} and \TRule{\False}{\lforall} rules, the
eigenvariable condition requires that the !!{constant}~$a$ does not
occur anywhere in the branches above the respective inference. It is
necessary to ensure that the system is sound. Without this condition,
the following would be a closed !!{tableau} for
$\lexists[x][\formula{A}(x)] \lif \lforall[x][\formula{A}(x)]$:
\begin{center}
\begin{tableau}{}
  [\sFmla{\False}{\lexists[x][\formula{A}(x)] \lif \lforall[x][\formula{A}(x)]}, just=\TAss
    [\sFmla{\True}{\lexists[x][\formula{A}(x)]},
      just={\TRule{\False}{\lif}[1]}
      [\sFmla{\False}{\lforall[x][\formula{A}(x)]},
        just={\TRule{\False}{\lif}[1]}
        [\sFmla{\True}{\formula{A}(a)},
          just={\TRule{\True}{\lexists}[2]}
          [\sFmla{\False}{\formula{A}(a)},
            just={\TRule{\False}{\lforall}[3]}, close]
        ]
      ]
    ]
  ]
\end{tableau}
\end{center}
However, $\lexists[x][\formula{A}(x)] \lif
\lforall[x][\formula{A}(x)]$ is not valid.
\end{explain}

\end{document}
