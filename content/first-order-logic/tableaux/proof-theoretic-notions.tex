% Part: first-order-logic
% Chapter: tableaux
% Section: proof-theoretic-notions

\documentclass[../../../include/open-logic-section]{subfiles}

\begin{document}

\olfileid{fol}{tab}{ptn}
\olsection{Proof-Theoretic Notions}

\begin{editorial}
This section collects the definitions of the provability relation
and consistency for tableaux.
\end{editorial}

\begin{explain}
Just as we've defined a number of important semantic notions
(validity, entailment, satisfiabilty), we now define corresponding
\emph{proof-theoretic notions}.  These are not defined by appeal to
satisfaction of !!{sentence}s in !!{structure}s, but by appeal to the
existence of certain closed !!{tableau}x.  It was an important
discovery, due to G\"odel, that these notions coincide.  That they do
is the content of the \emph{completeness theorem}.
\end{explain}


\begin{defn}[Theorems]
A !!{sentence}~$!A$ is a \emph{theorem} if there is a closed
!!{tableau} for~$\sFmla{\False}{!A}$.  We write $\Proves !A$ if $!A$
is a theorem and $\Proves/ !A$ if it is not.
\end{defn}

\begin{defn}[!!^{derivability}]
!!^a{sentence} $!A$ is \emph{!!{derivable} from} a set of
!!{sentence}s~$\Gamma$, $\Gamma \Proves !A$, iff there is a
finite set $\{!B_1, \dots, !B_n\} \subseteq \Gamma$
and a closed !!{tableau} for the set
\[
\{
  \sFmla{\False}{!A},
  \sFmla{\True}{!B_1}, \dots,
  \sFmla{\True}{!B_n},  
  \}
\]
If $!A$ is not !!{derivable} from $\Gamma$ we write $\Gamma \Proves/
!A$.
\end{defn}

\begin{defn}[Consistency]
A set of !!{sentence}s~$\Gamma$ is \emph{inconsistent} iff there is a
finite set $\{!B_1, \dots, !B_n\} \subseteq \Gamma$ and a closed
!!{tableau} for the set
\[
\{
  \sFmla{\True}{!B_1}, \dots,
  \sFmla{\True}{!B_n},  
  \}.
\]
If $\Gamma$ is not inconsistent, we say it is \emph{consistent}.
\end{defn}

\begin{prop}[Reflexivity]
\ollabel{prop:reflexivity}
If $!A \in \Gamma$, then $\Gamma \Proves !A$.
\end{prop}

\begin{proof}
If $!A \in \Gamma$, $\{!A\}$ is a finite subset of~$\Gamma$ and the !!{tableau}
\begin{oltableau}
  [\sFmla{\False}{\formula{A}}, just = \TAss
    [\sFmla{\True}{\formula{A}}, just = \TAss,close]
  ]
\end{oltableau}
is closed.
\end{proof}
  

\begin{prop}[Monotony]
\ollabel{prop:monotony}
If $\Gamma \subseteq \Delta$ and $\Gamma \Proves !A$, then $\Delta
\Proves !A$.
\end{prop}

\begin{proof}
Any finite subset of~$\Gamma$ is also a finite subset of~$\Delta$.
\end{proof}

\begin{prop}[Transitivity]
\ollabel{prop:transitivity}
If $\Gamma \Proves !A$ and $\{!A\} \cup
\Delta \Proves !B$, then $\Gamma \cup \Delta \Proves !B$.
\end{prop}

\begin{proof}
If $\{!A\} \cup \Delta \Proves !B$, then there is a finite subset $\Delta_0 =
\{!C_1, \dots, !C_n\} \subseteq \Delta$ such that
\begin{align*}
\{\sFmla{\False}{!B}, & \sFmla{\True}{!A}, \sFmla{\True}{!C_1},
\dots, \sFmla{\True}{!C_n}\}
\intertext{has a closed !!{tableau}. If $\Gamma \Proves !A$ then there
  are $!D_1$, \dots, $!D_m$ such that}
\{\sFmla{\False}{!A}, & \sFmla{\True}{!D_1},
\dots, \sFmla{\True}{!D_m}\}
\end{align*}
has a closed !!{tableau}.

Now consider the !!{tableau} with assumptions
\[
\sFmla{\False}{!B},
\sFmla{\True}{!C_1}, \dots, \sFmla{\True}{!C_n},
\sFmla{\True}{!D_1}, \dots, \sFmla{\True}{!D_m}.
\]
Apply the \Cut{} rule on~$!A$. This generates two branches, one has
$\sFmla{\True}{!A}$ in it, the other $\sFmla{\False}{!A}$. Thus,
on the one branch, all of
\[
\{\sFmla{\False}{!B}, \sFmla{\True}{!A},
\sFmla{\True}{!C_1}, \dots, \sFmla{\True}{!C_n}\}
\]
are available. Since there is a closed !!{tableau} for these
assumptions, we can attach it to that branch; every branch through
$\sFmla{\True}{!A_1}$ closes. On the other branch, all of
\[
\{\sFmla{\False}{!A}, \sFmla{\True}{!D_1}, \dots,
\sFmla{\True}{!D_m}\}
\]
are available, so we can also complete the other side to obtain a
closed !!{tableau}.  This shows $\Gamma \cup \Delta \Proves !B$.
\end{proof}

Note that this means that in particular if $\Gamma \Proves !A$ and $!A
\Proves !B$, then $\Gamma \Proves !B$. It follows also that if $!A_1,
\dots, !A_n \Proves !B$ and $\Gamma \Proves !A_i$ for each~$i$, then
$\Gamma \Proves !B$.

\begin{prop}
\ollabel{prop:incons}
$\Gamma$ is inconsistent iff $\Gamma \Proves !A$ for every
  !!{sentence}~$!A$.
\end{prop}

\begin{proof}
Exercise.
\end{proof}

\begin{prob}
Prove \olref[fol][tab][ptn]{prop:incons}
\end{prob}

\begin{prop}[Compactness]
  \ollabel{prop:proves-compact}
  \begin{enumerate}
  \item If $\Gamma \Proves !A$ then there is a finite subset $\Gamma_0
    \subseteq \Gamma$ such that $\Gamma_0 \Proves !A$.
  \item If every finite subset of~$\Gamma$ is
    consistent, then $\Gamma$ is consistent.
  \end{enumerate}
\end{prop}

\begin{proof}
  \begin{enumerate}
    \item If $\Gamma \Proves !A$, then there is a finite subset
      $\Gamma_0 = \{!B_1, \dots, !B_n\}$ and a closed !!{tableau} for
      \[
      \sFmla{\False}{!A}, \sFmla{\True}{!B_1}, \dotsm \sFmla{\True}{!B_n}
      \]
      This !!{tableau} also shows $\Gamma_0 \Proves !A$.
    \item If $\Gamma$ is inconsistent, then for some finite subset
      $\Gamma_0 = \{!B_1, \dots, !B_n\}$ there is a closed !!{tableau}
      for
      \[
      \sFmla{\True}{!B_1}, \dotsm \sFmla{\True}{!B_n}
      \]
      This closed !!{tableau} shows that $\Gamma_0$ is inconsistent.
  \end{enumerate}
\end{proof}

\end{document}
