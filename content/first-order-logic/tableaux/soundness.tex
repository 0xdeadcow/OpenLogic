% Part: first-order-logic
% Chapter: tableaux
% Section: soundness.tex

\documentclass[../../../include/open-logic-section]{subfiles}

\begin{document}

\olfileid{fol}{tab}{sou}
\olsection{Soundness}

\begin{explain}
A !!{derivation} system, such as tableaux, is \emph{sound}
if it cannot !!{derive} things that do not actually hold.  Soundness is
thus a kind of guaranteed safety property for !!{derivation} systems.
Depending on which proof theoretic property is in question, we would
like to know for instance, that
\begin{enumerate}
\item every !!{derivable} !!{sentence} is valid;
\item if a !!{sentence} is !!{derivable} from some others, it is also a
  consequence of them;
\item if a set of !!{sentence}s is inconsistent, it is unsatisfiable.
\end{enumerate}
These are important properties of a !!{derivation} system.  If any of them do
not hold, the !!{derivation} system is deficient---it would !!{derive} too much.
Consequently, establishing the soundness of a !!{derivation} system is of the
utmost importance.

Because all these proof-theoretic properties are defined via closed
!!{tableau}s of some kind or other, proving (1)--(3) above requires
proving something about the semantic properties of closed
!!{tableau}s.  We will first define what it means for !!a{signed
  formula} to be satisfied in a structure, and then show that if a
!!{tableau} is closed, no structure satisfies all its assumptions.
(1)--(3) then follow as corollaries from this result.
\end{explain}

\begin{defn}
A !!{structure}~$\Struct M$ \emph{satisfies} !!a{signed formula}
$\sFmla{\True}{!A}$ iff $\Sat{M}{!A}$, and it satisfies
$\sFmla{\False}{!A}$ iff $\Sat/{M}{!A}$. $\Struct{M}$ satisfies a set
of !!{signed formula}s~$\Gamma$ iff it satisfies every $\sFmla{S}{!A}
\in \Gamma$. $\Gamma$~is \emph{satisfiable} if there is !!a{structure}
that satisfies it, and \emph{unsatisfiable} otherwise.
\end{defn}

\begin{thm}[Soundness]
  \ollabel{thm:tableau-soundness}
  If $\Gamma$ has a closed !!{tableau}, $\Gamma$ is unsatisfiable.
\end{thm}

\begin{proof}
Let's call a branch of !!a{tableau} satisfiable iff the set of
!!{signed formula}s on it is satisfiable, and let's call a !!{tableau}
satisfiable if it contains at least one satisfiable branch.

We show the following: Extending a satisfiable !!{tableau} by one of
the rules of inference always results in a satisfiable !!{tableau}.
This will prove the theorem: any closed !!{tableau} results by
applying rules of inference to the !!{tableau} consisting only of
assumptions from~$\Gamma$. So if $\Gamma$ were satisfiable, any
!!{tableau} for it would be satisfiable. A closed !!{tableau},
however, is clearly not satisfiable: every branch contains both
$\sFmla{\True}{!A}$ and $\sFmla{\False}{!A}$, and no structure can
both satisfy and not satisfy~$!A$.

Suppose we have a satisfiable !!{tableau}, i.e., !!a{tableau} with at
least one satisfiable branch. Applying a rule of inference either adds
!!{signed formula}s to a branch, or splits a branch in two. If the
!!{tableau} has a satisfiable branch which is not extended by the rule
application in question, it remains a satisfiable branch in the
extended !!{tableau}, so the extended tableau is satisfiable. So we
only have to consider the case where a rule is applied to a
satisfiable branch.

Let $\Gamma$ be the set of !!{signed formula}s on that branch, and
let $\sFmla{S}{!A} \in \Gamma$ be the !!{signed formula} to which the
rule is applied. If the rule does not result in a split branch, we
have to show that the extended branch, i.e., $\Gamma$ together with
the conclusions of the rule, is still satisfiable. If the rule results
in split branch, we have to show that at least one of the two
resulting branches is satisfiable.
  
First, we consider the possible inferences with only one premise.
\begin{enumerate}
\item The branch is expanded by applying $\TRule{\True}{\lnot}$ to
  $\sFmla{\True}{\lnot !B} \in \Gamma$. Then the extended branch
  contains the !!{signed formula}s $\Gamma \cup
  \{\sFmla{\False}{!B}\}$.  Suppose $\Sat{M}{\Gamma}$. In particular,
  $\Sat{M}{\lnot !B}$. Thus, $\Sat/{M}{!B}$, i.e., $\Struct{M}$
  satisfies $\sFmla{\False}{!B}$. 
\item The branch is expanded by applying $\TRule{\False}{\lnot}$ to
  $\sFmla{\False}{\lnot !B} \in \Gamma$: Exercise.
\item The branch is expanded by applying $\TRule{\True}{\land}$ to
  $\sFmla{\True}{!B \land !C} \in \Gamma$, which results in two new
  !!{signed formula}s on the branch: $\sFmla{\True}{!B}$ and
  $\sFmla{\True}{!C}$. Suppose $\Sat{M}{\Gamma}$, in particular
  $\Sat{M}{!B \land !C}$. Then $\Sat{M}{!B}$ and $\Sat{M}{!C}$. This
  means that $\Struct{M}$ satisfies both $\sFmla{\True}{!B}$ and
  $\sFmla{\True}{!C}$.
\item The branch is expanded by applying $\TRule{\False}{\lor}$ to
  $\sFmla{\True}{!B \lor !C} \in \Gamma$: Exercise.
\item The branch is expanded by applying $\TRule{\False}{\lif}$ to
  $\sFmla{\True}{!B \lif !C} \in \Gamma$: This results in two new
  !!{signed formula}s on the branch: $\sFmla{\True}{!B}$ and
  $\sFmla{\False}{!C}$. Suppose $\Sat{M}{\Gamma}$, in particular
  $\Sat/{M}{!B \lif !C}$. Then $\Sat{M}{!B}$ and $\Sat/{M}{!C}$. This
  means that $\Struct{M}$ satisfies both $\sFmla{\True}{!B}$ and
  $\sFmla{\False}{!C}$.
\item The branch is expanded by applying $\TRule{\True}{\lforall}$ to
  $\sFmla{\True}{\lforall[x][!B(x)]} \in \Gamma$: This results in a
  new !!{signed formula}~$\sFmla{\True}{!A(t)}$ on the branch.
  Suppose $\Sat{M}{\Gamma}$, in particular,
  $\Sat{M}{\lforall[x][!A(x)]}$.  By
  \olref[syn][sem]{prop:quant-terms}, $\Sat{M}{!A(t)}$.  Consequently,
  $\Struct{M}$ satisfies $\sFmla{\True}{!A(t)}$.
\item The branch is expanded by applying $\TRule{\False}{\lforall}$ to
  $\sFmla{\False}{\lforall[x][!B(x)]} \in \Gamma$: This results in a
  new !!{signed formula}~$\sFmla{\False}{!A(a)}$ where $a$ is
  !!a{constant} not occurring in~$\Gamma$. Since $\Gamma$ is
  satisfiable, there is a $\Struct{M}$ such that $\Sat{M}{\Gamma}$, in
  particular $\Sat/{M}{\lforall[x][!B(x)]}$. We have to show that
  $\Gamma \cup \{\sFmla{\False}{!A(a)}\}$ is satisfiable. To do this,
  we define a suitable~$\Struct{M'}$ as follows.

  By \olref[syn][ass]{prop:sat-quant}, $\Sat/{M}{\lforall[x][!B(x)]}$
  iff for some $s$, $\Sat/{M}{!B(x)}[s]$. Now let $\Struct{M'}$ be
  just like $\Struct{M}$, except $\Assign{a}{M'} = s(x)$.  By
  \olref[syn][ext]{cor:extensionality-sent}, for any
  $\sFmla{\True}{!C} \in \Gamma$, $\Sat{M'}{!C}$, and for any
  $\sFmla{\False}{!C} \in \Gamma$, $\Sat/{M'}{!C}$, since $a$ does not
  occur in~$\Gamma$.

  By \olref[syn][ext]{prop:extensionality}, $\Sat/{M'}{!A(x)}[s]$.  By
  \olref[syn][ext]{prop:ext-formulas}, $\Sat/{M'}{!A(a)}[s]$.  Since
  $!A(a)$ is !!a{sentence}, by
  \olref[syn][ass]{prop:sentence-sat-true}, $\Sat/{M'}{!A(a)}$, i.e.,
  $\Struct{M'}$ satisfies $\sFmla{\False}{!A(a)}$.
\item The branch is expanded by applying $\TRule{\True}{\lexists}$ to
  $\sFmla{\True}{\lexists[x][!B(x)]} \in \Gamma$: Exercise.
\item The branch is expanded by applying $\TRule{\False}{\lexists}$ to
  $\sFmla{\False}{\lexists[x][!B(x)]} \in \Gamma$:
\end{enumerate}
Now let's consider the possible inferences with two premises.
\begin{enumerate}
\item The branch is expanded by applying $\TRule{\False}{\land}$ to
  $\sFmla{\False}{!B \land !C} \in \Gamma$, which results in two
  branches, a left one continuing through $\sFmla{\False}{!B}$ and a
  right one through $\sFmla{\False}{!C}$. Suppose $\Sat{M}{\Gamma}$,
  in particular $\Sat/{M}{!B \land !C}$.  Then $\Sat/{M}{!B}$ or
  $\Sat/{M}{!C}$. In the former case, $\Struct{M}$ satisfies
  $\sFmla{\False}{!B}$, i.e., $\Struct{M}$ satisfies the formulas on
  the left branch. In the latter,  $\Struct{M}$ satisfies
  $\sFmla{\False}{!C}$, i.e., $\Struct{M}$ satisfies the formulas on
  the right branch.
\item The branch is expanded by applying $\TRule{\True}{\lor}$ to
  $\sFmla{\True}{!B \lor !C} \in \Gamma$: Exercise.
\item The branch is expanded by applying $\TRule{\True}{\lif}$ to
  $\sFmla{\True}{!B \lif !C} \in \Gamma$: Exercise.
  \item The branch is expanded by \Cut: This results in two branches,
    one containing $\sFmla{\True}{!B}$, the other containing
    $\sFmla{\False}{!B}$. Since $\Sat{M}{\Gamma}$ and either
    $\Sat{M}{!B}$ or $\Sat/{M}{!B}$, $Struct{M}$ satisfies either the
    left or the right branch.
\end{enumerate}
\end{proof}

\begin{prob}
Complete the proof of \olref[fol][tab][sou]{thm:tableau-soundness}.
\end{prob}

\begin{cor}
\ollabel{cor:weak-soundness}
If $\Proves !A$ then $!A$ is valid.
\end{cor}

\begin{cor}
\ollabel{cor:entailment-soundness}
If $\Gamma \Proves !A$ then $\Gamma \Entails !A$.
\end{cor}

\begin{proof}
  If $\Gamma \Proves !A$ then for some $!B_1$, \dots, $!B_n \in
  \Gamma$, $\{\sFmla{\False}{!A}, \sFmla{\True}{!B_1}, \dots,
  \sFmla{\True}{!B_n}\}$ has a closed !!{tableau}. By
  \olref{thm:tableau-soundness}, every !!{structure} $\Struct M$
  either makes some $!B_i$ false or makes $!A$ true.  Hence, if
  $\Sat{M}{\Gamma}$ then also $\Sat{M}{!A}$.
\end{proof}

\begin{cor}
\ollabel{cor:consistency-soundness}
If $\Gamma$ is satisfiable, then it is consistent.
\end{cor}

\begin{proof}
We prove the contrapositive.  Suppose that $\Gamma$ is not consistent.
Then there are $!B_1$, \dots, $!B_n \in \Gamma$ and a closed
!!{tableau} for $\{\sFmla{\True}{!B}, \dots, \sFmla{\True}{!B}\}$.  By
\olref{thm:tableau-soundness}, there is no $\Struct{M}$ such that
$\Sat{M}{!B_i}$ for all $i=1$, \dots,~$n$.  But then $\Gamma$ is not
satisfiable.
\end{proof}

\end{document}
