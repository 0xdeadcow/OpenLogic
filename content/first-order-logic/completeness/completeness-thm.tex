% Part: first-order-logic
% Chapter: completeness
% Section: completeness-theorem

\documentclass[../../../include/open-logic-section]{subfiles}

\begin{document}

\olfileid{fol}{com}{cth}
\olsection{The Completeness Theorem}

\begin{thm}[Completeness Theorem]
\ollabel{thm:completeness}
Let $\Gamma$ be a set of !!{sentence}s.  If
$\Gamma$ is consistent, it is satisfiable.
\end{thm}

\begin{proof}
Suppose $\Gamma$ is consistent.  By \olref[lin]{lem:lindenbaum}, there
is a $\Gamma^* \supseteq \Gamma$ which is maximally consistent and
saturated.  If $\Gamma$ does not contain~$\eq$, then by
\olref[mod]{lem:truth}, $\Sat{M(\Gamma^*)}{!A}$ iff $!A \in \Gamma^*$.
From this it follows in particular that for all $!A \in \Gamma$,
$\Sat{M(\Gamma^*)}{!A}$, so $\Gamma$ is satisfiable.  If $\Gamma$ does
contain~$\eq$, then by \olref[ide]{lem:truth}, $\Sat{M/_\approx}{!A}$
iff $!A \in \Gamma^*$ for all sentences~$!A$.  In particular,
$\Sat{M/_\approx}{!A}$ for all $!A \in \Gamma$, so $\Gamma$ is
satisfiable.
\end{proof}

\begin{cor}[Completeness Theorem, Second Version]
\ollabel{cor:completeness}
For all $\Gamma$ and $!A$ sentences: if $\Gamma \Entails !A$ then
$\Gamma \Proves !A$.
\end{cor}

\begin{proof}
Note that the $\Gamma$'s in \olref{cor:completeness} and
\olref{thm:completeness} are universally quantified.  To make sure we
do not confuse ourselves, let us restate \olref{thm:completeness}
using a different variable: for any set of sentences~$\Delta$, if
$\Delta$ is consistent, it is satisfiable.  By contraposition, if
$\Delta$ is not satisfiable, then $\Delta$ is inconsistent.  We will
use this to prove the corollary.

Suppose that $\Gamma \Entails !A$.  Then $\Gamma \cup \{\lnot !A\}$ is
unsatisfiable by \olref[syn][sem]{prop:entails-unsat}.  Taking $\Gamma
\cup \{\lnot !A\}$ as our $\Delta$, the previous version of
\olref{thm:completeness} gives us that $\Gamma \cup \{\lnot !A\}$ is
inconsistent.  By
\tagrefs{prfSC/{fol:seq:ptn:prop:prov-incons},prfND/{fol:ntd:ptn:prop:prov-incons}},
$\Gamma \Proves !A$.
\end{proof}

\begin{prob}
Use \olref[fol][com][cth]{cor:completeness} to prove
\olref[fol][com][cth]{thm:completeness}, thus showing that the two
formulations of the completeness theorem are equivalent.
\end{prob}

\begin{prob}
In order for a !!{derivation} system to be complete, its rules must be
strong enough to prove every unsatisfiable set inconsistent.  Which of
the rules of $\Log{LK}$ were necessary to prove completeness?  Are any
of these rules not used anywhere in the proof?  In order to answer
these questions, make a list or diagram that shows which of the rules
of $\Log{LK}$ were used in which results that lead up to the proof of
\olref[fol][com][cth]{thm:completeness}.  Be sure to note any tacit
uses of rules in these proofs.
\end{prob}

\end{document}
