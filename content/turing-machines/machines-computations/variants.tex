% Part: turing-machines
% Chapter: machines-computations
% Section: variants

\documentclass[../../../include/open-logic-section]{subfiles}

\begin{document}

\olfileid{tur}{mac}{var}
\olsection{Variants of Turing Machines}

There are in fact many possible ways to define Turing machines, of
which ours is only one.  In some ways, our definition is more liberal
than others. We allow arbitrary finite alphabets, a more restricted
definition might allow only two tape symbols, $\TMstroke$ and
$\TMblank$.  We allow the machine to write a symbol to the tape and
move at the same time, other definitions allow either writing or
moving.  We allow the possibility of writing without moving the tape
head, other definitions leave out the $\TMstay$ ``instruction.''  In
other ways, our definition is more restrictive. We assumed that the
tape is infinite in one direction only, other definitions allow the
tape to be infinite both to the left and the right. In fact, one can
even allow any number of separate tapes, or even an infinite grid
of squares.  We represent the instruction set of the Turing machine by
a transition function; other definitions use a transition relation
where the machine has more than one possible instruction in any given
situation.

This last relaxation of the definition is particularly interesting.
In our definition, when the machine is in state~$q$ reading
symbol~$\sigma$, $\delta(q, \sigma)$ determines what the new symbol,
state, and tape head position is.  But if we allow the instruction set
to be a relation between current state-symbol pairs $\tuple{q,
  \sigma}$ and new state-symbol-direction triples $\tuple{q', \sigma',
  D}$, the action of the Turing machine may not be uniquely
determined---the instruction relation may contain both $\tuple{q,
  \sigma, q', \sigma', D}$ and $\tuple{q, \sigma, q'', \sigma'', D'}$.
In this case we have a \emph{non-deterministic} Turing machine.  These
play an important role in computational complexity theory.

There are also different conventions for when a Turing machine halts:
we say it halts when the transition function is undefined, other
definitions require the machine to be in a special designated halting
state. We have explained in \olref[hal]{sec} why requiring a designated
halting state is not a restriction which impacts what Turing machines can
compute.  Since the tapes of our Turing machines are infinite in one
direction only, there are cases where a Turing machine can't properly
carry out an instruction: if it reads the leftmost square and is
supposed to move left. According to our definition, it just stays put
instead of ``falling off'', but we could have defined it so that it
halts when that happens. This definition is also equivalent: we could
simulate the behavior of a Turing machine that halts when it attempts
to move left from square~$0$ by deleting every transition
$\delta(q,\TMendtape) = \tuple{q',\sigma,\TMleft}$---then instead of
attempting to move left on~$\TMendtape$ the machine
halts.\footnote{This doesn't \emph{quite} work, since nothing prevents
us from writing and reading $\TMendtape$ on squares other than
square~$1$ (see \olref[una]{ex:mover}). We can get around that by
adding a second $\TMendtape'$ symbol we use instead for such a
purpose.}

There are also different ways of representing numbers (and hence the
input-output function computed by a Turing machine): we use unary
representation, but you can also use binary representation. This
requires two symbols in addition to $\TMblank$ and $\TMendtape$.

What is perhaps harder to see is that we gain noadditional computing
power by allowing a tape that is infinite in both directions, or even
multiple tapes. The reason is, roughly, that a Turing machine with a
single infinite tape can simulate multiple or two-way infinite tapes.
E.g., we could use the even squares for the squares of tape~$1$ (or
the ``positive'' squares of a two-way infinite tape) and the off
squares for the squares of tape~$2$ (or the ``negative'' squares).

Now here is an interesting fact: none of these variations matters as
to which functions are Turing computable. \emph{If a function is Turing
computable according to one definition, it is Turing computable
according to all of them.}  

\end{document}
