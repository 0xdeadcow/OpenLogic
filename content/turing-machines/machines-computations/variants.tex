% Part: turing-machines
% Chapter: machines-computations
% Section: variants

\documentclass[../../../include/open-logic-section]{subfiles}

\begin{document}

\olfileid{tur}{mac}{var}
\olsection{Variants of Turing Machines}

There are in fact many possible ways to define Turing machines, of
which ours is only one.  We allow arbitrary finite alphabets, a more
restricted definition might allow only two tape symbols, $\TMstroke$
and $\TMblank$.  We allow the machine to write a symbol to the tape
and move at the same time, other definitions allow either writing or
moving.  We allow the possibility of writing without moving the tape
head, other definitions leave out the $\TMstay$ ``instruction.''  Our
definition assumes that the tape is infinite in one direction only,
other definitions allow the tape to be infinite both to the left and
the right. In fact, we might even allow any number of separate tapes,
or even an infinite grid of squares.  We represent the instruction set
of the Turing machine by a transition function; other definitions use
a transition relation.

This last relaxation of the definition is particularly interesting.
In our definition, when the machine is in state~$q$ reading
symbol~$\sigma$, $\delta(q, \sigma)$ determines what the new symbol,
state, and tape head position is.  But if we allow the instruction set
to be a relation between current state-symbol pairs $\tuple{q,
  \sigma}$ and new state-symbol-direction triples $\tuple{q', \sigma',
  D}$, the action of the Turing machine may not be uniquely
determined---the instruction relation may contain both $\tuple{q,
  \sigma, q', \sigma', D}$ and $\tuple{q, \sigma, q'', \sigma'', D'}$.
In this case we have a \emph{non-deterministic} Turing machine.  These
play an important role in computational complexity theory.

There are also different conventions for when a Turing machine halts:
we say it halts when the transition function is undefined, other
definitions require the machine to be in a special designated halting
state.  And there are differnt ways of representing numbers: we use
unary representation, but you can also use binary representation (this
requires two symbols in addition to $\TMblank$).

Now here is an interesting fact: none of these variations matters as
to which functions are Turing computable. If a function is Turing
computable according to one definition, it is Turing computable
according to all of them.

\end{document}
