% Part: turing-machines 
% Chapter: machines-computations 
% Section: history

\documentclass[../../../include/open-logic-section]{subfiles}

\begin{document}

\olfileid{tur}{mac}{his} 
\olsection{History}

%% This section was taken out of the introduction, to be reintegrated when
%% we have a history tag/env.

Alan Turing invented Turing machines in 1936. While his interest at the
time was the decidability of first-order logic, the paper has been
described as a definitive paper on the foundations of computer design
(Copeland 1993, 10). Notice that this was a full five years before the
first working general purpose computer was built in 1941 (by the German
Konrad Zuse in his parents living room), seven years before Turing and his
colleagues at Bletchley Park built the code-breaking Colossus (1943), nine
years before the American ENIAC (1945), twelve years before the first
British general purpose computer the Manchester Mark I was built in
Manchester (1948) and thirteen years before the Americans first tested the
BINAC (1949). The Manchester Mark I has the distinction of being the first
stored-program computer - previous machines had to be rewired by hand for
each new task.

\end{document}