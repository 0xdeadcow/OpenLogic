% Part: turing-machines
% Chapter: machines-computations
% Section: halting-states

\documentclass[../../../include/open-logic-section]{subfiles}

\begin{document}

\olfileid{tur}{tur}{tur}
\olsection{Halting States}

\begin{explain}
Although we have defined our machines to halt only when there
is no instruction to carry out, common representations of Turing
machines have a dedicated \emph{halting state}, $h$, such that
$h \in Q$.

The idea behind a halting state is simple: when the machine has
finished operation (it is ready to accept input, or has finished writing
the output), it goes into a state $h$ where it halts. Some
machines have two halting states, one that accepts input and one
that rejects input. 
\end{explain}

\begin{ex}\emph{Halting States}.
To elucidate this concept, let us begin with an alteration of the
even machine. Instead of having the machine halt in state $q_0$
if the input is even, we can add an instruction to send the machine
into a halt state.
\[
\begin{tikzpicture}[->,>=stealth',shorten >=1pt,auto,node distance=2.8cm,
                    semithick]
  \tikzstyle{every state}=[fill=none,draw=black,text=black]

  \node[initial, state]         (A)                     {$q_0$};
  \node[state]         (B) [right of=A] {$q_1$};
  \node[state]         (C) [below of=A] {$h$};

  \path (A) edge [bend left] node {\TMtrans{\TMstroke}{\TMstroke}{R}} (B)
  	    edge node {\TMtrans{\TMblank}{\TMblank}{N}} (C)
        (B) edge [loop above] node {\TMtrans{\TMblank}{\TMblank}{R}} (B)
            edge [bend left] node {\TMtrans{\TMstroke}{\TMstroke}{R}} (A);
\end{tikzpicture}
\]

Let us further expand the example. When the machine determines that the input
is odd, it never halts. We can alter the machine to
include a \emph{reject} state by replacing the looping instruction with an
instruction to go to a reject state~$r$.
\[
\begin{tikzpicture}[->,>=stealth',shorten >=1pt,auto,node distance=2.8cm,
                    semithick]
  \tikzstyle{every state}=[fill=none,draw=black,text=black]

  \node[initial,state]         (A)                     {$q_0$};
  \node[state]         (B) [right of=A] {$q_1$};
  \node[state]         (C) [below of=A] {$h$};
  \node[state]         (D) [below of=B] {$r$};

  \path (A) edge [bend left] node {\TMtrans{\TMstroke}{\TMstroke}{R}} (B)
  	    edge node {\TMtrans{\TMblank}{\TMblank}{N}} (C)
        (B) edge node {\TMtrans{\TMblank}{\TMblank}{N}} (D)
            edge [bend left] node {\TMtrans{\TMstroke}{\TMstroke}{R}} (A);
\end{tikzpicture}
\]
\end{ex}

\begin{explain}
Adding a dedicated halting state can be advantageous in cases like
this, where it makes explicit when the machine accepts/rejects certain
inputs.  However, it is important to note that no computing power is
gained by adding a dedicated halting state. Similarly, a less
formal notion of halting has its own advantages. The definition of
halting used so far in this chapter makes the proof of the
\emph{Halting Problem} intuitive and easy to demonstrate. For this
reason, we continue with our original definition.
\end{explain}

\end{document}
