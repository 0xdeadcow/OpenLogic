% Part: computability
% Chapter: tm-computations
% Section: church-turing-thesis

\documentclass[../../../include/open-logic-section]{subfiles}

\begin{document}

\olfileid{cmp}{tur}{ctt}
\olsection{Church-Turing Thesis}

Turing machines are supposed to be a precise replacement for the concept
of an effective procedure. Turing took it that anyone who grasped the
concept
of an effective procedure and the concept of a Turing machine would have
the intuition that anything that could be done via an effective procedure
could be done by Turing machine. This claim is given support by the fact
that all the other proposed precise replacements for the concept of an
effective procedure turn out to be extensionally equivalent to the concept
of a Turing machine---that is, they can perform exactly the same set of
tasks. This claim is called the Church-Turing thesis, since Church
formulated the same claim with respect to the lambda calculus in the same
year.

\begin{defn}
\emph{The Church-Turing Thesis}: Anything computable via an effective
procedure is Turing-computable.
\end{defn}

Given the thesis, it follows from the fact that~$\fn{h}$ is not
Turing-computable that there is no effective procedure for
computing~$\fn{h}$, and thus that the halting problem is unsolvable.


\end{document}
