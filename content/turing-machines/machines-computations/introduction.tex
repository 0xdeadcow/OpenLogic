% Part: turing-machines 
% Chapter: machines-computations 
% Section: introduction

\documentclass[../../../include/open-logic-section]{subfiles}

\begin{document}

\olfileid{tur}{mac}{int} 
\olsection{Introduction}

\begin{explain}
Even though the term ``Turing machine'' evokes the image of
a physical machine with moving parts, strictly speaking a Turing machine is
a purely mathematical construct. It is perhaps best to think of a Turing
machine as a program for a special kind of imaginary mechanism. This
mechanism consists of a \emph{tape} and a \emph{read-write head}. In our
version of Turing machines, the tape is infinite in one direction (to the
right), and it is divided into \emph{squares}, each of which may contain a
symbol from a finite \emph{alphabet}. Such alphabets can contain any number
of different symbols, but we will mainly make do with three: $\TMendtape$,
$\TMblank$, and $\TMstroke$. When the mechanism is started, the tape is
empty (i.e., each square contains the symbol $\TMblank$) except for the
leftmost square, which contains $\TMendtape$, and a finite number of
squares which contain the \emph{input}. At any time, the mechanism is in
one of a finite number of \emph{states}. At the outset, the head scans the
leftmost square and in a specified \emph{initial state}. At each step of
the mechanism's run, the content of the square currently scanned together
with the state the mechanism is in and the Turing machine program determine
what happens next. The Turing machine program consists of a list of
5-tuples $\tuple{q_i, \sigma, q_j, \sigma', D}$. Whenever the
mechanism is in state $q_i$ and reads symbol $\sigma$, it replaces the
symbol on the current square with $\sigma'$, the head moves left, right, or
stays put according to whether $D$ is $\TMleft$, $\TMright$, or $\TMstay$,
and the mechanism goes into state~$q_j$.

For instance, consider the situation below:

\begin{center}
  \usetikzlibrary{calc, chains, shapes, decorations.pathmorphing}
  \begin{tikzpicture}
  \tikzstyle{tmhead}=[arrow box,draw,minimum size=3ex,arrow box
arrows={east:.25cm, west:.25cm}]

  \tikzset{tape/.style={minimum size=3ex, draw}}
  \begin{scope}[start chain=0 going right, node distance=0pt]
    \foreach \x [count=\i] in
             {$\TMendtape$,$\TMstroke$,$\TMstroke$,$\TMstroke$,$\TMblank$,$\TMstroke$,$\TMstroke$,$\TMstroke$,$\TMstroke$,$\TMblank$,$\TMblank$,$\TMblank$} {
    \ifnum\i=12 % if last node reset outer sep to 0pt
      \node [on chain=0, tape, outer sep=0pt] (n\i) {\x};
      \draw (n\i.north east) -- ++(.2,0) decorate [decoration={zigzag, segment length=.5ex, amplitude=.3ex}] {-- ($(n\i.south east)+(+.2,0)$)} -- (n\i.south east) -- cycle;
     \else
      \node [on chain=0, tape] (n\i) {\x};
     \fi
     \ifnum\i=1 % if first node draw a thick line at the left
      \draw [line width=2pt] (n\i.north west) -- (n\i.south west);
     \fi
             }
             \node [tmhead,yshift=-.3cm] at (n3.south) (head) {$q_1$};

%   \node [right=.25cm of n12] {$\cdots$};
%   \node [tape, above left=.25cm and 1cm of n1] (q3) {$q_3$};
%   \draw [>=latex, ->] (q3) -| (n5);
  \end{scope}
 \end{tikzpicture}
\end{center}

The tape of the Turing machine contains the end-of-tape symbol
$\TMendtape$ on the leftmost square, followed by three $\TMstroke$'s,
a $\TMblank$, four more $\TMstroke$'s, and the rest of the tape is
filled with $\TMblank$'s.  The head is reading the third square from
the left, which contains a $\TMstroke$, and is in state~$q_1$---we say
``the machine is reading a $\TMstroke$ in state~$q_1$.''  If the
program of the turing machine contains, say, the tuple $\tuple{q_1,
\TMstroke, q_5, \TMblank, \TMright}$, we would now replace the
$\TMstroke$ on the third square with a~$\TMblank$, move right to the
fourth square, and change the state of the machine to~$q_5$.

We say that the machine \emph{halts} when it encounters some state,
$q_n$, and symbol, $\sigma$ such that there is no instruction
$\tuple{q_n, \sigma, \quad, \quad, \quad}$. In other words, the machine
has no instruction to carry out, and at that point, it ceases
operation. Halting is sometimes represented by a specific halt
state~$h$.  This will be demonstrated in more detail later on.
\end{explain}


\end{document}
