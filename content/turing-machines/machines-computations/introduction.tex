% Part: first-order-logic
% Chapter: completeness
% Section: introduction

\documentclass[../../../include/open-logic-section]{subfiles}

\begin{document}

\olfileid{cmp}{tur}{int}
\olsection{Introduction}

\begin{explain}
Even though the term ``Turing machine'' evokes the image of a physical
machine with moving parts, strictly speaking a Turing machine is a
purely mathematical construct.  It is perhaps best to think of a
Turing machine as a program for a special kind of imaginary mechanism.
This mechanism consists of a \emph{tape} and a \emph{read-write head}.
In our version of Turing machines, the tape is infinite in one
direction (to the right), and it is divided into \emph{squares}, each
of which may contain a symbol from a finite \emph{alphabet}.  Such
alphabets can contain any number of different symbols, but we will
mainly make do with three: $\TMendtape$, $\TMblank$, and
$\mid$.  When he mechanism is started, the tape is empty (i.e., each
square contains the symbol $\TMblank$) except for the
leftmost square, which contains $\TMendtape$, and a finite number
of squares which contain the \emph{input}.  At any time, the mechanism
is in one of a finite number of \emph{states}.  At the outset, the
head scans the leftmost square and in a specified \emph{initial
  state}.  At each step of the mechanism's run, the content of the
square currently scanned together with the state the mechanism is in
and the Turing machine program determine what happens next.  The Turing
machine program consists of a list of 5-tuples $\langle q_i, \sigma,
q_j, \sigma', D\rangle$.  Whenever the mechanism is in state $q_i$ and
reads symbol $\sigma$, it replaces the symbol on the current square
with $\sigma'$, the head moves left, right, or stays put according to
whether $D$ is $L$, $R$, or $N$, and the mechanism goes into
state~$q_j$.  When the mechanism enters state~$h$ we say it
\emph{halts}, and the contents of the tape at that point is its
\emph{output}.
\end{explain}


\end{document}
