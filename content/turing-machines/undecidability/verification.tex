% Part: turing-machines 
% Chapter: undecidability 
% Section: verification

\documentclass[../../../include/open-logic-section]{subfiles}

\begin{document}

\olfileid{tms}{und}{ver} 
\olsection{Verifying the Representation}

\begin{explain}
In order to verify that our representation works, we first have to
make sure that if $M$ halts on input~$w$, then $!T(M, w) \lif !E(M,
w)$ is valid. We can do this simply by proving that $!T(M, w)$ implies
a description of the configuration of $M$ for each step of the
execution of~$M$ on input~$w$. If $M$ halts on input~$w$, then for
some~$n$, $M$ will be in a halting configuration at step~$n$ (and be
scanning square~$m$, for some $m$). Hence, $!T(M, w)$ implies $\Obj
Q_q(\num{m}, \num{n}) \land \Obj S_\sigma(\num{m}, \num{n})$ for
some~$q$ and $\sigma$ such that $\delta(q, \sigma)$ is undefined.
\end{explain}

\begin{defn} 
Let $!C(M, w, n)$ be the !!{sentence}
\[ 
\Obj Q_q(\num{m}, \num{n}) \land \Obj S_{\sigma_0}(\num{0}, \num{n})
\land \dots \land \Obj S_{\sigma_k}(\num{k}, \num{n}) \land
\lforall[x][(\num{k} < x \lif \Obj S_\TMblank(x, \num{n}))]
\] 
where $q$ is the state
of $M$ at time~$n$, $M$ is scanning square~$m$ at time~$n$, square~$i$
contains symbol~$\sigma_i$ at time~$n$ for $0 \le i \le k$ and $k$ is the
right-most non-blank square of the tape at time~$m$. 
\end{defn}

Suppose that $M$ does halt for input~$w$. Then there is some
time~$n$, state~$q$, square~$m$, and symbol~$\sigma$ such that:
\begin{enumerate} 
\item At time $n$ the machine is in state~$q$ scanning square~$m$ on
  which~$\sigma$ appears.
\item There transition function $\delta(q, \sigma)$ is undefined.
\end{enumerate}

$!C(M, w, n)$ will be the description of this time and will include
the clauses $\Obj Q_{q}(\num{m}, \num{n})$ and $\Obj
S_{\sigma}(\num{m}, \num{n})$. These clauses together imply $!E(M,
w)$:
\[
\lexists[x][\lexists[y][(\bigvee_{\tuple{q, \sigma} \in
      X}(\Obj Q_q(x, y) \land \Obj S_\sigma(x, y)))]]
\]
since $\Obj Q_{q'}(\num{m}, \num{n}) \land S_{\sigma'}(\num{m},
\num{n}) \Entails \bigvee_{\tuple{q, \sigma} \in X} (\Obj Q_q(\num{m},
\num{n}) \land \Obj S_{\sigma}(\num{m}, \num{n})$, as
$\tuple{q',\sigma'} \in X$.

So if $M$ halts for input $w$, then there is some time $n$ such that
$!C(M, w, n) \Entails !E(M,w)$

Since consequence is transitive, it is sufficient to show that for any
time~$n$, $!T(M, w) \Entails !C(M, w, n)$.

\begin{lem}
\ollabel{lem:config}
For each $n$, $!T(M, w) \Entails !C(M, w, n)$. 
\end{lem}

\begin{proof}
If $n = 0$, then the conjuncts of $!C(M, w, 0)$ are also conjuncts of
$!T(M, w)$, so entailed by it.

Inductive hypothesis: If $M$ has not halted before time $n$, then
$!T(M,w) \Entails !C(M, w, n)$.

Suppose $n > 0$ and at time~$n$, $M$ started on $w$ is in state~$q$
scanning square~$m$, and the content of the tape is $\sigma_0$, \dots,
$\sigma_k$.

Suppose that $M$ has not halted before time $n+1$. If $!T(M,n)$ is
true in !!a{structure}~$\Struct M$, the inductive hypothesis tells us
that $!C(M, w, n)$ is true in $\Struct M$ also. In particular, $\Obj
Q_q(\num{m}, \num{n})$ and $\Obj S_\sigma(\num{m}, \num{n})$ are true
in $\Struct M$.

Since $M$ does not halt at~$n$, there must be an instruction of one of the
following three forms in the program of~$M$: 

\begin{enumerate} 
\item \ollabel{right} $\delta(q, \sigma) = \tuple{q', \sigma', \TMright}$

\item \ollabel{left} $\delta(q, \sigma) = \tuple{q', \sigma', \TMleft}$

\item \ollabel{stay} $\delta(q, \sigma) = \tuple{q', \sigma', \TMstay}$
\end{enumerate}

We will consider each of these three cases in turn.  First, assume
that $m \le k$.

\begin{enumerate} 
\item Suppose there is an instruction of the form~\olref{right}.
  By \olref[rep]{defn:tm-descr}, \olref[rep]{rep-right}, this means that
\[ 
\lforall[x][\lforall[y][((\Obj Q_{q}(x,
y) \land \Obj S_\sigma(x, y)) \lif (\Obj Q_{q'}(x',
y') \land \Obj S_{\sigma'}(x, y') \land !A(x, y)))]]
\] 
is a conjunct of $!T(M,w)$. This entails the following
!!{sentence}, through universal instantiation: 
\[ 
(\Obj Q_{q}(\num{m}, \num{n}) \land \Obj S_{\sigma}(\num{m}, \num{n}))
\lif (\Obj Q_{q'}(\num{m}', \num{n}') \land \Obj S_{\sigma'}(\num{m},
\num{n}') \land !A(m, n)).
\]
This in turn entails,
\begin{multline*}
\Obj Q_{q'}(\num{m}', \num{n}') \land \Obj S_{\sigma'}(\num{m},
\num{n}') \land {}\\
\Obj S_{\sigma_0}(\num{0}, \num{n}') \land \dots \land
\Obj S_{\sigma_k}(\num{k}, \num{n}') \land {}\\
\lforall[x][(\num{k} < x
  \lif \Obj S_\TMblank(x, \num{n}'))]
\end{multline*}
The first line comes directly from the consequent of the preceding
conditional. The each conjunct in the middle line---which excludes
$S_{\sigma_m}(\num{m},\num{n'})$---follows from the corresponding
conjunct in~$!C(M, w, n)$ together with $!A(m, n)$. The last line
follows from the corresponding conjunct in $C!(M, w, n)$, $\num{m} < x
\lif \num{k} < x$, and $!A(m, n)$.  Together, this just is $!C(M, w, n')$.

\item Suppose there is an instruction of the form~\olref{left}.
  Then, by \olref[rep]{defn:tm-descr}, \olref[rep]{rep-left},
\[
\lforall[x][\lforall[y][((\Obj Q_{q}(x', y) \land \Obj
S_{\sigma}(x', y)) \lif (\Obj Q_{q'}(x, y') \land \Obj
S_{\sigma'}(x', y') \land !A(x, y)))]] 
\] 
is a conjunct of $!T(M,w)$, which entails the following !!{sentence}: 
\[ 
(\Obj Q_{q}(\num{m}', \num{n}) \land \Obj S_{\sigma}(\num{m}', \num{n})
\lif (\Obj Q_{q'}(\num{m}, \num{n}') \land \Obj S_{\sigma'}(\num{m}',
\num{n}') \land !A(x, y)),
\]
which in turn implies
\begin{multline*}
  \Obj Q_{q'}(\num{m}, \num{n}') \land \Obj S_{\sigma'_m}(\num{m},
  \num{n}') \land {}\\
  \Obj S_{\sigma_0}(\num{0}, \num{n}')\dots \land
  \Obj S_{\sigma_k}(\num{k}, \num{n}')  \land {} \\
  \lforall[x][(\num{k} < x
  \lif \Obj S_\TMblank(x, \num{n}'))]
\end{multline*}
as before. But this just is $!C(M, w, n')$.

\item Case \olref{stay} is left as an exercise.
\end{enumerate}
If $m > k$ and $\sigma' \neq \TMblank$, the last instruction has
written a non-blank symbol to the right of the right-most non-blank
square~$k$ at time~$n$. In this case, $!C(M, w, n')$ has the form
\begin{multline*}
\Obj Q_{q'}(\num{m}', \num{n}') \land {}\\
\Obj S_{\sigma_0}(\num{0}, \num{n}') \land \dots \land
\Obj S_{\sigma_k}(\num{k}, \num{n}') \land {}\\
\Obj S_{\TMblank}(\num{k+1}, \num{n}') \land \dots \land
\Obj S_{\TMblank}(\num{m-1}, \num{n}') \land {}\\
\Obj S_{\sigma'}(\num{m},
\num{n}') \land {}\\
\lforall[x][(\num{m} < x
  \lif \Obj S_\TMblank(x, \num{n}'))]
\end{multline*}
For $k < i < m$, $\Obj S_{\TMblank}(\num{i}, \num{n})$ follows from
the conjunct $\lforall[x][(\num{k} < x \lif \Obj S_\TMblank(x,
  \num{n}))]$ of $!C(M, w, n)$ and the fact that $!T(M, w) \Entails
\num{k} < \num{i}$ if $k < i$. $\Obj S_{\TMblank}(\num{i}, \num{n}')$
then follows from $A(m, n)$ and $\num{i} < \num{m}$.  From
$\lforall[x][(\num{k} < x \lif \Obj S_\TMblank(x, \num{n}))]$ we get
$\lforall[x][(\num{m} < x \lif \Obj S_\TMblank(x, \num{n}))]$ since
$\num{k}<\num{m}$ and $<$ is transitive.  From that plus $!A(m, n)$ we
get $\lforall[x][(\num{m} < x \lif \Obj S_\TMblank(x, \num{n}'))].$
Similarly for cases (2) and (3).

We have shown that for any~$n$, $!T(M, w) \Entails !C(M, w, n)$.
\end{proof}

\begin{lem}
\ollabel{lem:valid-if-halt}
If $M$ halts on input~$w$, then $!T(M, w) \lif
!E(M, w)$ is valid.
\end{lem}

\begin{proof}
By \olref{lem:config}, we know that, for any time~$n$, the
description~$!C(M, w, n)$ of the configuration of $M$ at time~$n$ is a
consequence of $!T(M, w)$.  Suppose $M$ halts after $k$ steps. It will
be scanning square~$m$, say. Then $!C(M, w, k)$ contains as conjuncts
both $\Obj Q_q(\num{m}, \num{k})$ and $\Obj S_\sigma(\num{m},
\num{k})$ with $\delta(q,\sigma)$ undefined. Thus, $!C(M, w, k)
\Entails !E(M, w)$. But then $!T(M, w) \Entails !E(M, w)$ and
therefore $!T(M, w) \lif !E(M, w)$ is valid.
\end{proof}

\begin{prob}
Complete case~\olref[tms][und][ver]{stay} of the proof of
\olref[tms][und][ver]{lem:config}.
\end{prob}

\begin{explain} 
To complete the verification of our claim, we also have to
establish the reverse direction: if $!T(M, w) \lif !E(M, w)$ is valid, then
$M$ does in fact halt when started on input~$m$. 
\end{explain}

\begin{lem}
\ollabel{lem:halt-if-valid}
If $\Entails !T(M, w) \lif !E(M, w)$, then $M$ halts on input~$w$.
\end{lem}

\begin{proof} 
Consider the $\Lang L_M$-!!{structure} $\Struct M$ with
domain $\Nat$ which interprets $\Obj 0$ as $0$, $'$ as the successor
function, and $<$ as the less-than relation, and the predicates $\Obj Q_q$
and $\Obj S_\sigma$ as follows:
\begin{align*}
  \Assign{\Obj Q_q}{M} & =
\Setabs{\tuple{m, n}}{\text{started on~$w$, at step~$n$ $M$ is in state $q$
  scanning square~$m$}} \\
\Assign{\Obj S_\sigma}{M} & = \{(m, n) :
\text{started on~$w$ after $n$ steps, square~$m$ of $M$ contains
  symbol~$\sigma$}\}
\end{align*}
In other words, we construct the !!{structure}~$\Struct{M}$ so that it
describes what $M$ started on input~$w$ actually does, step by step.
Clearly, $\Sat{M}{!T(M, w)}$. If $\Entails !T(M, w) \lif !E(M, w)$,
then also $\Sat{M}{!E(M, w)}$, i.e.,
\[
\Sat{M}{\lexists[x][\lexists[y][(\bigvee_{\tuple{q, \sigma} \in
      X}(\Obj Q_q(x, y) \land \Obj S_\sigma(x, y)))]]}.
\]
As $\Domain{M} = \Nat$, there must be $m$, $n \in \Nat$ so that
$\Sat{M}{\Obj Q_q(\num{m}, \num{n}) \land \Obj S_\sigma(\num{m},
  \num{n})}$ for some $q$ and $\sigma$ such that $\delta(q, \sigma)$
is undefined. By the definition of $\Struct M$, this means that $M$
started on input~$w$ after~$n$ steps is in state~$q$ and reading
symbol~$\sigma$, and the transition function is undefined, i.e., $M$
has halted.
\end{proof}

\end{document}
