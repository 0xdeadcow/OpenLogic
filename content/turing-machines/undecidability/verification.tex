% Part: turing-machines 
% Chapter: undecidability 
% Section: verification

\documentclass[../../../include/open-logic-section]{subfiles}

\begin{document}

\olfileid{tms}{und}{ver} 
\olsection{Verifying the Representation}

\begin{explain}
 In order to verify that our representation works, we first
have to make sure that if $M$ halts on input~$w$, then $T(m, w) \lif H(M,
w)$ is valid. We can do this simly by proving that $T(m, w)$ implies a
description of the configuration of $M$ for each step of the execution of
$M$ on input $w$. If $M$ halts on input $w$, then for some $n$, $M$ will be
in a halting configuration at step $n$ (and be scanning square~$m$, for
some $m$). Hence, $T(M, w)$ implies $\Obj Q_h(\overline m, \overline n)$.
\end{explain}

\begin{defn} 
Let $C(M, w, n)$ be the sentence 
\[ 
\Obj Q_q(\overline m,
\overline n) \land \Obj S_{\sigma_0}(\overline 0, \overline n) \land \dots
\land \Obj S_{\sigma_k}(\overline k, \overline n) 
\] 
where $q$ is the state
of $M$ at time~$n$, $M$ is scanning square~$m$ at time~$n$, square $i$
contains symbol~$\sigma_i$ at time~$n$ for $0 \le i \le k$ and $k$ is the
right-most non-blank square of the tape at time~$m$. 
\end{defn}

\begin{proof} Suppose that $M$ does halt for input $w$. Then there is some
time $n$, state $q$ and square $m$ and symbol $\sigma$ such that:

\begin{enumerate} 
\item There is no instruction beginning $\tuple{q_q, \sigma}$ 
\item At time $n$ the machine is in state $q$ scanning a square
$m$ on which $\sigma$ appears. 
\end{enumerate}

C(M, w, n) will be the description of this time and will include the
clauses $Q_q(\overline m, \overline n)$ and $S_{\sigma}(\overline m,
\overline n)$.

These clauses together imply $H(M,w)$, since by definition one of the
disjuncts will be: \[ \lexists[\Obj x]\lexists[\Obj y][(\Obj Q_q(\Obj x,
\Obj y) \land S_\sigma(\Obj x, \Obj y))] \]

So if $M$ halts for input $w$, then there is some time $n$ such that $\Sat{
C(M, w, n)}{H(M,w)}$

Since consequence is transitive, it is sufficient to show that for any time
n, $\Sat{T(M, w)}{C(M, w, n)}$

\begin{lem} 
For each $n$, $T(M, w)$ implies $C(M, w, n)$. 
\end{lem}

By induction on $n$.

If $n = 0$, then $C(M, w, n)$ is a conjunct of $T(M, w)$, so implied by it.
Inductive hypothesis: If $M$ has not halted before time $n$, then $T(M,w)$
implies $C(M, w, n)$.

Suppose $n > 0$ and at time $n$, $M$ started on $w$ is in state~$q$
scanning square $m$, and the content of the tape is $\sigma_0$, \dots,
$\sigma_k$.

Suppose that $M$ has not halted before time $n+1$. Since $T(M,n)$ is true
in its intended interpretation $\Struct M$, the inductive hypothesis tells
us that $C(M, w, n)$ is true in $\Struct M$ also. In particular,
$Q_q(\overline m, \overline n)$ and $S_{\sigma_m}(\overline m, \overline
n)$ are true in $\Struct M$.

Since M does not halt at n, there must be an instruction of one of the
following three forms in the program of M: 

\begin{enumerate} 
\item $\tuple{q_q, \sigma_m , q_{q'}, \sigma'_{m}, \TMright}$

\item $\tuple{q_q, \sigma_m , q_{q'}, \sigma'_{m}, \TMleft}$

\item $\tuple{q_q, \sigma_m , q_{q'}, \sigma'_{m}, \TMstay}$
\end{enumerate}

We will consider each of these three cases in turn.

\begin{enumerate} 
\item Suppose there is an instruction of the form of (1).
This means that 
\[ 
\lforall[\Obj x] \lforall[\Obj y]( (\Obj Q_{q}(\Obj x,
\Obj y) \land \Obj S_{\sigma_m}(\Obj x, \Obj y)) \lif (\Obj Q_{q'}(\Obj x',
\Obj y') \land \Obj S_{\sigma'_{m}}(\Obj x, \Obj y') \land A(\Obj x, \Obj
y))) 
\] 
is a member of $T(M,w)$. This in turn implies the following
sentence, through universal instantiation: 
\[ 
(\Obj Q_{q}(\overline m,
\overline n) \land \Obj S_{\sigma_m}(\overline m, \overline n)) \lif (\Obj
Q_{q'}(\overline m', \overline n') \land \Obj S_{\sigma'_{m}}(\overline m,
\overline n') \land A(\Obj x, \Obj y))) \] Which in turn implies, \[ \Obj
Q_{q'}(\overline m', \overline n') \land \Obj S_{\sigma'_m}(\overline m,
\overline n') \land \Obj S_{\sigma_0}(\overline 0, \overline n')\dots \land
\Obj S_{\sigma_k}(\overline k, \overline n') 
\] 
But this just is $C(M, w, n')$.

\item Suppose there is an instruction of the form of 2. So, 
\[
\lforall[\Obj x] \lforall[\Obj y]( (\Obj Q_{q}(\Obj x', \Obj y) \land \Obj
S_{\sigma_m}(\Obj x, \Obj y)) \lif (\Obj Q_{q'}(\Obj x, \Obj y') \land \Obj
S_{\sigma'_m}(\Obj x, \Obj y') \land A(\Obj x, \Obj y))) 
\] 
is an element of $T(M,w)$, which implies the following sentence: 
\[ 
(\Obj Q_{q}(\overline
m', \overline n) \land \Obj S_{\sigma_m}(\overline m, \overline n) \lif
(\Obj Q_{q'}(\overline m, \overline n') \land \Obj S_{\sigma'_m}(\overline
m, \overline n') \land A(\Obj x, \Obj y))),
\]
which in turn implies \[ \Obj
Q_{q'}(\overline m, \overline n')\land \Obj S_{\sigma'_m}(\overline m,
\overline n') \land \Obj S_{\sigma_0}(\overline 0, \overline n')\dots \land
\Obj S_{\sigma_k}(\overline k, \overline n') 
\] 
But this just is $C(M, w, n')$.

\item This case is left as an exercise.

Given the above cases, we have shown that for any $n$, $T(M, w) \lif C(M,
w, n)$.

Based on this lemma, we know that, for any time~$n$, the description of
time~$n$ is a consequence of $T(M, w)$, and therefore, that $T(M, w) \lif
H(M, w)$ is valid.
\end{enumerate} 
\end{proof}

\begin{explain} 
To complete the verification of our claim, we also have to
establish the reverse direction: if $T(M, w) \lif H(M, w)$ is valid, then
$M$ does in fact halt when started on input~$m$. 
\end{explain}

\begin{lem}
If $T(M, w)$ entails $H(M, w)$, then $M$ halts on input~$w$.
\end{lem}

\begin{proof} 
Consider the $\Lang L_M$-!!{structure} $\Struct M$ with
domain $\Nat$ which inerprets $\Obj 0$ as $0$, $'$ as the successor
function, and $<$ as the lest-than relation, and the predicates $\Obj Q_q$
and $\Obj S_\sigma$ as follows:
\begin{align*}
  \Assign{\Obj Q_q}{M} & =
\{(m, n) : \text{after $n$ steps, $M$ started on $w$ is in state $q$
  scanning square~$m$}\} \\
\Assign{\Obj S_\sigma}{M} & = \{(m, n) :
\text{after $n$ steps, $M$ started on $w$ has symbol~$\sigma$ on
  square~$m$}\}
\end{align*}
Clearly, $\Sat{M}{T(M, w)}$. If $T(M, w)$
implies $H(M, w)$, then $\Sat{M}{H(M, w)}$, i.e.,
\[
\Sat{M}{\lexists[\Obj x]\lexists[\Obj y][\Obj{Q}_h(\Obj x, \Obj y)]}.
\]
As $\Domain{M} = \Nat$, there must be $m$, $n \in \Nat$ so that
$\Sat{M}{\Obj Q_h(\overline{n}, \overline{m})}$. By the definition of
$\Struct M$, this means that $M$ started on input~$w$ is in state~$h$
after~$m$ steps, i.e., has halted.
\end{proof}

\end{document}
