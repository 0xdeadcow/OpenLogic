% Part: turing-machines
% Chapter: undecidability
% Section: enumerating-tms

\documentclass[../../../include/open-logic-section]{subfiles}

\begin{document}

\olfileid{tur}{und}{enu}
\olsection{Enumerating Turing Machines}

\begin{explain}
We can show that the set of all Turing-machines is
!!{enumerable}. This follows from the fact that each Turing machine
can be finitely described.  The set of states and the tape vocabulary
are finite sets.  The transition function is a partial function from
$Q \times \Sigma$ to $Q \times \Sigma \times \{\TMleft, \TMright,
\TMstay\}$, and so likewise can be specified by listing its values for
the finitely many argument pairs for which it is defined.  Of course,
strictly speaking, the states and vocabulary can be anything; but the
\emph{behavior} of the Turing machine is independent of which objects
serve as states and vocabulary. So we may assume, for instance, that
the states and vocabulary symbols are natural numbers, or that the
states and vocabulary are all strings of letters and digits.

Suppose we fix a !!{denumerable} vocabulary for specifying Turing
machines: $\sigma_0 = \TMendtape$, $\sigma_1 = \TMblank$, $\sigma_2 =
\TMstroke$, $\sigma_3$, \dots, $\TMright$, $\TMleft$, $\TMstay$,
$q_0$, $q_1$, \dots. Then any Turing machine can be specified by some
finite string of symbols from this alphabet (though not every finite
string of symbols specifies a Turing machine). For instance, suppose
we have a Turing machine $M = \tuple{Q, \Sigma, q, \delta}$ where
\begin{align*}
  Q & =
  \{q'_0, \dots, q'_n\} \subseteq \{q_0, q_1, \dots \} \text{ and}\\
  \Sigma & = \{\TMendtape, \sigma'_1, \sigma'_2, \dots, \sigma'_m\}
  \subseteq \{\sigma_0, \sigma_1, \dots\}.
\end{align*}
We could specify it by the string
\[
q'_0 q'_1 \dots q'_n \TMendtape \sigma'_1 \dots \sigma'_m \TMendtape q
\TMendtape S(\sigma'_0,q'_0)\TMendtape \dots \TMendtape S(\sigma'_m, q'_n)
\]
where $S(\sigma'_i, q'_j) $ is the string $\sigma'_i q'_j
\delta(\sigma'_i, q'_j)$ if $\delta(\sigma'_i,q'_j)$ is defined, and
$\sigma'_i q'_j$ otherwise.
\end{explain}

\begin{thm}
There are functions from $\Nat$ to $\Nat$ which are not Turing
computable.
\end{thm}

\begin{proof}
We know that the set of finite strings of symbols from
!!a{denumerable} alphabet is !!{enumerable}. This gives us that the
set of descriptions of Turing machines, as a subset of the finite
strings from the !!{enumerable} vocabulary $\{q_0, q_1, \dots,
\TMendtape, \sigma_1, \sigma_2, \dots\}$, is itself enumerable.  Since
every Turing computable function is computed by some (in fact, many)
Turing machines, this means that the set of all Turing computable
functions from $\Nat$ to $\Nat$ is also enumerable.

On the other hand, the set of all functions from $\Nat$ to $\Nat$ is
not !!{enumerable}. This follows immediately from the fact that not
even the set of all functions of one argument from $\Nat$
to the set $\{0,1\}$ is !!{enumerable}.  If all functions
were computable by some Turing machine we could enumerate the set of
all functions. So there are some functions that are not
Turing-computable. 
\end{proof}

\end{document}
