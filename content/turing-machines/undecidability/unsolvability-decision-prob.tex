% Part: turing-machines
% Chapter: undecidability
% Section: unsolvability-decision-problem

\documentclass[../../../include/open-logic-section]{subfiles}

\begin{document}

\olfileid{tms}{und}{dec}
\olsection{Unsolvability of the Decision Problem}

We say that first-order logic is \emph{decidable} iff there is an effective
mechanical method for determining whether or not a !!{sentence} is
valid. We also know, from the Church-Turing thesis, that every
function which is computable is Turing computable. We have proved that
the function~$h(m,w)$ that halts with an output $\TMstroke$ if the
Turing-machine described by $m$ halts on input $w$ and outputs
$\TMblank$ otherwise, is not Turing-computable. Via the Church-Turing
thesis, $h$~is not computable.

In order to show that first-order logic is undecidable, we prove the following.

\begin{quote}
Given input~$w$ and a Turing machine~$M$ we can effectively describe
!!a{sentence} $!T$ representing~$M$ and~$w$ and a !!{sentence} $!H$
expressing ``$M$ eventually halts'' such that:
\begin{center}
  $\Entails !H \lif !T$ iff $M$ halts for input~$w$.
\end{center}
\end{quote}

This gives us the theorem that if first-order logic is decidable, then
the halting problem is solvable. However, we already know that the
halting problem is not solvable.  Therefore, first-order logic is not
decidable.

\end{document}
