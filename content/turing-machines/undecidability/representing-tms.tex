% Part: turing-machines
% Chapter: undecidability
% Section: representing-tms

\documentclass[../../../include/open-logic-section]{subfiles}

\begin{document}

\olfileid{tms}{und}{rep}
\olsection{Representing Turing Machines}

\begin{explain}
In order to represent Turing machines and their behavior by a !!{sentence}
of first-order logic, we have to define a suitable language. The
language consists of two parts: !!{predicate}s for describing
configurations of the machine, and expressions for counting execution
steps (``moments'') and positions on the tape. The latter require an
initial moment, $\Obj 0$, a ``successor'' function which is
traditionally written as a postfix $\prime$, and an ordering $x
< y$ of ``before.''
\end{explain}

\begin{defn}
Given a Turing machine $M = \tuple{Q, \Sigma, q_0, \delta}$, the
language~$\Lang L_M$ consists of:
\begin{enumerate}
\item A two-place !!{predicate} $\Obj Q_q(x, y)$ for every state~$q \in
  Q$.  Intuitively, $\Obj Q_q(\num{n}, \num{m})$ expresses ``after $m$
  steps, $M$ is in state~$q$ scanning the $n$th square.''
\item A two-place !!{predicate} $\Obj S_\sigma(x, y)$ for every
  symbol~$\sigma\in \Sigma$.  Intuitively, $\Obj S_\sigma(\num{n},
  \num{m})$ expresses ``after $m$ steps, the $n$th square contains
  symbol~$\sigma$.''
\item A constant $\Obj 0$
\item A one-place function $\prime$
\item A two-place predicate $<$
\end{enumerate}
\end{defn}

For each number $n$ there is a canonical term $\num{n}$, the
\emph{numeral} for~$n$, which represents it in~$\Lang L_M$. $\num{0}$
is $\Obj 0$, $\num{1}$ is $\Obj 0'$, $\num{2}$ is $\Obj 0''$, and so
on. More formally:
\begin{align*}
\num{0} & = \Obj 0 \\
\num{n+1} &= \num{n}'
\end{align*}

The !!{sentence}s describing the operation of the Turing machine~$M$ on
input $w = \sigma_{i_1}\dots\sigma_{i_n}$ are the following:
\begin{enumerate}
\item Axioms describing numbers:
\begin{enumerate}
\item A !!{sentence} that says that the successor function is injective:
\[
\lforall[x][\lforall[y][
    (\eq[x'][y'] \lif \eq[x][y])]]
\]
\item A !!{sentence} that says that every number is less than its successor:
\[
\lforall[x][(x < x')]
\]
\item A !!{sentence} that ensures that $<$ is transitive:
\[
\lforall[x][\lforall[y][\lforall[z][
      ((x < y \land y < z) \lif x < z)]]]
\]
\end{enumerate}
\item Axioms describing the input configuration:
\begin{enumerate}
\item $M$ is in the inital state~$q_0$ at time~0, scanning square~1:
\[
\Obj Q_{q_0}(\num{1}, \num{0})
\]
\item The first $n+1$ squares contain the symbols $\TMendtape$,
  $\sigma_{1}$, \dots, $\sigma_{n}$:
\[
\Obj S_\TMendtape(\num{0}, \num{0}) \land
\Obj S_{\sigma_1}(\num{1}, \num{0}) \land
\dots \land
\Obj S_{\sigma_n}(\num{n}, \num{0})
\]
\item Otherwise, the tape is empty:
\[
\lforall[x][(\num{n} < x \lif \Obj S_\TMblank(x, \num{0}))]
\]
\end{enumerate}
\item Axioms describing the transition from one configuration to
  the next:

For the following, let $A(x, y)$ be the conjunction of all !!{sentence}s
of the form
\[
\lforall[z][
  (((z < x \lor x < z) \land \Obj S_\sigma(z, y))
  \lif \Obj S_\sigma(z, y'))]
\]
where $\sigma \in \Sigma$.  We use $A(\num{n},\num{m})$ to express
``other than at square~$n$, the tape after $m+1$ steps is the same as
after $m$ steps.''
\begin{enumerate}
\item For every instruction $\delta(q_i, \sigma) = \tuple{q_j,
  \sigma', \TMleft}$, the !!{sentence}:
\[
\lforall[x][\lforall[y][
    ((\Obj Q_{q_i}(x', y) \land \Obj S_{\sigma}(x, y)) \lif
   (\Obj Q_{q_j}(x, y') \land \Obj S_{\sigma'}(x, y') \land
A(x, y)))]]
\]
\item For every instruction $\delta(q_i, \sigma) = \tuple{q_j,
  \sigma', \TMright}$, the !!{sentence}:
\[
\lforall[x][\lforall[y][(
   (\Obj Q_{q_i}(x, y) \land \Obj S_{\sigma}(x, y)) \lif
   (\Obj Q_{q_j}(x', y') \land \Obj S_{\sigma'}(x, y') \land
A(x, y)))]]
\]
\item For every instruction $\delta(q_i, \sigma) = \tuple{q_j,
  \sigma', \TMstay}$, the !!{sentence}:
\[
\lforall[x][\lforall[y][(
   (\Obj Q_{q_i}(x, y) \land \Obj S_{\sigma}(x, y)) \lif
   (\Obj Q_{q_j}(x, y') \land \Obj S_{\sigma'}(x, y') \land
A(x, y)))]]
\]
\end{enumerate}
\end{enumerate}
%%(Probably also need axioms saying every square as exactly one symbol on
%%it at all times, machine always in exactly one state.)

Let $!T(M, w)$ be the conjunction of all the above !!{sentence}s for Turing
machine~$M$ and input~$w$

In order to express that $M$ eventualy halts, we have to find a
!!{sntence} that says ``after some number of steps, the transition
function will not be undefined.''  Let $X$ be the set of all pairs
$\tuple{q, \sigma}$ such that $\delta(q, \sigma)$ is undefined.  Let
$!H(M, w)$ then be the !!{sentence}
\[
\lexists[x][\lexists[y][(\bigvee_{\tuple{q, \sigma} \in
      X}(\Obj Q_q(x, y) \land \Obj S_\sigma(x, y)))]]
\]

If we use a Turing machne with a designated with halting state~$h$, it
is even easier: then the !!{sentence} $!H(M, w)$
\[
\lexists[x][\lexists[y][\Obj Q_h(x, y)]]
\]
expresses that the machine eventually halts.

\end{document}
