% Part: turing-machines 
% Chapter: undecidability
% Section: introduction

\documentclass[../../../include/open-logic-section]{subfiles}

\begin{document}

\olfileid{tur}{und}{int} 
\olsection{Introduction}

It might seem obvious that not every function, even every arithmetical
function, can be computable. There are just too many, whose behavior
is too complicated.  Functions defined from the decay of radioactive
particles, for instance, or other chaotic or random behavior. Suppose
we start counting 1-second intervals from a given time, and define the
function $f(n)$ as the number of particles in the universe that decay
in the $n$-th 1-second interval after that initial moment.  This seems
like a candidate for a function we cannot ever hope to compute.

But it is one thing to not be able to imagine how one would compute
such functions, and quite another to actually prove that they are
uncomputable.  In fact, even functions that seem hopelessly
complicated may, in an abstract sense, be computable.  For instance,
suppose the universe is finite in time---some day, in the very distant
future the universe will contract into a single point, as some
cosmological theories predict. Then there is only a finite (but
incredibly large) number of seconds from that initial moment for which
$f(n)$ is defined.  And any function which is defined for only finitely
many inputs is computable: we could list the outputs in one big table,
or code it in one very big Turing machine state transition diagram.

We are often interested in special cases of functions whose values give
the answers to yes/no questions.  For instance, the question ``is $n$
a prime number?'' is associated with the function
\[
\fn{isprime}(n) = \begin{cases}
  1 & \text{if $n$ is prime}\\
  0 & \text{otherwise.}
  \end{cases}
\]
We say that a yes/no question can be \emph{effectively decided}, if
the associated $1/0$-valued function is effectively computable.

To prove mathematically that there are functions which cannot be
effectively computed, or problems that cannot effectively decided, it
is essential to fix a specific model of computation, and show about it
that there are functions it cannot compute or problems it cannot
decide.  We can show, for instance, that not every function can be
computed by Turing machines, and not every problem can be decided by
Turing machines.  We can then appeal to the Church-Turing thesis to
conclude that not only are Turing machines not powerful enough to
compute every function, but no effective procedure can.

The key to proving such negative results is the fact that we can
assign numbers to Turing machines themselves.  The easiest way to do
this is to enumerate them, perhaps by fixing a specific way to write
down Turing machines and their programs, and then listing them in a
systematic fashion.  Once we see that this can be done, then the
existence of Turing-uncomputable functions follows by simple
cardinality considerations: the set of functions from $\Nat$ to~$\Nat$ (in
fact, even just from $\Nat$ to $\{0, 1\}$) are !!{nonenumerable}, but
since we can enumerate all the Turing machines, the set of Turing-computable
functions is only !!{denumerable}.

We can also define \emph{specific} functions and problems which we can
prove to be uncomputable and undecidable, respectively.  One such
problem is the so-called \emph{Halting Problem.} Turing machines can
be finitely described by listing their instructions.  Such a
description of a Turing machine, i.e., a Turing machine program, can
of course be used as input to another Turing machine.  So we can
consider Turing machines that decide questions about other Turing
machines.  One particularly interesting question is this: ``Does the
given Turing machine eventually halt when started on input~$n$?''  It
would be nice if there were a Turing machine that could decide this
question: think of it as a quality-control Turing machine which
ensures that Turing machines don't get caught in infinite loops and
such.  The interestign fact, which Turing proved, is that there cannot
be such a Turing machine. There cannot be a single Turing machine
which, when started on input consisting of a description of a Turing
machine $M$ and some number~$n$, will always halt with either output
$1$ or $0$ according to whether $M$ machine would have halted when
started on input $n$ or not.

Once we have examples of specific undecidable problems we can use them
to show that other problems are undecidable, too.  For instance, one
celebrated undecidable problem is the question, ``Is the first-order
!!{formula}~$!A$ valid?''.  There is no Turing machine which, given as
input a first-order !!{formula}~$!A$, is guaranteed to halt with
output $1$ or $0$ according to whether $!A$ is valid or not.
Historically, the question of finding a procedure to effectively solve
this problem was called simply ``the'' decision problem; and so we say
that the decision problem is unsolvable.  Turing and Church proved this
result independently at around the same time, so it is also called the
Church-Turing Theorem.

\end{document}
