% Part: counterfactuals
% Chapter: minimal-change-semantics
% Section: transitivity

\documentclass[../../../include/open-logic-section]{subfiles}

\begin{document}

\olfileid{cnt}{min}{tra}

\olsection{Transitivity}

For the material conditional, the chain rule holds: $!A \lif !B, !B
\lif !C \Entails !A \lif !C$. In other words, the material conditional
is transitive. Is the same true for counterfactuals? Consider the
following example due to Stalnaker.
\begin{quote}
  If J.~Edgar Hoover had been born a Russian, he would have been a Communist.

  If J.~Edgar Hoover were a Communist, he would have been be a traitor.

  Therefore, If J.~Edgar Hoover had been born a Russian, he would have
  been be a traitor.
\end{quote}
If Hoover had been born (at the same time he actually did), not in the
United States, but in Russia, he would have grown up in the Soviet
Union and become a Communist (let's assume). So the first premise is
true. Likewise, the second premise, considered in isolation is
true. The conclusion, however, is false: in all likelihood, Hoover
would have been a fervent Communist if he had been born in the USSR,
and not been a traitor (to his country).  The intuitive assignment of
truth values is borne out by the Stalnaker-Lewis account. The closest
possible world to ours with the only change being Hoover's place of
birth is the one where Hoover grows up to be a good citizen of the
USSR. This is the closest possible world where the antecedent of the
first premise and of the conclusion is true, and in that world Hoover
is a loyal member of the Communist party, and so not a traitor. To
evaluate the second premise, we have to look at a different world,
however: the closest world where Hoover is a Communist, which is one
where he was born in the United States, turned, and thus became a
traitor.\footnote{Of course, to appreciate the force of the example we
  have to take on board some metaphysical and political assumptions,
  e.g., that it is possible that Hoover could have been born to
  Russian parents, or that Communists in the US of the 1950s were
  traitors to their country.}

\begin{prob}
  Find a convincing, intuitive example for the failure of transitivity
  of counterfactuals.
\end{prob}

\begin{ex}\ollabel{ex:trans-counterex}
  The sphere semantics invalidates the inference, i.e., we have $p
  \cif q, q \cif r \Entails/ p \cif r$. Consider the model $\mModel{M}
  = \tuple{W, O, V}$ where $W = \{w, w_1, w_2\}$, $O_w = \{\{w\}, \{w,
  w_1\}, \{w, w_1, w_2\}\}$, $V(p) = \{w_2\}$, $V(q) = \{w_1, w_2\}$,
  and $V(r) = \{w_1\}$. There is a $p$-admitting sphere $S = \{w, w_1,
  w_2\}$ and $p \lif q$ is true at all worlds in it, so $\mSat{M}{p
    \cif q}[w]$. There is also a $q$-admitting sphere $S' = \{w,
  w_1\}$ and $\mSat/{M}{q \lif r}$ is true at all worlds in it, so
  $\mSat{M}{q \cif r}[w]$. However, the $p$-admitting sphere $\{w,
  w_1, w_2\}$ contains a world, namely~$w_2$, where $\mSat/{M}{p \lif
    r}[w_2]$.
\end{ex}

\begin{prob}
Draw the sphere diagram corresponding to the counterexample in
\olref[cnt][min][tra]{ex:trans-counterex}.
\end{prob}

\begin{prob}
  In \olref[cnt][min][tra]{ex:trans-counterex}, world $w_2$ is where
  Hoover is born in Russia, is a communist, and not a traitor, and
  $w_1$ is the world where Hoover is born in the US, is a communist,
  and a traitor. In this model, $w_1$ is closer to~$w$ than $w_2$
  is. Is this necessary? Can you give a counterexample that does
  not assume that Hoover's being born in Russia is a more remote
  possibility than him being a Communist?
\end{prob}

\end{document}
