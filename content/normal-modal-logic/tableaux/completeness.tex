% Part: normal-modal-logic
% Chapter: tableaux
% Section: completeness

\documentclass[../../../include/open-logic-section]{subfiles}

\begin{document}

\olfileid{mod}{tab}{cpl}
      
\olsection{Completeness for \Log{K}}

\begin{explain}
  To show that the method of !!{tableau}s is complete, we have to show
  that whenever there is no closed !!{tableau} to show $\Gamma \Proves
  !A$, then $\Gamma \Entails/ !A$, i.e., there is a countermodel. But
  ``there is no closed !!{tableau}'' means that every way we could try
  to construct one has to fail to close. The trick is to see that if
  every such way fails to close, then a specific, \emph{systematic and
    exhaustive} way also fails to close. And this systematic and
  exhaustive way would close if a closed !!{tableau} exists. The
  single tableau will contain, among its open branches, all the
  information required to define a countermodel. The countermodel
  given by an open branch in this tableau will contain the all the
  prefixes used on that branch as the worlds, and !!a{propositional
    variable}~$p$ is true at $\sigma$ iff $\sFmla{\True}{p}[\sigma]$
  occurs on the branch.
\end{explain}

\begin{defn}
  A branch in !!a{tableau} is called complete if, whenever it contains
  a prefixed !!{formula} $\sFmla{S}{!A}[\sigma]$ to which a rule can
  be applied, it also contains
  \begin{enumerate}
    \item the prefixed !!{formula}s that are the corresponding
      conclusions of the rule, in the case of propositional stacking
      rules;
    \item one of the corresponding conclusion !!{formula}s in the case
      of propositional branching rules;
    \item at least one possible conclusion in the case of modal rules
      that require a new prefix;
    \item the corresponding conclusion for every prefix occurring on
      the branch in the case of modal rules that require a used
      prefix.
  \end{enumerate}
\end{defn}

\begin{explain}
For instance, a complete branch contains $\sFmla{\True}{!B}[\sigma]$
and $\sFmla{\True}{!C}[\sigma]$ whenever it contains $\sFmla{\True}{!B
  \land !C}$. If it contains $\sFmla{\True}{!B \lor !C}[\sigma]$ it
contains at least one of $\sFmla{\False}{!B}[\sigma]$ and
$\sFmla{\True}{!C}[\sigma]$. If it contains \iftag{prvBox}
{$\sFmla{\False}{\Box}[\sigma]$} {$\sFmla{\True}{\Diamond}[\sigma]$}
it also contains \iftag{prvBox} {$\sFmla{\False}{\Box}[\sigma.n]$}
{$\sFmla{\True}{\Diamond}[\sigma.n]$} for at least one~$n$.  And
whenever it contains \iftag{prvBox} {$\sFmla{\True}{\Box}[\sigma]$}
{$\sFmla{\False}{\Diamond}[\sigma]$} it also contains \iftag{prvBox}
{$\sFmla{\True}{\Box}[\sigma.n]$}
{$\sFmla{\False}{\Diamond}[\sigma.n]$} for every~$n$ such that
$\sigma.n$ is used on the branch.
\end{explain}

\begin{prop}\ollabel{prop:complete-tableau}
  Every finite $\Gamma$ has a !!{tableau} in which every branch is complete.
\end{prop}

\begin{proof}
  Consider an open branch in !!a{tableau} for~$\Gamma$. There are
  finitely many prefixed !!{formula}s in the branch to which a rule
  could be applied. In some fixed order (say, top to bottom), for each
  of these prefixed !!{formula}s for which the conditions (1)--(4) do
  not already hold, apply the rules that can be applied to it to
  extend the branch. In some cases this will result in branching;
  apply the rule at the tip of each resulting branch for all remaining
  prefixed !!{formula}s. Since the number of prefixed !!{formula}s is
  finite, and the number of used prefixes on the branch is finite,
  this procedure eventually results in (possibly many) branches
  extending the original branch. Apply the procedure to each, and
  repeat. But by construction, every branch is closed.
\end{proof}

\begin{thm}[Completeness]
  \ollabel{thm:tableau-completeness}
  If $\Gamma$ has no closed !!{tableau}, $\Gamma$ is satisfiable.
\end{thm}

\begin{proof}
By the proposition, $\Gamma$ has !!a{tableau} in which every branch is
complete. Since it has no closed !!{tableau}, it thas has !!a{tableau} in
which at least one branch is open and complete. Let $\Delta$ be the
set of prefixed !!{formula}s on the branch, and $P(\Delta)$ the set of
prefixes occurring in it.

We define a model~$\mModel{M(\Delta)} = \tuple{P(\Delta), R, V}$ where
the worlds are the prefixes occurring in~$\Delta$, the accessibility
relation is given by:
\[
R\sigma\sigma' \quad \text{iff} \quad
\sigma'=\sigma.n \quad \text{for some~$n$}
\]
and
\[
V(p) = \Setabs{\sigma}{\sFmla{\True}{p}[\sigma] \in \Delta}.
\]
We show by induction on~$!A$ that if $\sFmla{\True}{!A}[\sigma] \in
\Delta$ then $\mSat{M(\Delta)}{!A}[\sigma]$, and if
$\sFmla{\False}{!A}[\sigma] \in \Delta$ then
$\mSat/{M(\Delta)}{!A}[\sigma]$.

\begin{enumerate}
  \item \indcase{!A}{p}{If $\sFmla{\True}{\indfrm}[\sigma] \in \Delta$
    then $\sigma \in V(p)$ (by definition of~$V$) and so
    $\mSat{M(\Delta)}{\indfrm}[\sigma]$.

    If $\sFmla{\False}{\indfrm}[\sigma] \in \Delta$ then
    $\sFmla{\True}{\indfrm}[\sigma] \notin \Delta$, since the branch
    would otherwise be closed. So $\sigma \notin V(p)$ and thus
    $\mSat/{M(\Delta)}{\indfrm}[\sigma]$.}
  \item \indcase{!A}{\lnot !B}{\iftag{probNot}{Exercise.}{If
      $\sFmla{\True}{\indfrm}[\sigma] \in \Delta$, then
      $\sFmla{\False}{!B}[\sigma] \in \Delta$ since the branch is
      complete. By induction hypothesis,
      $\mSat/{M(\Delta)}{!B}[\sigma]$ and thus
      $\mSat{M(\Delta)}{\indfrm}[\sigma]$.

      If $\sFmla{\False}{\indfrm}[\sigma] \in \Delta$, then
      $\sFmla{\True}{!B}[\sigma] \in \Delta$ since the branch is
      complete. By induction hypothesis,
      $\mSat{M(\Delta)}{!B}[\sigma]$ and thus
      $\mSat/{M(\Delta)}{\indfrm}[\sigma]$.}}
  \item \indcase{!A}{!B \land !A}{\iftag{probAnd}{Exercise.}{If
      $\sFmla{\True}{\indfrm}[\sigma] \in \Delta$, then both
      $\sFmla{\True}{!B}[\sigma] \in \Delta$ and
      $\sFmla{\True}{!C}[\sigma] \in \Delta$ since the branch is
      complete. By induction hypothesis,
      $\mSat{M(\Delta)}{!B}[\sigma]$ and
      $\mSat{M(\Delta)}{!C}[\sigma]$. Thus
      $\mSat{M(\Delta)}{\indfrm}[\sigma]$.

      If $\sFmla{\False}{\indfrm}[\sigma] \in \Delta$, then either
      $\sFmla{\False}{!B}[\sigma] \in \Delta$ or
      $\sFmla{\False}{!C}[\sigma] \in \Delta$ since the branch is
      complete. By induction hypothesis, either
      $\mSat/{M(\Delta)}{!B}[\sigma]$ or
      $\mSat/{M(\Delta)}{!B}[\sigma]$. Thus
      $\mSat/{M(\Delta)}{\indfrm}[\sigma]$.}}
  \item \indcase{!A}{!B \lor !A}{\iftag{probOr}{Exercise.}{If
      $\sFmla{\True}{\indfrm}[\sigma] \in \Delta$, then either
      $\sFmla{\True}{!B}[\sigma] \in \Delta$ or
      $\sFmla{\True}{!C}[\sigma] \in \Delta$ since the branch is
      complete. By induction hypothesis, either
      $\mSat{M(\Delta)}{!B}[\sigma]$ or
      $\mSat{M(\Delta)}{!C}[\sigma]$. Thus
      $\mSat{M(\Delta)}{\indfrm}[\sigma]$.

      If $\sFmla{\False}{\indfrm}[\sigma] \in \Delta$, then both
      $\sFmla{\False}{!B}[\sigma] \in \Delta$ and
      $\sFmla{\False}{!C}[\sigma] \in \Delta$ since the branch is
      complete. By induction hypothesis, both
      $\mSat/{M(\Delta)}{!B}[\sigma]$ and
      $\mSat/{M(\Delta)}{!B}[\sigma]$. Thus
      $\mSat/{M(\Delta)}{\indfrm}[\sigma]$.}}
  \item \indcase{!A}{!B \lif !A}{\iftag{probIf}{Exercise.}{If
      $\sFmla{\True}{\indfrm}[\sigma] \in \Delta$, then either
      $\sFmla{\False}{!B}[\sigma] \in \Delta$ or
      $\sFmla{\True}{!C}[\sigma] \in \Delta$ since the branch is
      complete. By induction hypothesis, either
      $\mSat/{M(\Delta)}{!B}[\sigma]$ or
      $\mSat{M(\Delta)}{!C}[\sigma]$. Thus
      $\mSat{M(\Delta)}{\indfrm}[\sigma]$.

      If $\sFmla{\False}{\indfrm}[\sigma] \in \Delta$, then both
      $\sFmla{\True}{!B}[\sigma] \in \Delta$ and
      $\sFmla{\False}{!C}[\sigma] \in \Delta$ since the branch is
      complete. By induction hypothesis, both
      $\mSat{M(\Delta)}{!B}[\sigma]$ and
      $\mSat/{M(\Delta)}{!B}[\sigma]$. Thus
      $\mSat/{M(\Delta)}{\indfrm}[\sigma]$.}}
  \item \indcase{!A}{\Box !B}{\iftag{probBox}{Exercise.}{If
      $\sFmla{\True}{\indfrm}[\sigma] \in \Delta$, then, since the
      branch is complete, $\sFmla{\True}{!B}[\sigma.n] \in \Delta$ for
      every $\sigma.n$ used on the branch, i.e., for every $\sigma'
      \in P(\Delta)$ such that $R\sigma\sigma'$.  By induction
      hypothesis, $\mSat{M(\Delta)}{!B}[\sigma']$ for every $\sigma'$
      such that $R\sigma\sigma'$. Therefore,
      $\mSat{M(\Delta)}{\indfrm}[\sigma]$.

      If $\sFmla{\False}{\indfrm}[\sigma] \in \Delta$, then for some
      $\sigma.n$, $\sFmla{\False}{!B}[\sigma.n] \in \Delta$ since the
      branch is complete.  By induction hypothesis,
      $\mSat/{M(\Delta)}{!B}[\sigma.n]$.  Since $R\sigma(\sigma.n)$,
      there is a~$\sigma'$ such that
      $\mSat/{M(\Delta)}{!B}[\sigma']$. Thus
      $\mSat/{M(\Delta)}{\indfrm}[\sigma]$.}}
  \item \indcase{!A}{\Diamond !B}{\iftag{probDiamond}{Exercise.}{If
      $\sFmla{\True}{\indfrm}[\sigma] \in \Delta$, then for some
      $\sigma.n$, $\sFmla{\True}{!B}[\sigma.n] \in \Delta$ since the
      branch is complete. By induction hypothesis,
      $\mSat{M(\Delta)}{!B}[\sigma.n]$.  Since $R\sigma(\sigma.n)$,
      there is a~$\sigma'$ such that
      $\mSat{M(\Delta)}{!B}[\sigma']$. Thus
      $\mSat{M(\Delta)}{\indfrm}[\sigma]$.

      If $\sFmla{\False}{\indfrm}[\sigma] \in \Delta$, then, since the
      branch is complete, $\sFmla{\False}{!B}[\sigma.n] \in \Delta$
      for every $\sigma.n$ used on the branch, i.e., for every
      $\sigma' \in P(\Delta)$ such that $R\sigma\sigma'$.  By
      induction hypothesis, $\mSat/{M(\Delta)}{!B}[\sigma']$ for every
      $\sigma'$ such that $R\sigma\sigma'$. Therefore,
      $\mSat/{M(\Delta)}{\indfrm}[\sigma]$.}}
\end{enumerate}
Since $\Gamma \subseteq \Delta$, $\mSat{M(\Delta)}{\Gamma}$.
\end{proof}

\begin{prob}
Complete the proof of \olref[mod][tab][cpl]{thm:tableau-completeness}.
\end{prob}

\begin{cor}
\ollabel{cor:entailment-completeness}
If $\Gamma \Entails !A$ then $\Gamma \Proves !A$.
\end{cor}

\begin{cor}
\ollabel{cor:weak-completeness}
If $!A$ is true in all models, then $\Proves !A$.
\end{cor}

\end{document}
