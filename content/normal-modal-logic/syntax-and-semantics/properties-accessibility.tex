% Part: normal-modal-logic
% Chapter: syntax-and-semantics
% Section: properties-accessibility

\documentclass[../../../include/open-logic-section]{subfiles}

\begin{document}

\olfileid{mod}{syn}{acc}

\olsection{Properties of Accessibility Relations}

\begin{defn}
  We single out the following five potential properties of an
  accessibility relation:
 \begin{center}
    \begin{tabular}{| l || l |}
      \hline
      {\emph{$R$ is called \dots}} & {\emph{\dots if it satisfies:}} \\
      \hline \hline
      ``serial''   & $\forall u \exists v\, Ruv$; \\
      \hline
      ``reflexive'' & $\forall w \, Rww$; \\
      \hline
      ``symmetric'' & $\forall u\forall v(Ruv \lif Rvu)$; \\
      \hline
      ``transitive'' & $ \forall u \forall v \forall w (Ruv \land Rvw
      \lif Ruw)$; \\
      \hline 
      ``euclidean'' & $ \forall w \forall u \forall v (Rwu \land Rwv
      \lif Ruv)$.\\
      \hline
    \end{tabular}
  \end{center}
\end{defn}

\begin{thm}\ollabel{thm:soundschemas}
  Let $\mModel{M} = \tuple{W, R, V}$ be a model. Then:
 \begin{enumerate}
 \item If $R$ is serial then schema \Ax{D}, i.e., $\Box !A \lif
   \Diamond !A$, is true in $\mModel{M}$;
 \item If $R$ is reflexive then schema \Ax{T}, i.e., $\Box !A \lif
   !A$, is true in $\mModel{M}$;
 \item If $R$ is symmetric then schema \Ax{B}, i.e., $!A \lif
   \Box\Diamond !A$, is true in $\mModel{M}$;
 \item If $R$ is transitive then schema \Ax{4}, i.e., $\Box
   !A \lif \Box \Box !A$, is true in $\mModel{M}$;
 \item If $R$ is euclidean then schema \Ax{5}, i.e., $\Diamond
   !A \lif \Box \Diamond !A$, is true in $\mModel{M}$.
  \end{enumerate}
\end{thm}

\begin{proof}
  Here is the case for \Ax{B}: to show that the schema is true in a
  model we need to show that all of its instances are true all worlds
  in the model. So let $!A \lif \Box\Diamond !A$ be a given instance
  of \Ax{B}, and let $w \in W$ be an arbitrary world. Suppose the
  antecedent $!A$ is true at $w$, in order to show that $\Box \Diamond
  !A$ is true at $w$. So we need to show that $\Diamond !A$ is true at
  all $w'$ accessible from $w$. Now, for any $w'$ such that $Rww'$ we
  have, using the hypothesis of symmetry, that also $Rw'w$ (see
  \olref{fig:Bsymm}). Since $\mSat{M}{!A}[w]$, we have
  $\mSat{M}{\Diamond !A}[w']$. Since $w'$ was an arbitrary world such
  that $Rww'$, we have $\mSat{M}{\Box\Diamond!A}[w]$.
\end{proof}

\begin{prob}
  Complete the proof of \olref[mod][syn][acc]{thm:soundschemas}
\end{prob}

\begin{figure}
  \begin{center}
    \begin{tikzpicture}[node distance=2cm, auto, thick]
      \node (w1) at (0, 0) [label=90:$w$, label=below:$\formula{A}$]{$\bullet$}; 
      \node (w2) at (2, 0) [label=45:$w'$, label=below:{$\Diamond\formula{A}$}]{$\bullet$}; 
      \draw[->, bend left] (w1) to node {$R$} (w2); 
      \draw[->, bend left] (w2) to node {} (w1); 
      \draw [rounded corners] (-1.25,-1) -- ++(0,3)  -- ++(4.25,0) -- ++(0,-3) --  cycle;
      \path node at (-0.6,1.5) {$\mModel{M}$};
      \path node at (-0.6, 0) {$\Box\Diamond\formula{A}$};
    \end{tikzpicture}
  \end{center}
\caption{The argument from symmetry.}\ollabel{fig:Bsymm}
\end{figure}

Notice that the converse implications of \olref{thm:soundschemas} do
not hold: it's not true that if a model verifies a schema, then the
accessibility relation of that model has the corresponding property (a
counterexample is provided by \olref{ex:reflexive}).

\begin{ex}\ollabel{ex:reflexive}
  Let $\mModel{M} = \tuple{W, R, V}$ be a model such that $W = \{u, v
  \}$, where worlds $u$ and $v$ are related by $R$: i.e., both $Ruv$
  and $Rvu$.  Suppose that for all $p$: $u \in V(p) \Leftrightarrow v
  \in V(p)$. Then:
  \begin{enumerate}
  \item For all $!A$: $\mSat{M}{!A}[u]$ if and only if
    $\mSat{M}{!A}[v]$ (use induction on $!A$).
  \item Schema \Ax{T} is true in $\mModel{M}$.
  \end{enumerate}
  Since $\mModel{M}$ is not reflexive (it is, in fact,
  \emph{irreflexive}), the converse of \olref{thm:soundschemas}
  fails in the case of \Ax{T} (similar arguments can be given for
  some---though not all---the other schemas mentioned in
  \olref{thm:soundschemas}).
\end{ex}

\begin{prob}
  Prove the claims in \olref[mod][syn][acc]{ex:reflexive}.
\end{prob}

\end{document}
