% Part: normal-modal-logic
% Chapter: syntax-and-semantics
% Section: language-modal-logic

\documentclass[../../../include/open-logic-section]{subfiles}

\begin{document}

\olfileid{mod}{syn}{lan}

\olsection{The Language of Basic Modal Logic}

The basic language of modal logic contains a set $\Var$ of
!!{propositional variable}s $\Obj p_1$, $\Obj p_2$, \dots, the
familiar logical connectives $\lnot$ (``not''), $\land$ (``and''),
$\lor$ (``or''), $\lif$, (``if \dots then''), the symbols $\ltrue$
(the truth symbol) and $\lfalse$ (the falsity symbol), as well as the
two basic modalities $\Box$ and $\Diamond$.

\begin{defn}
\emph{!!^{formula}s} of the basic modal language are inductively
  defined as follows:
\begin{enumerate}
\item Every propositional variable $\Obj p_i$ is an (atomic) !!{formula}.
\item $\ltrue$ is an (atomic) !!{formula}
\item $\lfalse$ is an (atomic) !!{formula}.
\item If $!A$ is a formula, so is $\lnot !A$.
\item If $!A$ and $!B$ are formulas, so are $(!A \land !B)$, $(!A \lor
  !B)$, $(!A \lif !B)$, and $(!A \liff !B)$.
\item If $!A$ is a formula, so is $\Box !A$.
\item Nothing else is !!a{formula}.
\end{enumerate}
If a !!{formula}~$!A$ does not contain $\Box$, we say
it is \emph{modal-free}. 
\end{defn}

$\Diamond A$ abbreviates $\lnot \Box \lnot !A$. So for instance,
$\Diamond\Box p \lif \Diamond\Diamond p$ is short for
$\lnot\Box\lnot\Box p \lif \lnot\Box\lnot\lnot \Box\lnot p$.

\end{document}
