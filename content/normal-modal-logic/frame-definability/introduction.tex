% Part: normal-modal-logic
% Chapter: frame-correspondence
% Section: introduction

\documentclass[../../../include/open-logic-section]{subfiles}

\begin{document}

\olfileid{mod}{frd}{int}

\olsection{Introduction}

One question that interests modal logicians is the relationship
between the accessibility relation and the truth of certain
!!{formula}s in models with that accessibility relation. For instance,
suppose the accessibility relation is reflexive, i.e., for every $w
\in W$, $Rww$. In other words, every world is accessible from
itself. That means that when $\Box !A$ is true at a world~$w$, $w$
itself is among the accessible worlds at which~$!A$ must therefore be
true. So, if the accessibility relation~$R$ of~$\mModel{M}$ is
reflexive, then whatever world $w$ and formula $!A$ we take, $\Box !A
\lif !A$ will be true there (in other words, the schema $\Box p \lif
p$ and all its substitution instances are true in~$\mModel{M}$).

The converse, however, is false. It's not the case, e.g., that if
$\Box p \lif p$ is true in~$\mModel{M}$, then $R$ is reflexive. For we
can easily find a non-reflexive model~$\mModel{M}$ where $\Box p \lif
p$ is true at all worlds: take the model with a single world~$w$, not
accessible from itself, but with $w \in V(p)$. By picking the truth
value of $p$ suitably, we can make $\Box !A \lif !A$ true in a model
that is not reflexive.

The solution is to remove the variable assignment~$V$ from the
equation. If we require that $\Box p \lif p$ is true at all worlds
in~$\mModel{M}$, regardless of which worlds are in~$V(p)$, then it is
necessary that $R$ is reflexive. For in any non-reflexive model, there
will be at least one world~$w$ such that not $Rww$. If we set $V(p) =
W \setminus \{w\}$, then $p$ will be true at all worlds other
than~$w$, and so at all worlds accessible from~$w$ (since $w$ is
guaranteed not to be accessible from~$w$, and $w$ is the only world
where~$p$ is false). On the other hand, $p$ is false at $w$, so $\Box
p \lif p$ is false at~$w$.

This suggests that we should introduce a notation for model structures
without a valuation: we call these \emph{frames}. A frame $\mModel{F}$
is simply a pair $\tuple{W, R}$ consisting of a set of worlds with an
accessibility relation. Every model $\tuple{W, R, V}$ is then, as we
say, \emph{based on} the frame $\tuple{W, R}$. Conversely, a frame
determines the class of models based on it; and a class of frames
determines the class of models which are based on any frame in the
class. And we can define $\mSat{F}{!A}$, the notion of !!a{formula}
being \emph{valid} in a frame as: $\mSat{M}{!A}$ for all $\mModel{M}$
based on~$\mModel{F}$.

With this notation, we can establish correspondence relations between
!!{formula}s and classes of frames: e.g., $\mSat{F}{\Box p \lif p}$
if, and only if, $\mModel{F}$ is reflexive.

\end{document}
