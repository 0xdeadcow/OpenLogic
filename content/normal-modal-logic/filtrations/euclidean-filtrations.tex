% Part: normal-modal-logic
% Chapter: filtrations
% Section: euclidean-filtrations

\documentclass[../../../include/open-logic-section]{subfiles}

\begin{document}

\olfileid{nml}{fil}{euc}

\olsection{Filtrations of Euclidean Models}

The approach of \olref[acc]{sec} does not work in the case of models
that are euclidean or serial and euclidean. Consider the model at the
top of \olref{fig:ser-eucl}, which is both euclidean and serial. Let
$\Gamma = \{p, \Box p \}$. When taking a filtration through $\Gamma$,
then $[w_1] = [w_3]$ since $w_1$ and $w_3$ are the only worlds that
agree on~$\Gamma$. Any filtration will also have the arrow inherited
from $\mModel{M}$, as depicted in \olref{fig:ser-eucl2}. That model
isn't euclidean. Moreover, we cannot add arrows to that model in order
to make it euclidean. We would have to add double arrows between
$[w_2]$ and $[w_4]$, and then also between $w_2$ and~$w_5$. But $\Box
p$ is supposed to be true at~$w_2$, while $p$ is false at~$w_5$.

\begin{figure}[htpb]
  \centering
  \begin{tikzpicture}[modal]
    \node[world] (w1)
          [label={left: \mFalse{p}},
            label={below: $\mSat{{}}{\Box p}$}] {$w_1$}; 
    \node[world] (w2)
          [label={right: \mTrue{p}},
            label = {below: $\mSat{{}}{\Box p}$},
            right=of w1] {$w_2$}; 
    \draw[->] (w1) to (w2);
    \node[world] (w3)
          [label={left: \mFalse{p}},
            label={below: $\mSat{{}}{\Box p}$},
            below=of w1] {$w_3$}; 
    \node[world] (w4)
          [label={right:\mTrue{p}},
            label={below: $\mSat/{{}}{\Box p}$},
            right=of w3] {$w_4$};
    \draw[->] (w3) to (w4);
    \draw[reflexive above] (w4) to (w4);     
    \node[world] (w5)
          [label={right: \mFalse{p}},
            label=below:$\mSat/{{}}{\Box p}$,
            right=of w4] {$w_5$}; 
    \draw[->, bend left] (w4) to (w5);
    \draw[->, bend left] (w5) to (w4);
    \draw[reflexive above] (w5) to (w5);     
  \end{tikzpicture}
  \caption{A serial and euclidean model.}
  \ollabel{fig:ser-eucl}
\end{figure}

\begin{figure}[ht]
  \centering
  \begin{tikzpicture}[modal,every text node part/.style={align=center}]
    \node[world] (w1)
          [label={left: \mFalse{p}},
            label={right:$[w_1]=[w_3]$},
            label={below: $\mSat{{}}{\Box p}$}] {$[w_1]$}; 
    \node[world] (w2)
         [label={right: \mTrue{p}},
           label = {below: $\mSat{{}}{\Box p}$},
           right=of w1, yshift=1.5cm] {$[w_2]$}; 
    \draw[->] (w1) to (w2);
    \node[world] (w4)
         [label={right:\mTrue{p}},
           label={below: $\mSat/{{}}{\Box p}$},
           right=of w1,yshift=-1.5cm] {$[w_4]$};
    \draw[->] (w1) to (w4);
    \draw[reflexive above] (w4) to (w4);     
    \node[world] (w5)
         [label={right: \mFalse{p}},
             label=below:$\mSat/{{}}{\Box p}$,
             right=of w4] {$[w_5]$}; 
    \draw[->, bend left] (w4) to (w5);
    \draw[->, bend left] (w5) to (w4);
    \draw[reflexive above] (w5) to (w5);     
  \end{tikzpicture}
  \caption{The filtration of the model in \olref{fig:ser-eucl}.}
  \ollabel{fig:ser-eucl2}
\end{figure}

In particular, to obtain a euclidean filtration it is not enough to
consider filtrations through arbitrary $\Gamma$'s closed under
sub-!!{formula}s. Instead we need to consider sets $\Gamma$ that are
\emph{modally closed} (see \olref[pre]{defn:modallyclosed}). Such sets
of sentences are infinite, and therefore do not immediately yield a
finite model property or the decidability of the corresponding system.

\begin{thm}\ollabel{thm:modal-closed-filt}
  Let $\Gamma$ be modally closed, $\mModel{M}=\tuple{W,R,V}$, and
  $\mModel{M^*} = \tuple{W^*,R^*,V^*}$ be a coarsest filtration of
  $\mModel{M}$.
  \begin{enumerate}
  \item If $\mModel{M}$ is symmetric, so is $\mModel{M^*}$.
  \item If $\mModel{M}$ is transitive, so is $\mModel{M^*}$.
  \item If $\mModel{M}$ is euclidean, so is $\mModel{M^*}$.
  \end{enumerate}
\end{thm}

\begin{proof}
\begin{enumerate}
  \item If $\mModel{M^*}$ is a coarsest filtration, then by definition
    $R^*[u][v]$ holds if and only if $C_1(u,v)$. For transitivity,
    suppose $C_1(u,v)$ and $C_1(v,w)$; we have to show
    $C_1(u,w)$. \iftag{prvBox}{Suppose $\mSat{M}{\Box !A}[u]$; then
      $\mSat{M}{\Box\Box!A}[u]$ since \Ax{4} is valid in all
      transitive models; since $\Box\Box!A \in \Gamma$ by closure,
      also by $C_1(u,v)$, $\mSat{M}{\Box!A}[v]$ and by $C_1(v,w)$,
      also $\mSat{M}{!A}[w]$. }{}\iftag{prvDiamond}{Suppose
      $\mSat{M}{!A}[w]$; then $\mSat{M}{\Diamond !A}[v]$ by
      $C_1(v,w)$, since $\Diamond !A \in \Gamma$ by modal closure. By
      $C_1(u,v)$, we get $\mSat{M}{\Diamond\Diamond !A}[u]$ since
      $\Diamond\Diamond!A \in \Gamma$ by modal closure. Since
      $\Ax{4}_\Diamond$ is valid in all transitive models,
      $\mSat{M}{\Diamond!A}[u]$.}{}
  \item Exercise. Use the fact that both \Ax{5} and $\Ax{5_\Diamond}$
    are valid in all euclidean models.
  \item Exercise. Use the fact that \Ax{B} and $\Ax{B_\Diamond}$ are
    valid in all symmetric models.
\end{enumerate}
\end{proof}

\begin{prob}
  Complete the proof of \olref[nml][fil][euc]{thm:modal-closed-filt}.
\end{prob}

\end{document}
