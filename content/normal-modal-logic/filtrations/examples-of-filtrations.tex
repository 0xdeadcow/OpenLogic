% Part: normal-modal-logic
% Chapter: filtrations
% Section: examples-of-filtrations

\documentclass[../../../include/open-logic-section]{subfiles}

\begin{document}

\olfileid{mod}{fil}{exf}

\olsection{Examples of Filtrations}

We have not yet shown that there are any filtrations. But indeed, for
any model $\mModel{M}$, there are many filtrations of $\mModel{M}$
through $\Gamma$. We identify two, in particular: the finest and
coarsest filtrations. Filtrations of the same models will differ in
their accessibility relation (as \olref[fil]{defn:filtration} stipulates
directly what $W^*$ and~$V^*$ should be). The finest filtration
will have as few related worlds as possible, whereas the coarsest will
have as many as possible.

\begin{defn}
  Where $\Gamma$ is closed under subformulas, the \emph{finest}
  filtration $\mModel{M^*}$ of a model $\mModel{M}$ is defined by putting:
  \[
  R^*[u][v] \quad \text{if and only if} \quad \exists u'\in [u] \;
  \exists v' \in [v] : Ru'v'.
  \]
\end{defn}

\begin{prop}\ollabel{prop:finest}
  The finest filtration $\mModel{M^*}$ is indeed a filtration. 
\end{prop}

\begin{proof}
  We need to check that $R^*$, so defined, satisfies
  \olref[fil]{defn:filtration}\olref[fil]{defn:filtration-R}. We check
  the three conditions in turn.

  If $Ruv$ then since $u \in [u]$ and $v \in [v]$, also $R^*[u][v]$,
  so \olref[fil]{defn:filtration-R1} is satisfied.
  
  \iftag{prvBox}{\iftag{probBox}{We leave the verification of
      \olref[fil]{defn:filtration-R2} as an exercise.}{For
        \olref[fil]{defn:filtration-R2}, suppose $\Box!A \in \Gamma$,
        $R^*[u][v]$, and $\mSat{M}{\Box!A}[u]$. By definition of
        $R^*$, there are $u' \equiv u$ and $v' \equiv v$ such that
        $Ru'v'$. Since $u$ and $u'$ agree on $\Gamma$, also
        $\mSat{M}{\Box!A}[u']$, so that $\mSat{M}{!A}[v']$. By closure
        of $\Gamma$ under sub-!!{formula}s, $v$ and $v'$ agree on
        $!A$, so $\mSat{M}{!A}[v]$, as desired.}}{}
  
  \iftag{prvDiamond}{\iftag{probDiamond}{We leave the verification of
      \olref[fil]{defn:filtration-R3} as an exercise.}{To verify
        \olref[fil]{defn:filtration-R3}, suppose $\Diamond!A \in
        \Gamma$, $R^*[u][v]$, and $\mSat{M}{!A}[v]$.  By definition of
        $R^*$, there are $u' \equiv u$ and $v' \equiv v$ such that
        $Ru'v'$. Since $v$ and $v'$ agree on~$\Gamma$, and $\Gamma$ is
        closed under sub-!!{formula}s, also $\mSat{M}{!A}[v']$, so
        that $\mSat{M}{\Diamond !A}[u']$.  Since $u$ and $u'$ also
        agree on $\Gamma$, $\mSat{M}{\Diamond !A}[u]$.}}{}
\end{proof}

\begin{probtag}{probBox,probDiamond}
  Complete the proof of \olref[mod][fil][exf]{prop:finest}.
\end{probtag}

\begin{defn}
  Where $\Gamma$ is closed under subformulas, the \emph{coarsest}
  filtration~$\mModel{M^*}$ of a model~$\mModel{M}$ is defined by
  putting $R^*[u][v]$ if and only if
  \iftag{notprvBox,notprvDiamond}{the following condition is
    met:}{\emph{both} of the following conditions are met:}
  \begin{tagenumerate}{prvBox,prvDiamond}
  \tagitem{prvBox}{\ollabel{defn:coarsest-Box}If $\Box!A \in \Gamma$ and $\mSat{M}{\Box!A}[u]$ then
    $\mSat{M}{!A}[v]$\iftag{prvDiamond}{;}{.}}{}
  \tagitem{prvDiamond}{\ollabel{defn:coarsest-Diamond}If $\Diamond!A \in \Gamma$ and $\mSat{M}{!A}[v]$ then
    $\mSat{M}{\Diamond!A}[u]$.}{}
  \end{tagenumerate}
\end{defn}

\begin{prop}
  The coarsest filtration $\mModel{M^*}$ is indeed a filtration. 
\end{prop}

\begin{proof}
  Given the definition of $R^*$, the only condition that is left to
  verify is the implication from $Ruv$ to $R^*[u][v]$. So assume
  $Ruv$. \iftag{prvBox}{Suppose $\Box!A \in \Gamma$ and
    $\mSat{M}{\Box!A}[u]$; then obviously
    $\mSat{M}{!A}[v]$\iftag{prvDiamond}{, and
      \olref{defn:coarsest-Box} is
      satisfied. }{.}}{}\iftag{prvDiamond}{Suppose $\Diamond!A \in
    \Gamma$ and $\mSat{M}{!A}[v]$. Then $\mSat{M}{\Diamond!A}[u]$
    since~$Ruv$\iftag{prvBox}{, and \olref{defn:coarsest-Diamond} is
      satisfied}{}.}{}
\end{proof}

\begin{ex}
  Let $W = \Int^+$, $Rnm$ iff $m = n + 1$, and $V(p) = \Setabs{2n}{n \in
    \Nat}$. The model $\mModel{M} = \tuple{W, R, V}$ is depicted in
  \olref{fig:ex-filtration}. The worlds are $1$, $2$, etc.; each
  world can access exactly one other world---its successor, and $p$ is
  true at all and only the even numbers.

  \begin{figure}
    \centering
    \begin{tikzpicture}[modal]
      \node (1) [label=below:\mFalse{p}] {$1$};
      \node (2) [label=below:\mTrue{p}, right=of 1] {$2$};
      \node (3) [label=below:\mFalse{p}, right=of 2] {$3$};
      \node (4) [label=below:\mTrue{p}, right=of 3] {$4$};
      \node[phantom] (5) [right=of 4] {};
      \draw[->] (1) to (2);
      \draw[->] (2) to (3);
      \draw[->] (3) to (4);
      \draw[dotted] (4) to (5);
    \end{tikzpicture}

    \begin{tikzpicture}[modal]
      \node (1) [label=below:\mFalse{p}] {$[1]$};
      \node (2) [label=below:\mTrue{p}, right=of 1] {$[2]$};
      \draw[->,bend left] (1) to (2);
      \draw[->,bend left] (2) to (1);
      
      \node (3) [label=below:\mFalse{p},right=of 2] {$[1]$};
      \node (4) [label=below:\mTrue{p}, right=of 3] {$[2]$};
      \draw[->,bend left] (3) to (4);
      \draw[->,bend left] (4) to (3);
      \draw[->,reflexive above] (4) to (4);
    \end{tikzpicture}
    \caption{An infinite model and its filtrations.}
    \ollabel{fig:ex-filtration}
  \end{figure}
  
  Now let $\Gamma$ be the set of sub-!!{formula}s of~$\Box p \lif p$,
  i.e., $\{p, \Box p, \Box p \lif p\}$. $p$ is true at all and only
  the even numbers, $\Box p$ is true at all and only the odd numbers,
  so $\Box p \lif p$ is true at all and only the even numbers. In
  other words, every odd number makes $\Box p$ true and $p$ and $\Box
  p \lif p$ false; every even number makes $p$ and $\Box p \lif p$
  true, but $\Box p$ false. So $W^* = \{ [1], [2] \}$, where $[1] =
  \{1, 3, 5, \dots\}$ and $[2] = \{2, 4, 6, \dots\}$. Since $2 \in
  V(p)$, $[2] \in V^*(p)$; since $1 \notin V(p)$, $[1] \notin
  V^*(p)$. So $V^*(p) = \{[2]\}$.

  Any filtration based on $W^*$ must have an accessibility relation
  that includes $\tuple{[1], [2]}, \tuple{[2],[1]}$: since $R12$, we
  must have $R^*[1][2]$ by
  \olref[fil]{defn:filtration}\olref[fil]{defn:filtration-R1}, and since $R23$ we
  must have $R^*[2][3]$, and $[3]=[1]$. It cannot include
  $\tuple{[1],[1]}$: if it did, we'd have $R^*[1][1]$, $\mSat{M}{\Box
    p}[1]$ but $\mSat{M}{p}[1]$, contradicting
  \olref[fil]{defn:filtration-R1}. Nothing requires or rules out that
  $R^*[2][2]$. So, there are two possible filtrations of~$\mModel{M}$,
  corresponding to the two accessibility relations
  \[
  \{\tuple{[1],[2]}, \tuple{[2],[1]}\} \text{ and }
  \{\tuple{[1],[2]}, \tuple{[2],[1]}, \tuple{[2],[2]}\}.
  \]
  In either case, $p$ and $\Box p \lif p$ are false and $\Box p$ is
  true at~$[1]$; $p$ and $\Box p \lif p$ are true and $\Box p$ is
  false at~$[2]$.
\end{ex}


\begin{prob}
  Consider the following model $\mModel{M} = \tuple{W, R, V}$ where
  $W = \Bin^* \setminus \{1, \emptyseq\}$, the set of sequences of
  $0$s and~$1$s starting with~$0$, with $R\sigma\sigma'$ iff $\sigma'
  = \sigma 0$ or $\sigma' = \sigma 1$, and $V(p) = \Setabs{\sigma
    0}{\sigma \in \Bin^*}$ and $V(q) = \Setabs{\sigma 1}{\sigma \in
      \Bin^* \setminus \{1\}}$. Here's a picture:
  \begin{center}
    \begin{tikzpicture}[modal,
        node distance=2cm
      ]

      \node (0) [label={below:\mTrue{p}\\\mFalse{q}},font=\tiny] {$0$};
      \node (00) [label={below:\mTrue{p}\\\mFalse{q}},
        above right=of 0,font=\tiny] {$00$};
      \node (000) [label={below:\mTrue{p}\\\mFalse{q}},
        right=of 00, yshift=.8cm, font=\tiny] {$000$};
      \node[phantom] (0000) [right=of 000, yshift=.5cm] {};
      \node[phantom] (0001) [right=of 000, yshift=-.5cm] {};
      \node (001) [label={below:\mFalse{p}\\\mTrue{q}},
        right=of 00, yshift=-.8cm, font=\tiny] {$001$};
      \node[phantom](0010) [right=of 001, yshift=.5cm] {};
      \node[phantom](0011) [right=of 001, yshift=-.5cm] {};
      \node[world] (01) [label={below:\mFalse{p}\\\mTrue{q}},
        below right=of 0,font=\tiny] {$01$};
      \node[world] (010) [label={below:\mTrue{p}\\\mFalse{q}},
        right=of 01, yshift=.8cm,font=\tiny] {$010$};
      \node[phantom] (0100) [right=of 010, yshift=.5cm] {};
      \node[phantom] (0101) [right=of 010, yshift=-.5cm] {};
      \node (011) [label={below:\mFalse{p}\\\mTrue{q}},
        right=of 01, yshift=-.8cm,font=\tiny] {$011$};
      \node[phantom] (0110) [right=of 011, yshift=.5cm] {};
      \node[phantom] (0111) [right=of 011, yshift=-.5cm] {};

      \draw[->] (0) to (00);
      \draw[->] (0) to (01);
      \draw[->] (00) to (000);
      \draw[dotted] (000) to (0000);
      \draw[dotted] (000) to (0001);
      \draw[->] (00) to (001);
      \draw[dotted] (001) to (0010);
      \draw[dotted] (001) to (0011);
      \draw[->] (01) to (010);
      \draw[dotted] (010) to (0100);
      \draw[dotted] (010) to (0101);
      \draw[->] (01) to (011);
      \draw[dotted] (011) to (0110);
      \draw[dotted] (011) to (0111);
    \end{tikzpicture}
  \end{center}
  We have $\mSat/{M}{\Box(p \lor q) \lif (\Box p \lor \Box q)}[w]$ for
  every~$w$.

  Let $\Gamma$ be the set of sub-!!{formula}s of~$\Box(p \lor q) \lif
  (\Box p \lor \Box q)$. What are $W^*$ an $V^*$?  What is the
  accessibility relation of the finest filtration of~$\mModel{M}$? Of the
  coarsest?
\end{prob}

\end{document}
