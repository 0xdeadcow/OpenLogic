% Part: normal-modal-logic
% Chapter: sequent-calculus
% Section: introduction

\documentclass[../../../include/open-logic-section]{subfiles}

\begin{document}

\olfileid{nml}{seq}{int}

\olsection{Introduction}

The sequent calculus for propositional logic can be extended by
additional rules that deal
with~\iftag{prvBox}{$\Box$\iftag{prvDiamond}{ and~}{}}{}%
\iftag{prvDiamond}{$\Diamond$}{}. For instance, for~$\Log{K}$, we
have \Log{LK} plus:
\[
  \iftag{prvBox}{\iftag{prvDiamond}{% <> and [] primitive
    \Axiom$\Gamma \fCenter \Delta, !A$
    \RightLabel{$\Box$}
    \UnaryInf$\Box\Gamma \fCenter \Diamond\Delta, \Box!A$
    \DisplayProof  
    \qquad
    \Axiom$!A, \Gamma \fCenter \Delta$
    \RightLabel{$\Diamond$}
    \UnaryInf$\Diamond!A, \Box\Gamma \fCenter \Diamond\Delta$
    \DisplayProof
}{% only Box primitive
\Axiom$\Gamma \fCenter !A$
\RightLabel{$\Box$}
\UnaryInf$\Box\Gamma \fCenter \Box!A$
\DisplayProof
}}{% only <> primitive
  \Axiom$!A \fCenter \Delta$
  \RightLabel{$\Diamond$}
  \UnaryInf$\Diamond!A \fCenter \Diamond\Delta$
  \DisplayProof
}
\]
For extensions of~$\Log{K}$, additional rules have to be added as
well.

Not every modal logic has such a sequent calculus.  Even $\Log{S5}$,
which is semantically simple (it can be defined without using
accessibility relations at all) is not known to have a sequent
calculus that results from~$\Log{LK}$ which is complete without the
rule~\Cut. However, it has a cut-free complete \emph{hypersequent}
calculus.

\end{document}
