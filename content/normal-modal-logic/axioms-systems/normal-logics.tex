% Part: normal-modal-logic
% Chapter: axioms-systems
% Section: normal-logics

\documentclass[../../../include/open-logic-section]{subfiles}

\begin{document}

\olfileid{mod}{prf}{nor}

\olsection{Normal Modal Logics}

Not every set of modal !!{formula}s can easily be characterized as
those !!{formula}s derivable from a set of axioms. We want modal
logics to be well-behaved. First of all, everything we can !!{derive}
in classical propositional logic should still be !!{derivable}, of
course taking into account that the !!{formula}s may now contain also
\iftag{prvBox}{$\Box$\iftag{prvDiamond}{
    and~}{}}{}\iftag{prvDiamond}{$\Diamond$}{}. To this end, we
require that a modal logic contain all tautological instances and be
closed under modus ponens.

\begin{defn}
  A \emph{modal logic} is a set~$\Sigma$ of modal !!{formula}s which
  \begin{enumerate}
  \item contains all tautologies, and
  \item is closed under substitution, i.e., if $!A \in \Sigma$, and
    $!D_1$, \dots, $!D_n$ are !!{formula}s, then
    \[
    \SSubst{!A}{\subst{!D_1}{p_1}, \dots, \subst{!D_n}{p_n}} \in \Sigma,
    \]
    \item is closed under \emph{modus ponens}, i.e., if $!A$ and $!A
      \lif !B \in \Sigma$, then $!B \in \Sigma$.
  \end{enumerate}
\end{defn}

In order to use the relational semantics for modal logics, we also
have to require that all !!{formula}s valid in all modal models are
included. It turns out that this requirement is met as soon as all
instances of \Ax{K}\iftag{prvDiamond}{ and~\Dual{}}{} are
!!{derivable}, and whenever !!a{formula}~$!A$ is !!{derivable}, so
is~$\Box !A$. A modal logic that satisfies these conditions is called
\emph{normal}. (Of course, there are also non-normal modal logics, but
the usual relational models are not adequate for them.)

\begin{defn}
  A modal logic $\Sigma$ is \emph{normal} if it contains
  \begin{align*}
    \tag{\Ax{K}} & \Box(p \lif q) \lif (\Box p \lif \Box q),
    \iftag{prvDiamond}{\\
      \tag{\Dual} & \Diamond p \liff \lnot\Box\lnot p}{}
  \end{align*}
  and is closed under \emph{necessitation}, i.e., if $!A \in
  \Sigma$, then $\Box !A \in \Sigma$.
\end{defn}

Observe that while tautological implication is ``fine-grained'' enough
to preserve \emph{truth at a world}, the rule \Nec{} only preserves
\emph{truth in a model} (and hence also validity in a frame or in a
class of frames).

\begin{prop}\ollabel{prop:rk}
  Every normal modal logic is closed under rule \RK,
  \begin{prooftree}
    \AxiomC{$!A_1 \lif (!A_2 \lif \cdots (!A_{n-1} \lif !A_n)\cdots)$}
    \RightLabel{\RK}
    \UnaryInfC{$\Box!A_1 \lif (\Box!A_2 \lif \cdots (\Box!A_{n-1}
      \lif \Box!A_n)\cdots).$}
  \end{prooftree}
\end{prop}

\begin{proof}
  By induction on~$n$: If $n = 1$, then the rule is just \Nec, and
  every normal modal logic is closed under \Nec.

  Now suppose the result holds for $n-1$; we show it holds for~$n$.

  Assume
  \begin{align*}
  & !A_1 \lif (!A_2 \lif \cdots (!A_{n-1} \lif !A_n)\cdots) \in \Sigma
  \intertext{By the induction hypothesis, we have}
  & \Box!A_1 \lif (\Box!A_2 \lif \cdots \Box(!A_{n-1} \lif !A_n)\cdots)
  \in \Sigma
  \intertext{Since $\Sigma$ is a normal modal logic, it contains all
    instances of~$\Ax{K}$, in particular}
  & \Box(!A_{n-1} \lif !A_n) \lif (\Box!A_{n-1} \lif \Box!A_n) \in \Sigma
  \intertext{Using modus ponens and suitable tautological instances we get}
  & \Box!A_1 \lif (\Box!A_2 \lif \cdots (\Box!A_{n-1}
  \lif \Box!A_n)\cdots) \in \Sigma. \qedhere
  \end{align*}
\end{proof}

\begin{prop}\ollabel{prop:notDiamondBot}
  Every normal modal logic $\Sigma$ contains~$\lnot\Diamond\lfalse$.
\end{prop}

\begin{prob}
  Prove \olref[mod][prf][nor]{prop:notDiamondBot}.
\end{prob}

\begin{prop}
  Let $!A_1$, \dots, $!A_n$ be !!{formula}s. Then there is a
  smallest modal logic $\Sigma$ containing all instances of
  $!A_1$, \dots, $!A_n$.
\end{prop}

\begin{proof}
  Given $!A_1$, \dots, $!A_n$, define $\Sigma$ as the
  intersection of all normal modal logics containing all instances of
  $!A_1$, \dots, $!A_n$. The intersection is non-empty as
  $\Frm[L]$, the set of all !!{formula}s, is such a modal
  logic.
\end{proof}

\begin{defn}
The smallest normal modal logic containing $!A_1$, \dots, $!A_n$ is
called a \emph{modal system} and denoted by $\Log{K} !A_1 \dots
!A_n$. The smallest normal modal logic is denoted by~\Log{K}.
\end{defn}

\end{document}
