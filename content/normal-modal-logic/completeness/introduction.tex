% Part: normal-modal-logic
% Chapter: completeness
% Section: introduction

\documentclass[../../../include/open-logic-section]{subfiles}

\begin{document}

\olfileid{nml}{com}{int}

\olsection{Introduction}

If $\Sigma$ is a modal system, then the soundness theorem establishes
that if $\Sigma \Proves !A$, then $!A$ is valid in any class
$\mClass{C}$ of models in which all instances of all !!{formula}s
in~$\Sigma$ are valid. In particular that means that if $\Log{K}
\Proves !A$ then $!A$ is true in all models; if $\Log{KT} \Proves !A$
then $!A$ is true in all reflexive models; if $\Log{KD} \Proves !A$
then $!A$ is true in all serial models, etc.

Completeness is the converse of soundness: that \Log{K} is complete
means that if !!a{formula}~$!A$ is valid, $\Proves !A$, for instance.
Proving completeness is a lot harder to do than proving soundness.  It
is useful, first, to consider the contrapositive: \Log{K} is complete
iff whenever $\Proves/ !A$, there is a countermodel, i.e., a
model~$\mModel{M}$ such that $\mSat/{M}{!A}$. Equivalently (negating
$!A$), we could prove that whenever $\Proves/ \lnot !A$, there is a
model of~$!A$.  In the construction of such a model, we can use
information contained in $!A$.  When we find models for specific
!!{formula}s we often do the same: e.g., if we want to find a
countermodel to $p \lif \Box q$, we know that it has to contain a
world where $p$ is true and $\Box q$ is false. And a world where $\Box
q$ is false means there has to be a world accessible from it where $q$
is false. And that's all we need to know: which worlds make the
!!{propositional variable}s true, and which worlds are accessible from
which worlds.

In the case of proving completeness, however, we don't have a specific
!!{formula}~$!A$ for which we are constructing a model. We want to
establish that a model exists for every $!A$ such that
$\Proves/[\Sigma] \lnot !A$. This is a minimal requirement, since
\emph{if} $\Proves[\Sigma] \lnot !A$, by soundness, there is no model
for~$!A$ (in which $\Sigma$ is true).  Now note that $\Proves/[\Sigma]
\lnot !A$ iff $!A$ is $\Sigma$-consistent.  (Recall that $\Sigma
\Proves/[\Sigma] \lnot !A$ and $!A \Proves/[\Sigma] \lfalse$ are
equivalent.)  So our task is to construct a model for every
$\Sigma$-consistent !!{formula}.

The trick we'll use is to find a $\Sigma$-consistent set of
!!{formula}s that contains~$!A$, but also other formulas which tell us
what the world that makes $!A$ true has to look like. Such sets are
\emph{complete} $\Sigma$-consistent sets. It's not enough to
construct a model with a single world to make~$!A$ true, it will have
to contain multiple worlds and an accessibility relation. The complete
$\Sigma$-consistent set containing~$!A$ will also contain other
!!{formula}s of the form $\Box !B$ and~$\Diamond !C$. In all
accessible worlds, $!B$ has to be true; in at least one, $!C$ has to
be true. In order to accomplish this, we'll simply take \emph{all}
possible complete $\Sigma$-consistent sets as the basis for the set of
worlds. A tricky part will be to figure out when a complete
$\Sigma$-consistent set should count as being accessible from another
in our model.

We'll show that in the model so defined, $!A$ is true at a
world---which is also a complete $\Sigma$-consistent set---iff $!A$ is
!!a{element} of that set.  If $!A$ is $\Sigma$-consistent, it will be
!!a{element} of at least one complete $\Sigma$-consistent set (a fact
we'll prove), and so there will be a world where $!A$~is true. So we
will have a single model where every $\Sigma$-consistent
!!{formula}~$!A$ is true at some world.  This single model is the
\emph{canonical} model for~$\Sigma$.

\end{document}
