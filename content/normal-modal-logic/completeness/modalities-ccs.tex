% Part: normal-modal-logic
% Chapter: completeness
% Section: modalities-ccs

\documentclass[../../../include/open-logic-section]{subfiles}

\begin{document}

\olfileid{mod}{com}{mod}

\olsection{Modalities and Complete Consistent Sets}
  
When we construct a model whose set of worlds is given by the complete
consistent sets in some normal modal logic $\Sigma$, we will also need
to define an accessibility relation between such ``worlds.'' The next
few lemmas give us the tools to do so. As noted, $\Sigma$ will be a
normal modal logic throughout.

\begin{defn}
  If $\Gamma$ is a set of !!{formula}s, let
  \begin{align*}
    \Box\Gamma & = \Setabs{\Box !B}{!B \in \Gamma}\\
    \Diamond\Gamma & = \Setabs{\Diamond !B}{!B \in \Gamma}\\
  \end{align*}
\end{defn}

\begin{lem}\ollabel{lem:Gamma-proves1}
  If $\Gamma \Proves[\Sigma] !A$ then $\Box\Gamma \Proves[\Sigma] \Box
  !A$.
\end{lem}

\begin{proof}
  If $\Gamma \Proves[\Sigma] !A$ then there are $!B_1$, \dots, $!B_k
  \in \Gamma$ such that $\Sigma \Proves !B_1 \lif (!B_2 \lif \cdots
  (!B_n \lif !A)\cdots)$. Since $\Sigma$ is normal, by rule \RK{},
  $\Sigma \Proves \Box!B_1 \lif (\Box!B_2 \lif \cdots (\Box!B_n \lif
  \Box!A)\cdots)$, where obviously $\Box!B_1$, \dots, $\Box!B_k \in
  \Box\Gamma$. Hence, by definition, $\Box\Gamma \Proves[\Sigma] \Box
  !A$.
\end{proof}

\begin{lem}\ollabel{lem:Gamma-proves2}
  If $\Setabs{!B}{\Box !B \in \Gamma} \Proves[\Sigma] !A$ then
  $\Gamma \Proves[\Sigma] \Box!A$.
\end{lem}

\begin{proof}
  Let $\Delta = \Setabs{!B}{\Box !B \in \Gamma}$, so that $\Delta
  \Proves[\Sigma] !A$; then by \olref{lem:Gamma-proves1},
  $\Box\Delta \Proves \Box!A$. But obviously
  $\Box\Delta \subseteq \Gamma$, so also $\Gamma
  \Proves[\Sigma] \Box!A$ by Monotony.
\end{proof}

\begin{thm}\ollabel{thm:box-phiGamma}
  If $\Gamma$ is complete $\Sigma$-consistent, then $\Box !A \in
  \Gamma$ if and only if for every complete $\Sigma$-consistent
  $\Delta$ such that $\Setabs{!B}{\Box !B \in \Gamma} \subseteq
  \Delta$, it holds that $!A \in \Delta$.
\end{thm}

\begin{proof}
  The left-to-right half of the theorem is obvious. For the converse,
  suppose $\Box!A \notin \Gamma$. Since $\Gamma$ is deductively
  closed, $\Gamma \Proves/[\Sigma] \Box !A$, and by
  \olref{lem:Gamma-proves2}, $\Setabs{!B}{\Box!B \in \Gamma}
  \Proves/[\Sigma] !A$. By
  \olref[prf][con]{prop:consistencyfacts}\olref[prf][con]{prop:consistencyfacts-b},
  $\Setabs{!B}{\Box!B \in \Gamma} \cup \{ \lnot!A \}$ is
  $\Sigma$-consistent, so that by Lindenbaum's Lemma there is a
  complete $\Sigma$-consistent set~$\Delta$ such that
  $\Setabs{!B}{\Box!B \in \Gamma} \cup \{ \lnot!A \} \subseteq
  \Delta$. By consistency, $!A \notin \Delta$, and the theorem is
  proved.
\end{proof}

\begin{lem}\ollabel{lem:Gamma-proves3}
  Suppose $\Gamma$ and $\Delta$ are complete
  $\Sigma$-consistent. Then: $\Setabs{!A}{\Box!A \in \Gamma}
    \subseteq \Delta$ if and only if $\Diamond\Delta \subseteq \Gamma$.
\end{lem}

\begin{proof}
  ``Only if'' direction: Assume $\Setabs{!A}{\Box!A \in \Gamma}
  \subseteq \Delta$ and suppose $!A \in \Delta$.  In order to
  show $\Diamond!A \in \Gamma$ it suffices to show $\Box\lnot!A
  \notin \Gamma$ for then by maximality $\lnot\Box\lnot !A \in
  \Gamma$. Now, if $\Box\lnot!A \in \Gamma$ then by hypothesis
  $\lnot!A \in \Delta$, against the consistency of $\Delta$
  (since $!A \in \Delta$). Hence  $\Box\lnot!A \notin
  \Gamma$, as required.

  ``If'' direction: Assume $\Diamond\Delta
  \subseteq \Gamma$. We argue contrapositively: suppose $!A
  \notin \Delta$ in order to show $\Box!A \notin \Gamma$. If
  $!A \notin \Delta$ then by maximality $\lnot!A \in \Delta$
  and so by hypothesis $\Diamond\lnot!A \in \Gamma$. But in a
  normal modal logic $\Diamond\lnot!A$ is equivalent to
  $\lnot\Box !A$, and if the latter is in $\Gamma$ by consistency
  $\Box!A \notin\Gamma$, as required.
\end{proof}

\begin{cor}
  If $\Gamma$ is complete $\Sigma$-consistent, then $\Diamond !A \in
  \Gamma$ if and only if for some complete $\Sigma$-consistent
  $\Delta$ such that $\Diamond\Delta \subseteq
  \Gamma$, it holds that $!A \in \Delta$.
\end{cor}

\begin{proof}
Suppose $\Gamma$ is complete $\Sigma$-consistent, and argue as
follows:
\begin{align*}
  \Diamond!A \in \Gamma \Leftrightarrow &
  \lnot\Box\lnot !A \in \Gamma, && \text{re-writing}; \\
  \Leftrightarrow & \Box\lnot!A \notin \Gamma, && \Gamma
  \text{ is $\Sigma$-consistent}; \\ 
  \Leftrightarrow & \exists\Delta \left[ \Delta \text{ is
      complete $\Sigma$-consistent} {} \land {}\right.\\
    & \qquad \left.\Setabs{ !B }{ \Box !B \in \Gamma }
  \subseteq \Delta  \land \lnot!A \notin\Delta\right], &&
   \text{\olref{thm:box-phiGamma}}; \\
 \Leftrightarrow & \exists\Delta \left[ \Delta \text{ is
     complete $\Sigma$-consistent} \land {}\right.\\
   & \qquad \left.\Diamond\Delta \subseteq \Gamma \land  \lnot!A \notin\Delta\right], && 
   \text{\olref{lem:Gamma-proves3}}; \\
\Leftrightarrow & \exists\Delta \left[ \Delta \text{ is
    complete $\Sigma$-consistent} {} \land {}\right.\\
  & \qquad \left.\Diamond \Delta \subseteq \Gamma  \land !A \in\Delta\right], && \Delta \text{ is complete}. \qedhere  
\end{align*}
\end{proof}

\end{document}
