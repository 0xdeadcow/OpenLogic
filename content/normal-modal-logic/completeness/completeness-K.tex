% Part: normal-modal-logic
% Chapter: completeness
% Section: completeness-K

\documentclass[../../../include/open-logic-section]{subfiles}

\begin{document}

\olfileid{mod}{com}{cmk}

\olsection{Determination and Completeness for  \Log{K}}

We are now prepared to use the canonical model to establish
completeness. Completeness follows from the fact that the !!{formula}s
true in the canonical for~$\Sigma$ are exactly the
$\Sigma$-!!{derivable} ones. Models with this property are said to
\emph{determine}~$\Sigma$.

\begin{defn}
  A model $\mModel{M}$ \emph{determines} a normal modal
  logic~$\Sigma$ precisely when $\mSat{M}{!A}$ if and only if $\Sigma
  \Proves !A$, for all !!{formula}s~$!A$.
\end{defn}

\begin{thm}[Determination]\ollabel{thm:determination}
   $\mSat{M^\Sigma}{!A}$ if and only if $\Sigma \Proves !A$.
\end{thm}

\begin{proof}
  If $\mSat{M^\Sigma}{!A}$, then for every complete
  $\Sigma$-consistent $\Delta$, we have
  $\mSat{M^\Sigma}{!A}[\Delta]$. Hence, by the Truth Lemma, $!A \in
  \Delta$ for every complete $\Sigma$-consistent $\Delta$, whence by
  \olref[lin]{cor:provability-characterization} (with $\Gamma =
  \emptyset$), $\Sigma \Proves !A$.

  Conversely, if $\Sigma \Proves !A$ then by
  \olref[ccs]{prop:ccs-properties}\olref[ccs]{prop:ccs-closed}, every
  complete $\Sigma$-consistent $\Delta$ contains~$!A$, and hence by
  the Truth Lemma, $\mSat{M^\Sigma}{!A}[\Delta]$ for every~$\Delta \in
  W^\Sigma$, i.e., $\mSat{M^\Sigma}{!A}$.
\end{proof}

Since the canonical model for \Log{K} determines~\Log{K}, we
immediately have completeness of~\Log{K} as a corollary:

\begin{cor}\ollabel{cor:Kcomplete}
  The basic modal logic \Log{K} is complete with respect to the
  class of all models, i.e., if $\Entails !A$ then $\Log{K}
  \Proves !A$.
\end{cor}

\begin{proof}
  Contrapositively, if $\Log{K} \Proves/ !A$ then by Determination
  $\mSat/{M^{\Log{K}}}{!A}$ and hence $!A$ is not valid.
\end{proof}

For the general case of completeness of a system $\Sigma$ with respect
to a class of models, e.g., of \Log{KTB4} with respect to the class of
reflexive, symmetric, transitive models, determination alone is not
enough.  We must also show that the canonical model for the system
$\Sigma$ is a member of the class, which does not follow obviously
from the canoncial model construction---nor is it always true!

\end{document}
