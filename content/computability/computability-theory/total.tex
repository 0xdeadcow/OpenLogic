% Part: computability
% Chapter: computability-theory
% Section: total

\documentclass[../../../include/open-logic-section]{subfiles}

\begin{document}

\olfileid{cmp}{thy}{tot}
\olsection{Totality is Undecidable}

Let us consider one more example of using the $s$-$m$-$n$ theorem to show
that something is noncomputable. Let $\fn{Tot}$ be the set of indices
of total computable functions, i.e.
\[
\fn{Tot} = \Setabs{x}{\text{for every $y$, $\cfind{x}(y)\defined$}}.
\]

\begin{prop}
\ollabel{prop:total}
$\fn{Tot}$ is not computable.
\end{prop}

\begin{proof}
To see that $\fn{Tot}$ is not computable, it suffices to show that $K$
is reducible to it. Let $h(x,y)$ be defined by
\[
h(x,y) \simeq
\begin{cases}
0 & \text{if $x \in K$} \\
\text{undefined} & \text{otherwise}
\end{cases}
\]
Note that $h(x,y)$ does not depend on $y$ at all. It should
not be hard to see that $h$~is partial computable: on input $x, y$, the
we compute~$h$ by first simulating the function~$\cfind{x}$ on input~$x$; if
this computation halts, $h(x,y)$ outputs $0$ and halts. So
$h(x,y)$ is just $Z(\umin{s}{T(x,x,s))}$, where $Z$ is the constant zero
function.

Using the $s$-$m$-$n$ theorem, there is a primitive recursive
function~$k(x)$ such that for every $x$ and~$y$,
\[
\cfind{k(x)}(y) =
\begin{cases}
0 & \text{if $x \in K$} \\
\text{undefined} & \text{otherwise}
\end{cases}
\]
So $\cfind{k(x)}$ is total if $x \in K$, and undefined otherwise. Thus,
$k$ is a reduction of $K$ to $\fn{Tot}$.
\end{proof}

\begin{digress}
It turns out that $\fn{Tot}$ is not even computably enumerable---its
complexity lies further up on the ``arithmetic hierarchy.''  But we
will not worry about this strengthening here.
\end{digress}

\end{document}

