% Part: computability
% Chapter: computability-theory
% Section: prop-reduce

\documentclass[../../../include/open-logic-section]{subfiles}

\begin{document}

\olfileid{cmp}{thy}{ppr}
\olsection{Properties of Reducibility}

The intuition behind writing $A \leq_m B$ is that $A$ is ``no harder
than'' $B$. The following two propositions support this intuition.

\begin{prop}
\ollabel{prop:trans-red}
If $A \leq_m B$ and $B \leq_m C$, then $A \leq_m C$.
\end{prop}

\begin{proof}
Composing a reduction of $A$ to $B$ with a reduction of $B$ to
$C$ yields a reduction of $A$ to $C$. (You should check the details!{})
\end{proof}

\begin{prop}
\ollabel{prop:reduce}
Let $A$ and $B$ be any sets, and suppose $A$ is many-one reducible to~$B$.
\begin{enumerate}
\item If $B$ is computably enumerable, so is~$A$.
\item If $B$ is computable, so is~$A$.
\end{enumerate}
\end{prop}

\begin{proof}
Let $f$ be a many-one reduction from $A$ to $B$. For the first
claim, just check that if $B$ is the domain of a partial function~$g$,
then $A$ is the domain of~$g \circ f$:
\begin{align*}
x \in A & \text{iff } f(x) \in B \\
& \text{iff }  g(f(x)) \defined.
\end{align*}

For the second claim, remember that if $B$~is computable then $B$
and~$\Complement{B}$ are computably enumerable. It is not hard to
check that $f$~is also a many-one reduction of $\Complement{A}$ to
$\Complement{B}$, so, by the first part of this proof, $A$ and
$\Complement{A}$ are computably enumerable. So $A$ is computable as
well. (Alternatively, you can check that $\Char{A} = \Char{B} \circ
f$; so if $\Char{B}$ is computable, then so is~$\Char{A}$.)
\end{proof}

\begin{digress}
A more general notion of reducibility called \emph{Turing
  reducibility} is useful in other contexts, especially for proving
undecidability results. Note that by \olref[cmp]{cor:comp-k}, the
complement of~$K_0$ is not reducible to~$K_0$, since it is not
computably enumerable. But, intuitively, if you knew the answers to
questions about $K_0$, you would know the answer to questions about
its complement as well. A set $A$ is said to be Turing reducible
to~$B$ if one can determine answers to questions in~$A$ using a
computable procedure that can ask questions about~$B$. This is more
liberal than many-one reducibility, in which (1) you are only allowed
to ask one question about $B$, and (2) a ``yes'' answer has to
translate to a ``yes'' answer to the question about $A$, and similarly
for ``no.'' It is still the case that if $A$~is Turing reducible
to~$B$ and $B$~is computable then $A$~is computable as well (though,
as we have seen, the analogous statement does not hold for computable
enumerability).

You should think about the various notions of reducibility we have
discussed, and understand the distinctions between
them. We will, however, only deal with many-one reducibility in this
chapter. Incidentally, both types of reducibility discussed in the
last paragraph have analogues in computational complexity, with the
added requirement that the Turing machines run in polynomial time: the
complexity version of many-one reducibility is known as \emph{Karp
  reducibility}, while the complexity version of Turing reducibility
is known as \emph{Cook reducibility}.
\end{digress}

\end{document}
