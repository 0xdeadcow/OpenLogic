% Part: computability
% Chapter: computability-theory
% Section: introduction

\documentclass[../../../include/open-logic-section]{subfiles}

\begin{document}

\olfileid{cmp}{thy}{int}
\olsection{Introduction}

The branch of logic known as \emph{Computability Theory} deals with
issues having to do with the computability, or relative computability,
of functions and sets. It is a evidence of Kleene's influence
that the subject used to be known as \emph{Recursion Theory}, and
today, both names are commonly used.

Let us call a function~$f\colon \Nat \pto \Nat$ \emph{partial
  computable} if it can be computed in some model of computation. If
$f$ is total we will simply say that $f$ is \emph{computable}. A
relation $R$ with computable characteristic function~$\Char{R}$ is
also called computable. If $f$ and $g$ are partial functions, we will
write $f(x) \fdefined$ to mean that $f$ is defined at $x$, i.e., $x$ is
in the domain of $f$; and $f(x) \fundefined$ to mean the opposite,
i.e., that $f$ is not defined at~$x$. We will use $f(x) \simeq g(x)$
to mean that either $f(x)$ and $g(x)$ are both undefined, or they are
both defined and equal.

One can explore the subject without having to refer to a specific
model of computation. To do this, one shows that there is a universal
partial computable function, $\fn{Un}(k, x)$. This allows us to
enumerate the partial computable functions. We will adopt the notation
$\cfind{k}$ to denote the $k$-th unary partial computable function,
defined by $\cfind{k}(x) \simeq \fn{Un}(k, x)$. (Kleene used $\{ k \}$
for this purpose, but this notation has not been used as much
recently.)  Slightly more generally, we can uniformly enumerate the
partial computable functions of arbitrary arities, and we will use
$\cfind{k}[n]$ to denote the $k$-th $n$-ary partial recursive
function.

Recall that if $f(\vec x, y)$ is a total or partial function, then
$\umin{y}{f (\vec x, y)}$ is the function of $\vec x$ that returns the
least $y$ such that $f(\vec x, y) = 0$, assuming that all of $f(\vec
x, 0), \ldots, f(\vec x, y-1)$ are defined; if there is no such $y$,
$\umin{y}{f (\vec x, y)}$ is undefined. If $R(\vec x, y)$ is a
relation, $\umin{y}{R(\vec x, y)}$ is defined to be the least $y$ such
that $R(\vec x, y)$ is true; in other words, the least $y$ such that
{\em one minus} the characteristic function of $R$ is equal to zero at
$\vec x, y$.

To show that a function is computable, there are
two ways one can proceed:
\begin{enumerate}
\item Rigorously: describe a Turing machine or partial recursive
  function explicitly, and show that it computes the function you have
  in mind;
\item Informally: describe an algorithm that computes it, and appeal to
  Church's thesis.
\end{enumerate}
There is no fine line between the two; a detailed description of
an algorithm should provide enough information so that it is
relatively clear how one could, in principle, design the right Turing
machine or sequence of partial recursive definitions. Fully rigorous
definitions are unlikely to be informative, and we will try to find a
happy medium between these two approaches; in short, we will try to
find intuitive yet rigorous proofs that the precise definitions could
be obtained.

\end{document}
