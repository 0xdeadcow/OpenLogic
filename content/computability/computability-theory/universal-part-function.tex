% Part: computability
% Chapter: computability-theory
% Section: universal-part-function

\documentclass[../../../include/open-logic-section]{subfiles}

\begin{document}

\olfileid{cmp}{thy}{uni}
\olsection{The Universal Partial Computable Function}

\begin{thm}
\ollabel{thm:univ-comp}
There is a universal partial computable function $\fn{Un}(k,x)$. In other
words, there is a function $\fn{Un}(k,x)$ such that:
\begin{enumerate}
\item $\fn{Un}(k,x)$ is partial computable.
\item If $f(x)$ is any partial computable function, then there is a
natural number $k$ such that $f(x) \simeq \fn{Un}(k,x)$ for every $x$.
\end{enumerate}
\end{thm}

\begin{proof}
Let $\fn{Un}(k,x) \simeq U(\umin{s}{T(k,x,s)})$ in Kleene's
normal form theorem.
\end{proof}

\begin{explain}
This is just a precise way of saying that we have an effective
enumeration of the partial computable functions; the idea is that if
we write $f_k$ for the function defined by $f_k(x) = \fn{Un}(k,x)$,
then the sequence $f_0$, $f_1$, $f_2$, \dots includes all the partial
computable functions, with the property that $f_k(x)$ can be computed
``uniformly'' in $k$ and $x$. For simplicity, we am using a binary
function that is universal for unary functions, but by coding
sequences of numbers we can easily generalize this to more arguments.
For example, note that if $f(x,y,z)$ is a 3-place partial recursive
function, then the function $g(x) \simeq f((x)_0, (x)_1, (x)_2)$ is a
unary recursive function.
\end{explain}

\end{document}
