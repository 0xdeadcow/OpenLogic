% Part: computability
% Chapter: computability-theory
% Section: applications-fixed-point

\documentclass[../../../include/open-logic-section]{subfiles}

\begin{document}

\olfileid{cmp}{thy}{apf}
\olsection{Applying the Fixed-Point Theorem}

The fixed-point theorem essentially lets us define partial computable
functions in terms of their indices. For example, we can find an
index $e$ such that for every $y$,
\[
\cfind{e}(y) = e + y.
\]
As another example, one can use the proof of the fixed-point theorem
to design a program in Java or C++ that prints itself out.

Remember that if for each $e$, we let $W_e$ be the domain of $\cfind{e}$,
then the sequence $W_0$, $W_1$, $W_2$,~\dots enumerates the computably
enumerable sets. Some of these sets are computable. One can ask if
there is an algorithm which takes as input a value $x$, and, if $W_x$
happens to be computable, returns an index for its characteristic
function. The answer is ``no,'' there is no such algorithm:

\begin{thm}
There is no partial computable function $f$ with the following
property: whenever $W_e$ is computable, then $f(e)$ is defined and
$\cfind{f(e)}$ is its characteristic function.
\end{thm}

\begin{proof}
Let $f$ be any computable function; we will construct an $e$
such that $W_e$ is computable, but $\cfind{f(e)}$ is not its
characteristic function. Using the fixed point theorem, we can find an
index $e$ such that
\[
\cfind{e}(y) \simeq
\begin{cases}
0 & \text{if $y=0$ and $\cfind{f(e)}(0) \fdefined = 0$} \\
\text{undefined} & \text{otherwise.}
\end{cases}
\]
That is, $e$ is obtained by applying the fixed-point theorem to the
function defined by
\[
g(x,y) \simeq
\begin{cases}
0 & \text{if $y=0$ and $\cfind{f(x)}(0) \fdefined = 0$} \\
\text{undefined} & \text{otherwise.}
\end{cases}
\]
Informally, we can see that $g$ is partial computable, as follows: on
input $x$ and $y$, the algorithm first checks to see if $y$ is equal
to~$0$. If it is, the algorithm computes $f(x)$, and then uses the
universal machine to compute $\cfind{f(x)}(0)$. If this last computation
halts and returns~$0$, the algorithm returns~$0$; otherwise, the
algorithm doesn't halt.

But now notice that if $\cfind{f(e)}(0)$ is defined and equal to $0$,
then $\cfind{e}(y)$ is defined exactly when $y$ is equal to $0$, so $W_e =
\{ 0 \}$. If $\cfind{f(e)}(0)$ is not defined, or is defined but not
equal to $0$, then $W_e = \emptyset$. Either way, $\cfind{f(e)}$ is not
the characteristic function of~$W_e$, since it gives the wrong answer
on input $0$.
\end{proof}

\end{document}
