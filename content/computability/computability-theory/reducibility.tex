% Part: computability
% Chapter: computability-theory
% Section: reducibility

\documentclass[../../../include/open-logic-section]{subfiles}

\begin{document}

\olfileid{cmp}{thy}{red}
\olsection{Reducibility}

\begin{explain}
We now know that there is at least one set, $K_0$, that is computably
enumerable but not computable. It should be clear that there are
others. The method of reducibility provides a powerful method of
showing that other sets have these properties, without constantly
having to return to first principles.

Generally speaking, a ``reduction'' of a set $A$ to a set $B$ is a
method of transforming answers to whether or not elements are in~$B$
into answers as to whether or not elements are in~$A$. We will focus
on a notion called ``many-one reducibility,'' but there are many other
notions of reducibility available, with varying properties. Notions of
reducibility are also central to the study of computational
complexity, where efficiency issues have to be considered as well. For
example, a set is said to be ``NP-complete'' if it is in NP and every
NP problem can be reduced to it, using a notion of reduction that is
similar to the one described below, only with the added requirement
that the reduction can be computed in polynomial time.

We have already used this notion implicitly. Define the set $K$ by
\[
K = \Setabs{x}{\cfind{x}(x) \defined},
\]
i.e., $K = \Setabs{x}{x \in W_x}$. Our proof that the halting problem
in unsolvable, \olref[hlt]{thm:halting-problem}, shows most directly
that $K$ is not computable. Recall that $K_0$ is the set
\[
K_0 = \Setabs{\tuple{e, x}}{\cfind{e}(x) \defined }.
\]
i.e. $K_0 = \Setabs{\tuple{x,e}}{x \in W_e}$. It is easy to extend any
proof of the uncomputability of $K$ to the uncomputability of $K_0$:
if $K_0$ were computable, we could decide whether or not an element
$x$ is in $K$ simply by asking whether or not the pair $\tuple{x, x}$
is in $K_0$. The function $f$ which maps $x$ to $\tuple{x, x}$ is an
example of a \emph{reduction} of $K$ to $K_0$.
\end{explain}

\begin{defn}
Let $A$ and $B$ be sets. Then $A$ is said to be \emph{many-one
  reducible} to~$B$, written $A \leq_m B$, if there is a computable
function~$f$ such that for every natural number~$x$,
\[
x \in A \quad \text{if and only if} \quad f(x) \in B.
\]
If $A$ is many-one reducible to $B$ and vice-versa, then $A$ and $B$
are said to be \emph{many-one equivalent}, written $A \equiv_m B$.
\end{defn}

If the function $f$ in the definition above happens to be injective,
$A$ is said to be \emph{one-one reducible} to $B$. Most of the
reductions described below meet this stronger requirement, but we will
not use this fact.

\begin{digress}
It is true, but by no means obvious, that one-one reducibility really
is a stronger requirement than many-one reducibility. In other words,
there are infinite sets $A$ and $B$ such that $A$ is many-one
reducible to $B$ but not one-one reducible to $B$.
\end{digress}

\end{document}
