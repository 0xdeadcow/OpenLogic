% Part: computability
% Chapter: computability-theory
% Section: no-universal-function

\documentclass[../../../include/open-logic-section]{subfiles}

\begin{document}

\olfileid{cmp}{thy}{nou}
\olsection{No Universal Computable Function}

\begin{thm}
\ollabel{thm:no-univ}
There is no universal computable function. In other words, the
universal function $\fn{Un'}(k, x) = \cfind{k}(x)$ is not
computable.
\end{thm}

\begin{proof}
This theorem says that there is no {\em total} computable
function that is universal for the total computable functions. The
proof is a simple diagonalization: if $\fn{Un}'(k,x)$ were total and
computable, then
\[
d(x) = \fn{Un}'(x, x) + 1
\]
would also be total and computable. However, for every $k$, $d(k)$ is
not equal to $\fn{Un}'(k,k)$.
\end{proof}

\begin{explain}
Theorem~\olref[uni]{thm:univ-comp} above shows that we can get around
this diagonalization argument, but only at the expense of allowing
partial functions. It is worth trying to understand what goes wrong
with the diagonalization argument, when we try to apply it in the
partial case. In particular, the function $h(x) = \fn{Un}(x,x)+1$
\emph{is} partial recursive. Suppose $h$ is the $k$-th function in the
enumeration; what can we say about~$h(k)$?
\end{explain}

\end{document}
