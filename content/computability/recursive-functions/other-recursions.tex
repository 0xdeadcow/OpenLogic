% Part: computability
% Chapter: recursive-functions
% Section: other-recursions

\documentclass[../../../include/open-logic-section]{subfiles}

\begin{document}

\olfileid{cmp}{rec}{ore}
\olsection{Other Recursions}

Using pairing and sequencing, we can justify more exotic (and
useful) forms of primitive recursion. For example, it is often useful
to define two functions simultaneously, such as in the following
definition:
\begin{eqnarray*}
f_0(0,\vec z) & = & k_0(\vec z) \\
f_1(0,\vec z) & = & k_1(\vec z) \\
f_0(x+1,\vec z) & = & h_0(x,f_0(x,\vec z),f_1(x,\vec z),\vec z) \\
f_1(x+1,\vec z) & = & h_1(x,f_0(x,\vec z),f_1(x,\vec z),\vec z)
\end{eqnarray*}
This is an instance of \emph{simultaneous recursion}. Another useful
way of defining functions is to give the value of $f(x+1,\vec z)$ in
terms of \emph{all} the values $f(0,\vec z)$, \dots,~$f(x,\vec z)$, as in
the following definition:
\begin{eqnarray*}
f(0,\vec z) & = & g(\vec z) \\
f(x+1,\vec z) & = & h(x, \tuple{f(0,\vec z), \dots, f(x,\vec z)},
\vec z).
\end{eqnarray*}
The following schema captures this idea more succinctly:
\[
f(x,\vec z) = h(x,\tuple{f(0,\vec z), \dots, f(x-1,\vec z)})
\]
with the understanding that the second argument to $h$ is just the
empty sequence when $x$ is $0$. In either formulation, the idea is
that in computing the ``successor step,'' the function $f$ can make
use of the entire sequence of values computed so far.
This is known as a \emph{course-of-values} recursion. For a particular
example, it can be used to justify the following type of definition:
\begin{eqnarray*}
f(x,\vec z) & = & \left\{
\begin{array}{ll}
  h(x,f(k(x,\vec z),\vec z),\vec z) & \text{if $k(x,\vec z) < x$} \\
  g(x,\vec z) & \text{otherwise}
\end{array}\right.
\end{eqnarray*}
In other words, the value of $f$ at $x$ can be computed in terms of
the value of $f$ at \emph{any} previous value, given by $k$.

You should think about how to obtain these functions using ordinary
primitive recursion. One final version of primitive recursion is more
flexible in that one is allowed to change the \emph{parameters} (side
values) along the way:
\begin{eqnarray*}
f(0,\vec z) & = & g(\vec z) \\
f(x+1,\vec z) & = & h(x,f(x,k(\vec z)),\vec z)
\end{eqnarray*}
This, too, can be simulated with ordinary primitive recursion. (Doing
so is tricky. For a hint, try unwinding the computation by hand.)

Finally, notice that we can always extend our ``universe'' by defining
additional objects in terms of the natural numbers, and defining
primitive recursive functions that operate on them. For example, we
can take an integer to be given by a pair $\langle m, n \rangle$ of natural
numbers, which, intuitively, represents the integer $m-n$. In other
words, we say
\[
\fn{Integer}(x) \defiff \fn{length}(x) = 2
\]
and then we define the following:
\begin{enumerate}
\item $\fn{iequal}(x,y)$
\item $\fn{iplus}(x,y)$
\item $\fn{iminus}(x,y)$
\item $\fn{itimes}(x,y)$
\end{enumerate}
Similarly, we can define a rational number to be a pair $\langle x, y \rangle$
of integers with $y \neq 0$, representing the value $x / y$. And we
can define $\fn{qequal}$, $\fn{qplus}$, $\fn{qminus}$, $\fn{qtimes}$,
$\fn{qdivides}$, and so on.

\end{document}
