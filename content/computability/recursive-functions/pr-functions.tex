% Part: computability
% Chapter: recursive-functions
% Section: pr-functions

\documentclass[../../../include/open-logic-section]{subfiles}

\begin{document}

\olfileid{cmp}{rec}{prf}
\olsection{Primitive Recursion Functions}

Let us record again how we can define new functions from existing ones
using primitive recursion and composition.

\begin{defn}
  \ollabel{defn:primitive-recursion} Suppose $f$ is a $k$-place
  function ($k\ge 1$) and $g$ is a $(k+2)$-place function. The
  function defined by \emph{primitive recursion from $f$ and $g$} is
  the $(k+1)$-place function~$h$ defined by the equations
  \begin{align*}
    h(x_0,\dots,x_{k-1},0) & =  f(x_0,\dots,x_{k-1}) \\
    h(x_0,\dots,x_{k-1},y+1) & = g(x_0,\dots,x_{k-1}, y, h(x_0,\dots,x_{k-1}, y))
  \end{align*}
\end{defn}

\begin{defn}
  \ollabel{defn:composition} Suppose $f$ is a $k$-place function, and
  $g_0$, \dots, $g_{k-1}$ are $k$ functions which are all
  $n$-place. The function defined by \emph{composition from $f$ and
    $g_0$, \dots,~$g_{k-1}$} is the $n$-place function~$h$ defined by
  \[
  h(x_0, \dots, x_{n-1}) =
  f(g_0(x_0, \dots, x_{n-1}), \dots, g_{k-1}(x_0, \dots, x_{n-1})).
  \]
\end{defn}

In addition to $\Succ$ and the projection functions
\[
\Proj{n}{i}(x_0,\dots,x_{n-1}) = x_i,
\]
for each natural number $n$ and $i < n$, we will include among the
primitive recursive functions the function~$\Zero(x) = 0$.

\begin{defn}
  The set of primitive recursive functions is the set of functions 
  from $\Nat^n$ to $\Nat$, defined inductively by the
  following clauses:
  \begin{enumerate}
  \item $\Zero$ is primitive recursive.
  \item $\Succ$ is primitive recursive.
  \item Each projection function $\Proj{n}{i}$ is primitive recursive.
  \item If $f$ is a $k$-place primitive recursive function and $g_0$,
    \dots,~$g_{k-1}$ are $n$-place primitive recursive functions, then
    the composition of $f$ with $g_0$, \dots,~$g_{k-1}$ is primitive
    recursive.
  \item If $f$ is a $k$-place primitive recursive function and $g$ is
    a $k+2$-place primitive recursive function, then the function
    defined by primitive recursion from $f$ and $g$ is primitive
    recursive.
\end{enumerate}
\end{defn}

\begin{explain}
Put more concisely, the set of primitive recursive functions is the
smallest set containing $\Zero$, $\Succ$, and the projection
functions~$\Proj{n}{j}$, and which is closed under composition and
primitive recursion.

Another way of describing the set of primitive recursive functions is
by defining it in terms of ``stages.'' Let $S_0$ denote the set of
starting functions: $\Zero$, $\Succ$, and the projections. These are
the primitive recursive functions of stage~$0$. Once a stage $S_i$ has
been defined, let $S_{i+1}$ be the set of all functions you get by
applying a single instance of composition or primitive recursion to
functions already in~$S_i$. Then
\[
S = \bigcup_{i \in \Nat} S_i
\]
is the set of all primitive recursive functions
\end{explain}

Let us verify that $\Add$ is a primitive recursive function.

\begin{prop}
  The addition function $\Add(x,y) = x+y$ is primitive recursive.
\end{prop}

\begin{proof}
We already have a primitive recursive definition of $\Add$ in terms of
two functions $f$ and~$g$ which matches the format of
\olref{defn:primitive-recursion}:
\begin{align*}
  \Add(x_0, 0) & = f(x_0) = x_0\\
  \Add(x_0, y+1) & = g(x_0, y, \Add(x_0, y)) =
  \Succ(\Add(x_0, y))
\end{align*}
So $\Add$ is primitive recursive provided $f$ and $g$ are as
well. $f(x_0) = x_0 = \Proj{1}{0}(x_0)$, and the projection functions
count as primitive recursive, so $f$ is primitive recursive. The
function $g$ is the three-place function $g(x_0, y, z)$ defined by
\[
g(x_0, y, z) = \Succ(z).
\]
This does not yet tell us that $g$ is primitive recursive, since $g$
and $\Succ$ are not quite the same function: $\Succ$ is one-place, and
$g$ has to be three-place. But we can define $g$ ``officially'' by
composition as
\[
g(x_0, y, z) = \Succ(\Proj{3}{2}(x_0, y, z))
\]
Since $\Succ$ and $\Proj{3}{2}$ count as primitive recursive
functions, $g$ does as well, since it can be defined by composition
from primitive recursive functions.
\end{proof}

\begin{prop}
  \ollabel{prop:mult-pr}
  The multiplication function $\Mult(x,y) = x \cdot y$ is primitive recursive.
\end{prop}

\begin{proof}
  Exercise.
\end{proof}

\begin{prob}
  Prove \olref[cmp][rec][prf]{prop:mult-pr} by showing that the
  primitive recursive definition of $\Mult$ is can be put into the
  form required by \olref[cmp][rec][prf]{defn:primitive-recursion} and
  showing that the corresponding functions $f$ and~$g$ are primitive
  recursive.
\end{prob}

\begin{ex}
Here's our very first example of a primitive recursive definition:
\begin{align*}
h(0) & =  1 \\
h(y+1) & =  2 \cdot h(y).
\end{align*}
This function cannot fit into the form required by
\olref{defn:primitive-recursion}, since $k=0$. The definition also
involves the constants $1$ and $2$. To get around the first problem,
let's introduce a dummy argument and define the function~$h'$:
\begin{align*}
h'(x_0, 0) & =  f(x_0) = 1 \\
h'(x_0, y+1) & =  g(x_0, y, h'(x_0, y)) = 2 \cdot h'(x_0, y).
\end{align*}
The function $f(x_0) = 1$ can be defined from $\Succ$ and $\Zero$ by
composition: $f(x_0) = \Succ(\Zero(x_0))$. The function $g$ can be
defined by composition from $g'(z) = 2 \cdot z$ and projections:
\begin{align*}
g(x_0, y, z) & = g'(\Proj{3}{2}(x_0, y, z))
\intertext{and $g'$ in turn can be defined by composition as}
g'(z) & = \Mult(g''(z), \Proj{1}{0}(z))
\intertext{and}
g''(z) & = \Succ(f(z)),
\end{align*}
where $f$ is as above: $f(z) = \Succ(\Zero(z))$. Now that we have $h'$
we can use composition again to let $h(y) =
h'(\Proj{1}{0}(y),\Proj{1}{0}(y))$. This shows that $h$ can be defined
from the basic functions using a sequence of compositions and
primitive recursions, so $h$ is primitive recursive.
\end{ex}

\end{document}
