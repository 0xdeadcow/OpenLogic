% Part: computability
% Chapter: recursive-functions
% Section: pr-relations

\documentclass[../../../include/open-logic-section]{subfiles}

\begin{document}

\olfileid{cmp}{rec}{prr}
\olsection{Primitive Recursive Relations}


\begin{defn}
A relation $R(\vec x)$ is said to be primitive recursive if its characteristic
function,
\[
\Char{R}(\vec x) = \left\{
  \begin{array}{ll}
  1 & \mbox{if $R(\vec x)$} \\
  0 & \mbox{otherwise}
  \end{array}
\right.
\]
is primitive recursive.
\end{defn}

In other words, when one speaks of a primitive recursive relation
$R(\vec x)$, one is referring to a relation of the form $\Char{R}(\vec
x) = 1$, where $\Char{R}$ is a primitive recursive function which, on
any input, returns either 1 or 0. For example, the relation
$\fn{IsZero}(x)$, which holds if and only if $x = 0$, corresponds to the
function $\Char{\fn{IsZero}}$, defined using primitive recursion by
\[
\Char{\fn{IsZero}}(0) = 1, \quad \Char{\fn{IsZero}}(x+1) = 0.
\]

It should be clear that one can compose relations with other primitive
recursive functions. So the following are also primitive recursive:
\begin{enumerate}
\item The equality relation, $x = y$, defined by $\fn{IsZero}(\left|x -
  y\right|)$
\item The less-than relation, $x \leq y$, defined by $\fn{IsZero}(x
  \tsub y)$
\end{enumerate}


\begin{prop}
  The set of primitive recursive relations is closed under boolean
  operations, that is, 
  if $P(\vec x)$ and $Q(\vec x)$ are primitive recursive, so are
  \begin{enumerate}
  \item $\lnot P(\vec x)$
  \item $P(\vec x) \land Q(\vec x)$
  \item $P(\vec x) \lor Q(\vec x)$
  \item $P(\vec x) \lif Q(\vec x)$
  \end{enumerate}
\end{prop}

\begin{proof}
  Suppose $P(\vec x)$ and $Q(\vec x)$ are primitive recursive, i.e.,
  their characteristic functions $\Char{P}$ and $\Char{Q}$ are.  We
  have to show that the characteristic functions of $\lnot P(\vec x)$,
  etc., are also primitive recursive.
  \[
  \Char{\lnot P}(\vec x) = \begin{cases}
    0 & \text{if $\Char{P}(\vec x) = 1$}\\
    1 & \text{otherwise}
  \end{cases}
  \]
  We can define $\Char{\lnot P}(\vec x)$ as $1 \tsub \Char{P}(\vec x)$.
  \[
  \Char{P \land Q}(\vec x) = \begin{cases}
    1 & \text{if $\Char{P}(\vec x) = \Char{Q}(\vec x) = 1$}\\
    0 & \text{otherwise}
  \end{cases}
  \]
  We can define $\Char{P \land Q}(\vec x)$ as $\Char{P}(\vec x) \cdot
  \Char{Q}(\vec x)$ or as $\fn{min}(\Char{P}(\vec x), \Char{Q}(\vec x))$.

  Similarly, $\Char{P \lor Q}(\vec x) = \fn{max}(\Char{P}(\vec x),
  \Char{Q}(\vec x))$ and $\Char{P \lif Q}(\vec x) = \fn{max}(1 \tsub
  \Char{P}(\vec x), \Char{Q}(\vec x))$.
\end{proof}

\begin{prop}
  The set of primitive recursive relations is closed under bounded
  quantification, i.e., if $R(\vec x, z)$ is a primitive recursive
  relation, then so are the relations $\bforall{z < y}{R(\vec x, z)}$
  and $\bexists{z < y}{R(\vec x, z)}$.

  ($\bforall{z < y}{R(\vec x, z)}$ holds of $\vec x$ and $y$ if and
  only if $R(\vec x, z)$ holds for every~$z$ less than~$y$, and
  similarly for $\bexists{z < y}{R(\vec x, z)}$.)
\end{prop}

\begin{proof}
  By convention, we take $\bforall{z < 0}{R(\vec x, z)}$ to be true
  (for the trivial reason that there are no $z$ less than $0$) and
  $\bexists{z < 0}{R(\vec x, z)}$ to be false. A universal quantifier
  functions just like a finite product or iterated minimum, i.e., if
  $P(\vec x, y) \defiff \bforall{z < y}{R(\vec x, z)}$ then
  $\Char{P}(\vec x, y)$ can be defined by
  \begin{align*}
    \Char{P}(\vec x, 0) & = 1\\
    \Char{P}(\vec x, y+1) & =
    \fn{min}(\Char{P}(\vec x, y), \Char{R}(\vec x, y))).
  \end{align*}
  Bounded existential quantification can similarly be defined using
  $\fn{max}$. Alternatively, it can be defined from bounded universal
  quantification, using the equivalence $\bexists{z < y}{R(\vec x, z)} \liff
  \lnot \bforall{z < y}{\lnot R(\vec x, z)}$. Note that, for example, a
  bounded quantifier of the form $\bexists{x \leq y}{\dots x\dots}$ is
  equivalent to $\bexists{x < y+1}{\dots x \dots}$.
\end{proof}
  
Another useful primitive recursive function is the conditional
function, $\fn{cond}(x,y,z)$, defined by
\[
\fn{cond}(x,y,z) = \begin{cases}
  y & \text{if $x = 0$} \\
  z & \text{otherwise}.
\end{cases}
\]
This is defined recursively by
\[
\fn{cond}(0,y,z) = y, \quad \fn{cond}(x+1,y,z) = z.
\]
One can use this to justify definitions of primitive recursive functions
by cases from primitive recursive relations:

\begin{prop}
If $g_0(\vec x)$, \dots,~$g_m(\vec x)$ are primitive recursive functions, and $R_0(\vec
x)$, \dots, $R_{m-1}(\vec x)$ are primitive recursive relations, then
the function $f$ defined by
\[
f(\vec x) = \begin{cases}
    g_0(\vec x) & \text{if $R_0(\vec{x})$} \\
    g_1(\vec x) & \text{if $R_1(\vec{x})$ and not $R_0(\vec{x})$} \\
    \vdots & \\
    g_{m-1}(\vec x) & \text{if $R_{m-1}(\vec{x})$ and none of the
      previous hold}
    \\
    g_m(\vec x) & \mbox{otherwise}
\end{cases}
\]
is also primitive recursive.
\end{prop}

\begin{proof}
  When $m = 1$, this is just the function defined by
  \[
  f(\vec x) = \fn{cond}(\Char{\lnot R_0}(\vec x),g_0(\vec x),g_1(\vec
  x)).
  \]
  For $m$ greater than $1$, one can just compose definitions of this
  form.
\end{proof}
\end{document}
