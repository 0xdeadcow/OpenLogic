% Part: computability
% Chapter: recursive-functions
% Section: pr-functions-computable

\documentclass[../../../include/open-logic-section]{subfiles}

\begin{document}

\olfileid{cmp}{rec}{cmp}
\olsection{Primitive Recursive Functions are Computable}

Suppose a function $h$ is defined by primitive recursion
\begin{eqnarray*}
h(\vec x, 0) & = & f(\vec x) \\
h(\vec x, y) & = & g(\vec x, y, h(\vec x, y))
\end{eqnarray*}
and suppose the functions $f$ and $g$ are computable.  (We use $\vec x$ to abbreviate $x_0$, \dots, $x_{k-1}$.) Then $h(\vec
x, 0)$ can obviously be computed, since it is just $f(\vec x)$ which we
assume is computable.  $h(\vec x, 1)$ can then also be computed, since
$1 = 0 + 1$ and so $h(\vec x, 1)$ is just
\begin{align*}
 h(\vec x, 1) & = g(\vec x, 0, h(\vec x, 0)) =  g(\vec x, 0, f(\vec x)).
\intertext{We can go on in this way and  compute}
h(\vec x, 2) & = g(\vec x, 1, h(\vec x, 1)) = g(\vec x, 1, g(\vec x, 0, f(\vec x)))\\
h(\vec x, 3) & = g(\vec x, 2, h(\vec x, 2)) = g(\vec x, 2, g(\vec x, 1, g(\vec x, 0, f(\vec x))))\\
h(\vec x, 4) & = g(\vec x, 3, h(\vec x, 3)) = g(\vec x, 3, g(\vec x, 2, g(\vec x, 1, g(\vec x, 0, f(\vec x)))))\\
& \vdots
\end{align*}
Thus, to compute $h(\vec x, y)$ in general, successively compute
$h(\vec x, 0)$, $h(\vec x, 1)$, \dots, until we reach $h(\vec x, y)$.

Thus, a primitive recursive definition yields a new computable function if the
functions $f$ and $g$ are computable.  Composition of functions also
results in a computable function if the functions $f$ and $g_i$ are
computable.

Since the basic functions $\Zero$, $\Succ$, and $\Proj{n}{i}$ are computable,
and composition and primitive recursion yield computable functions
from computable functions, this means that every primitive recursive
function is computable.

\end{document}
