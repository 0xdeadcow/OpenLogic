% Part: computability
% Chapter: recursive-functions
% Section: pr-functions-computable

\documentclass[../../../include/open-logic-section]{subfiles}

\begin{document}

\olfileid{cmp}{rec}{cmp}
\olsection{Primitive Recursive Functions are Computable}

Suppose a function $h$ is defined by primitive recursion
\begin{eqnarray*}
h(0,\vec z) & = & f(\vec z) \\
h(x+1, \vec z) & = & g(x, h(x, \vec z), \vec z)
\end{eqnarray*}
and suppose the functions $f$ and $g$ are computable.  Then $h(0, \vec
z)$ can obviously be computed, since it is just $f(\vec z)$ which we
assume is computable.  $h(1, \vec z)$ can then also be computed, since
$1 = 0 + 1$ and so $h(1, \vec z)$ is just
\[
 g(0, h(0, \vec z), \vec z) =  g(0, f(\vec z), \vec z).
\]
We can go on in this way and  compute
\begin{align*}
h(2, \vec z) & = g(1, g(0, f(\vec z), \vec z), \vec z)\\
h(3, \vec z) & = g(2, g(1, g(0, f(\vec z), \vec z), \vec z), \vec z)\\
h(4, \vec z) & = g(3, g(2, g(1, g(0, f(\vec z), \vec z), \vec z), \vec z), \vec z)\\
& \vdots
\end{align*}
Thus, to compute $h(x, \vec z)$ in general, successively compute $h(0,
\vec z)$, $h(1, \vec z)$, \dots, until we reach $h(x, \vec z)$.

Thus, primitive recursion yields a new computable function if the
functions $f$ and $g$ are computable.  Composition of functions also
results in a computable function if the functions $f$ and $g_i$ are
computable.

Since the basic functions $\Zero$, $\Succ$, and $\Proj{n}{i}$ are computable,
and composition and primitive recursion yield computable functions
from computable functions, his means that every primitive recursive
function is computable.

\end{document}
