% Part: computability
% Chapter: recursive-functions
% Section: bounded-minimization

\documentclass[../../../include/open-logic-section]{subfiles}

\begin{document}

\olfileid{cmp}{rec}{bmi}
\olsection{Bounded Minimization}

\begin{explain}
It is often useful to define a function as the least number satisfying
some property or relation~$P$. If $P$ is decidable, we can compute
this function simply by trying out all the possible numbers, $0$, $1$,
$2$, \dots, until we find the least one satisfying~$P$.  This kind of
unbounded search takes us out of the realm of primitive recursive
functions. However, if we're only interested in the least number
\emph{less than some independently given bound}, we stay primitive
recursive. In other words, and a bit more generally, suppose we have a
primitive recursive relation~$R(x,z)$. Consider the function that maps
$x$ and~$y$ to the least $z < y$ such that $R(x, z)$. It, too, can be
computed, by testing whether $R(x, 0)$, $R(x, 1)$, \dots, $R(x, y-1)$.
But why is it primitive recursive?
\end{explain}

\begin{prop}
If $R(\vec x, z)$ is primitive recursive, so is the function
$m_R(\vec{x}, y)$ which returns the least~$z$ less than~$y$ such that
$R(\vec x, z)$ holds, if there is one, and $y$ otherwise.  We will
write the function~$m_R$ as
\[
\bmin{z < y}{R(\vec{x}, z)},
\]
\end{prop}

\begin{proof}
Note than there can be no~$z < 0$ such that $R(\vec{x}, z)$ since
there is no $z < 0$ at all.  So $m_R(\vec x, 0) = 0$.

In case the bound is of the form $y + 1$ we have three cases: (a)
There is a $z < y$ such that $R(\vec{x}, z)$, in which case
$m_R(\vec{x}, z) = m_R(\vec{x}, y)$. (b) There is no such~$z<y$ but
$R(\vec{x}, y)$ holds, then $m_R(\vec x, y+1) = y$. (c) There is no $z
< y+1$ such that $R(\vec{x}, z)$, then $m_R(\vec{z}, y+1) = y+1$.
Note that there is a $z<y$ such that $R(\vec x, z)$ iff $m_R(\vec x,
y) \neq y$.  So,
\begin{align*}
m_R(\vec{x}, 0) & = 0\\
m_R(\vec{x}, y+1) & =
\begin{cases}
m_R(\vec{z}, y) & \text{if $m_R(\vec x, y) \neq y$}\\
y & \text{if $m_R(\vec x, y) = y$ and $R(\vec{x}, y)$}\\
y+1 & \text{otherwise.}
\end{cases}
\end{align*}
\end{proof}

\begin{prob}
Suppose $R(\vec x, z)$ is primitive recursive. Define the function
$m'_R(\vec{x}, y)$ which returns the least~$z$ less than~$y$ such that
$R(\vec{x}, z)$ holds, if there is one, and $0$ otherwise, by
primitive recursion from~$\Char{R}$.
\end{prob}

\end{document}
