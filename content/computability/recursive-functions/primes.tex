% Part: computability
% Chapter: recursive-functions
% Section: primes

\documentclass[../../../include/open-logic-section]{subfiles}

\begin{document}

\olfileid{cmp}{rec}{pri}
\olsection{Primes}

Bounded quantification and bounded minimization provide us with a good
deal of machinery to show that natural functions and relations are
primitive recursive. For example, consider the relation ``$x$
divides $y$'', written $x \mid y$.  The relation $x \mid y$ holds if
division of $y$ by~$x$ is possible without remainder, i.e., if $y$ is
an integer multiple of~$x$.  (If it doesn't hold, i.e., the remainder
when dividing $x$ by $y$ is $> 0$, we write $x \nmid y$.) In other
words, $x \mid y$ iff for some~$z$, $x \cdot z = y$.  Obviously, any
such $z$, if it exists, must be $\leq y$. So, we have that $x \mid y$
iff for some $z \le y$, $x \cdot z = y$.  We can define the relation
$x \mid y$ by bounded existential quantification from $=$ and
multiplication by
\[
x \mid y \defiff \bexists{z \leq y}{(x \cdot z) = y}.
\]
We've thus shown that $x \mid y$ is primitive recursive.

A natural number~$x$ is \emph{prime} if it is neither $0$ nor $1$ and
is only divisible by $1$ and itself. In other words, prime numbers are
such that, whenever $y \mid x$, either $y = 1$ or~$y=x$.  To test if
$x$~is prime, we only have to check if $y \mid x$ for all $y \le x$,
since if $y > x$, then automatically~$y \nmid x$.  So, the relation
$\fn{Prime}(x)$, which holds iff $x$ is prime, can be defined by
\[
\fn{Prime}(x) \defiff x \geq 2 \land \bforall{y \leq x}{(y \mid x \lif y
  = 1 \lor y = x)}
\]
and is thus primitive recursive.

The primes are $2$, $3$, $5$, $7$, $11$, etc. Consider the function
$p(x)$ which returns the $x$th prime in that sequence, i.e., $p(0) =
2$, $p(1) = 3$, $p(2) = 5$, etc. (For convenience we will often write
$p(x)$ as $p_x$ ($p_0=2$, $p_1=3$, etc.)

If we had a function
$\fn{nextPrime(x)}$, which returns the first prime number larger
than~$x$, $p$~can be easily defined using primitive recursion:
\begin{align*}
  p(0) & = 2\\
  p(x+1) & = \fn{nextPrime}(p(x))
\end{align*}
Since $\fn{nextPrime}(x)$ is the least $y$ such that $y > x$ and
$y$~is prime, it can be easily computed by unbounded search. But it
can also be defined by bounded minimization, thanks to a result due to
Euclid: there is always a prime number between $x$ and $x\fac+1$.
\[
  \fn{nextPrime(x)} =
  \bmin{y \leq x\fac+1}{(y > x \land \fn{Prime}(y))}.
\]
This shows, that $\fn{nextPrime}(x)$ and hence $p(x)$ are (not just
computable but) primitive recursive.

(If you're curious, here's a quick proof of Euclid's theorem. Suppose
$p_n$ is the largest prime $\le x$ and consider the product $p =
p_0\cdot p_1 \cdot \dots \cdot p_n$ of all primes~$\le x$. Either
$p+1$ is prime or there is a prime between $x$ and~$p+1$.  Why?
Suppose $p+1$ is not prime. Then some prime number $q \mid p+1$ where
$q < p+1$. None of the primes $\le x$ divide $p+1$. (By definition
of~$p$, each of the primes $p_i \le x$ divides~$p$, i.e., with
remainder~$0$. So, each of the primes $p_i \le x$ divides $p+1$ with
remainder~$1$, and so $p_i \nmid p+1$.)  Hence, $q$ is a prime $>
x$ and $< p+1$.  And $p \le x\fac$, so there is a prime $> x$ and $\le
x\fac+1$.)

\begin{prob}
Define integer division $d(x, y)$ using bounded minimization.
\end{prob}

\end{document}
