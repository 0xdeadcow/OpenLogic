% Part: computability
% Chapter: recursive-functions
% Section: examples

\documentclass[../../../include/open-logic-section]{subfiles}

\begin{document}

\olfileid{cmp}{rec}{exa}
\olsection{Examples of Primitive Recursive Functions}


Here are some examples of primitive recursive functions:
\begin{enumerate}
\item Constants: for each natural number $n$, $n$ is a 0-ary primitive
  recursive function, since it is equal to $S(S(\dots S(0)))$.

\item The identity function: $\fn{id}(x) = x$, i.e.\ $\Proj{1}{0}$

\item Addition, $x+y$

\item Multiplication, $x \cdot y$

\item Exponentiation, $x^y$ (with $0^0$ defined to be $1$)

\item Factorial, $x\fac$

\item The predecessor function, $\fn{pred}(x)$, defined by
\[
\fn{pred}(0) = 0, \quad \fn{pred}(x+1) = x
\]

\item Truncated subtraction, $x \tsub y$, defined by
\[
x \tsub 0 = x, \quad x \tsub (y+1) = \fn{pred}(x \tsub y)
\]

\item Maximum, $\fn{max}(x,y)$, defined by
\[
\fn{max}(x,y) \defis x + (y \tsub x)
\]

\item Minimum, $\fn{min}(x,y)$

\item Distance between $x$ and $y$, $\left|x-y\right|$
\end{enumerate}

\begin{prob}
Show that \[f(x, y) =
2^{2^{\iddots^{2^{x}}}}\raisebox{1ex}{\bigg\rbrace}
\raisebox{1ex}{\text {$y$ $2$'s}}\] is primitive recursive.
\end{prob}

\begin{prob}
Show that $d(x, y) = \lfloor x/y \rfloor$ (i.e., division, where you
disregard everything after the decimal point) is primitive
recursive. When $y = 0$, we stipulate $d(x, y) = 0$. Give an explicit
definifion of $d$ using primitive recursion and composition. You will
have detour through an axiliary function---you cannot use recursion on
the arguments $x$ or $y$ themselves.
\end{prob}

The set of primitive recursive functions is further closed under the
following two operations:
\begin{enumerate}
\item Finite sums: if $f(x,\vec z)$ is primitive recursive, then so is the
function
\[
g(y,\vec z) \defis \sum_{x = 0}^y f(x,\vec z).
\]
\item Finite products: if $f(x,\vec z)$ is primitive recursive, then so is the
function
\[
h(y,\vec z) \defis \prod_{x = 0}^y f(x,\vec z).
\]
\end{enumerate}
For example, finite sums are defined recursively by the equations
\[
g(0,\vec z) = f(0,\vec z), \quad g(y+1,\vec z) = g(y,\vec z) +
f(y+1,\vec z).
\]
We can also define boolean operations, where $1$ stands for true, and
$0$ for false:
\begin{enumerate}
\item Negation, $\fn{not}(x) \defis 1 \tsub x$
\item Conjunction, $\fn{and}(x,y) \defis x \cdot y$
\end{enumerate}
Other classical boolean operations like $\fn{or}(x,y)$ and
$\fn{ifthen}(x,y)$ can be defined from these in the usual way.

\end{document}
