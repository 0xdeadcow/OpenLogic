% Part: computability
% Chapter: recursive-functions
% Section: examples

\documentclass[../../../include/open-logic-section]{subfiles}

\begin{document}

\olfileid{cmp}{rec}{exa}
\olsection{Examples of Primitive Recursive Functions}


We already have some examples of primitive recursive functions: the
addition and multiplication functions~$\Add$ and $\Mult$.  The
identity function $\fn{id}(x) = x$ is primitive recursive, since it is
just~$\Proj{1}{0}$. The constant functions $\fn{const}_n(x) = n$ are
primitive recursive since they can be defined from $\Zero$ and $\Succ$
by successive composition. This is useful when we want to use
constants in primitive recursive definitions, e.g., if we want to
define the function $f(x) = 2 \cdot x$ can obtain it by composition
from $\fn{const}_n(x)$ and multiplication as $f(x) =
\Mult(\fn{const}_2(x), \Proj{1}{0}(x))$. We'll make use of this trick
from now on.

\begin{prop}
  The exponentiation function $\fn{exp}(x, y) = x^y$ is primitive recursive.
\end{prop}

\begin{proof}
  We can define $\fn{exp}$ primitive recursively as
  \begin{align*}
    \fn{exp}(x, 0) & = 1\\
    \fn{exp}(x, y+1) & = \Mult(x, \fn{exp}(x,y)).
    \intertext{Strictly speaking, this is not a recursive definition
      from primitive recursive functions. Officially, though, we
      have:}
    \fn{exp}(x, 0) & = f(x)\\
    \fn{exp}(x, y+1) & = g(x, y, \fn{exp}(x,y)).
    \intertext{where}
    f(x) & = \Succ(\Zero(x)) = 1\\
    g(x, y, z) & = \Mult(\Proj{3}{0}(x, y, z), \Proj{3}{2}(x, y, z) = x \cdot z
  \end{align*}
  and so $f$ and $g$ are defined from primitive recursive functions by
  composition.
\end{proof}

\begin{prop}
  The predecessor function $\fn{pred}(y)$ defined by
  \[
  \fn{pred}(y) = \begin{cases}
    0 & \text{if $y=0$}\\
    y-1 & \text{otherwise}
  \end{cases}
  \]
  is primitive recursive.
\end{prop}

\begin{proof}
 Note that 
 \begin{align*}
   \fn{pred}(0) & = 0 \text{ and}\\
   \fn{pred}(y+1) & = y.
 \end{align*}
 This is almost a primitive recursive definition.  It does not,
 strictly speaking, fit into the pattern of definition by primitive
 recursion, since that pattern requires at least one extra
 argument~$x$. It is also odd in that it does not actually use
 $\fn{pred}(y)$ in the definition of $\fn{pred}(y+1)$. But we can
 first define $\fn{pred}'(x, y)$ by
 \begin{align*}
   \fn{pred}'(x, 0) & = \Zero(x) = 0,\\
   \fn{pred}'(x, y+1) & = \Proj{3}{1}(x, y, \fn{pred'}(x, y)) = y.
 \end{align*}
and then define $\fn{pred}$ from it by composition, e.g., as
$\fn{pred}(x) = \fn{pred}'(\Zero(x), \Proj{1}{0}(x))$.
\end{proof}

\begin{prop}
  The factorial function $\fn{fac}(x) = \fact{x} = 1 \cdot 2 \cdot 3
  \cdot \dots \cdot x$ is primitive recursive.
\end{prop}

\begin{proof}
  The obvious primitive recursive definition is
  \begin{align*}
    \fn{fac}(0) &= 1\\
    \fn{fac}(y+1) & = \fn{fac}(y) \cdot (y+1).
    \intertext{Officially, we have to first define a two-place function $h$}
    h(x, 0) & = \fn{const}_1(x)\\
    h(x, y) & = g(x, y, h(x, y))
    \intertext{where $g(x, y, z) = \Mult(\Proj{3}{2}(x, y, z),
      \Succ(\Proj{3}{1}(x, y, z)$ and then let}
    \fn{fac}(y) & = h(\Proj{1}{0}(y), \Proj{1}{0}(y))
  \end{align*}
  From now on we'll be a bit more laissez-faire and not give the official
  definitions by composition and primitive recursion.
\end{proof}

\begin{prop}
  Truncated subtraction, $x \tsub y$, defined by
  \[
  x \tsub y = \begin{cases}
    0 & \text{if $x > y$}\\
    x-y & \text{otherwise}
  \end{cases}
  \]
  is primitive recursive.
\end{prop}

\begin{proof}
  We have:
  \begin{align*}
    x \tsub 0 & = x\\
    x \tsub (y+1) & = \fn{pred}(x \tsub y) \qedhere
  \end{align*}
\end{proof}

\begin{prop}
 The distance between $x$ and $y$, $\left|x-y\right|$, is primitive recursive.
\end{prop}

\begin{proof}
  We have $\left| x-y \right| = (x \tsub y) + (y \tsub x)$, so
  the distance can be defined by composition from $+$ and $\tsub$,
  which are primitive recursive.
\end{proof}

\begin{prop}
  The maximum of $x$ and $y$, $\fn{max}(x,y)$, is primitive recursive.
\end{prop}

\begin{proof}
  We can define $\fn{max}(x,y)$ by composition from $+$ and $\tsub$ by
  \[
  \fn{max}(x,y) \defis x + (y \tsub x).
  \]
  If $x$ is the maximum, i.e., $x \ge y$, then $y \tsub x = 0$, so $x
  + (y \tsub x) = x + 0 = x$. If $y$ is the maximum, then $y \tsub x =
  y - x$, and so $x + (y \tsub x) = x + (y - x) = y$.
\end{proof}

\begin{prop}
  \ollabel{prop:min-pr}
  The minimum of $x$ and $y$, $\fn{min}(x,y)$, is primitive recursive.
\end{prop}

\begin{proof}
  Prove \olref[cmp][rec][exa]{prop:min-pr}.
\end{proof}

\begin{prob}
Show that \[
f(x, y) =
2^{(2^{\iddots^{2^{x}}})}\raisebox{1ex}{\bigg\rbrace}
\raisebox{1ex}{\text {$y$ $2$'s}}\] is primitive recursive.
\end{prob}

\begin{prob}
Show that integer division $d(x, y) = \lfloor x/y \rfloor$ (i.e.,
division, where you disregard everything after the decimal point) is
primitive recursive. When $y = 0$, we stipulate $d(x, y) = 0$. Give an
explicit definition of~$d$ using primitive recursion and
composition.
\end{prob}

\begin{prop}
The set of primitive recursive functions is closed under the
following two operations:
\begin{enumerate}
\item Finite sums: if $f(\vec x, z)$ is primitive recursive, then so
is the function
\[
g(\vec x, y) \defis \sum_{z = 0}^y f(\vec x, z).
\]
\item Finite products: if $f(\vec x, z)$ is primitive recursive, then
so is the function
\[
h(\vec x, y) \defis \prod_{z = 0}^y f(\vec x, z).
\]
\end{enumerate}
\end{prop}

\begin{proof}
For example, finite sums are defined recursively by the equations
\begin{align*}
  g(\vec x, 0) & = f(\vec x, 0)\\
  g(\vec x, y+1) & = g(\vec x, y) + f(\vec x, y+1).\qedhere
\end{align*}
\end{proof}


\end{document}
