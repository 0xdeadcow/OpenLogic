% Part: computability
% Chapter: recursive-functions
% Section: noations-pr-functions

\documentclass[../../../include/open-logic-section]{subfiles}

\begin{document}

\olfileid{cmp}{rec}{not}
\olsection{Primitive Recursion Notations}

One advantage to having the precise inductive description of the primitive
recursive functions is that we can be systematic in describing them.
For example, we can assign a ``notation'' to each such function, as
follows. Use symbols $\Zero$, $\Succ$, and $\Proj{n}{i}$ for zero,
successor, and the projections. Now suppose $f$ is defined by
composition from a $k$-place function~$h$ and $n$-place functions $g_0$,
\dots,~$g_{k-1}$, and we have assigned notations $H$, $G_0$,
\dots,~$G_{k-1}$ to the latter functions. Then, using a new symbol
$\fn{Comp}_{k,n}$, we can denote the function $f$ by
$\fn{Comp}_{k,n}[H,G_0,\dots,G_{k-1}]$. For the functions defined by
primitive recursion, we can use analogous notations of the form
$\fn{Rec}_k[G,H]$, where $k+1$ is the arity of the function being
defined. With this setup, we can denote the addition function by
\[
\fn{Rec}_2[\Proj{1}{0},\fn{Comp}_{1,3}[\Succ,\Proj{3}{2}]].
\]
Having these notations sometimes proves useful.

\begin{prob}
  Give the complete primitive recursive notation for $\Mult$.
\end{prob}

\end{document}
