% Part: computability
% Chapter: recursive-functions
% Section: partial-functions

\documentclass[../../../include/open-logic-section]{subfiles}

\begin{document}

\olfileid{cmp}{rec}{par}
\olsection{Partial Recursive Functions}


To motivate the definition of the recursive functions, note that our
proof that there are computable functions that are not primitive
recursive actually establishes much more. The argument was
simple: all we used was the fact was that it is possible to enumerate
functions $f_0,f_1,\dots$ such that, as a function of $x$ and $y$,
$f_x(y)$ is computable. So the argument applies to \emph{any class of
  functions that can be enumerated in such a way}. This puts us in a
bind: we would like to describe the computable functions explicitly;
but any explicit description of a collection of computable functions
cannot be exhaustive!

The way out is to allow \emph{partial} functions to come into play. We
will see that it \emph{is} possible to enumerate the partial
computable functions.\iftag{TMs}{ In fact, we already pretty much know
  that this is the case, since it is possible to enumerate Turing
  machines in a systematic way.}{} We will come back to our diagonal
argument later, and explore why it does not go through when partial
functions are included.

The question is now this: what do we need to add to the primitive
recursive functions to obtain all the partial recursive functions? We
need to do two things:
\begin{enumerate}
\item Modify our definition of the primitive recursive functions to
  allow for partial functions as well.
\item \emph{Add} something to the definition, so that some new partial
  functions are included.
\end{enumerate}

The first is easy. As before, we will start with zero, successor, and
projections, and close under composition and primitive recursion. The
only difference is that we have to modify the definitions of
composition and primitive recursion to allow for the possibility that
some of the terms in the definition are not defined. If $f$ and $g$
are partial functions, we will write $f(x) \fdefined$ to mean that $f$
is defined at $x$, i.e., $x$ is in the domain of $f$; and $f(x)
\fundefined$ to mean the opposite, i.e., that $f$ is not defined at~$x$.
We will use $f(x) \simeq g(x)$ to mean that either $f(x)$ and $g(x)$
are both undefined, or they are both defined and equal. We will use these
notations for more complicated terms as well. We will adopt the
convention that if $h$ and $g_0$, \dots,~$g_k$ all are partial functions,
then
\[
h(g_0(\vec x),\dots,g_k(\vec x))
\]
is defined if and only if each $g_i$ is defined at $\vec x$, and $h$
is defined at $g_0(\vec x)$, \dots,~$g_k(\vec x)$. With this
understanding, the definitions of composition and primitive recursion
for partial functions is just as above, except that we have to replace
``$=$'' by ``$\simeq$''.

What we will add to the definition of the primitive recursive
functions to obtain partial functions is the \emph{unbounded search
  operator}. If $f(x,\vec z)$ is any partial function on the natural
numbers, define $\mu x \; f(x,\vec z)$ to be
\begin{quote}
  the least $x$ such that $f(0,\vec z), f(1,\vec z), \dots, f(x,\vec
  z)$ are all defined, and $f(x,\vec z) = 0$, if such an $x$ exists
\end{quote}
with the understanding that $\mu x \; f(x,\vec z)$ is undefined
otherwise. This defines $\mu x \; f(x,\vec z)$ uniquely.

\begin{explain}
Note that our definition makes no reference to\iftag{TMs}{ Turing
  machines, or}{} algorithms, or any specific computational model. But
like composition and primitive recursion, there is an operational,
computational intuition behind unbounded search. When it
comes to the computability of a partial function, arguments
where the function is undefined correspond to inputs for which the
computation does not halt. The procedure for computing $\mu x \;
f(x,\vec z)$ will amount to this: compute $f(0,\vec z), f(1,\vec z),
f(2,\vec z)$ until a value of 0 is returned. If any of the
intermediate computations do not halt, however, neither does the
computation of $\mu x \; f(x,\vec z)$.
\end{explain}

If $R(x,\vec z)$ is any relation, $\mu x \; R(x,\vec z)$ is defined to
be $\mu x \; (1 \tsub \Char{R}(x,\vec z))$. In other words, $\mu x \;
R(x,\vec z)$ returns the least value of $x$ such that $R(x,\vec z)$
holds. So, if $f(x,\vec z)$ is a total function, $\mu x \; f(x,\vec
z)$ is the same as $\mu x \; (f(x,\vec z) = 0)$. But note that our
original definition is more general, since it allows for the
possibility that $f(x,\vec z)$ is not everywhere defined (whereas, in
contrast, the characteristic function of a relation is always total).

\begin{defn}
The set of \emph{partial recursive functions} is the smallest set of
partial functions from the natural numbers to the natural numbers (of
various arities) containing zero, successor, and projections, and
closed under composition, primitive recursion, and unbounded search.
\end{defn}

Of course, some of the partial recursive functions will happen to be
total, i.e., defined for every argument.

\begin{defn}
\ollabel{defn:recursive-fn}
The set of \emph{recursive functions} is
the set of partial recursive functions that are total.
\end{defn}

A recursive function is sometimes called ``total recursive'' to
emphasize that it is defined everywhere.

\end{document}


