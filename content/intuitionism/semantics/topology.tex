\begin{document}
\begin{defn}[Topology]
  A \emph{topology space} is a pair $\tuple{X, \mathcal{O}}$ where
  $\mathcal{O}$ is a collection of subsets of $X$, such that
  \begin{enumerate}
  \item $\emptyset, X \in \mathcal{O}$
  \item if $U,V \in \mathcal{O}$ then $U \cap V \in \mathcal{O}$
  \item if $U_i \in \mathcal{O}$ for $i \in I$, then $\cup\{U_i | i \in 
    I_i\} \in \mathcal{O}$
  \end{enumerate}
\end{defn}

In plain English, a topology is a set $X$ endowed with a collection
$\mathcal{O}$ of subsets (called open sets), that closed under finite
intersections and arbitrary unions.

We may write $X$ for a topology if the collection of open sets can be
inferred from the context; note that, still, only after $X$ is endowed
with open sets can it be called a topology.

\begin{defn}
  An interpretation of IPC in a topology $X$ is a function $V$ assigning
  an open set in $X$ to each propositional variable, extended to composite
  formulas as follows:
  \begin{enumerate}
  \item $V(!A \land !B) = V(!A) \cap V(!B)$
  \item $V(!A \lor !B) = V(!A) \cup V(!B)$
  \item $V(\lfalse) = \emptyset$
  \item $V(!A \lif !B) = Int((X - V(!A)) \cup V(!B))$
  \end{enumerate}
\end{defn}

Note how we preserve the property that subsets assigned to formulas are
always open sets. The $Int()$ in the last rule is a function converting a
subset into its interior, which is defined as the union of all open
sets contained in it (thus also an open set).

We now give some intuitions of this definition; the final
justification is, of course, left to formal proofs. We maintain the
invariant that, $!A \Proves !B$ iff $V(!A) \subset V(!B)$, rendering
the definition for $\land$ and $\lor$ immediate. $\lfalse$ is mapped
to $\emptyset$, because $\emptyset$ is subset of every subset, just
like every formula is derivable from $\lfalse$. The definition for
$\lif$ is tricky: by our intuition of $!A \lif !B$, it is like the ``weakest''
formula such that, along with $!A$, can prove $!B$, that is $(!A
\lif !B) \land !A \Proves !B $. So $V(!A \lif !B)$ should be the
greatest open set such that $V(!A \lif !B) \cap V(!A)
\subset V(!B)$, leading to our definition.

\begin{defn}
  
\end{defn}

\begin{proof}
\end{proof}

\begin{defn}
  An interpretation of IQC in a topology $X$ is an interpretation of
  the IPC fragment, plus for each predicate a function assigning an
  open set in $X$ to each 
\end{defn}

\end{document}