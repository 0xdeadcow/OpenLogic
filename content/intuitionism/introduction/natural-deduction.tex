\documentclass[../../../include/open-logic-section]{subfiles}

\begin{document}

\olsection{Natural deduction}

In this section we present intuitionistic predicate logic in natural
deduction fashion, with some intuitive motivation of rules:
\begin{enumerate}
\item[Conjunction]
\begin{gather*}
  \AxiomC{$!A_1$}
  \AxiomC{$!A_2$}
  \RightLabel{$\Intro{\land}$}
  \BinaryInfC{$!A_1 \land !A_2$}
  \DisplayProof
  \quad
  \AxiomC{$!A_1 \land !A_2$}
  \RightLabel{$\Elim{\land}$}
  \UnaryInfC{$!A_i$}
  \DisplayProof
  \;
  i \in \{1,2\}
\end{gather*}
$!A_1 \land !A_1$ is true if both $!A_1$ and $!A_2$ is true;
conversely, $!A_i$ is true if $!A_1 \land !A_2$ is
true ($i \ in \{1,2\}$).
\item[Disjunction]
\begin{gather*}
  \AxiomC{$!A_i$}
  \RightLabel{\Intro{\lor}}
  \UnaryInfC{$!A_1 \lor !A_2$}
  \DisplayProof
  \;
  i \in \{1,2\}
  \quad
  \AxiomC{$!A_1 \lor !A_2$}
  \AxiomC{$\Discharge{!A_1}{u}$}
  \DeduceC{$!C$}
  \AxiomC{$\Discharge{!A_2}{u}$}
  \DeduceC{$!C$}
  \DischargeRule{$\Elim{\lor}$}{u}
  \TrinaryInfC{$!C$}
  \DisplayProof
\end{gather*}
$!A_1 \lor !A_2$ is true if $!A_1$ or $!A_2$ is true; $!C$ is true if
$!A_1 \lor !A_2$ is true, $!C$ is true assuming $!A_1$ and $!C$ is true
assuming $!A_2$.
\item[Implication]
\begin{gather*}
  \AxiomC{$\Discharge{!A}{}$}
  \DeduceC{$!B$}
  \RightLabel{$\Intro{\lif}$}
  \UnaryInfC{$!A \lif !B$}
  \DisplayProof
  \quad
  \AxiomC{$!A$}
  \AxiomC{$!A \lif !B$}
  \RightLabel{$\Elim{\lif}$}
  \BinaryInfC{$!B$}
  \DisplayProof
\end{gather*}
$!A \lif !B$ is true under some assumptions if $!B$ is true under
those assumptions plus $!A$; $!B$ is true if $!A \lif !B$ is true and
$!A$ is true.
\item[Absurdity]
\begin{gather*}
  \AxiomC{$\lfalse$}
  \RightLabel{$\Elim{\lfalse}$}
  \UnaryInfC{$!A$}
  \DisplayProof
\end{gather*}
There is no way for $\lfalse$ to be true; however if we have $\lfalse$
true, then everything is true. Intuitively speaking, $\lfalse$ being
true is so strong, contains so much information, that we can prove
everything true.
\end{enumerate}

With the new machinery, we can now construct proofs in a more
structural and formal manner. Let's redo some proofs in the last
section:

\begin{enumerate}
\item $\Proves !A \lif (!A \lif \lfalse) \lif \lfalse$
  \begin{gather*}
    \AxiomC{$\Discharge{!A}{2}$}
    \AxiomC{$\Discharge{!A \lif \lfalse}{1}$}
    \BinaryInfC{$\lfalse$}
    \DischargeRule{}{1}
    \UnaryInfC{$(!A \lif \lfalse)\lif \lfalse$}
    \DischargeRule{}{2}
    \UnaryInfC{$!A \lif (!A \lif \lfalse) \lif \lfalse$}
    \DisplayProof
  \end{gather*}
\item $\Proves (!A \land !B \lif !C) \lif !A \lif !B \lif !C$
  \begin{gather*}
    \AxiomC{$\Discharge{!A}{2}$}
    \AxiomC{$\Discharge{!B}{1}$}
    \BinaryInfC{$!A \land !B$}
    \AxiomC{$\Discharge{!A \land !B \lif !C}{3}$}
    \BinaryInfC{$!C$}
    \DischargeRule{}{1}
    \UnaryInfC{$!B \lif !C$}
    \DischargeRule{}{2}
    \UnaryInfC{$!A \lif !B \lif !C$}
    \DischargeRule{}{3}
    \UnaryInfC{$(!A \land !B \lif !C) \lif !A \lif !B \lif !C$}
    \DisplayProof
  \end{gather*}

\item $\Proves (!A \land (!A \lif \lfalse))\lif \lfalse$
  \begin{gather*}
    \AxiomC{$\Discharge{!A \land (!A \lif \lfalse)}{1}$}
    \UnaryInfC{$!A$}
    \AxiomC{$\Discharge{!A \land (!A \lif \lfalse)}{1}$}
    \UnaryInfC{$!A \lif \lfalse$}
    \BinaryInfC{$\lfalse$}
    \DischargeRule{}{1}
    \UnaryInfC{$(!A \land (!A \lif \lfalse))\lif \lfalse$}
    \DisplayProof
  \end{gather*}

\item $\Proves ((!A \lor (!A \lif \lfalse))\lif \lfalse)\lif \lfalse$
  \begin{gather*}
    \AxiomC{$\Discharge{!A}{1}$}
    \UnaryInfC{$!A \lor (!A \lif \lfalse)$}
    \AxiomC{$\Discharge{(!A \lor (!A \lif \lfalse))\lif \lfalse}{2}$}
    \BinaryInfC{$\lfalse$}
    \DischargeRule{}{1}
    \UnaryInfC{$!A \lif \lfalse$}
    \UnaryInfC{$!A \lor (!A \lif \lfalse)$}
    \AxiomC{$\Discharge{(!A \lor (!A \lor \lfalse))\lif \lfalse}{2}$}
    \BinaryInfC{$\lfalse$}
    \DischargeRule{}{2}
    \UnaryInfC{$((!A \lor (!A \lif \lfalse))\lif \lfalse)\lif \lfalse$}
    \DisplayProof
  \end{gather*}
  
\end{enumerate}

\end{document}