\documentclass[../../../include/open-logic-section]{subfiles}


\begin{document}
\olsection{Normalization}

In this section we prove that, via some reduction order, any
deduction can be reduced to a normal deduction, which is called the
\emph{normalization property}. As we already noted, we will adopt proof
terms to make the proofs more formal; once we prove the normalization
for proof terms, the normalization for natural deductions follows
immediately by ``propositions as types''.

Firstly we define some functions; complexity of formulas is measured by
length:
\begin{align*}
  len(p) &= 0\\
  len(!A \land !B) &= len(!A) + len(!B) + 1\\
  len(!A \lor !B) &= len(!A) + len(!B) + 1 \\
  len(!A \lif !B) &= len(!A) + len(!B) + 1
\end{align*}

and a function to  measure complexity of redices:
\begin{align*}
  cutrank((\lambd[\typeof{x}{!A}][\typeof{N}{!B}])Q) &= len(!A) + len(!B) + 1 \\
  cutrank(\ande{i}{\andi{\typeof{M}{!A}}{\typeof{N}{!B}}}) &= len(!A) + len(!B) + 1 \\
  cutrank(\ore{\ori{i}{!A_{\bar{i}}}{\typeof{M}{!A_i}}}{\typeof{x_1}{!A_1}}{\typeof{N_1}{!C}}{\typeof{x_2}{!A_2}}{\typeof{N_2}{!C}}) &=
  len(!A) + len(!B) + 1
\end{align*}
it should really be called $redexrank$, but we follow the convention
in other texts.

And the complexity of a proof term is measured by the most complex
redex in it, and $0$ if it's normal:
\begin{equation*}
  maxrank(M) = max(\{cutrank{N} | N \text{is a sub term of} M \text{and
    a redex}\} \cup \{0\})
\end{equation*}

\begin{lem}\ollabel{lem:subst}
  If $\Subst{M}{N^{!A}}{x^{!A}}$ is a redex and $M \not\equiv x$, then one of the
  following cases is true:
  \begin{enumerate}
  \item $M$ is a redex,
  \item $M$ is of the form $\ande{i}{x}$, and $N$ is of the form
    $\andi{P}{Q}$
  \item $M$ is of the form $\ore{x}{x_1}{P_1}{x_2}{P_2}$, and $N$ is of
    the form $\ori{i}{}{Q}$
  \item $M$ is of the form $x Q$, and $N$ is of the form $\lambd[x][P]$
  \end{enumerate}
  also, for the first case, $cutrank(\Subst{M}{N}{x}) = cutrank(M)$;
  for the other cases, $cutrank(\Subst{M}{N}{x}) = len(!A)$.
\end{lem}
\begin{proof}
  Proof by cases.

  \begin{enumerate}
  \item
    If $M$ is a single variable $y$, either $y \equiv x$,  or $y \not\equiv x$
    and $\Subst{y}{N}{x}$ can't be a redex; both cases violate
    the assumption and vacuously justify the lemma.
    
  \item
    If $M$ is of the form $\andi{N_1}{N_2}$, or $\lambd[x][N]$, or
    $\ori{i}{N}$, the substitution cannot be a redex in any case.

  \item If $M$ is of the form $\ande{i}{P}$, we consider two cases. If $P$ is of
    the form $\andi{P_1}{P_2}$, then $M$ is a redex, and apparently
    $\Subst{M}{N}{x}$ is also a redex, and both $cutrank$ should be
    equal. If $P$ is a single variable, it must be $x$ to make the
    substitution a redex, and $N$ must be of the form $\andi{Q_1}{Q_2}$; now
    consider $\Subst{P_i(x)}{pair(P,Q)}{x}$, which is $P_i(pair(P,Q))$,
    whose $cutrank$ equal to $cutrank(x)$, which is $len(!A)$.
    
  \end{enumerate}
  The cases of $\ore{N}{x_1}{N_1}{x_2}{N_2}$ and $P Q$ are similar.
\end{proof}

\begin{rem}
  Recall there are two sides: logic and program. The above lemma, which
  is stated on program side, would be more apparent if we observe from logic
  side; that is, think substitution as replacing assumptions with
  deductions: if a deduction with assumptions replaced has a cut at
  bottom, either the cut exists before the replacing, or there was
  an assumption as a premise of an elimination rule, which is
  replaced by a deduction whose last rule is the corresponding
  introduction rule.
\end{rem}

\begin{lem}
  If $M$ contracts to $M'$, and $cutrank(M) > cutrank(N)$ for all 
  proper sub-term $N$ of $M$ ($N$ is also a redex of course), then
  $cutrank(M) > maxrank(M')$.
\end{lem}
\begin{proof}
  Proof by cases.
  \begin{enumerate}
  \item
    If $M$ is of the form $p_i(pair(M_1, M_2))$, then $M'$ is of the
    form $M_i$; since any sub-term of $M_i$ is also proper sub-term of $M$,
    it follows immediately.

  \item
    If $M$ is of the form $(\lambd[x^{!A}][N])Q^{!A}$, then $M'$ is of the form
    $\Subst{N}{Q^{!A}}{x^{!A}}$ (after the substitution is done); consider each of
    the redices in $M'$, either there is corresponding redex in $N$ with
    equal $cutrank$, which is less than $cutrank(M)$ by assumption; or the $cutrank$ equals to $len(!A)$, which by
    definition is less than $cutrank((\lambd[x^{!A}][N])Q)$.
    
  \item
    If $M$ is of the form $case(in_i N^{!A_i}, x_1^{!A_1}.N_1^{!C}, x_2^{!A_2}.N_2^{!C})$, then $M'$ is
    of the form $\Subst{N_i}{N}{x_i^{!A_i}}$ (after the substitution is
    down); with similar arguments, for each of the recides in $M'$,
    either there is corresponding redex in $N_i$ with equal $cutrank$,
    which is less than $cutrank(M)$ by assumption; or the $cutrank$
    equals to $len(!A_i)$, which by definition is less than
    $cutrank(case(in_i N^{!A_i}, x_1^{!A_1}.N_1^{!C},
    x_2^{!A_2}.N_2^{!C}))$.
  \end{enumerate}
\end{proof}

\begin{thm}
  All proof terms reduces to normal form; all deduction reduces to
  normal deduction.
\end{thm}
\begin{proof}
  The second follows from the first immediately. Prove the first by
  complete induction on $m=maxrank(M)$, where $M$ is a proof term.
  \begin{enumerate}

  \item
    For $m=0$, $M$ is already normal.
    
  \item
    Otherwise, we do another deeper induction on $n$, the number of sub-redices of
    $M$ with $cutrank$ equal to $m$.
    \begin{enumerate}
    \item
      For the base case($n=1$) we select the redex to contract as follows: select any
      redex $N$ such that $m = cutrank(N) > cutrank(P)$ for any proper
      sub-term $P$ (which is also a redex of course); we can find such a redex since we can just select $P$
      if its $cutrank$ is greater.

      Let $N'$ denote the reductum of $N$. Now by lemma $maxrank(N') <
      maxrank(N)$, thus we can see that $n$, the number of redices with
      $cutrank=m$ is decreased. So $m$ is decreased (by $1$ or more), and
      we can apply the outer inductive hypothesis;

    \item
      For induction case($n>1$) the process is similar, except that $n$
      is only decreased to a positive number and thus $m$ does not change. We simply
      apply the inner inductive hypothesis.
    \end{enumerate}
\end{enumerate}
\end{proof}

Actually the normalization of terms is not specific to the reduction
order we gave; we can prove that, whatever order we chose, the term
can always be reduced to normal form; this property is called
\emph{strong normalization}.
\end{document}