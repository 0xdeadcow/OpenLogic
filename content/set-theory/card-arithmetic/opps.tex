\documentclass[../../../include/open-logic-section]{subfiles}

\begin{document}

\olfileid{sth}{card-arithmetic}{opps}
\olsection{Defining the Basic Operations}

Since we do not need to keep track of order, cardinal arithmetic is
rather easier to define than ordinal arithmetic. We will define
addition, multiplication, and exponentiation simultaneously. 

\begin{defn}
When $\cardfont{a}$ and $\cardfont{b}$ are cardinals:
\begin{align*}
	\cardfont{a} \cardplus \cardfont{b} &:= 
	\card{\cardfont{a} \disjointsum \cardfont{b}}\\
	\cardfont{a} \cardtimes \cardfont{b} &:= 
	\card{\cardfont{a} \times \cardfont{b}}\\
	\cardexpo{\cardfont{a}}{\cardfont{b}} &:= 
	\card{\funfromto{\cardfont{b}}{\cardfont{a}}}
\end{align*}
where $\funfromto{X}{Y} = \Setabs{f}{f \text{ is a function } X \to Y}$.
(It is easy to show that $\funfromto{X}{Y}$ exists for any sets $X$
and $Y$; we leave this as an exercise.) 
\end{defn}

\begin{prob}
Prove in $\Zminus$ that $\funfromto{X}{Y}$ exists for any sets $X$
and~$Y$. Working in $\ZF$, compute $\setrank{\funfromto{X}{Y}}$ from
$\setrank{X}$ and $\setrank{Y}$, in the manner of
\olref[sth][ord-arithmetic][using-addition]{rankcomputation}. 
\end{prob}

It might help to explain this definition. Concerning addition: this
uses the notion of disjoint sum, $\disjointsum$, as defined in
\olref[ord-arithmetic][add]{defdissum}; and it is easy
to see that this definition gives the right verdict for finite cases.
Concerning multiplication: \olref[sfr][set][pai]{cardnmprod} tells us
that if $A$ has $n$ members and $B$ has $m$ members then $A \times B$
has $n \cdot m$ members, so our definition simply generalises the idea
to transfinite multiplication. Exponentiation is similar: we are
simply generalising the thought from the finite to the transfinite.
Indeed, in certain ways, transfinite cardinal arithmetic looks much
more like ``ordinary'' arithmetic than does transfinite ordinal
arithmetic:

\begin{prop}\ollabel{cardplustimescommute}
$\cardplus$ and $\cardtimes$ are commutative and associative. 
\end{prop}

\begin{proof}
For commutativity, by
\olref[cardinals][cardsasords]{lem:CardinalsBehaveRight} it
suffices to observe that $\cardeq{(\cardfont{a} \disjointsum
\cardfont{b})}{(\cardfont{b} \disjointsum \cardfont{a})}$ and
$\cardeq{(\cardfont{a} \times \cardfont{b})}{(\cardfont{b} \times
\cardfont{a})}$. We leave associativity as an exercise.
\end{proof}

\begin{prob}
Prove that $\cardplus$ and $\cardtimes$ are associative.
\end{prob}

\begin{prop}
$A$ is infinite iff $\card{A} \cardplus 1 = 1 \cardplus \card{A} = \card{A}$.
\end{prop}

\begin{proof}
As in
\olref[cardinals][classing]{generalinfinitycharacter}, from
\olref[ord-arithmetic][using-addition]{ordinfinitycharacter} and
\olref[cardinals][cardsasords]{lem:CardinalsBehaveRight}. 
\end{proof}

This explains why we need to use different symbols for ordinal versus
cardinal addition/multiplication: these are genuinely \emph{different}
operations. This next pair of results shows that ordinal versus
cardinal exponentiation are also different operations. (Recall that
\olref[z][infinity-again]{defnomega} entails that $2 = \{0,
1\}$):

\begin{lem}\ollabel{lem:SizePowerset2Exp}
$\card{\Pow{A}} = \cardexpo{2}{\card{A}}$, for any $A$.
\end{lem}

\begin{proof}
For each subset $B \subseteq A$, let $\chi_B \in \funfromto{A}{2}$ be given by:
\begin{align*}
	\chi_{B}(x) &:=
	\begin{cases}
		1 & \text{if }x\in B\\
		0 & \text{otherwise.}
	\end{cases}
\end{align*}
Now let $f(B) = \chi_B$; this defines !!a{bijection} $f \colon \Pow{A}
\to \funfromto{A}{2}$. So $\cardeq{\Pow{A}}{\funfromto{A}{2}}$. Hence
$\cardeq{\Pow{A}}{\funfromto{\card{A}}{2}}$, so that
$\card{\Pow{A}} = \card{\funfromto{\card{A}}{2}} =
2^{\card{A}}$.
\end{proof}

This snappy proof essentially subsumes the discussion of
\olref[sfr][siz][red-alt]{sec}. There, we showed how to ``reduce'' the
uncountability of $\Pow{\omega}$ to the uncountability of the set of
infinite binary strings, $\Bin^\omega$. In effect, $\Bin^{\omega}$ is
just $\funfromto{\omega}{2}$; and the preceding proof showed that the
reasoning we went through in \olref[sfr][siz][red-alt]{sec} will go
through using any set~$A$ in place of~$\omega$. The result also yields
a quick fact about cardinal exponentiation:

\begin{cor}\ollabel{cantorcor}
$\cardfont{a} < \cardexpo{2}{\cardfont{a}}$ for any cardinal~$\cardfont{a}$.
\end{cor}

\begin{proof}
From Cantor's Theorem (\olref[sfr][siz][car]{thm:cantor}) and
\olref{lem:SizePowerset2Exp}.
\end{proof}
\noindent
So $\omega < \cardexpo{2}{\omega}$. But note: this is a result about
\emph{cardinal} exponentiation. It should be contrasted with
\emph{ordinal} exponentation, since in the latter case $\omega =
\ordexpo{2}{\omega}$ (see \olref[ord-arithmetic][expo]{sec}).

Whilst we are on the topic of cardinal exponentiation, we can also be
a bit more precise about the ``way'' in which $\Real$ is
!!{nonenumerable}.

\begin{thm}\ollabel{continuumis2aleph0}
$\card{\Real} = \cardexpo{2}{\omega}$
\end{thm}

\begin{proof}[Proof skeleton]
There are plenty of ways to prove this. The most straightforward is to
argue that $\cardle{\Pow{\omega}}{\Real}$ and
$\cardle{\Real}{\Pow{\omega}}$, and then use Schr\"oder-Bernstein to
infer that $\cardeq{\Real}{\Pow{\omega}}$, and
\olref[card-arithmetic][opps]{lem:SizePowerset2Exp} to infer
that $\card{\Real} = \cardexpo{2}{\omega}$. We leave it as an
(illuminating) exercise to define injections $f \colon
\Pow{\omega} \to \Real$ and $g \colon \Real \to \Pow{\omega}$.
\end{proof}

\begin{prob}
Complete the proof of
\olref[sth][card-arithmetic][opps]{continuumis2aleph0}, by
showing that $\cardle{\Pow{\omega}}{\Real}$ and
$\cardle{\Real}{\Pow{\omega}}$.
\end{prob}

\end{document}