\documentclass[../../../include/open-logic-section]{subfiles}

\begin{document}

\olfileid{sth}{card-arithmetic}{fix}
\olsection{$\aleph$-Fixed Points}

In \olref[spine][]{chap}, we suggested that Replacement stands in need
of justification, because it forces the hierarchy to be rather tall.
Having done some cardinal arithmetic, we can give a little
illustration of the height of the hierarchy. 

Evidently $0 < \aleph_0$, and $1 < \aleph_1$, and $2 < \aleph_2$\ldots
and, indeed, the difference in size only gets \emph{bigger} with every
step. So it is tempting to conjecture that $\kappa< \aleph_\kappa$
for every ordinal $\kappa$. 

But this conjecture is \emph{false}, given $\ZFC$. In fact, we can
prove that there are \emph{$\aleph$-fixed-points}, i.e.,
cardinals $\kappa$ such that $\kappa=\aleph_\kappa$. 

\begin{prop}\ollabel{alephfixed}
There is an $\aleph$-fixed-point.
\end{prop}

\begin{proof}
Using recursion, define:
\begin{align*}
	\kappa_0 &= 0\\
	\kappa_{n+1} &= \aleph_{\kappa_n}\\
	\kappa&= \bigcup_{n < \omega}\kappa_n
\end{align*}
Now $\kappa$ is a cardinal by
\olref[cardinals][classing]{unioncardinalscardinal}. But now:
\[
	\kappa= \bigcup_{n < \omega} \kappa_{n+1} = 
	\bigcup_{n < \omega}\aleph_{\kappa_n} = 
	\bigcup_{\alpha < \kappa}\aleph_\alpha = \aleph_\kappa
\]
%	By construction, $\kappa$ is the least cardinal greater than each
%	$\kappa_n$. So $\aleph_\kappa$ is the least cardinal greater than
%	each $\aleph_{\kappa_n}$, and hence greater than each
%	$\kappa_{n+1}$. But equally $\kappa$ is the least cardinal greater
%	than each $\kappa_{n+1} = \aleph_{\kappa_n}$. So $\kappa=
%	\aleph_\kappa$.
\end{proof}

Boolos once wrote an article about exactly the $\aleph$-fixed-point we
just constructed. After noting the existence of $\kappa$, at the start
of his article, he said:
\begin{quote}
	[$\kappa$ is] a \emph{pretty big} number, by the lights of those
	with no previous exposure to  set theory,  so big, it seems to me,
	that  it calls into question the truth of any theory, one of whose
	assertions is the claim that there are at least $\kappa$ objects.
	\citep[p.~257]{Boolos2000}
\end{quote}
And he ultimately concluded his paper by asking:
\begin{quote}
	[do] we  suspect that,  however  it  may  have  been  at  the
	beginning  of  the  story,  by  the  time  we have come thus  far
	the wheels  are  spinning  and  we  are no longer  listening  to
	a description  of  anything  that  is the case?
	\citep[p.~268]{Boolos2000}
\end{quote}
If we have, indeed, outrun ``anything that is the case'', then we must
point the finger of blame directly at Replacement. For it is this
axiom which allows our proof to work. In which case, one assumes,
Boolos would need to revisit the claim he made, a few decades earlier,
that Replacement has ``no undesirable'' consequences (see
\olref[replacement][extrinsic]{sec}).

But is the existence of $\kappa$ so bad? It might help, here, to
consider Russell's \emph{Tristram Shandy paradox}. Tristram Shandy
documents his life in his diary, but it takes him a year to record a
single day. With every passing year, Tristram falls further and
further behind: after one year, he has recorded only one day, and has
lived 364 days unrecorded days; after two years, he has only recorded
two days, and has lived 728 unrecorded days; after three years, he has
only recorded three days, and lived 1092 unrecorded
days \dots\footnote{Forgetting about leap years.} Still, if Tristram
is \emph{immortal}, Tristram will manage to record every day, for he
will record the $n$th day on the $n$th year of his life. And so, ``at
the end of time'', Tristram will have a complete diary. 

Now: why is this so different from the thought that $\alpha$ is
smaller than $\aleph_\alpha$---and indeed, increasingly, desperately
smaller---up until $\kappa$, at which point, we catch up, and $\kappa
= \aleph_\kappa$?

Setting that aside, and assuming we accept $\ZFC$, let's close with a
little more fun concerning fixed-point constructions. The next three
results establish, intuitively, that there is a (non-trivial) point at
which the hierarchy is as wide as it is tall:

\begin{prop}\ollabel{bethfixed}
There is a $\beth$-fixed-point, i.e., a $\kappa$ such that $\kappa=
\beth_\kappa$.
\end{prop}

\begin{proof}
As in \olref{alephfixed}, using ``$\beth$'' in place of ``$\aleph$''. 
\end{proof}

\begin{prop}\ollabel{stagesize}
$\card{V_{\omega+\alpha}} = \beth_{\alpha}$. If $\omega \ordtimes
\omega \leq \alpha$, then $\card{V_\alpha} = \beth_\alpha$.
\end{prop}

\begin{proof}
The first claim holds by a simple transfinite induction. The second
claim follows, since if $\omega \ordtimes \omega \leq \alpha$ then
$\omega + \alpha = \alpha$. To establish this, we use facts about
ordinal arithmetic from \olref[ord-arithmetic][]{chap}. First note
that $\omega \ordtimes \omega = \omega \ordtimes (1 \ordplus \omega) =
(\omega  \ordtimes 1) \ordplus (\omega\ordtimes\omega) = \omega
\ordplus (\omega \ordtimes \omega)$. Now if $\omega \ordtimes \omega
\leq \alpha$, i.e., $\alpha = (\omega\ordtimes\omega) \ordplus \beta$
for some $\beta$, then $\omega \ordplus \alpha = \omega \ordplus
((\omega \ordtimes \omega) \ordplus \beta) = (\omega \ordplus (\omega
\ordtimes \omega)) \ordplus \beta = (\omega \ordtimes \omega) \ordplus
\beta = \alpha$. 
\end{proof}

\begin{cor}
There is a $\kappa$ such that $\card{V_\kappa} = \kappa$.
\end{cor}

\begin{proof}
Let $\kappa$ be a $\beth$-fixed point, as given by \olref{bethfixed}.
Clearly $\omega \ordtimes \omega < \kappa$. So $\card{V_\kappa} =
\beth_\kappa= \kappa$ by \olref{stagesize}.
\end{proof}

There are as many stages beneath $V_\kappa$ as there are !!{element}s
of $V_\kappa$. Intuitively, then, $V_\kappa$ is as wide as it is tall.
This is very Tristram-Shandy-esque: we move from one stage to the next
by taking \emph{powersets}, thereby making our hierarchy \emph{much}
bigger with each step. But, ``in the end'', i.e., at stage $\kappa$,
the hierarchy's width catches up with its height. 

One might ask: \emph{How often does the hierarchy's width match its
height?} The answer is: \emph{As often as there are ordinals.} But
this needs a little explanation. 

We define a term $\tau$ as follows. For any $A$, let:
\begin{align*}
	\tau_0(A) & \defis \card{A}\\
	\tau_{n+1}(A) & \defis \beth_{\tau_n(A)}\\
	\tau(A) & \defis \bigcup_{n < \omega}\tau_n(A)
\intertext{As in \olref{bethfixed}, $\tau(A)$ is a
$\beth$-fixed point for any $A$, and trivially $\card{A} < \tau(A)$.
So now consider this recursive definition:}
	W_0 &\defis 0\\
	W_{\alpha + 1} & \defis \tau(W_\alpha)\\
	W_\alpha & \defis \bigcup_{\beta < \alpha} W_\beta
	\text{, when $\alpha$ is a limit}
\end{align*}
The construction is defined for all ordinals. Intuitively, then,
$W$ is ``!!a{injection}'' from the ordinals to $\beth$-fixed points.
And, exactly as before, 
$V_{W_\alpha}$ is as wide as it is tall, for any $\alpha$.

\end{document}