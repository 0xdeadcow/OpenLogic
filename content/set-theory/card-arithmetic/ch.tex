\documentclass[../../../include/open-logic-section]{subfiles}

\begin{document}

\olfileid{sth}{card-arithmetic}{ch}

\olsection{The Continuum Hypothesis}

The previous result hints (correctly) that cardinal exponentiation
would be quite \emph{easy}, if infinite cardinals are guaranteed to
``play straightforwardly'' with powers of $2$, i.e., (by
\olref[opps]{lem:SizePowerset2Exp}) with taking powersets. But we
cannot assume that infinite cardinals \emph{do} play straightforwardly 
powersets. 

To start unpacking this, we introduce some nice notation.

\begin{defn}
Where $\cardsucc{\cardfont{a}}$ is the least cardinal strictly greater
than $\cardfont{a}$, we define two infinite sequences:
\begin{align*}
	\aleph_{0} &:= \omega & 		
	\beth_{0} &:= \omega\\
	\aleph_{\alpha \ordplus 1} &:= \cardsucc{(\aleph_{\alpha})} &
	\beth_{\alpha+1} &:= \cardexpo{2}{\beth_{\alpha}}\\
	\aleph_{\alpha} &:= \bigcup_{\beta< \alpha} \aleph_{\beta} &
	\beth_{\alpha} &:= \bigcup_{\beta < \alpha}\beth_{\beta} & \text{when $\alpha$ is a limit ordinal}.
	\end{align*}
\end{defn}

The definition of $\cardsucc{\cardfont{a}}$ is in order, since
\olref[cardinals][classing]{lem:NoLargestCardinal} tells us that, for each
cardinal $\cardfont{a}$, there is some cardinal greater than
$\cardfont{a}$, and Transfinite Induction guarantees that there is a
\emph{least} cardinal greater than $\cardfont{a}$. The rest of the
definition of $\cardfont{a}$ is provided by transfinite recursion. 

Cantor introduced this ``$\aleph$'' notation; this is \emph{aleph},
the first letter in the Hebrew alphabet and the first letter in the
Hebrew word for ``infinite''. Peirce introduced the ``$\beth$''
notation; this is \emph{beth}, which is the second letter in the
Hebrew alphabet.\footnote{Peirce used this notation in a letter to
Cantor of December 1900. Unfortunately, Peirce also gave a bad
argument there that $\beth_\alpha$ does not exist for $\alpha \geq
\omega$.} Now, these notations provide us with infinite cardinals.

\begin{prop}
$\aleph_\alpha$ and $\beth_\alpha$ are cardinals, for every
ordinal $\alpha$. 
\end{prop}

\begin{proof}
Both results hold by a simple transfinite induction. $\aleph_0 =
\beth_0 = \omega$ is a cardinal by
\olref[cardinals][classing]{omegaisacardinal}. Assuming $\aleph_\alpha$ and
$\beth_\alpha$ are both cardinals, $\aleph_{\alpha+1}$ and
$\beth_{\alpha+1}$ are explicitly defined as cardinals. And the union
of a set of cardinals is a cardinal, by
\olref[cardinals][classing]{unioncardinalscardinal}.
\end{proof}
\noindent
Moreover, every infinite cardinal is an $\aleph$:

\begin{prop}
If $\cardfont{a}$ is an infinite cardinal, then $\cardfont{a} =
\aleph_\gamma$ for some unique $\gamma$.
\end{prop}

\begin{proof}
By transfinite induction on cardinals. For induction, suppose that if
$\cardfont{b} < \cardfont{a}$ then $\cardfont{b} =
\aleph_{\gamma_\cardfont{b}}$. If $\cardfont{a} =
\cardsucc{\cardfont{b}}$ for some $\cardfont{b}$, then $\cardfont{a} =
\cardsucc{(\aleph_{\gamma_\cardfont{b}})}=
\aleph_{\gamma_\cardfont{b}+1}$. If $\cardfont{a}$ is not the
successor of any cardinal, then since cardinals are ordinals
$\cardfont{a} = \bigcup_{\cardfont{b} < \cardfont{a}} \cardfont{b} =
\bigcup_{\cardfont{b} < \cardfont{a}}{\aleph_{\gamma_\cardfont{b}}}$,
so $\cardfont{a} = \aleph_\gamma$ where $\gamma =
\bigcup_{\cardfont{b} < \cardfont{a}}\gamma_\cardfont{b}$. 
\end{proof}

Since every infinite cardinal is an $\aleph$, this prompts us to ask:
is every infinite cardinal a~$\beth$? Certainly if that \emph{were}
the case, then the infinite cardinals would ``play straightforwardly''
with the operation of taking powersets. Indeed, we would have the
following:

\
\\\emph{Generalized Continuum Hypothesis} (GCH). $\aleph_\alpha  = \beth_\alpha$, for all $\alpha$. 

\
\\Moreover, if GCH held, then we could make some considerable
simplifications with cardinal exponentiation. In particular, we could
show that when $\cardfont{b} < \cardfont{a}$, the value of
$\cardexpo{\cardfont{a}}{\cardfont{b}}$ is trapped by
$\cardfont{a}\leq \cardexpo{\cardfont{a}}{\cardfont{b}} \leq
\cardsucc{\cardfont{a}}$. We could then go on to give precise
conditions which determine which of the two possibilities obtains
(i.e., whether $\cardfont{a} = \cardexpo{\cardfont{a}}{\cardfont{b}}$
or $\cardexpo{\cardfont{a}}{\cardfont{b}} =
\cardsucc{\cardfont{a}}$).\footnote{The condition is dictated by
\emph{cofinality}.}

But GCH is a \emph{hypothesis}, not a \emph{theorem}. In fact,
\citet{Godel1938} proved that if $\ZFC$ is consistent, then so is
$\ZFC + \text{GCH}$. But it later turned out that we can equally add
$\lnot$GCH to $\ZFC$. Indeed, consider the simplest non-trivial
\emph{instance} of GCH, namely: 

\
\\\emph{Continuum Hypothesis} (CH). $\aleph_1 = \beth_1$. 

\
\\
\citet{Cohen1963} proved that if $\ZFC$ is consistent then so is $\ZFC
+ \lnot\text{CH}$. So the Continuum Hypothesis is independent from $\ZFC$.

The Continuum Hypothesis is so-called, since ``the continuum'' is
another name for the real line, $\Real$.
\olref[opps]{continuumis2aleph0} tells us that $\card{\Real} =
\beth_1$. So the Continuum Hypothesis states that there is no cardinal
between the cardinality of the natural numbers, $\aleph_0 = \beth_0$,
and the cardinality of the continuum, $\beth_1$.

Given the \emph{independence} of (G)CH from $\ZFC$, what should say
about their \emph{truth}? Well, there is \emph{much} to say. Indeed,
and much fertile recent work in set theory has been directed at
investigating these issues. But two very quick points are certainly worth
emphasising. 

First: it does not \emph{immediately} follow from these formal
independence results that either GCH or CH is \emph{indeterminate} in
truth value. After all, maybe we just need to add more axioms, which
strike us as natural, and which will settle the question one way or
another. G\"odel himself suggested that this was the right response. 

Second: the independence of CH from $\ZFC$ is certainly
\emph{striking}, but it is certainly not \emph{incredible} (in the
literal sense). The point is simply that, for all $\ZFC$ tells us,
moving from cardinals to their successors may involve a less blunt
tool than simply taking powersets.
 %The operation of taking powersets moves us from one stage of the
 %hierarchy to its successor stage (from $V_{\alpha}$ to
 %$V_{\alpha+1}$). 

With those two observations made, if you want to know more, you will
now have to turn to the various philosophers and mathematicians with
horses in the race. (Though you may want to start with the very nice
discussion in \citeauthor{Potter2004} \citeyear[\S15.6]{Potter2004}.) 

\end{document}