\documentclass[../../../include/open-logic-section]{subfiles}

\begin{document}

\olfileid{sth}{ord-arithmetic}{mult}

\olsection{Ordinal Multiplication}

We now turn to ordinal multiplication, and we approach this much like
ordinal addition. So, suppose we want to multiply $\alpha$ by~$\beta$.
To do this, you might imagine a rectangular grid, with width $\alpha$
and height $\beta$; the product of $\alpha$ and $\beta$ is now the
result of moving along each row, then moving through the next
row\ldots until you have moved through the entire grid. Otherwise put,
the product of $\alpha$ and $\beta$ arises by replacing \emph{each}
element in $\beta$ with a copy of $\alpha$.  

To make this formal, we simply use the reverse lexicographic ordering
on the Cartesian product of $\alpha$ and $\beta$:

\begin{defn}
For any ordinals $\alpha, \beta$, their product $\alpha \ordtimes \beta = \ordtype{\alpha \times \beta, \rlexless}$.
\end{defn}

We must again confirm that this is a well-formed definition:

\begin{lem}\ollabel{ordtimeslessiswo}
$\tuple{\alpha \times \beta, \rlexless}$ is a well-order, for any
ordinals $\alpha$ and $\beta$.
\end{lem}

\begin{proof}
Exactly as for \olref[add]{ordsumlessiswo}.
\end{proof}

And it is also not hard to prove that multiplication behaves thus:

\begin{lem}\ollabel{ordtimesrecursion}
For any ordinals $\alpha, \beta$:
\begin{align*}
	\alpha \ordtimes 0 &= 0\\
	\alpha \ordtimes (\beta \ordplus 1) &= 
		(\alpha \ordtimes \beta) \ordplus \alpha\\
	\alpha  \ordtimes \beta &= 
		\supstrict_{\delta < \beta}(\alpha \ordtimes \delta) && 
		\text{when $\beta$ is a limit ordinal}.
\end{align*}
\end{lem}

\begin{proof}
Left as an exercise.
\end{proof}

Indeed, just as in the case of addition, we could have defined ordinal
multiplication via these recursion equations, rather than offering a
direct definition. Equally, as with addition, certain behaviour is
familiar:

\begin{lem}\ollabel{ordinalmultiplicationisnice}
If $\alpha, \beta, \gamma$ are ordinals, then:
\begin{enumerate}
	\item\ollabel{ordtimes1} if $\alpha \neq 0$ and $\beta < \gamma$,
	then $\alpha \ordtimes \beta < \alpha \ordtimes \gamma$;
	\item\ollabel{ordtimes2} if $\alpha \neq 0$ and $\alpha \ordtimes
	\beta = \alpha\ordtimes\gamma$, then $\beta = \gamma$;
	\item\ollabel{ordtimes3}  $\alpha \ordtimes (\beta \ordtimes
	\gamma) = (\alpha \ordtimes \beta) \ordtimes \gamma$;
	\item\ollabel{ordtimes4}  If $\alpha \leq \beta$, then $\alpha
	\ordtimes \gamma \leq \beta \ordtimes \gamma$;
	\item\ollabel{ordtimes5}  $\alpha \ordtimes (\beta \ordplus
	\gamma) = (\alpha \ordtimes \beta )\ordplus (\alpha\ordtimes
	\gamma)$.
\end{enumerate}
\end{lem}

\begin{proof}
Left as an exercise.
\end{proof}

You can prove (or look up) other results, to your heart's content.
But, given
\olref[ord-arithmetic][add]{ordsumnotcommute}, the
following should not come as a surprise:

\begin{prop}
Ordinal multiplication is not commutative: $2 \ordtimes \omega  =
\omega < \omega \ordtimes 2$
\end{prop}

\begin{proof}
$2 \ordtimes \omega = \supstrict_{n < \omega}(2\ordtimes  n) = \omega \in \supstrict_{n < \omega}(\omega \ordplus n) = \omega \ordplus \omega = \omega \ordtimes 2$.
\end{proof}

Again, the intuitive rationale is quite straightforward. To compute $2
\ordtimes \omega$, you replace each natural number with two entities.
You would get the same order type if you simply inserted all the
``half'' numbers into the natural numbers, i.e., you considered the
natural ordering on $\Setabs{\nicefrac{n}{2}}{n \in \omega}$. And, put
like that, the order type is plainly the same as that of $\omega$
itself. But, to compute $\omega \ordtimes 2$, you place down two
copies of $\omega$, one after the other. 

\begin{prob}
Prove
\olref[sth][ord-arithmetic][mult]{ordtimeslessiswo},
\olref[sth][ord-arithmetic][mult]{ordtimesrecursion},
and
\olref[sth][ord-arithmetic][mult]{ordinalmultiplicationisnice}
\end{prob}

\end{document}