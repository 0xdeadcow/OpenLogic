\documentclass[../../../include/open-logic-section]{subfiles}

\begin{document}

\olfileid{sth}{spine}{zf}
\olsection{$\Z$ and $\ZF$: A Milestone}

With Foundation, we reach another important milestone. We have
considered theories $\Zminus$ and $\ZFminus$, which we said were
certain theories ``minus'' a certain something. That certain something
is Foundation. So:

\begin{defn}
The theory $\Z$ adds Foundation to $\Zminus$. So its axioms are
Extensionality, Union, Pairs, Powersets, Infinity, Foundation, and all
instances of the Separation scheme.

The theory $\ZF$ adds Foundation to $\ZFminus$. Otherwise put, $\ZF$
adds all instances of Replacement to~$\Z$.
\end{defn}

Still, one question might have occurred to you. If Regularity is
equivalent over $\ZFminus$ to Foundation, and Regularity's
justification is clear, why bother to go around the houses, and take
Foundation as our basic axiom, rather than Regularity? 

Setting aside historical reasons (to do with who formulated what and
when), the basic reason is that Foundation can be presented without
employing the definition of the~$V_\alpha$s. That definition relied
upon all of the work of \olref[recursion]{sec}: we
needed to prove Transfinite Recursion, to show that it was justified.
But our proof of Transfinite Recursion employed \emph{Replacement}.
So, whilst Foundation and Regularity are equivalent modulo $\ZFminus$,
they are not equivalent modulo $\Zminus$. 

Indeed, the matter is more drastic than this simple remark suggests.
Though it goes well beyond this book's remit, it turns out that both
$\Zminus$ and $\Z$ are too weak to define the $V_\alpha$s. So, if you
are working only in $\Z$, then Regularity (as we have formulated it)
does not even make \emph{sense}. This is why our official axiom is
Foundation, rather than Regularity. 

From now on, we will work in $\ZF$ (unless otherwise stated), without
any further comment. 

\end{document}