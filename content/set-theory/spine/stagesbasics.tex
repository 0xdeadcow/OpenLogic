\documentclass[../../../include/open-logic-section]{subfiles}

\begin{document}

\olfileid{sth}{spine}{Valphabasic}
\olsection{Basic Properties of Stages}

To bring out the foundational importance of the definition of the
$V_\alpha$s, we will present a few basic results about them. We start with a definition:\footnote{There's no standard terminology for ``potent''; this is the name used by \citet{ButtonLT1}.}
\begin{defn}
	The set $A$ is \emph{potent} iff $\forall x((\exists y \in A)x \subseteq y \lif x \in A$. 
\end{defn}
\begin{lem}\ollabel{Valphabasicprops}
For each ordinal $\alpha$:
\begin{enumerate}
	\item\ollabel{Valphatrans} Each $V_\alpha$ is transitive.
	\item\ollabel{Valphapotent} Each $V_\alpha$ is potent.
	\item\ollabel{Valphacum} If $\gamma \in \alpha$, then $V_\gamma
	\in V_\alpha$ (and hence also $V_\gamma \subseteq V_\alpha$ by
	\olref{Valphatrans})
\end{enumerate}
\end{lem}

\begin{proof}
We prove this by a (simultaneous) transfinite induction.  For
induction, suppose that \olref{Valphatrans}--\olref{Valphacum} holds
for each ordinal $\beta < \alpha$. 

The case of $\alpha = \emptyset$ is trivial. 

Suppose $\alpha = \ordsucc{\beta}$. To show \olref{Valphacum}, if
$\gamma \in \alpha$ then $V_\gamma \subseteq V_\beta$ by hypothesis,
so $V_\gamma \in \Pow{V_\beta} = V_\alpha$. To show
\olref{Valphapotent}, suppose $A \subseteq B \in V_\alpha$ i.e., $A
\subseteq B \subseteq V_\beta$; then $A \subseteq V_\beta$ so $A \in
V_\alpha$. To show \olref{Valphatrans}, note that if $x \in A \in
V_\alpha$ we have $A \subseteq V_\beta$, so $x \in V_\beta$, so $x
\subseteq V_\beta$ as $V_\beta$ is transitive by hypothesis, and so $x
\in V_\alpha$. 

Suppose $\alpha$ is  a limit ordinal. To show \olref{Valphacum}, if
$\gamma \in \alpha$ then $\gamma \in \ordsucc{\gamma} \in \alpha$, so
that $V_\gamma \in V_{\ordsucc{\gamma}}$ by assumption, hence
$V_\gamma \in \bigcup_{\beta \in \alpha} V_\beta = V_\alpha$. To show
\olref{Valphatrans} and \olref{Valphapotent}, just observe that a
union of transitive (respectively, potent) sets is transitive
(respectively, potent). 
\end{proof}

\begin{lem}\ollabel{Valphanotref}
For each ordinal $\alpha$, $V_\alpha \notin V_\alpha$.
\end{lem}

\begin{proof}
By transfinite induction. Evidently $V_\emptyset \notin V_\emptyset$. 

If $V_{\ordsucc{\alpha}} \in V_{\ordsucc{\alpha}} = \Pow{V_\alpha}$,
then $V_{\ordsucc{\alpha}} \subseteq V_\alpha$; and since $V_\alpha
\in V_{\ordsucc{\alpha}}$ by \olref{Valphabasicprops}, we have
$V_\alpha \in V_\alpha$. Conversely: if $V_\alpha \notin V_\alpha$
then $V_{\ordsucc{\alpha}} \notin V_{\ordsucc{\alpha}}$

If $\alpha$ is a limit and $V_\alpha \in V_\alpha = \bigcup_{\beta \in
\alpha}V_\beta$, then $V_\alpha \in V_\beta$ for some $\beta \in
\alpha$; but then also $V_\beta \in V_\alpha$ so that $V_\beta \in
V_\beta$ by \olref{Valphabasicprops} (twice). Conversely, if $V_\beta
\notin V_\beta$ for all $\beta \in \alpha$, then $V_\alpha \notin
V_\alpha$.
\end{proof}

\begin{cor}
For any ordinals $\alpha, \beta$: $\alpha \in \beta$ iff $V_\alpha \in V_\beta$
\end{cor}

\begin{proof}
\Olref{Valphabasicprops} gives one direction. Conversely, suppose $V_\alpha \in V_\beta$. Then $\alpha \neq \beta$ by \olref{Valphanotref}; and $\beta \notin \alpha$, for otherwise we would have $V_\beta \in V_\alpha$ and hence $V_\beta \in V_\beta$ by \olref{Valphabasicprops} (twice), contradicting \olref{Valphanotref}. So $\alpha \in \beta$ by Trichotomy.
\end{proof}
\noindent
All of this allows us to think of each $V_\alpha$ as the $\alpha$th
stage of the hierarchy. Here is why.

Certainly our $V_\alpha$s can be thought of as being formed in an
\emph{iterative} process, for our use of ordinals tracks the notion of
iteration. Moreover, if one stage is formed before the other, i.e.,
$V_\beta \in V_\alpha$, i.e., $\beta \in \alpha$, then our process of
formation is \emph{cumulative}, since $V_\beta \subseteq V_\alpha$.
Finally, we are indeed forming \emph{all} possible collections of sets
that were available at any earlier stage, since any successor stage
$V_{\ordsucc{\alpha}}$ is the power-set of its predecessor $V_\alpha$.

In short: with $\ZFminus$, we are \emph{almost} done, in articulating
our vision of the cumulative-iterative hierarchy of sets. (Though, of
course, we still need to justify Replacement.)

\end{document}