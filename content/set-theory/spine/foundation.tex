\documentclass[../../../include/open-logic-section]{subfiles}

\begin{document}

\olfileid{sth}{spine}{foundation}

\olsection{Foundation}

We have almost articulated the vision of the iterative-cumulative
hierarchy in~$\ZFminus$. ``Almost'', because there is a wrinkle. Nothing in
$\ZFminus$ guarantees that \emph{every} set is in some $V_\alpha$,
i.e., that every set is formed at some stage. 

Now, there is a fairly straightforward (mathematical) sense in which
we don't \emph{care} whether there are sets outside the hierarchy. (If
there are any there, we can simply ignore them.) But we have motivated
our \emph{concept} of set with the thought that every set is formed at
some stage (see \stageshier{} in \olref[z][story]{sec}.) So
we will want to preclude the possibility of sets which fall outside of
the hierarchy. Accordingly, we must add a new axiom, which ensures
that every set occurs somewhere in the hierarchy. 

Since the $V_\alpha$s are our stages, we might simply consider adding
the following as an axiom:

\
\\\emph{Regularity.} $\forall A \exists \alpha\, A \subseteq V_\alpha$

\ \\This is \citeauthor{VonNeumann1925}'s approach
(\citeyear{VonNeumann1925}). However, for reasons I will explain in
the next section, we will instead adopt an alternative axiom:

\begin{axiom}[Foundation]
$(\forall A \neq \emptyset)(\exists B \in A)A \cap B = \emptyset$.
\end{axiom}

With some effort, we can show (in $\ZFminus$) that Foundation entails Regularity:
\begin{defn}
For each set $A$, let:
\begin{align*}
	\text{cl}_0(A) &= A,\\
	\text{cl}_{n+1}(A) &= \bigcup \text{cl}_n(A),\\
	\text{trcl}(A) &= \bigcup_{n < \omega} \text{cl}_{n}(A).
\end{align*}
\end{defn}

We call $\text{trcl}(A)$ the \emph{transitive closure} of $A$. The name is apt:

\begin{prop}\ollabel{subsetoftrcl}
$A \subseteq \trcl{A}$ and $\trcl{A}$ is a transitive set. 
\end{prop}

\begin{proof}
Evidently $A = \text{cl}_0(A) \subseteq \text{trcl}(A)$. And if $x
\in b \in \trcl{A}$,  then $b \in \text{cl}_n(A)$ for some $n$, so $x
\in \text{cl}_{n+1}(A) \subseteq \text{trcl}(A)$. 
\end{proof}

\begin{lem}\ollabel{lem:TransitiveWellFounded}
If $A$ is a transitive set, then there is some $\alpha$ such that $A
\subseteq V_\alpha$.
\end{lem}

\begin{proof}
Recalling the definition of ``$\supstrict(X)$'' from
\olref[ordinals][opps]{defsupstrict}, define:
\begin{align*}
	D  &= \Setabs{x \in A}{\forall \delta\ x \nsubseteq V_\delta}\\
	\alpha &= \supstrict\Setabs{\delta}{(\exists x \in A)
	(x \subseteq V_\delta \land (\forall \gamma \in \delta)x \nsubseteq V_\gamma)}
\end{align*}
Suppose $D = \emptyset$. So if $x \in A$, then there is some $\delta
\in \alpha$ such that $x \subseteq V_\delta$, so $x \in V_\alpha$ by
\olref[spine][Valphabasic]{Valphabasicprops}. Hence $A \subseteq
V_{\alpha}$, as required. 

So it suffices to show that $D = \emptyset$. For reductio, suppose
otherwise. By Foundation, there is some $B \in D$ such that $D \cap B
= \emptyset$. If $x \in B$ then $x \in A$, since $A$ is transitive,
and since $x \notin D$, it follows that $\exists \delta\ x \subseteq
V_\delta$. So now let
\[
	\beta = \supstrict\Setabs{\delta}{(\exists x \in b)(x \subseteq V_\delta \land (\forall \gamma < \delta)x \nsubseteq V_\gamma)}.
\]
As before, $B \subseteq V_\beta$, contradicting the claim that $B \in
D$.	
\end{proof}

\begin{thm}\ollabel{zfentailsregularity}
Regularity holds.
\end{thm}

\begin{proof}
Fix $A$; now $A \subseteq \trcl{A}$ by \olref{subsetoftrcl}, which is transitive. So there is some $\alpha$ such that $A \subseteq \trcl{A} \subseteq V_\alpha$ by \olref{lem:TransitiveWellFounded}
\end{proof}

These results show that $\ZFminus$ proves the conditional
$\emph{Foundation}\Rightarrow\emph{Regularity}$. In
\olref[spine][rank]{zfminusregularityfoundation}, we will show that
$\ZFminus$ proves $\emph{Regularity}\Rightarrow\emph{Foundation}$. As
such, Foundation and Regularity are \emph{equivalent} (modulo
$\ZFminus$). But this means that, given $\ZFminus$, we can justify
Foundation by noting that it is equivalent to Regularity. And we can
justify Regularity immediately on the basis of \stageshier{}. 

\end{document}