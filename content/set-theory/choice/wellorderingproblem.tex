\documentclass[../../../include/open-logic-section]{subfiles}

\begin{document}

\olfileid{sth}{choice}{woproblem}
\olsection{The Well-Ordering Problem}

Evidently rather a lot hangs on whether we accept Well-Ordering. But
the discussion of this principle has tended to focus on an equivalent
principle, the Axiom of Choice. So we will now turn our attention to
that (and prove the equivalence). 

In \citeyear{Cantor1883}, Cantor expressed his support for the Axiom
of Well-Ordering, calling it ``a law of thought which appears to me to
be fundamental, rich in its consequences, and particularly remarkable
for its general validity'' (cited in \citeauthor{Potter2004}
\citeyear[p.~243]{Potter2004}). But Cantor ultimately became convinced
that the ``Axiom'' was in need of proof. So did the mathematical
community. 

The problem was ``solved'' by Zermelo in \citeyear{Zermelo1904}. To
explain his solution, we need some definitions. 

\begin{defn}
A function $f$ is a \emph{choice function} iff $f(x) \in x$ for all $x \in \dom{f}$. We say that $f$ is a \emph{choice function for $A$} iff $f$ is a choice function with $\dom{f} = A \setminus \{\emptyset\}$.
\end{defn}

Intuitively, for every (non-empty) set $x \in A$, a choice function
for $A$ \emph{chooses} a particular element, $f(x)$, from $x$. The
Axiom of Choice is then:

\begin{axiom}[Choice]
	Every set has a choice function.
\end{axiom}

Zermelo showed that Choice entails well-ordering, and vice versa:

\begin{thm}[in $\ZF$]\ollabel{thmwochoice}
Well-Ordering and Choice are equivalent.
\end{thm}

\begin{proof}
\emph{Left-to-right.} Let $A$ be a set of sets. Then $\bigcup A$
exists by the Axiom of Union, and so by Well-Ordering there is some
$<$ which well-orders $\bigcup A$. Now let $f(x) = \text{the $<$-least
member of }x$. This is a choice function for $A$.

\emph{Right-to-left.} Fix $A$. By Choice, there is a choice function,
$f$,  for $\Pow{A} \setminus \{\emptyset\}$. Using Transfinite
Recursion, define a function:
\begin{align*}
	g(0) &= f(A)\\
	g(\alpha) &= 
		\begin{cases}
			\text{stop!{}} &\text{if }A = \funimage{g}{\alpha}\\
			f(A \setminus \funimage{g}{\alpha}) & \text{otherwise}\\	
		\end{cases}
\end{align*}
The indication to ``stop!'' is just a shorthand for what would
otherwise be a more long-winded definition. That is, when $A =
\funimage{g}{\alpha}$ for the first time, let $g(\delta) = A$ for all
$\delta \leq \alpha$. Now, in the first instance, we can only be sure that this defines a \emph{term} (see the remarks after \olref[sth][spine][recursion]{transrecursionschema}); but we will show that we indeed have a function.

Since $f$ is a choice function, for each $\alpha$ (when defined) we have $g(\alpha) =
f(A \setminus \funimage{g}{\alpha}) \in A \setminus
\funimage{g}{\alpha}$; i.e., $g(\alpha) \notin \funimage{g}{\alpha}$.
So if $g(\alpha) = g(\beta)$ then $g(\beta) \notin
\funimage{g}{\alpha}$, i.e., $\beta \notin \alpha$, and similarly
$\alpha \notin \beta$. So $\alpha = \beta$, by Trichotomy. So $g$ is
!!{injective}.

Next, observe that we do stop!{}, i.e.\ that there is some (least) ordinal $\alpha$ such that $A = g[\alpha]$. For suppose otherwise; then as $g$ is !!{injective} we would have $\cardless{\alpha}{\Pow{A} \setminus \{\emptyset\}}$ for
every ordinal $\alpha$, contradicting \olref[hartogs]{HartogsLemma}. Hence also $\ran{g} = A$.

Assembling these facts, $g$ is !!a{bijection} from some ordinal to $A$. Now $g$ can be used to well-order $A$.
\end{proof}

So Well-Ordering and Choice stand or fall together. But the question
remains: do they stand or fall?

\end{document}