\documentclass[../../../include/open-logic-section]{subfiles}

\begin{document}

\olfileid{sth}{replacement}{refproofs}
\olsection{Appendix: Results surrounding Replacement}

In this section, we will prove Reflection within $\ZF$. We will also
prove a sense in which Reflection is equivalent to Replacement. And we
will prove an interesting consequence of all this, concerning the
strength of Reflection/Replacement. \emph{Warning: this is easily the
most advanced bit of mathematics in this textbook.} 

We'll start with a lemma which, for brevity, employs the notational
device of \emph{overlining} to deal with sequences of variables or
objects. So: ``$\overline{a}_k$'' abbreviates ``$a_{k_1}$, \dots,
$a_{k_n}$'', where $n$ is determined by context.

\begin{lem}\ollabel{lemreflection}
For each $1 \leq i \leq k$, let $\phi_i(\overline{v}_i, x)$ be a
formula. Then for each $\alpha$
there is some $\beta > \alpha$ such that, for any $\overline{a}_1,
\ldots, \overline{a}_k \in V_\beta$ and each $1 \leq i \leq k$:
\[
	\exists x\phi_i(\overline{a}_i, x) \rightarrow (\exists x \in V_\beta) \phi_i(\overline{a}_i, x)
\]
\end{lem}

\begin{proof}
We define a term $\mu$ as follows: $\mu(\overline{a}_1, \ldots,
\overline{a}_k)$ is the least stage, $V$, which satisfies all of the
following conditionals, for $1 \leq i \leq k$:
\[
\exists x\phi_i(\overline{a}_i, x) \rightarrow (\exists x \in V) \phi_i(\overline{a}_i, x))
\]
It is easy to confirm that $\mu(\overline{a}_1, \ldots, \overline{a}_k)$ exists for all $\overline{a}_1, \ldots, \overline{a}_k$. Now, using Replacement and our recursion theorem, define:
\begin{align*}
	S_0 & = V_{\alpha+1}\\
	S_{n+1} & = S_n \cup \bigcup
	\Setabs{\mu(\overline{a}_1, \ldots, \overline{a}_k)}
	{\overline{a}_1, \ldots, \overline{a}_k \in S_n} \\
	S &= \bigcup_{m < \omega} S_n.
\end{align*}
Each $S_n$, and hence $S$ itself, is a stage after $V_\alpha$. Now fix
$\overline{a}_1$, \dots,~$\overline{a}_k \in S$; so there is some $n <
\omega$ such that $\overline{a}_1$, \dots, $\overline{a}_k \in S_n$.
Fix some $1 \leq i \leq k$, and suppose that $\exists x
\phi_i(\overline{a}_i,x)$. So $(\exists x \in \mu(\overline{a}_1,
\ldots, \overline{a}_k))\phi_i(\overline{a}_i, x)$ by construction, so
$(\exists x \in S_{n+1})\phi_i(\overline{a}_i, x)$ and hence
$(\exists x \in S)\phi_i(\overline{a}_i, x)$. So $S$ is our $V_\beta$.
\end{proof}
\noindent 
We can now prove \olref[sth][replacement][ref]{reflectionschema} quite
straightforwardly:

\begin{proof}[Proof] 
Fix $\alpha$. Without loss of generality, we can assume $\phi$'s only
connectives are $\exists$, $\lnot$ and $\land$ (since these are
expressively adequate). Let $\psi_1, \ldots, \psi_k$ enumerate each of
$\phi$'s subformulas according to complexity, so that $\psi_k = \phi$.
By \olref{lemreflection}, there is a $\beta > \alpha$ such that, for
any $\overline{a}_i \in V_\beta$ and each $1 \leq i \leq k$:
\begin{align}\label{reflectionnicelybehaved}
	\exists x\psi_i(\overline{a}_i, x) \rightarrow 
	(\exists x \in V_\beta) \psi_i(\overline{a}_i, x)\tag{*}
\end{align}
By induction on complexity of $\psi_i$, we will show that
$\psi_i(\overline{a}_i) \leftrightarrow
\psi_i^{V_\beta}(\overline{a}_i)$, for any  $\overline{a}_i \in
V_\beta$. 	If $\psi_i$ is atomic, this is trivial. The biconditional
also establishes that, when $\psi_i$ is a negation or conjunction of
subformulas satisfying this property, $\psi_i$ itself satisfies this
property. So the only interesting case concerns quantification. Fix
$\overline{a}_i \in V_\beta$; then:
\begin{align*}
	(\exists x \psi_i(\overline{a}_i, x))^{V_\beta}
	&\text{ iff }
	(\exists x \in V_\beta)\psi_i^{V_\beta}(\overline{a}_i, x)
	&&\text{by definition}\\
	&\text{ iff }
	(\exists x \in V_\beta)\psi_i(\overline{a}_i,  x)
	&&\text{by hypothesis}\\
	&\text{ iff }
	\exists x \psi_i(\overline{a}_i, x)
	&&\text{by \eqref{reflectionnicelybehaved}}
\end{align*}
This completes the induction; the result follows as $\psi_k = \phi$.
\end{proof}

We have proved Reflection in $\ZF$. Our proof essentially
followed \citet{Montague1961}. We now want to prove in $\Z$ that
Reflection entails Replacement. The proof follows \citet{Levy1960},
but with a simplification. 

Since we are working in $\Z$, we cannot present Reflection in exactly
the form given above. After all, we formulated Reflection using the
``$V_\alpha$'' notation, and that cannot be defined in $\Z$ (see
\olref[sth][spine][zf]{sec}). So instead we will offer an apparently
weaker formulation of Replacement, as follows:

\begin{defish}
\emph{Weak-Reflection.} For any formula $\phi$, there is a transitive
set $S$ such that $0$, $1$, and any parameters to $\phi$ are
!!{element}s of $S$, and $(\forall \overline{x} \in S)(\phi \liff
\phi^S)$.
\end{defish}

To use this to prove Replacement, we will first follow \citet[first
part of Theorem 2]{Levy1960} and show that we can ``reflect'' two
formulas at once:

\begin{lem}[in $\Z + \text{Weak-Reflection}$.]\ollabel{lem:reflect}
For any formulas $\psi, \chi$, there is a transitive set $S$ such that
$0$ and $1$ (and any parameters to the formulas) are !!{element}s of
$S$, and $(\forall \overline{x} \in S)((\psi \liff \psi^S) \land (\chi
\liff \chi^S))$.
\end{lem}

\begin{proof}
Let $\phi$ be the formula $(z = 0 \land \psi) \lor (z = 1 \land \chi)$. 

Here we use an abbreviation; we should spell out ``$z = 0$'' as
``$\forall t\, t \notin z$'' and ``$z =1$'' as ``$\forall s(s \in z
\liff \forall t\, t \notin s)$''. But since $0, 1 \in S$ and $S$ is
transitive, these formulas are \emph{absolute} for $S$; that is, they
will apply to the same object whether we restrict their quantifiers to
$S$.\footnote{More formally, letting $\xi$ be either of these
formulas, $\xi(z) \liff \xi^S(z)$.}

By Weak-Reflection, we have some appropriate $S$ such that:
\begin{align*}
	(\forall z, \overline{x} \in S)(&\phi \liff \phi^S)\\
	\text{i.e. }(\forall z, \overline{x} \in S)(&((z = 0 \land \psi) \lor (z = 1 \land \chi)) \liff {}\\
	&\phantom{(}((z = 0 \land \psi) \lor (z = 1 \land \chi))^S)\\
	\text{i.e. }(\forall z, \overline{x} \in S)(&((z = 0 \land \psi) \lor (z = 1 \land \chi))\liff {}\\
	&\phantom{(}((z = 0 \land \psi^S) \lor (z = 1 \land \chi^S)))\\
	\text{i.e. }(\forall \overline{x} \in S)(&(\psi \liff \psi^S) \land (\chi \liff \chi^S))
\end{align*}
The second claim entails the third because ``$z = 0$'' and ``$z=1$''
are absolute for $S$; the fourth claim follows since $0 \neq 1$.
\end{proof}\noindent We can now obtain Replacement, just by following and simplifying 
\citet[Theorem 6]{Levy1960}:

\begin{thm}[in $\Z$ + Weak-Reflection]\label{thm:replacement} 
For any formula $\phi(v,w)$, and any $A$, if $(\forall x \in A)\lexists![y][\phi(x,y)]$, then
$\Setabs{y}{(\exists x \in A)\phi(x,y}$ exists.
\end{thm}

\begin{proof}
Fix $A$ such that $(\forall x \in A)\lexists![y][\phi(x,y)]$, and
define formulas:
\begin{align*}
	\psi &\text{ is } (\phi(x, z) \land A = A)\\
	\chi &\text{ is } \lexists[y][\phi(x, y)]
\end{align*}
Using \olref{lem:reflect}, since $A$ is a parameter to $\psi$, there
is a transitive~$S$ such that $0, 1, A \in S$  (along with any other
parameters), and such that:
\[
	(\forall x,z \in S)((\psi \liff \psi^S) \land (\chi \liff \chi^S))
\]
So in particular:
\begin{align*}
	(\forall  x, z \in S)(&\phi(x, z) \liff \phi^S(x, z))\\
	(\forall x \in S)(&\exists y\phi(x, y) \liff (\exists y \in S)\phi^S(x, y)) 
\end{align*}
Combining these, and observing that $A \subseteq S$ since $A \in S$ and $S$ is transitive:
\begin{align*}
	(\forall x \in A)(&\exists y\phi(x, y) \liff (\exists y \in S)\phi(x, y))
\end{align*}
Now $(\forall x \in A)(\lexists![y \in S])\phi(x, y)$, because
$(\forall x \in A)\lexists![y][\phi(x, y)]$. Now Separation yields
$\Setabs{y \in S}{(\exists x \in A) \phi(x, y)} = \Setabs{y}{(\exists
x \in A) \phi(x, y)}$. 
\end{proof}

\end{document}