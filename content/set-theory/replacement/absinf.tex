\documentclass[../../../include/open-logic-section]{subfiles}

\begin{document}

\olfileid{sth}{replacement}{absinf}
\olsection{Replacement and ``Absolute Infinity''}

We will now put \limofsize{} behind us, and explore a different
family of (intrinsic) attempts to justify Replacement, which do take
seriously the idea of the sets as formed in stages.

When we first outlined the iterative process, we offered some
principles which explained what happens at each stage. These were
\stageshier, \stagesord, and \stagesacc. Later, we added some
principles which told us something about the number of stages:
\stagessucc{} told us that the process of set-formation never ends,
and \stagesinf{} told us that the process goes through an infinite-th
stage. 

It is reasonable to suggest that these two latter principles fall out
of some a broader principle, like:
\begin{enumerate}
	\item[] \stagesinex. There are absolutely infinitely many stages;
	the hierarchy is as tall as it could possibly be.
\end{enumerate}
Obviously this is an informal principle. But even if it is not
immediately \emph{entailed} by the cumulative-iterative conception of
set, it certainly seems \emph{consonant} with it. At the very least,
and unlike \limofsize, it retains the idea that sets are formed
stage-by-stage. 

The hope, now, is to leverage \stagesinex{} into a justification of
Replacement. So let us see how this might be done. 

In \olref[ordinals][idea]{sec}, we saw that it is easy to
construct a well-ordering which (morally) should be isomorphic to
$\omega+\omega$. Otherwise put, we can easily imagine a stage-by-stage
iterative process, whose order-type (morally) is $\omega+\omega$. As
such, if we have accepted \stagesinex, then we should surely accept
that there is at least an $\omega+\omega$-th stage of the hierarchy,
i.e., $V_{\omega+\omega}$, for the hierarchy surely \emph{could}
continue thus far. 

This thought generalizes as follows: for any well-ordering, the
process of building the iterative hierarchy should run at least as far
as that well-ordering. And we could guarantee this, just by treating
\olref[ordinals][ordtype]{thmOrdinalRepresentation} as an
\emph{axiom}. This would tell us that any well-ordering is isomorphic
to a von Neumann ordinal. Since each von Neumann ordinal will be equal
to its own rank, \olref[ordinals][ordtype]{thmOrdinalRepresentation}
will then tell us that, whenever we can describe a well-ordering in
our set theory, the iterative process of set building must outrun that
well-ordering. 

This idea certainly seems like a corollary of \stagesinex.
Unfortunately, if our aim is to extract Replacement from this idea,
then we face a simple, technical, barrier. By a result of
\citet{Montague1961}, Replacement is strictly stronger than
\olref[ordinals][ordtype]{thmOrdinalRepresentation}.\footnote{For more
discussion of this general idea, though, see
\citet[\S13.2]{Potter2004} and \citet{IncurvatiThesis} on the Axiom of
Ordinals.}

The upshot is that, if we are going to understand \stagesinex{} in
such a way as to yield Replacement, then it cannot \emph{merely} say
that the hierarchy outruns any well-ordering. It must make a stronger
claim than that. To this end, \cite{Shoenfield:AST} proposed a very
natural strengthening of the idea, as follows: the hierarchy is not
\emph{cofinal} with any set.\footnote{G\"odel seems to have proposed a
similar thought; see \cite[p.~223]{Potter2004}.} In slightly more detail:
if $\tau$ is a mapping which sends sets to stages of the hierarchy,
the image of any set $A$ under $\tau$ does not exhaust the hierarchy.
Otherwise put (schematically): 
\begin{enumerate}
	\item[] \stagescofin. If $A$ is a set and $\tau(x)$ is a stage for
	every $x \in A$, then there is a stage which comes after each
	$\tau(x)$ for $x \in A$.
\end{enumerate}
It is obvious that $\ZF$ proves a suitably formalised version of
\stagescofin. Conversely, we can informally argue that \stagescofin{}
justifies Replacement.\footnote{It would be harder to prove
Replacement using some formalisation of \stagescofin, since $\Z$ on
its own is not strong enough to define the stages, so it is not clear
how one would formalise \stagescofin. One good option, though, is to
work in the theory presented by \cite{Potter2004}, which \emph{can}
define stages.} For suppose $(\forall x \in A)\lexists![y][\phi(x,y)]$. Then for each $x \in A$, let $\sigma(x)$ be the $y$ such
that $\phi(x,y)$, and let $\tau(x)$ be the stage at which $\sigma(x)$
is first formed. By \stagescofin, there is a stage $V$ such that
$(\forall x \in A)\tau(x)\in V$. Now since each $\tau(x) \in V$ and
$\sigma(x) \subseteq \tau(x)$, by Separation we can obtain $\Setabs{y
\in V}{(\exists x \in A)\sigma(x) = y} = \Setabs{y}{(\exists x \in
A)\phi(x,y)}$.

\begin{prob}
Prove \stagescofin{} within $\ZF$.
\end{prob}

So \stagescofin{} vindicates Replacement. And it is at least plausible
that \stagesinex{} vindicates \stagescofin. For suppose \stagescofin{}
fails. So the hierarchy is cofinal with some set~$A$, i.e., we have a
map $\tau$ such that for any stage~$S$ there is some $x \in A$ such
that $S \in \tau(x)$. In that case, we do have a way to get a handle
on the supposed ``absolute infinity'' of the hierarchy: it is
\emph{exhausted} by the range of $\tau$ applied to $A$. And that
compromises the thought that the hierarchy is ``absolutely infinite''.
Contraposing: \stagesinex{} entails \stagescofin, which in turn
justifies Replacement.

This represents a genuinely promising attempt to provide an
intrinsic justification for Replacement. But whether it ultimately
works, or not, we will have to leave to you to decide.

\end{document}