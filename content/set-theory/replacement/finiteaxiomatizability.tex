\documentclass[../../../include/open-logic-section]{subfiles}

\begin{document}
	
\olfileid{sth}{replacement}{finiteaxiomatizability}
\olsection{Appendix: Finite axiomatizability}
We close this chapter by extracting some results from Replacement. The first result is due to \citet{Montague1961}; note that it is not a proof \emph{within} $\ZF$, but a proof \emph{about} $\ZF$:
\begin{thm}\ollabel{zfnotfinitely}
	$\ZF$ is not finitely axiomatizable. More generally: if $\Th{T}$ is finite and $\Th{T} \Proves \ZF$, then $\Th{T}$ is inconsistent.
	
	(Here, we tacitly restrict ourselves to first-order sentences whose only non-logical primitive is $\in$, and we write $\Th{T} \Proves \ZF$ to indicate that $\Th{T} \Proves \phi$ for all $\phi \in \ZF$.)
\end{thm}
\begin{proof}
	Fix finite $\Th{T}$ such that $\Th{T} \Proves \ZF$. So, $\Th{T}$ proves Reflection, i.e.\ \olref[sth][replacement][ref]{reflectionschema}. Since $\Th{T}$ is finite, we can rewrite it as a single conjunction, $\theta$. Reflecting with this formula, $\Th{T} \Proves \exists \beta(\theta \liff \theta^{V_\beta})$. Since trivially $\Th{T} \Proves \theta$, we find that $\Th{T} \Proves \exists \beta\ \theta^{V_\beta}$. 
	
	Now, let $\psi(X)$ abbreviate:
	\[
		\theta^X \land X\text{ is transitive} \land (\forall Y \in X)(Y\text{ is transitive}\lif \lnot \theta^{Y})
	\]
	roughly this says: $X$ is a transitive model of $\theta$, and $\in$-minimal in this regard. Now, recalling that $\Th{T} \Proves \exists \beta\ \theta^{V_\beta}$, by basic facts about ranks within $\ZF$ and hence within $\Th{T}$, we have:
	\begin{equation}
		\Th{T} \Proves \exists M \psi(M). \tag{*}\label{Mpsi}
	\end{equation}
	Using the first conjunct of $\psi(X)$, whenever $\Th{T} \Proves \sigma$, we have that $\Th{T} \Proves \forall X(\psi(X) \lif \sigma^X)$. So, by \eqref{Mpsi}:
	\begin{align*}
		\Th{T} &\Proves \forall X(\psi(X) \lif (\exists N \psi(N))^X)\\
	\intertext{Using this, and \eqref{Mpsi} again:}
		\Th{T} &\Proves \exists M(\psi(M) \land (\exists N \psi(N))^M)
	\intertext{In particular, then:}
		\Th{T} &\Proves \exists M(\psi(M) \land (\exists N \in M)((N\text{ is transitive})^N \land (\theta^N)^M))
	\intertext{So, by elementary reasoning concerning transitivity:}
		\Th{T} &\Proves \exists M(\psi(M) \land (\exists N \in M)(N\text{ is transitive} \land \theta^N))
	\end{align*} 
	So that $\Th{T}$ is inconsistent.\footnote{This ``elementary reasoning'' involves proving certain ``absoluteness facts'' for transitive sets.}
\end{proof}

Here is a similar result:

\begin{prop}\ollabel{finiteextensionofZ}
	Let $\Th{T}$ extend $\Z$ with finitely many new axioms. If $\Th{T} \Proves \ZF$, then $\Th{T}$ is inconsistent. (Here we use the same tacit restrictions as for \olref{zfnotfinitely}.)
\end{prop}
\begin{proof}
	Use $\theta$ for the conjunction of all of $\Th{T}$'s axioms \emph{except} for the (infinitely many) instances of Separation. Defining $\psi$ from $\theta$ as in \olref{zfnotfinitely}, we can show that $\Th{T} \Proves \exists M \psi(M)$. 
	
	As in \olref{zfnotfinitely}, we can establish the schema that, whenever $\Th{T} \Proves \sigma$, we have that $\Th{T} \Proves \forall X(\psi(X) \lif \sigma^X)$. We then finish our proof, exactly as in \olref{zfnotfinitely}.
	
	However, establishing the schema involves a little more work than in \olref{zfnotfinitely}. After all, the Separation-instances are in $\Th{T}$, but they are not conjuncts of $\theta$. However, we can overcome this obstacle by proving that $\Th{T} \Proves \forall X(X\text{ is transitive} \lif \sigma^X)$, for every Separation-instance $\sigma$. We leave this to the reader. 
\end{proof}
\begin{prob}
	Show that, for every Separation-instance $\sigma$, we have: $\Z \Proves \forall X(X\text{ is transitive} \lif \sigma^X)$. (We used this schema in \olref[sth][replacement][finiteaxiomatizability]{finiteextensionofZ}.)
\end{prob}
\begin{prob}
	Show that, for every $\phi \in \Z$, we have $\ZF \Proves \phi^{V_{\omega+\omega}}$.
\end{prob}
\begin{prob}
	Confirm the remaining schematic results invoked in the proofs of \olref[sth][replacement][finiteaxiomatizability]{zfnotfinitely} and  \olref[sth][replacement][finiteaxiomatizability]{finiteextensionofZ}.
\end{prob}

As remarked in \olref[sth][replacement][absinf]{sec}, this shows that Replacement is strictly stronger than
\olref[ordinals][ordtype]{thmOrdinalRepresentation}. Or, slightly more strictly: if $\Z$ + ``every well-ordering is isomorphic to a unique ordinal'' is consistent, then it fails to prove some Replacement-instance.


	%	By assumption, $\psi^{V_\alpha}$ has the form: 
	%	\[
	%		(\forall A \in V_\alpha)(\exists S \in V_\alpha)(\forall x \in V_\alpha)(x \in S \liff (\phi^{V_\alpha}(x) \land x \in A))
	%	\]
	%	To establish this holds, fix $A \in V_\alpha$. Using Separation, obtain: 
	%	\[
	%		S = \Setabs{x \in A}{\phi^{V_\alpha}(x)}
	%	\]
	%	Now $S \in V_\alpha$, since $S \subseteq A \in V_\alpha$, and clearly $(\forall x \in V_\alpha)(x \in S \liff (\phi^{V_\alpha}(x) \land x \in A)$.

\end{document}