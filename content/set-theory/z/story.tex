\documentclass[../../../include/open-logic-section]{subfiles}

\begin{document}

\olfileid{sth}{z}{story}
\olsection{The Story in More Detail}
	
In \olref[story][approach]{sec}, we quoted Schoenfield's
description of the process of set-formation. We now want to write down
a few more principles, to make this story a bit more precise. Here
they are:
\begin{enumerate}
\item[] \stageshier. Every set is formed at some stage.
\item[] \stagesord. Stages are ordered: some come \emph{before}
others.\footnote{We will actually assume---tacitly---that the stages
are \emph{well-ordered}. What this amounts to is explained in
\olref[ordinals][]{chap}. This is a substantial assumption. In fact,
using a very clever technique due to \citet{Scott1974}, this
assumption can be \emph{avoided} and then \emph{derived}. (This will
also explain why we should think that there is an initial stage.) We cannot go into that here; for more, see \citet{ButtonLT1}.} 
\item[] \stagesacc. For any stage $S$, and for any sets which were
formed \emph{before} stage~$S$: a set is formed at stage~$S$ whose
members are exactly those sets. Nothing else is formed at stage~$S$.
\end{enumerate}
These are informal principles, but we will be able to use them to
vindicate several of the axioms of Zermelo's set theory. 

(We should offer a word of caution. Although we will be presenting some
completely standard axioms, with completely standard names, the
italicized principles we have just presented have no particular names
in the literature. We simply monikers which we hope are helpful.)

\end{document}