\documentclass[../../../include/open-logic-section]{subfiles}

\begin{document}

\olfileid{sth}{ordinals}{replacement}
\olsection{Replacement}

In \olref[sth][ordinals][vn]{sec}, we motivated the introduction of ordinals by
suggesting that we could treat them as order-types, i.e., canonical
proxies for well-orderings. In order for that to work, we would need
to prove that \emph{every well-ordering is isomorphic to some
ordinal}. This would allow us to define $\ordtype{A, <}$ as the
ordinal $\alpha$ such that $\tuple{A, <} \isomorphic \alpha$. 

Unfortunately, we \emph{cannot} prove the desired result only the
Axioms we provided introduced so far. (We will see why in
\olref[replacement][strength]{sec}, but for now the point is: we can't.) We need a
new thought, and here it is:

\begin{axiom}[Scheme of Replacement]
For any formula $\phi(x, y)$, this is an axiom: 
	\\for any $A$, if $(\forall x \in A)\lexists![y][\phi(x,y)]$, {then} $\Setabs{y}{(\exists x \in A)\phi(x,y)}$ exists.
\end{axiom}
\noindent
As with Separation, this is a scheme: it yields infinitely many
axioms, for each of the infinitely many different $\phi$'s. And it can
equally well be (and normally is) written down thus:

\
\\\emph{For any formula $\phi(x,y)$ which does not contain ``$B$'', this is an axiom:
\\$\forall A[(\forall x \in A)\lexists![y][\phi(x,y)] \lif \exists B\forall y (y \in B \liff (\exists x \in A)\phi(x,y))]$}

\
\\ 
On first encounter, however, this is quite a tangled formula. The
following quick consequence of Replacement probably gives a
\emph{clearer} expression to the intuitive idea we are working with:

\begin{cor}
For any term $\tau(x)$, and any set $A$, this set exists:
\[
	\Setabs{\tau(x)}{x \in A} = \Setabs{y}{(\exists x \in A)y = \tau(x)}.
\]
\end{cor}

\begin{proof}
Since $\tau$ is a \emph{term}, $\forall x \lexists![y][\tau(x) = y]$.
A fortiori, $(\forall x \in A)\lexists![y][\tau(x) = y]$. So
$\Setabs{y}{(\exists x \in A)\tau(x) = y}$ exists by Replacement.
\end{proof}
\noindent
This suggests that ``Replacement'' is a good name for the Axiom: given
a set $A$, you can form a new set, $\Setabs{\tau(x)}{x \in A}$, by
replacing every member of $A$ with its image under~$\tau$. Indeed,
following the notation for the image of a set under a function, we
might write $\funimage{\tau}{A}$ for $\Setabs{\tau(x)}{x \in A}$.

Crucially, however, $\tau$ is a \emph{term}. It need not be (a name
for) a \emph{function}, in the sense of \olref[sfr][fun][rel]{sec},
i.e., a certain set of ordered pairs. After all, if $f$ is a function
(in that sense), then the set $\funimage{f}{A} = \Setabs{f(x)}{x \in
A}$ is just a particular subset of $\ran{f}$, and that is already
guaranteed to exist, just using the axioms of~$\Zminus$.\footnote{Just
consider $\Setabs{y \in \bigcup \bigcup f}{(\exists x \in A)y =
f(x)}$.} Replacement, by contrast, is a \emph{powerful} addition to
our axioms, as we will see in \olref[sth][replacement][]{chap}.

\end{document}