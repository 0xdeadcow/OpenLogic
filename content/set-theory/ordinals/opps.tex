\documentclass[../../../include/open-logic-section]{subfiles}

\begin{document}

\olfileid{sth}{ordinals}{opps}
\olsection{Successor and Limit Ordinals}

In the next few chapters, we will use ordinals a great deal. So it
will help if we introduce some simple notions. 

\begin{defn}
For any ordinal $\alpha$, its \emph{successor} is $\ordsucc{\alpha}
=\alpha \cup \{\alpha\}$. We say that $\alpha$ is a \emph{successor}
ordinal if $\ordsucc{\beta} = \alpha$ for some ordinal $\alpha$. We
say that $\alpha$ is a \emph{limit} ordinal iff $\alpha$ is neither
empty nor a successor ordinal.
\end{defn}

The following result shows that this is the right notion of
\emph{successor}:

\begin{prop}
For any ordinal $\alpha$:
\begin{enumerate}
	\item $\alpha \in \ordsucc{\alpha}$;
	\item $\ordsucc{\alpha}$ is an ordinal;
	\item there is no ordinal $\beta$ such that $\alpha \in \beta \in
	\ordsucc{\alpha}$.
	\end{enumerate}
\end{prop}

\begin{proof}
Trivially, $\alpha \in \alpha \cup \{\alpha\} = \ordsucc{\alpha}$.
Equally, $\ordsucc{\alpha}$ is a transitive set of ordinals, and hence
an ordinal by \olref[basic]{corordtransitiveord}.
And it is impossible that $\alpha \in \beta \in \ordsucc{\alpha}$,
since then either $\beta \in \alpha$ or $\beta = \alpha$,
contradicting \olref[basic]{ordordered}.
\end{proof}

This also licenses a variant of proof by transfinite induction:

\begin{thm}[Simple Transfinite Induction]\ollabel{simpletransrecursion}
Let $\phi(x)$ be a formula such that:\footnote{The formula may have
parameters.} 
\begin{enumerate}
	\item $\phi(\emptyset)$; and
	\item for any ordinal $\alpha$, if $\phi(\alpha)$ then
	$\phi(\ordsucc{\alpha})$; and 
	\item if $\alpha$ is a limit ordinal and $(\forall \beta \in
	\alpha)\phi(\beta)$, then $\phi(\alpha)$.
\end{enumerate}
Then $\forall \alpha \phi(\alpha)$.
\end{thm}

\begin{proof}
We prove the contrapositive. So, suppose there is some ordinal which
is $\lnot\phi$; let $\gamma$ be the least such ordinal. Then either
$\gamma = \emptyset$, or $\gamma = \ordsucc{\alpha}$ for some $\alpha$
such that $\phi(\alpha)$; or $\gamma$ is a limit ordinal and $(\forall
\beta \in \gamma)\phi(\beta)$.
\end{proof}

A final bit of notation will prove helpful.

\begin{defn}\ollabel{defsupstrict}
If $X$ is a  set of ordinals, then $\supstrict(X) = \bigcup_{\alpha
\in X} \ordsucc{\alpha}$.
\end{defn}

Here, ``lsub'' stands for ``least strict upper bound''.\footnote{Some
books use ``$\text{sup}(X)$'' for this.  But other books use
``$\text{sup}(X)$'' for the least \emph{non-strict} upper bound, i.e.\
simply $\bigcup X$. If $X$ has a greatest element, $\alpha$, these
notions come apart: the least \emph{strict} upper bound is
$\ordsucc{\alpha}$, whereas the least \emph{non-strict} upper bound is
just $\alpha$.}  The following result explains this:

\begin{prop}
If $X$ is a set of ordinals, $\supstrict(X)$ is the least ordinal
greater than every ordinal in $X$.
\end{prop}

\begin{proof}
Let $Y = \Setabs{\ordsucc{\alpha}}{\alpha \in X}$, so that
$\supstrict(X) = \bigcup Y$. Since ordinals are transitive and every
member of an ordinal is an ordinal, $\supstrict(X)$ is a transitive
set of ordinals, and so is an ordinal by
\olref[basic]{corordtransitiveord}. 

If $\alpha \in X$, then $\ordsucc{\alpha} \in Y$, so $\ordsucc{\alpha}
\subseteq \bigcup Y = \supstrict(X)$, and hence $\alpha \in
\supstrict(X)$. So $\supstrict(X)$ is strictly greater than every
ordinal in $X$.

Conversely, if $\alpha \in \supstrict(X)$, then $\alpha \in
\ordsucc{\beta} \in Y$ for some $\beta \in X$, so that $\alpha \leq
\beta \in X$. So $\supstrict(X)$ is the \emph{least} strict upper
bound on $X$.
\end{proof}

\end{document}