% Part: second-order-logic
% Chapter: sol-and-sets
% Section: cardinalities

\documentclass[../../../include/open-logic-section]{subfiles}

\begin{document}

\olfileid{sol}{set}{crd}

\olsection{Cardinalities of Sets}

\begin{explain}
Just as we can express that the domain is finite or infinite,
!!{enumerable} or !!{nonenumerable}, we can define the property of a
subset of~$\Domain{M}$ being finite or infinite, !!{enumerable} or
!!{nonenumerable}.
\end{explain}

\begin{prop}
The formula $\fn{Inf}(X) \ident$
\begin{multline*}
\lexists[u][(\lforall[x][\lforall[y][(\eq[u(x)][u(y)] \lif 
      \eq[x][y])]] \land {} \\\lexists[y][(X(y) \land \lforall[x][(X(x)
      \lif \eq/[y][u(x)])]])]
\end{multline*}
is satisfied with respect to a variable assignment~$s$ iff $s(X)$ is
infinite.
\end{prop}

\begin{prop}
The formula $\fn{Count}(X) \ident $
\begin{multline*}
\lexists[z][\lexists[u][(X(z) \land 
    \lforall[x][(X(x) \lif X(u(x)))] \land {} \\ \lforall[Y][((Y(z) \land
      \lforall[x][(Y(x) \lif Y(u(x)))]) \lif X = Y])]])
\end{multline*}
is satisfied with respect to a variable assignment~$s$ iff $s(X)$ is
!!{enumerable}.
\end{prop}

We know from Cantor's Theorem that there are !!{nonenumerable} sets,
and in fact, that there are infinitely many different levels of
infinite sizes.  Set theory develops an entire arithmetic of sizes of
sets, and assigns infinite cardinal numbers to sets.  The natural
numbers serve as the cardinal numbers measuring the sizes of finite
sets. The cardinality of !!{denumerable} sets is the first infinite
cardinality, called~$\aleph_0$ (``aleph-nought'' or
``aleph-zero''). The next infinite size is~$\aleph_1$. It is the
smallest size a set can be without being countable (i.e., of
size~$\aleph_0$).  We can define ``$X$ has size $\aleph_0$'' as
$\fn{Aleph}_0(X) \liff \fn{Inf}(X) \land \fn{Count}(X)$.  $X$ has size
$\aleph_1$ iff all its subsets are finite or have size~$\aleph_0$, but
is not itself of size~$\aleph_0$. Hence we can express this by the
formula $\fn{Aleph_1}(X) \ident \lforall[Y][(Y \subseteq X \lif
  (\lnot\fn{Inf}(Y) \lor \fn{Aleph}_0(Y)))] \land \lnot
\fn{Aleph}_0(X)$. Being of size $\aleph_2$ is defined similarly, etc.

There is one size of special interest, the so-called cardinality of
the continuum.  It is the size of $\Pow{\Nat}$, or, equivalently, the
size of~$\Real$. That a set is the size of the continuum can also be
expressed in second-order logic, but requires a bit more work.

\end{document}
