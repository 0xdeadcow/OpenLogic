% Part: second-order-logic
% Chapter: syntax-and-semantics
% Section: expressive-power

\documentclass[../../../include/open-logic-section]{subfiles}

\begin{document}

\olfileid{sol}{syn}{exp}
\olsection{Expressive Power}

\begin{explain}
Quantification over second-order variables is responsible for an
immense increase in the expressive power of the language over that of
first-order logic.  Second-order existential quantification lets us
say that functions or relations with certain properties exists. In
first-order logic, the only way to do that is to specify non-logical
symbol (i.e., !!a{function} or !!{predicate}) for this
purpose. Second-order universal quantification lets us say that all
subsets of, relations on, or functions from the !!{domain} to the
!!{domain} have a property.  In first-order logic, we can only say
that the subsets, relations, or functions assigned to one of the
non-logical symbols of the language have a property.  And when we say
that subsets, relations, functions exist that have a property, or that
all of them have it, we can use second-order quantification in
specifying this property as well. This lets us define relations not
definable in first-order logic, and express properties of the domain
not expressible in first-order logic.
\end{explain}

\begin{ex}
If $\Struct{M}$ is !!a{structure} for a language~$\Lang{L}$, a
relation~$R \subseteq \Domain{M}^2$ is definable in~$\Lang{L}$ if
there is some !!{formula}~$!A_R(\Obj{v_0}, \Obj{v_1})$ with only the
variables $\Obj v_0$ and $\Obj v_1$ free, such that $R(x, y)$ holds
(i.e., $\tuple{x,y} \in R$) iff $\Sat{M}{!A_R(\Obj{v_0},
  \Obj{v_1})}[s]$ for $s(\Obj{v_0}) = x$ and $s(\Obj{v_1}) = y$.  For
instance, in first-order logic we can define the identity
relation~$\Id{\Domain{M}}$ (i.e., $\Setabs{\tuple{x,x}}{x \in
  \Domain{M}}$) by the formula $\eq[\Obj v_0][\Obj v_1]$.  In
second-order logic, we can define this relation
\emph{without~$\eq$}. For if $x$ and $y$ are the same !!{element}
of~$\Domain{M}$, then they are !!{element}s of the same subsets
of~$\Domain{M}$ (since sets are determined by their
!!{element}s). Conversely, if $x$ and $y$ are different, then they are
not !!{element}s of the same subsets: e.g., $x \in \{x\}$ but $y
\notin \{x\}$ if $x \neq y$.  So ``being !!{element}s of the same
subsets of~$\Domain{M}$'' is a relation that holds of $x$ and $y$ iff
$x = y$. It is a relation that can be expressed in second-order logic,
since we can quantify over all subsets of~$\Domain{M}$. Hence, the
following !!{formula} defines $\Id{\Domain{M}}$:
\[
\lforall[X][(X(\Obj{v_0}) \liff X(\Obj{v_1}))]
\]
\end{ex}

\begin{prob}
Show that $\lforall[X][(X(\Obj{v_0}) \lif X(\Obj{v_1}))]$ (note:
$\lif$ not $\liff$!) defines $\Id{\Domain{M}}$.
\end{prob}

\begin{ex}
If $R$ is a two-place !!{predicate}, $\Assign{R}{M}$ is a two-place
relation on~$\Domain{M}$.  Its \emph{transitive closure}~$R^*$ is the
relation that holds between $x$ and $y$ if for some $z_1$, \dots,
$z_k$, $R(x,z_1)$, $R(z_1, z_2)$, \dots, $R(z_k,y)$ holds. This
includes the case if $k = 0$, i.e., if $R(x,y)$ holds. This means that
$R \subseteq R^*$. In fact, $R^*$ is the smallest relation that
includes~$R$ and that is transitive.  We can say in second-order logic
that $X$ is a transitive relation that includes~$R$:
\begin{multline*}
  !B_R(X) \ident \lforall[x][\lforall[y][(R(x,y) \lif X(x, y))]] \land {}\\
\lforall[x][\lforall[y][\lforall[z][((X(x,y) \land X(y,z)) \lif X(x,
      z))]]]
\end{multline*}
Here, somewhat confusingly, we use $R$ as the !!{predicate}
for~$R$. The first conjunct says that $R \subseteq X$ and the second
that $X$ is transitive.

To say that $X$ is the smallest such relation is to say that it is
itself included in every relation that includes $R$ and is
transitive. So we can define the transitive closure of~$R$ by the
!!{formula}
\[
R^*(X) \ident !B_R(X) \land \lforall[Y][(!B_R(Y) \lif
  \lforall[x][\lforall[y][(X(x, y) \lif Y(x,y))]])]
\]
$\Sat{M}{R^*(X)}[s]$ iff $s(X) = R^*$. The transitive closure of $R$
cannot be expressed in first-order logic.
\end{ex}


\end{document}
