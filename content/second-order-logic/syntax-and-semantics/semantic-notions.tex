% Part: second-order-logic
% Chapter: syntax-and-semantics
% Section: semantic-notions

\documentclass[../../../include/open-logic-section]{subfiles}

\begin{document}

\olfileid{sol}{syn}{sem}
\olsection{Semantic Notions}

\begin{explain}
The central logical notions of \emph{validity}, \emph{entailment}, and
\emph{satisfiability} are defined the same way for second-order logic
as they are for first-order logic, except that the underlying
satisfaction relation is now that for second-order !!{formula}s.  A
second-order !!{sentence}, of course, is !!a{formula} in which all
variables, including predicate and function variables, are bound.
\end{explain}

\begin{defn}[Validity]
A sentence $!A$ is \emph{valid}, $\Entails !A$, iff $\Sat{M}{!A}$ for every
!!{structure}~$\Struct M$.
\end{defn}

\begin{defn}[Entailment]
A set of sentences~$\Gamma$ \emph{entails} a sentence~$!A$, $\Gamma
\Entails !A$, iff for every !!{structure}~$\Struct M$ with
$\Sat{M}{\Gamma}$, $\Sat{M}{!A}$.
\end{defn}

\begin{defn}[Satisfiability]
A set of sentences~$\Gamma$ is \emph{satisfiable} if $\Sat{M}{\Gamma}$
for some !!{structure}~$\Struct M$.  If $\Gamma$ is not satisfiable it is
called \emph{unsatisfiable}.
\end{defn}

\end{document}
