% Part: second-order-logic
% Chapter: syntax-and-semantics
% Section: language-of-sol

\documentclass[../../../include/open-logic-section]{subfiles}

\begin{document}

\olfileid{sol}{syn}{lan}

\olsection{The Language of Second-Order Logic}

Like in first-order logic, expressions of second-order logic are built
up from a basic vocabulary containing \emph{!!{variable}s},
\emph{!!{constant}s}, \emph{!!{predicate}s} and sometimes
\emph{!!{function}s}.  From them, together with logical connectives,
quantifiers, and punctuation symbols such as parentheses and commas,
\emph{terms} and \emph{!!{formula}s} are formed.  The difference is
that in addition to varaibles for objects, second-order logic also
contains varaibles for relations and functions, and allows
quantification over them. So the logical symbols of second-order logic
are those of first-order logic, plus:

\begin{enumerate}
\item A !!{denumerable}s set of second-order relation !!{variable}s of
  every arity~$n$: $\Obj V_0^n$, $\Obj V_1^n$, $\Obj V_2^n$, \dots
\item A !!{denumerable}s set of second-order function !!{variable}s:
  $\Obj u_0^n$, $\Obj u_1^n$, $\Obj u_2^n$, \dots
\end{enumerate}

In first-order logic, the !!{identity}~$\eq$ is usually included. In
first-order logic, the non-logical symbols of a language~$\Lang{L}$
are crucial to allow us to express anything interesting. There are of
course !!{sentences} that use no non-logical symbols, but with
only~$\eq$ it is hard to say anything interesting.  In second-order
logic, since we have an unlimited supply of relation and function
variables, we can say anything we can say in a first-order language
even without a special supply of non-logical symbols.

\end{document}
