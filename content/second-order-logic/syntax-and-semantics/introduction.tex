% Part: second-order-logic
% Chapter: syntax-and-semantics
% Section: introduction

\documentclass[../../../include/open-logic-section]{subfiles}

\begin{document}

\olfileid{sol}{syn}{int}

\olsection{Introduction}

In first-order logic, we combine the non-logical symbols of a given
language, i.e., its !!{constant}s, !!{function}s, and !!{predicate}s,
with the logical symbols to express things about first-order
!!{structure}s.  This is done using the notion of satisfaction, which
relates !a{structure}~$\Struct{M}$, together with a variable
assignment~$s$, and !!a{formula}~$!A$: $\Sat{M}{!A}[s]$ holds iff what
$!A$ expresses when its !!{constant}s, !!{function}s, and
!!{predicate}s are interpreted as $\Struct{M}$ says, and its free
variables are interpreted as~$s$ says, is true.  The interpretation of
the !!{identity}~$\eq$ is built into the definition of
$\Sat{M}{!A}[s]$, as is the interpretation of~$\lforall$ and
$\lexists$. The former is always interpreted as the identity relation
on the !!{domain}~$\Domain{M}$ of the structure, and the quantifiers
are always interpreted as ranging over the entire !!{domain}.  But,
crucially, quantification is only allowed over elements of the
!!{domain}, and so only !!{variable}s are allowed to follow a
quantifier.

In second-order logic, both the language and the definition of
satisfaction are extended to include free and bound function and
predicate variables, and quantification over them.  These variables
are related to !!{function}s and !!{predicate}s the same way that
object variables are related to !!{constant}s.  They play the same
role in the formation of terms and !!{formula}s of second-order logic,
and quantification over them is handled in a similar way.  In the
\emph{standard} semantics, the second-order quantifiers range over all
possible objects of the right type ($n$-place functions from
$\Domain{M}$ to~$\Domain{M}$ for function variables, $n$-place
relations for predicate variables).  For instance, while
$\lforall[\Obj{v_0}][(\Obj{P^1_0}(\Obj v_0) \lor \lnot
  \Obj{P^1_0}(v_0))]$ is a formula in both first- and second-order
logic, in the latter we can also consider
$\lforall[\Obj{V^1_0}][\lforall[\Obj{v_0}][(\Obj{V^1_0}(\Obj v_0) \lor
    \lnot \Obj{V^1_0}(v_0))]]$ and
$\lexists[\Obj{V^1_0}][\lforall[\Obj{v_0}][(\Obj{V^1_0}(\Obj v_0) \lor
    \lnot \Obj{V^1_0}(v_0))]]$. Since these contain no free varaibles,
they are !!{sentence}s of second-order logic. Here, $\Obj{V^1_0}$ is a
second-order $1$-place predicate variable.  The allowable
interpretations of $\Obj{V^1_0}$ are the same that we can assign to a
$1$-place !!{predicate} like $\Obj{P^1_0}$, i.e., subsets
of~$\Domain{M}$.  Quantification over them then amounts to saying that
$\lforall[\Obj{v_0}][(\Obj{V^1_0}(\Obj v_0) \lor \lnot
  \Obj{V^1_0}(v_0))]$ holds for all ways of assigning a subset
of~$\Domain{M}$ as the value of $\Obj{V^1_0}$, or for at least one.
Since every set either contains or fails to contain a given object,
both are true in any !!{structure}.
\end{document}
