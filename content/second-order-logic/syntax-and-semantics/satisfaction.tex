% Part: second-order-logic
% Chapter: syntax-and-semantics
% Section: satisfaction

\documentclass[../../../include/open-logic-section]{subfiles}

\begin{document}

\olfileid{sol}{syn}{sat}

\olsection{Satisfaction}

\begin{explain}
To define the satisfaction relation $\Sat{M}{!A}[s]$ for second-order
!!{formula}s, we have to extend the definitions to cover second-order
variables.  The notion of !!a{structure} is the same for second-order
logic as it is for first-order logic. There is only a diffence for
variable assignments~$s$: these now must not just provide values for
the first-order variables, but also for the second-order variables.
\end{explain}

\begin{defn}[Variable Assignment]
A \emph{variable assignment}~$s$ for !!a{structure}~$\Struct{M}$ is a
function which maps each
\begin{enumerate}
\item object !!{variable}~$\Obj{v_i}$ to an element of~$\Domain M$,
  i.e., $s(\Obj{v_i}) \in \Domain{M}$
\item $n$-place relation variable~$\Obj{V_i^n}$ to an $n$-place
  relation on~$\Domain{M}$, i.e., $s(\Obj{V_i^n}) \subseteq \Domain{M}^n$;
\item $n$-place function variable~$\Obj{u_i^n}$ to an $n$-place
  function from $\Domain{M}$ to $\Domain{M}$, i.e.,
  $s(\Obj{u_i^n})\colon \Domain{M}^n \to \Domain{M}$;
\end{enumerate}
\end{defn}

\begin{explain}
!!^a{structure} assigns !!a{value} to each !!{constant} and
!!{function}, and a second-order variable assigns objects and
functions to each object and function variable. Together, they let us
assign a value to every term.
\end{explain}

\begin{defn}[\usetoken{S}{value} of a Term]
If $t$ is a term of the language~$\Lang L$, $\Struct M$ is
!!a{structure} for~$\Lang L$, and $s$ is !!a{variable} assignment
for~$\Struct M$, the \emph{!!{value}}~$\Value{t}{M}[s]$ is defined as
for first-order terms, plus the following clause:
\begin{quote}
\indcase{t}{\Atom{u}{t_1, \ldots, t_n}}{
\[
\Value{\indfrm}{M}[s] = s(u)(\Value{t_1}{M}[s], \ldots,
\Value{t_n}{M}[s]).
\]}
\end{quote}
\end{defn}

\begin{defn}[$x$-Variant]
If $s$ is !!a{variable} assignment for !!a{structure}~$\Struct M$,
then any !!{variable} assignment $s'$ for $\Struct M$ which differs
from $s$ at most in what it assigns to $x$ is called an
\emph{$x$-variant} of~$s$.  If $s'$ is an $x$-variant of $s$ we write
$s \sim_x s'$. (Similarly for second-order variables $X$ or $u$.)
\end{defn}


\begin{defn}[Satisfaction]
For second-order !!{formula}s~$!A$, the definition of satisfaction is
like \olref[fol][syn][sat]{defn:satisfaction} with the addition of:
\begin{enumerate}
\item \indcase{!A}{\Atom{X^n}{t_1, \dots, t_n}}{$\Sat{M}{\indfrm}[s]$
  iff $\langle \Value{t_1}{M}[s], \dots, \Value{t_n}{M}[s] \rangle \in
  s(X^n)$.}
\tagitem{prvAll}{%
  \indcase{!A}{\lforall[X][!B]}{$\Sat{M}{\indfrm}[s]$ iff for every
    $X$-variant $s'$ of $s$, $\Sat{M}{!B}[s']$.}}{}

\tagitem{prvEx}{%
  \indcase{!A}{\lexists[X][!B]}{$\Sat{M}{\indfrm}[s]$ iff there is an
    $X$-variant $s'$ of $s$ so that $\Sat{M}{!B}[s']$.}}{}

\tagitem{prvAll}{%
  \indcase{!A}{\lforall[u][!B]}{$\Sat{M}{\indfrm}[s]$ iff for every
    $u$-variant $s'$ of $s$, $\Sat{M}{!B}[s']$.}}{}

\tagitem{prvEx}{%
  \indcase{!A}{\lexists[u][!B]}{$\Sat{M}{\indfrm}[s]$ iff there is an
    $u$-variant $s'$ of $s$ so that $\Sat{M}{!B}[s']$.}}{}
\end{enumerate}
\end{defn}

\begin{ex}
  Consider the !!{formula} $\lforall[z][(\Atom{X}{z} \liff \lnot
    \Atom{Y}{z})]$. It contains no second-order quantifiers, but does
  contain the second-order variables $X$ and~$Y$ (here understood to
  be one-place). The corresponding first-order !!{sentence}
  $\lforall[z][(\Atom{P}{z} \liff \lnot \Atom{R}{z})]$ says that
  whatever falls under the interpretation of $P$ does not fall under
  the interpretation of~$R$ and vice versa. In !!a{structure}, the
  interpretation of !!a{predicate}~$P$ is given by the
  interpretation~$\Assign{M}{P}$. But for second-order !!{variable}s
  like $X$ and $Y$, the interpretation is provided, not by the
  !!{structure} itself, but by a !!{variable} assignment. Since the
  second-order formula is not !!a{sentence} (in includes freee
  variables $X$ and~$Y$), it is only satisfied relative to
  !!a{structure}~$\Struct{M}$ together with !!a{variable}
  assignment~$s$.

  $\Sat{M}{\lforall[z][(Xz \liff \lnot Yz)]}[s]$ whenever the
  !!{element}s of~$s(X)$ are not !!{element}s of~$s(Y)$, and vice
  versa, i.e., iff $s(Y) = \Domain{M} \setminus s(X)$. So for
  instance, take $\Domain{M} = \{1, 2, 3\}$. Since no !!{predicate}s,
  !!{function}s, or !!{constant}s are involved, the domain
  of~$\Struct{M}$ is all that is relevant. Now for $s_1(X) = \{1, 2\}$
  and $s_1(Y) = \{3\}$, we have $\Sat{M}{\lforall[z][(\Atom{X}{z}
      \liff \lnot \Atom{Y}{z})]}[s_1]$.

  By contrast, if we have $s_2(X) = \{1, 2\}$ and $s_2(Y) = \{2, 3\}$,
  $\Sat/{M}{\lforall[z][(\Atom{X}{z} \liff \lnot
      \Atom{Y}{z})]}[s_2]$. That's because there is a $z$-variant
  $s_2'$ of~$s_2$ with $s_2'(z) = 2$ where
  $\Sat{M}{\Atom{X}{z}}[s_2']$ (since $2 \in s_2'(X)$) but
  $\Sat/{M}{\lnot \Atom{Y}{z}}[s_2']$ (since also $s_2'(z) \in
  s_2'(Y)$).
\end{ex}

\begin{ex}
$\Sat{M}{\lexists[Y][(\lexists[y][\Atom{Y}{y}] \land
      \lforall[z][(\Atom{X}{z} \liff \lnot \Atom{Y}{z})])]}[s]$ if
  there is an $s' \sim_Y s$ such that
  $\Sat{M}{(\lexists[y][\Atom{Y}{y}] \land \lforall[z][(\Atom{X}{z}
      \liff \lnot \Atom{Y}{z})])}[s']$. And that is the case iff $s'(Y)
  \neq \emptyset$ (so that $\Sat{M}{\lexists[y][\Atom{Y}{y}]}[s']$)
  and, as in the previous example, $s'(Y) = \Domain{M} \setminus
  s'(X)$. In other words,
  $\Sat{M}{\lexists[Y][(\lexists[y][\Atom{Y}{y}] \land
      \lforall[z][(\Atom{X}{z} \liff \lnot \Atom{Y}{z})])]}[s]$ iff
  $\Domain{M} \setminus s(X)$ is non-empty, i.e., $s(X) \neq
  \Domain{M}$. So, the !!{formula} is satisfied, e.g., if $\Domain{M}
  = \{1, 2, 3\}$ and $s(X) = \{1, 2\}$, but not if $s(X) = \{1, 2, 3\}
  = \Domain{M}$.

  Since the !!{formula} is not satisfied whenever $s(X) = \Domain{M}$,
  the !!{sentence}
  \[
  \lforall[X][\lexists[Y][(\lexists[y][\Atom{Y}{y}] \land
      \lforall[z][(\Atom{X}{z} \liff \lnot \Atom{Y}{z})])]]
  \]
  is never satisfied: For any !!{structure}~$\Struct{M}$, the
  assignment~$s(X) = \Domain{M}$ will make the !!{sentence} false. On
  the other hand, the sentence
  \[
  \lexists[X][\lexists[Y][(\lexists[y][\Atom{Y}{y}] \land
      \lforall[z][(\Atom{X}{z} \liff \lnot \Atom{Y}{z})])]]
  \]
  is satisfied relative to any assignment~$s$, since we can always
  find an $X$-variant~$s'$ of~$s$ with $s'(X) \neq \Domain{M}$.
\end{ex}

\end{document}
