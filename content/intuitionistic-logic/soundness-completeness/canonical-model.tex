% part: intuitionistic-logic
% chapter: soundeness-completeness
% section: canonical-model

\documentclass[../../../include/open-logic-section]{subfiles}

\begin{document}

\olfileid{int}{sc}{mod}

\olsection{The Canonical Model}

The worls in our model will be finite sequences~$\sigma$ of natural
numbers, i.e., $\sigma \in \Nat^*$. Note that $\Nat^*$ is inductively
defined by:
\begin{enumerate}
\item $\emptyseq \in \Nat^*$.
\item If $\sigma \in \Nat^*$ and $n \in \Sigma$, then $\sigma.n \in
  \Nat^*$ (where $\sigma.n$ is $\sigma \concat \tuple{n}$).
\item Nothing else is in $\Nat^*$.
\end{enumerate}
So we can use $\Nat^*$ to give inductive definitions.

Let $\tuple{!B_1, !C_1}$, $\tuple{!B_2, !C_s}$, \dots, be an
enumeration of all pairs of !!{formula}s. Given a set of
!!{formula}s~$\Delta$, define $\Delta(\sigma)$ by induction as
follows:
\begin{enumerate}
\item $\Delta(\emptyseq) = \Delta$
\item $\Delta(\sigma.n) = {}$
  \[
  \begin{cases}
    (\Delta(\sigma) \cup \{!B_n\})^* &
    \text{if $\Delta(\sigma) \cup \{!B_n\} \Proves/ !C_n$} \\
    \Delta(\sigma) & \text{otherwise}
  \end{cases}
  \]
\end{enumerate}
Here by $(\Delta(\sigma) \cup \{!B_n\})^*$ we mean the prime set of
!!{formula}s which exists by \olref[lin]{lem:lindenbaum} applied to the
set $\Delta(\sigma) \cup \{!B_n\}$. Note that by this definition, if
$\Delta(\sigma) \cup \{!B_n\} \Proves/ !C_n$, then $\Delta(\sigma.n)
\Proves !B_n$ and $\Delta(\sigma.n) \Proves/ !C_n$.  Note also that
$\Delta(\sigma) \subseteq \Delta(\sigma.n)$ for any~$n$. If $\Delta$
is prime, then $\Delta(\sigma)$ is prime for all~$\sigma$.

\begin{defn}\ollabel{defn:canonical-model}
  Suppose $\Delta$ is prime.  Then the \emph{canonical model} for
  $\Delta$ is defined by:
  \begin{enumerate}
  \item $W = \Nat^*$, the set of finite sequences of natural numbers.
  \item $R$ is the partial order according to which $R\sigma\sigma'$
    iff $\sigma$ is an initial segment of~$\sigma'$ (i.e., $\sigma' =
    \sigma \concat \sigma''$ for some sequence~$\sigma''$).
  \item $V(p) = \Setabs{\sigma}{p \in \Delta(\sigma)}$.
  \end{enumerate}
\end{defn}

It is easy to verify that $R$ is indeed a partial order. Also, the
monotonicity condition on~$V$ is satisfied.  Since $\Delta(\sigma)
\subseteq \Delta(\sigma.n)$ we get $\Delta(\sigma)
\subseteq \Delta(\sigma')$ whenever $R\sigma\sigma'$ by induction
on~$\sigma$.

\end{document}
