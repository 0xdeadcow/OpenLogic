% part: intuitionistic-logic
% chapter: introduction
% section: axiomatic-derivations

\documentclass[../../../include/open-logic-section]{subfiles}

\begin{document}

\olfileid{int}{int}{axd}

\olsection{Axiomatic \usetoken{P}{derivation}}

Axiomatic !!{derivation}s for intuitionistic propositional logic are
the conceptually simplest, and historically first, !!{derivation}
systems. They work just as in classical propositional logic.

\begin{defn}[!!^{derivability}]
If $\Gamma$ is a set of !!{formula}s of $\Lang L$ then a
\emph{!!{derivation}} from $\Gamma$ is a finite sequence $!A_1$,
\dots,~$!A_n$ of !!{formula}s where for each $i \le n$ one of the
following holds:
\begin{enumerate}
\item $!A_i \in \Gamma$; or
\item $!A_i$ is an axiom; or
\item $!A_i$ follows from some $!A_j$ and $!A_k$ with $j < i$ and $k <
  i$ by modus ponens, i.e., $!A_k \ident !A_j \lif !A_i$.
\end{enumerate}
\end{defn}

\begin{defn}[Axioms]
The set of $\PAx$ of \emph{axioms} for the intuitionistic propositional logic 
are all !!{formula}s of the following forms:
\begin{align}
  & (!A \land !B) \lif !A \ollabel{ax:land1}\\
  & (!A \land !B) \lif !B \ollabel{ax:land2}\\
  & !A \lif (!B \lif (!A \land !B)) \ollabel{ax:land3}\\
  & !A \lif (!A \lor !B) \ollabel{ax:lor1}\\
  & !A \lif (!B \lor !A) \ollabel{ax:lor2}\\
  & (!A \lif !C) \lif ((!B \lif !C) \lif ((!A \lor !B) \lif !C)) \ollabel{ax:lor3}\\
  & !A \lif (!B \lif !A) \ollabel{ax:lif1}\\
  & (!A \lif (!B \lif !C)) \lif ((!A \lif !B) \lif (!A \lif !C)) \ollabel{ax:lif2}\\
  & \lfalse \lif !A \ollabel{ax:lfalse1}
\end{align}
\end{defn}

\begin{defn}[!!^{derivability}]
!!^a{formula} $!A$ is \emph{!!{derivable}} from $\Gamma$, written
$\Gamma \Proves !A$, if there is !!a{derivation} from $\Gamma$ ending
in $!A$.
\end{defn}

\begin{defn}[Theorems]
!!^a{formula}~$!A$ is a \emph{theorem} if there is !!a{derivation}
of~$!A$ from the empty set.  We write $\Proves !A$ if $!A$ is a
theorem and $\Proves/ !A$ if it is not.
\end{defn}

\begin{prop}
  If $\Gamma \Proves !A$ in intuitionistic logic, $\Gamma \Proves !A$ in
  classical logic. In particular, if $!A$ is an intuitionistic
  theorem, it is also a classical theorem.
\end{prop}

\begin{proof}
  Every intuitionistic axiom is also a classical axiom, so every
  !!{derivation} in intuitionistic logic is also !!a{derivation} in
  classical logic.
\end{proof}

\end{document}
