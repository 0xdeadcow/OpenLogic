% part: intuitionistic-logic
% chapter: propositions-as-types
% section: proofs-to-terms

\documentclass[../../../include/open-logic-section]{subfiles}

\begin{document}

\olfileid{int}{pty}{pt}

\olsection{Converting \usetoken{P}{derivation} to Proof Terms}

We will describe the process of converting natural deduction
!!{derivation}s to pairs. We will write a proof term to the left of
each formula in the !!{derivation}, resulting in expressions of the
form $M : !A$.  We'll then say that, $M$ \emph{witnesses}~$!A$.  Let's
call such an expression a \emph{judgment}.

First let us assign to each assumption a variable, with the
following constraints:
\begin{enumerate}
\item Assumptions !!{discharged} in the same step (that is, with the same
  number on the square bracket) must be assigned the same variable.
\item For assumptions not !!{discharged}, assumptions of different
  !!{formula}s should be assigned different variables.
\end{enumerate}
Such an assignment translates all assumptions of the form
\[
!A \qquad \text{into}\qquad x : !A.
\]
With assumptions all associated with variables (which are terms),
we can now inductively translate the rest of the deduction tree. The
modified natural deduction rules taking into account context and proof
terms are given below. Given the proof terms for the premise(s), we
obtain the corresponding proof term for conclusion.

\begin{gather*}
  \AxiomC{$M_1 : !A_1$}
  \AxiomC{$M_2 : !A_2$}
  \RightLabel{$\Intro{\land}$}
  \BinaryInfC{$\andi{M_1}{M_2}: !A_1 \land !A_2$}
  \DisplayProof\\
  \AxiomC{$M : !A_1 \land !A_2$}
  \RightLabel{$\Elim{\land}_1$}
  \UnaryInfC{$\ande{i}{M} : !A_1$}
  \DisplayProof
  \qquad
  \AxiomC{$M : !A_1 \land !A_2$}
  \RightLabel{$\Elim{\land}_2$}
  \UnaryInfC{$\ande{i}{M} : !A_2$}
  \DisplayProof
\end{gather*}

In $\Intro\land$ we assume we have $!A_1$ witnessed by term $M_1$ and
$!A_2$ witnessed by term $M_2$. We pack up the two terms into a pair
$\andi{M_1}{M_2}$ which witnesses $!A_1 \land !A_2$.

In $\Elim{\land}_i$ we assume that $M$ witnesses $!A_1 \land !A_2$.
The term witnessing $!A_i$ is $\ande{i}{M}$. Note that $M$ is not
necessary of the form $\andi{M_1}{M_2}$, so we cannot simply
assign~$M_1$ to the conclusion~$!A_i$.

Note how this coincides with the BHK interpretation. What the BHK
interpretation does not specify is how the function used as proof for
$!A \lif !B$ is supposed to be obtained. If we think of proof terms as
proofs or functions of proofs, we can be more explicit.
\begin{gather*}
  \AxiomC{$\Discharge{x:!A}{}$}
  \DeduceC{$N:!B$}
  \RightLabel{$\Intro{\lif}$}
  \UnaryInfC{$\lambd[\typeof{x}{!A}][N]:!A \lif !B$}
  \DisplayProof
  \qquad
  \AxiomC{$P : !A \lif !B$}
  \AxiomC{$Q : !A$}
  \RightLabel{$\Elim{\lif}$}
  \BinaryInfC{$PQ : !B$}
  \DisplayProof
\end{gather*}
The $\lambda$ notation should be understood as the same as in the
lambda calculus, and $PQ$ means applying $P$ to~$Q$.

\begin{gather*}
  \AxiomC{$M_1 : !A_1$}
  \RightLabel{$\Intro{\lor}_1$}
  \UnaryInfC{$\ori{1}{!A_1}{M_1} : !A_1 \lor !A_2$}
  \DisplayProof
  \qquad
  \AxiomC{$M_2 : !A_2$}
  \RightLabel{$\Intro{\lor}_2$}
  \UnaryInfC{$\ori{2}{!A_2}{M_2} : !A_1 \lor !A_2$}
  \DisplayProof
\\
  \AxiomC{$!M : A_1 \lor !A_2$}
  \AxiomC{$\Discharge{x_1 : !A_1}{}$}
  \DeduceC{$N_1 : !C$}
  \AxiomC{$\Discharge{x_2 : !A_2}{}$}
  \DeduceC{$N_2 : !C$}
  \RightLabel{$\Elim{\lor}$}
  \TrinaryInfC{$\ore{M}{x_1}{N_1}{x_2}{N_2} : !C$}
  \DisplayProof
\end{gather*}
The proof term $\ori{1}{!A_1}{M_1}$ is a term witnessing $!A_1 \lor
!A_2$, where $M_1$ witnesses $!A_1$.

The term $\ore{M}{x_1}{N_1}{x_2}{N_2}$ mimics the case clause in
programming languages: we already have the !!{derivation} of $!A \lor
!B$, !!a{derivation} of $!C$ assuming $!A$, and !!a{derivation} of
$!C$ assuming $!B$. The $case$ operator thus select the appropriate
proof depending on $M$; either way it's a proof of $!C$.

\begin{gather*}
  \AxiomC{$N : \lfalse$}
  \RightLabel{$\FalseInt$}
  \UnaryInfC{$\falsee{!A}{N} : !A$}
  \DisplayProof
\end{gather*}
$\falsee{!A}{N}$ is a term witnessing $!A$, whenever $N$ is a term
witnessing $\lfalse$.

Now we have a natural deduction !!{derivation} with all formulas
associated with a term. At each step, the relevant typing
context~$\Gamma$ is given by the list of assumptions remaining
!!{undischarged} at that step.  Note that $\Gamma$ is well defined:
since we have forbidden assumptions of different !!{undischarged}
assumptions to be assigned the same variable, there won't be any
disagreement about the formulas mapped to which a variable is mapped.

We now give some examples of such translations:

Consider the !!{derivation} of $\lnot\lnot(!A \lor \lnot !A)$, i.e.,
$((!A \lor (!A \lif \lfalse))\lif \lfalse)\lif \lfalse$. Its
translation is:

\begin{gather*}
  \AxiomC{$\Discharge{y : (!A \lor (!A \lif \lfalse))\lif \lfalse}{2}$}
  \AxiomC{$\Discharge{y: (!A \lor (!A \lif \lfalse))\lif \lfalse}{2}$}
  \AxiomC{$\Discharge{x:!A}{1}$}
  \UnaryInfC{$\ori{1}{!A \lif \lfalse}{x} : !A \lor (!A \lif \lfalse)$}
  \BinaryInfC{$y (\ori{1}{!A \lif \lfalse}{x}) : \lfalse$}
  \DischargeRule{}{1}
  \UnaryInfC{$\lambd[\typeof{x}{!A}][y (\ori{1}{!A \lif \lfalse}{x})] :
    !A \lif \lfalse$}
  \UnaryInfC{$\ori{2}{!A}{\lambd[\typeof{x}{!A}][y (\ori{1}{!A \lif
          \lfalse}{x})]} : !A \lor (!A \lif \lfalse)$}
  \BinaryInfC{$y (\ori{2}{!A}{\lambd[\typeof{x}{!A}][y \ori{1}{!A \lif
      \lfalse}{x}])} : \lfalse$}
  \DischargeRule{}{2}
  \UnaryInfC{$\lambd[\typeof{y}{(!A \lor (!A \lif \lfalse))\lif \lfalse}][y
      (\ori{2}{!A}{\lambd[\typeof{x}{!A}][y \ori{1}{!A \lif \lfalse}{x}]})] :
    ((!A \lor (!A \lif \lfalse))\lif \lfalse)\lif \lfalse$}
  \DisplayProof
\end{gather*}

The tree has no assumptions, so the context is empty; we get:
\begin{gather*}
  \Proves \lambd[\typeof{y}{(!A \lor (!A \lif \lfalse))\lif \lfalse}][y
    (\ori{2}{!A}{\lambd[\typeof{x}{!A}][y \ori{1}{!A \lif \lfalse}{x}]})] :
  ((!A \lor (!A \lif \lfalse))\lif \lfalse)\lif \lfalse
\end{gather*}
If we leave out the last $\Intro\lif$, the assumptions denoted by $y$
would be in the context and we would get:
\begin{gather*}
  y: ((!A \lor (!A \lif \lfalse))\lif \lfalse) \Proves y
  (\ori{2}{!A}{\lambd[\typeof{x}{!A}][y \ori{1}{!A \lif \lfalse}{x}]}) :
  \lfalse
  \end{gather*}

Another example: $\Proves !A \lif (!A \lif \lfalse) \lif \lfalse$

\begin{gather*}
  \AxiomC{$\Discharge{x:!A}{2}$}
  \AxiomC{$\Discharge{y:!A \lif \lfalse}{1}$}
  \BinaryInfC{$y x : \lfalse$}
  \DischargeRule{}{1}
  \UnaryInfC{$\lambd[\typeof{y}{!A \lif \lfalse}][y x]:
    (!A \lif \lfalse)\lif \lfalse$}
  \DischargeRule{}{2}
  \UnaryInfC{$\lambd[\typeof{x}{!A}][\lambd[\typeof{y}{!A \lif
          \lfalse}][y x]] : !A \lif (!A \lif \lfalse) \lif \lfalse$}
  \DisplayProof
\end{gather*}
Again all assumptions are !!{discharged} and thus the context is
empty, the resulting term is
\begin{gather*}
  \Proves \lambd[\typeof{x}{!A}][\lambd[\typeof{y}{!A \lif \lfalse}][y x]] :
  !A \lif (!A \lif \lfalse) \lif \lfalse
\end{gather*}
If we leave out the last two $\Intro\lif$ inferences, the assumptions
denoted by both $x$ and $y$ would be in context and we would get
\begin{gather*}
  x:!A, y:!A \lif \lfalse \Proves y x : \lfalse
\end{gather*}

\end{document}
