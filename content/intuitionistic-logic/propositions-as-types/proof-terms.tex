% part: intuitionistic-logic
% chapter: propositions-as-types
% section: proof-terms

\documentclass[../../../include/open-logic-section]{subfiles}

\begin{document}

\olfileid{int}{pty}{ter}

\olsection{Proof Terms}

We give the definition of proof terms, and then establish its relation
with natural deduction !!{derivation}s.

\begin{defn}[Proof terms]
  Proof terms are inductively generated by the following rules:
  \begin{enumerate}
  \item A single variable $x$ is a proof term.
  \item If $P$ and $Q$ are proof terms, then $PQ$ is also a proof
    term.
  \item If $x$ is a variable, $!A$ is !!a{formula}, and $N$ is a proof
    term, then $\lambd[\typeof{x}{!A}][N]$ is also a proof term.
  \item If $P$ and $Q$ are proof terms, then $\andi{P}{Q}$ is a proof
    term.
  \item If $M$ is a proof term, then $\ande{i}{M}$ is also a proof
    term, where $i$ is $1$ or~$2$.
  \item If $M$ is a proof term, and $!A$ is a formula, then
    $\ori{i}{!A}{!M}$ is a proof term, where $i$ is $1$ or~$2$.
  \item If $M, N_1, N_2$ is proof terms, and $x_1, x_2$ are variables,
    then $\ore{M}{x_1}{N_1}{x_2}{N_2}$ is a proof term.
  \item If $M$ is a proof term and $!A$ is a formula, then
    $\falsee{!A}{M}$ is proof term.
  \end{enumerate}
\end{defn}

Each of the above rules corresponds to an inference rule in natural
deduction.  Thus we can inductively assign proof terms to the
!!{formula}s in !!a{derivation}.  To make this assignment unique, we
must distinguish between the two versions of $\Elim\land$ and of
$\Intro\lor$.  For instance, the proof terms assigned to the
conclusion of $\Intro\lor$ must carry the information whether $!A \lor
!B$ is inferred from $!A$ or from~$!B$. Suppose $M$ is the term
assigned to~$!A$from which $!A \lor !B$ is inferred. Then the proof
term assigned to $!A \lor !B$ is $\ori{1}{!A}{!M}$. If we instead
infer $!B \lor !A$ then the proof term assigned is $\ori{2}{!A}{M}$.

The term $\lambd[\typeof{x}{!A}][N]$ is assigned to the conclusion of
$\Intro{\lif}$. The $!A$ represents the assumption being discharged;
only have we included it can we infer the formula of
$\lambd[\typeof{x}{!A}][N]$ based on the formula of $N$.

\begin{defn}[Typing context]
  A \emph{typing context} is a mapping from variables to formulas. We
  will call it simply the ``context'' if there is no confusion.  We
  write a context~$\Gamma$ as a set of pairs $\tuple{x, !A}$.
\end{defn}

A pair $\Gamma \Sequent M$ where $M$ is a proof term represents
!!a{derivation} of a formula with context~$\Gamma$.

\begin{defn}[Typing pair]
  A \emph{typing pair} is a pair $\tuple{\Gamma, M}$,
  where $\Gamma$ is a typing context and $M$ is a proof term. 
\end{defn}

Since in general terms only make sense with specific contexts, we will
speak simply of ``terms'' from now on instead of ``typing pair''; and
it will be apparent when we are talking about the literal term $M$.

\end{document}
