% Part: propositional-logic
% Chapter: syntax-and-semantics
% Section: formulas

\documentclass[../../../include/open-logic-section]{subfiles}

\begin{document}

\olfileid{pl}{syn}{fml}

\olsection{Propositional \usetoken{P}{formula}}

!!^{formula}s of propositional logic are built up from
\emph{!!{propositional
    variable}s}\iftag{prvFalse}{\iftag{prvTrue}{,}{ and} the
  propositional constant~$\lfalse$}{}\iftag{prvTrue}{\iftag{prvFalse}{
    and}{} the propositional constant~$\ltrue$}{} using \emph{logical
  connectives}.

\begin{enumerate}
\item !!^a{denumerable} set~$\PVar$ of !!{propositional variable}s $\Obj p_0$,
  $\Obj p_1$, \dots
\tagitem{prvFalse}{The propositional constant for !!{falsity}~$\lfalse$.}{}
\tagitem{prvTrue}{The propositional constant for !!{truth}~$\ltrue$.}{}
\item The logical connectives:
  \startycommalist
  \iftag{prvNot}{\ycomma $\lnot$ (negation)}{}%
  \iftag{prvAnd}{\ycomma $\land$ (conjunction)}{}%
  \iftag{prvOr}{\ycomma $\lor$ (disjunction)}{}%
  \iftag{prvIf}{\ycomma $\lif$ (!!{conditional})}{}%
  \iftag{prvIff}{\ycomma $\liff$ (!!{biconditional})}{}%
\item Punctuation marks: (, ), and the comma.
\end{enumerate}

We denote this language of propositional logic by $\Lang L_0$.

\iftag{defNot,defOr,defAnd,defIf,defIff,defTrue,defFalse,defEx,defAll}{%
In addition to the primitive connectives introduced
above, we also use the following \emph{defined} symbols:
\startycommalist
  \iftag{defNot}{\ycomma $\lnot$ (negation)}{}%
  \iftag{defAnd}{\ycomma $\land$ (conjunction)}{}%
  \iftag{defOr}{\ycomma $\lor$ (disjunction)}{}%
  \iftag{defIf}{\ycomma $\lif$ (!!{conditional})}{}%
  \iftag{defIff}{\ycomma $\liff$ (!!{biconditional})}{}%
  \iftag{defFalse}{\ycomma $\lfalse$ (!!{falsity})}{}%
  \iftag{defTrue}{\ycomma $\ltrue$ (!!{truth})}}{}.

\begin{tagblock}{defNot,defOr,defAnd,defIf,defIff,defTrue,defFalse,defEx,defAll}
\begin{explain}
A defined symbol is not officially part of the language, but is
introduced as an informal abbreviation: it allows us to abbreviate
formulas which would, if we only used primitive symbols, get quite
long.  This is obviously an advantage.  The bigger advantage, however,
is that proofs become shorter.  If a symbol is primitive, it has to be
treated separately in proofs. The more primitive symbols, therefore,
the longer our proofs.
\end{explain}
\end{tagblock}

% Alternate symbols

\begin{intro}
You may be familiar with different terminology and symbols than the
ones we use above. Logic texts (and teachers) commonly use either
$\sim$, $\neg$, and~!\ for ``negation'', $\wedge$, $\cdot$, and $\&$
for ``conjunction''.  Commonly used symbols for the ``conditional'' or
``implication'' are $\rightarrow$, $\Rightarrow$, and $\supset$.
\iftag{prvIff,defIff}{Symbols for ``biconditional,'' ``bi-implication,''
  or ``(material) equivalence'' are $\leftrightarrow$,
  $\Leftrightarrow$, and $\equiv$.}{}
\iftag{prvFalse,defFalse}{The $\lfalse$ symbol is variously called
  ``falsity,'' ``falsum,'' ``absurdity,'' or ``bottom.''}{}
\iftag{prvTrue,defTrue}{The $\ltrue$ symbol is variously called
  ``truth,'' ``verum,'' or ``top.''}{}
\end{intro}

\begin{defn}[Formula]
\ollabel{defn:formulas}
The set~$\Frm[L_0]$ of \emph{!!{formula}s} of propositional logic
is defined inductively as follows:
\begin{enumerate}
\tagitem{prvFalse}{$\lfalse$ is an atomic !!{formula}.}{}

\tagitem{prvTrue}{$\ltrue$ is an atomic !!{formula}.}{}

\item Every !!{propositional variable}~$\Obj p_i$ is an atomic
  !!{formula}.

\tagitem{prvNot}{If $!A$ is !!a{formula}, then $\lnot !A$ is a
  !!{formula}.}{}

\tagitem{prvAnd}{If $!A$ and $!B$ are !!{formula}s, then $(!A \land
  !B)$ is !!a{formula}.}{}

\tagitem{prvOr}{If $!A$ and $!B$ are !!{formula}s, then $(!A \lor !B)$
  is !!a{formula}.}{}

\tagitem{prvIf}{If $!A$ and $!B$ are !!{formula}s, then $(!A \lif !B)$
  is !!a{formula}.}{}

\tagitem{prvIff}{If $!A$ and $!B$ are !!{formula}s, then $(!A \liff !B)$
  is !!a{formula}.}{}

\tagitem{limitClause}{Nothing else is !!a{formula}.}{}
\end{enumerate}
\end{defn}

\begin{explain}
The definition of !!{formula}s is an
\emph{inductive definition}.  Essentially, we construct the set of
!!{formula}s in infinitely many stages.  In the initial stage, we
pronounce all atomic formulas to be formulas; this corresponds to the
first few cases of the definition, i.e., the cases for
\iftag{prvTrue}{$\ltrue$, }{}%
\iftag{prvFalse}{$\lfalse$, }{}%
$\Obj p_i$.  ``Atomic !!{formula}''
thus means any !!{formula} of this form.

The other cases of the definition give rules for constructing new
!!{formula}s out of !!{formula}s already constructed.  At the second
stage, we can use them to construct !!{formula}s out of atomic
!!{formula}s.  At the third stage, we construct new formulas from the
atomic formulas and those obtained in the second stage, and so on.  A
!!{formula} is anything that is eventually constructed at such a
stage, and nothing else.
\end{explain}

\begin{tagblock}{defNot,defOr,defAnd,defIf,defIff,defTrue,defFalse,defEx,defAll}
\begin{defn}
Formulas constructed using the defined operators are to be understood
as follows:

\begin{tagenumerate}{defTrue,defFalse,defNot,defOr,defAnd,defIf,defIff,defEx,defAll}

\tagitem{defTrue}{$\ltrue$ abbreviates
  \iftag{prvFalse}{$\lnot\lfalse$}{$(!A \lor \lnot !A)$ for some
    fixed atomic !!{formula}~$!A$}.}{}

\tagitem{defFalse}{$\lfalse$ abbreviates
  \iftag{prvTrue}{$\lnot\ltrue$}{$(!A \land \lnot !A)$ for some
    fixed atomic !!{formula}~$!A$}.}{}

\tagitem{defNot}{$\lnot !A$ abbreviates $!A \lif \lfalse$.}{}

\tagitem{defOr}{$!A \lor !B$ abbreviates
  \iftag{prvAnd}{$\lnot(\lnot !A \land \lnot !B)$}{$\lnot !A \lif
    !B$}.}{}

\tagitem{defAnd}{$!A \land !B$ abbreviates
  \iftag{prvOr}{$\lnot(\lnot !A \lor \lnot !B)$}{$\lnot (!A \lif
    \lnot !B)$}.}{}

\tagitem{defIf}{$!A \lif !B$ abbreviates
  \iftag{prvOr}{$\lnot !A \lor !B)$}{$\lnot (!A \land \lnot !B)$}.}{}

\tagitem{defIff}{$!A \liff !B$ abbreviates $(!A \lif !B) \land (!B
  \lif !A)$.}{}
\end{tagenumerate}
\end{defn}
\end{tagblock}

\begin{defn}[Syntactic identity]
The symbol $\ident$ expresses syntactic identity between strings of
symbols, i.e., $!A \ident !B$ iff $!A$ and $!B$ are strings of symbols
of the same length and which contain the same symbol in each place.
\end{defn}

The $\ident$ symbol may be flanked by strings obtained by
concatenation, e.g., $!A \ident (!B \lor !C)$ means: the string of
symbols~$!A$ is the same string as the one obtained by concatenating
an opening parenthesis, the string $!B$, the $\lor$ symbol, the
string~$!C$, and a closing parenthesis, in this order. If this is the
case, then we know that the first symbol of $!A$ is an opening
parenthesis, $!A$ contains $!B$ as a substring (starting at the second
symbol), that substring is followed by $\lor$, etc.

\end{document}
