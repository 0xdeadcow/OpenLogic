% Part: propositional-logic
% Chapter: propositional-logic
% Section: preliminaries

\documentclass[../../../include/open-logic-section]{subfiles}

\begin{document}

\olfileid{prp}{prp}{pre}

\olsection{Preliminaries}

The language $\Lang L_0$ of classical propositional logic
comprises as basic symbols (1) countably many propositional !!{variable}s
$p_0$, $p_1$,~\dots, as well as (2) symbols for the logical !!{operator}s
  \startycommalist
  \iftag{prvNot}{\ycomma $\lnot$ (negation)}{}%
  \iftag{prvAnd}{\ycomma $\land$ (conjunction)}{}%
  \iftag{prvOr}{\ycomma $\lor$ (disjunction)}{}%
  \iftag{prvIf}{\ycomma $\lif$ (!!{conditional})}{}%
  \iftag{prvIff}{\ycomma $\liff$ (!!{biconditional})}{},
\iftag{prvFalse}{the propositional constant for !!{falsity}~$\lfalse$,}{}
\iftag{prvTrue}{the propositional constant for !!{truth}~$\ltrue$,}{}
 and the two parentheses $($ and
$)$. We assume that these symbols are all distinct and no one occurs as
a part of another one. We refer to the set of the propositional
!!{variable}s as~$\PVar$.

\begin{defn} [Formulas]
The set $\Frm[L_0]$ of the !!{formula}s of the language
  $\Lang L_0$ is inductively defined as the smallest set of strings
  over the alphabet containing $\PVar$ and such that
  if $!A$ and $!B$ are in $\Frm[L_0]$, then so are:
\begin{enumerate}
\tagitem{prvNot} {$(\lnot !A$);}{}
\tagitem{prvIf} {$(!A \lif !B)$.}{}
\end{enumerate}
\end{defn}     
                        
\begin{thm} \ollabel{thm:induction}
\emph{Principle of induction on !!{formula}s}: If some
  property $P$ holds of all the propositional !!{variable}s and is such
  that it holds for $(\lnot !A)$ and $(!A \lif !B)$
  whenever it holds for $!A$ and $!B$, then $P$ holds of all
 !!{formula}s in $\Frm[L_0]$ .
\end{thm}

\begin{proof}
  Let $S$ be the collection of all !!{formula}s with property $P$, so
  that, in particular, $S \subseteq \Frm[L_0]$. Then $S$ contains
  the propositional !!{variable}s and is closed under the !!{operator}s;
  since $\Frm[L_0]$ is the smallest such class, also
  $\Frm[L_0]\subseteq S$. So $\Frm[L_0] = S$, and every
  formula has propery~$P$.
\end{proof}

\begin{prob} 
Prove that any !!{formula} in $\Frm[L_0]$ is \emph{balanced}, in
that it has as many left parentheses as right ones.
\end{prob}

\begin{prob} 
Prove that no !!{formula} begins with $\lnot$ and that no proper
initial segment of !!a{formula} is !!a{formula}.
\end{prob}

\begin{explain}
The !!{formula}s $(!A \lor !B)$ and $(!A \land !B)$ abbreviate
$((\lnot !A) \lif !B)$ and $\lnot(!A \lif (\lnot !B))$,
respectively. Similarly, $!A \equiv !B$ abbreviates $(!A \lif !B)
\land (!B \lif !A)$. Parentheses around $\lnot !A$ are usually dropped,
with the understanding that $\lnot$ binds the shortest !!{formula}
that follows it; outermost parentheses are likewise usually dropped.
\end{explain}
 
\begin{prop}[Unique Readability]
Any !!{formula} $!A$ in $ \Frm[L_0]$ has
exactly one parsing as one of the following
\begin{enumerate}
\item $p_n$ for some $p_n \in  \PVar$
\tagitem{prvNot}{$(\lnot !B)$ for some $!B$ in $\Frm[L_0]$;}{}
\tagitem{prvIf}{$(!B \lif !D)$ for some $!B, !D$ in
    $\Frm[L_0]$.}{}
\end{enumerate}
Moreover, such parsing is \emph{unique}, in that, e.g., $!A$ cannot
have the form $(\lnot !B)$ \emph{ in two different ways}.
\end{prop}

\begin{proof}
By induction on $!A$. For instance, suppose that $!A$ has
two distinct readings as $(!B \lif !D)$ and $(!B' \lif
!D')$. Then $!B$ and $!B'$ must be the same (or else one would
be a proper initial segment of the other); so is the two readings of
$!A$ are distinct it must be because $!D$ and $!D'$ are
distinct readings of the same sequence of symbols, which is impossible
by the inductive hypothesis. 
\end{proof}

\begin{thm}[Principle of Definition by Recursion] 
\ollabel{thm:rec} 
For any set $V$ and !!{function}s $\mathbf{i} \colon \PVar \to V$ and
$h_1, h_2$ from $V$ and $V \times V$, respectively, into $V$, there
exists exactly one function $f \colon \Frm[L_0] \to V$ satisfying the
following equations:
\begin{align*}
  f(p_n) &= \mathbf{i}(p_n) \\
  f(\lnot !A) &= h_1(f(!A))\\
  f(!A \lif !B) &= h_2(f(!A),f(!B))
\end{align*}
\end{thm}

\begin{proof}
Let $\mathcal{F}$ be the class of all functions $g$ such that:
\begin{itemize}
\item $\dom{g}$ is finite and closed under !!{subformula}s;
\item whenever $!A$ is in $\dom{g}$ then $g$ satisfies the
    equation corresponding to $!A$.
\end{itemize}
Put $f = \bigcup \mathcal{F}$. It is easy to see that: 
\begin{enumerate}
\item No two functions $g$ and $g'$ in $\mathcal{F}$ disagree on any
  of the arguments on which they are both defined. Hence, $f$ is
  well-defined as a function. (This requires unique readability.)
\item $f$ satisfies the equations.
\item $f$ is unique.
\item $f$ is total, i.e., $\dom{f} = \Frm[L_0]$.
\end{enumerate}
These are established by induction on !!{formula}s.
\end{proof}

\begin{defn} [Uniform Substitution]
If $!A$ and $!B$ are !!{formula}s, and $p_i$ is a propositional
!!{variable}, then $\Subst{!A}{!B}{p_i}$ denotes the result of
replacing each occurrence of $p_i$ by an occurrence of $!B$ in $!A$;
similarly, the simultaneous substitution of $p_1$, \dots,~$p_n$ by
!!{formula}s $!B_1$, \dots,~$\!B_n$ is denoted by
$\Subst{!A}{!B_1}{p_1},\ldots, \Subst{}{!B_n}{p_n}$.
\end{defn}

\begin{prob}
Give a mathematically rigorous definition of $\Subst{!A}{!B}{p_i}$
using~\olref{thm:rec}.
\end{prob}

\end{document}

