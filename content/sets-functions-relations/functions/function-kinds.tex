% Part: sets-functions-relations
% Chapter: functions
% Section: kinds

\documentclass[../../../include/open-logic-section]{subfiles}

\begin{document}

\olfileid{sfr}{fun}{kin}
\olsection{Kinds of Functions}

\begin{defn}
A function $f \colon X \rightarrow Y$ is \emph{!!{surjective}} iff $Y$
is also the range of~$f$, i.e., for every $y \in Y$ there is at least
one $x \in X$ such that~$f(x) = y$.
\end{defn}

\begin{explain}
If you want to show that a function is !!{surjective}, then you need
to show that every object in the codomain is the output of the
function given some input or other.
\end{explain}

\begin{defn}
A function $f \colon X \rightarrow Y$ is \emph{!!{injective}} iff for
each $y \in Y$ there is at most one $x \in X$ such that~$f(x) = y$.
\end{defn}

\begin{explain}
Any function pairs each possible input with a unique output. !!^a{injective}
function has a unique input for each possible output. If you want to
show that a function $f$ is !!{injective}, you need to show that for
any element $y$ of the codomain, if $f(x)=y$ and $f(w)=y$, then $x=w$.

A function which is neither !!{injective}, nor !!{surjective}, is the
constant function $f\colon \Nat \rightarrow \Nat$ where $f(x) = 1$.

A function which is both !!{injective} and !!{surjective} is the
identity function $f\colon \Nat \rightarrow \Nat$ where $f(x) = x$.

The successor function $f \colon \Nat \rightarrow \Nat$ where $f(x) =
x+1$ is !!{injective}, but not !!{surjective}.

The function
\[
f(x) =
\begin{cases}
  \frac{x}{2} & \text{if $x$ is even} \\
  \frac{x+1}{2} & \text{if $x$ is odd.}
\end{cases}
\]
is !!{surjective}, but not !!{injective}.
\end{explain}

\begin{defn}
A function $f \colon X \to Y$ is \emph{!!{bijective}} iff it is both
!!{surjective} and !!{injective}.  We call such a function
\article{bijection} \emph{!!{bijection}} from $X$ to~$Y$ (or between
$X$ and~$Y$).
\end{defn}

\end{document}
