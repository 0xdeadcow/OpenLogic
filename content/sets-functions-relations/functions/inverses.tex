% Part: sets-functions-relations
% Chapter: functions
% Section: inverses

\documentclass[../../../include/open-logic-section]{subfiles}

\begin{document}

\olfileid{sfr}{fun}{inv}
\olsection{Inverses of Functions}

\begin{explain}
We think of functions as maps. An obvious question to ask about
functions, then, is whether the mapping can be ``reversed.'' For
instance, the successor function $f(x) = x + 1$ can be reversed, in
the sense that the function $g(y) = y - 1$ ``undoes'' what $f$ does. 

But we must be careful. Although the definition of~$g$ defines a
function $\Int \to \Int$, it does not define a \emph{function} $\Nat
\to \Nat$, since $g(0) \notin \Nat$.  So even in simple cases, it is
not quite obvious whether a function can be reversed; it may depend on
the domain and codomain.

This is made more precise by the notion of an inverse of a function.
\end{explain}

\begin{defn}
A function $g \colon B \to A$ is an \emph{inverse} of a function $f
\colon A \to B$ if $f(g(y)) = y$ and $g(f(x)) = x$ for all $x \in A$
and $y \in B$.
\end{defn}

If $f$ has an inverse~$g$, we often write $f^{-1}$ instead of~$g$.

\begin{explain}
Now we will determine when functions have inverses. A good candidate
for an inverse of $f\colon A \to B$ is $g\colon B \to A$ ``defined
by''
\[
g(y) = \text{``the'' $x$ such that $f(x) = y$.}
\]
But the scare quotes around ``defined by'' (and ``the'') suggest that
this is not a definition.  At least, it will not always work, with
complete generality. For, in order for this definition to specify a
function, there has to be one and only one~$x$ such that $f(x) =
y$---the output of~$g$ has to be uniquely specified. Moreover, it has
to be specified for every $y \in B$.  If there are $x_1$ and $x_2 \in
A$ with $x_1 \neq x_2$ but $f(x_1) = f(x_2)$, then $g(y)$ would not be
uniquely specified for $y = f(x_1) = f(x_2)$. And if there is no~$x$
at all such that $f(x) = y$, then $g(y)$ is not specified at all.  In
other words, for $g$ to be defined, $f$~must be both !!{injective} and
!!{surjective}.

Let's go slowly. We'll divide the question into two: Given a
function~$f\colon A \to B$, when is there a function $g\colon B \to A$
so that $g(f(x)) = x$? Such a $g$ ``undoes'' what $f$ does, and is
called a \emph{left inverse} of~$f$. Secondly, when is there a
function $h\colon B \to A$ so that $f(h(y)) = y$? Such an $h$ is
called a \emph{right inverse} of~$f$---$f$ ``undoes'' what $h$~does.
\end{explain}

\begin{prop}
If $f\colon A \to B$ is !!{injective}, then there is a \emph{left
inverse}~$g\colon B \to A$ of~$f$ so that $g(f(x)) = x$ for all $x
\in A$.
\end{prop}

\begin{proof}
Suppose that $f\colon A \to B$ is !!{injective}. Consider a $y \in B$.
If $y \in \ran{f}$, there is an $x \in A$ so that $f(x) = y$. Because
$f$ is !!{injective}, there is only one such~$x \in A$. Then we can
define: $g(y) = x$, i.e., $g(y)$ is ``the'' $x \in A$ such that $f(x)
= y$.  If $y \notin \ran{f}$, we can map it to any~$a \in A$. So, we
can pick an $a \in A$ and define $g\colon B \to A$ by:
\[
g(y) = \begin{cases}
    x & \text{if $f(x) = y$}\\
    a & \text{if $y \notin \ran{f}$.}
\end{cases}
\]
It is defined for all $y \in B$, since for each such $y \in \ran{f}$
there is exactly one $x \in A$ such that $f(x) = y$. By definition, if
$y = f(x)$, then $g(y) = x$, i.e., $g(f(x)) = x$.
\end{proof}

\begin{prob}
Show that if $f\colon A \to B$ has a left inverse~$g$, then $f$~is
!!{injective}.
\end{prob}

\begin{prop}
    If $f\colon A \to B$ is !!{surjective}, then there is a
    \emph{right inverse}~$h\colon B \to A$ of~$f$ so that $f(h(y)) =
    y$ for all~$y \in B$.
\end{prop}

\begin{proof}
Suppose that $f\colon A \to B$ is !!{surjective}. Consider a $y \in
B$. Since $f$~is !!{surjective}, there is an $x_y \in A$ with $f(x_y)
= y$.  Then we can define: $h(y) = x_y$, i.e., for each $y \in B$ we
choose some $x \in A$ so that $f(x) = y$; since $f$~is !!{surjective}
there is always at least one to choose from.\footnote{Since $f$ is
!!{surjective}, for every~$y \in B$ the set $\Setabs{x}{f(x) = y}$ is
nonempty. Our definition of~$h$ requires that we choose a single $x$
from each of these sets. That this is always possible is actually not
obvious---the possibility of making these choices is simply assumed as an axiom. 
In other words, this proposition assumes the so-called Axiom of
Choice, an issue we will \oliflabeldef{cumul:::part}{revisit in \olref[sth][choice][]{chap}}{gloss over}. 
However, in many specific cases, e.g., when $A = \Nat$ or is finite, or when $f$ is !!{bijective},
the Axiom of Choice is not required. (In the particular case when $f$ is !!{bijective}, for each $y \in B$ the set 
$\Setabs{x}{f(x) = y}$ has exactly one !!{element}, so that there is no choice to make.)} 
By definition, if $x = h(y)$,
then $f(x) = y$, i.e., for any $y \in B$, $f(h(y)) = y$.
\end{proof}

\begin{prob}
Show that if $f\colon A \to B$ has a right inverse~$h$, then $f$~is
!!{surjective}.
\end{prob}

\begin{explain}
  By combining the ideas in the previous proof, we now get that every
  !!{surjection} has an inverse, i.e., there is a single function
  which is both a left and right inverse of~$f$.
\end{explain}

\begin{prop}\ollabel{prop:bijection-inverse}
If $f\colon A \to B$ is !!{bijective}, there is a
function~$f^{-1}\colon B \to A$ so that for all $x \in A$,
$f^{-1}(f(x)) = x$ and for all $y \in B$, $f(f^{-1}(y)) = y$.
\end{prop}

\begin{proof}
Exercise.
\end{proof}

\begin{prob}
Prove \olref[sfr][fun][inv]{prop:bijection-inverse}. You have to
define~$f^{-1}$, show that it is a function, and show that it is an
inverse of~$f$, i.e., $f^{-1}(f(x)) = x$ and $f(f^{-1}(y)) = y$ for
all $x \in A$ and $y \in B$.
\end{prob}

\begin{explain}
There is a slightly more general way to extract inverses. We saw in
\olref[kin]{sec} that every function $f$ induces !!a{surjection} $f'
\colon A \to \ran{f}$ by letting $f'(x) = f(x)$ for all $x \in A$.
Clearly, if $f$~is !!{injective}, then $f'$~is !!{bijective}, so that
it has a unique inverse by \olref{prop:bijection-inverse}. By a very
minor abuse of notation, we sometimes call the inverse of $f'$ simply
``the inverse of~$f$.''
\end{explain}

\begin{prop}\ollabel{prop:left-right}%
  Show that if $f\colon A \to B$ has a left inverse~$g$ and a right
  inverse~$h$, then $h = g$.
\end{prop}

\begin{proof}
  Exercise.
\end{proof}

\begin{prob}
  Prove \olref[sfr][fun][inv]{prop:left-right}.
\end{prob}

\begin{prop}\ollabel{prop:inverse-unique}
Every function~$f$ has at most one inverse.
\end{prop}

\begin{proof}
  Suppose $g$ and $h$ are both inverses of~$f$. Then in particular
  $g$~is a left inverse of~$f$ and $h$~is a right inverse. By
  \olref{prop:left-right}, $g = h$.
\end{proof}

\end{document}
