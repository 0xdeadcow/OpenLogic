% Part: sets-functions-relations
% Chapter: functions
% Section: inverses

\documentclass[../../../include/open-logic-section]{subfiles}

\begin{document}

\olfileid{sfr}{fun}{inv}
\olsection{Inverses of Functions}

\begin{explain}
One obvious question about functions is whether a given mapping can be
``reversed.'' For instance, the successor function $f(x) = x + 1$ can
be reversed in the sense that the function $g(y) = y - 1$ ``undos''
what $f$ does. But we must be careful: While the definition of $g$
defines a function $\Int \to \Int$, it does not define a function
$\Nat \to \Nat$ ($g(0) \notin \Nat$).  So even in simple cases, it is
not quite obvious if functions can be reversed, and that it may depend
on the domain and codomain.  Let's give a precise definition.
\end{explain}

\begin{defn}
A function $g \colon Y \to X$ is an \emph{inverse} of a function $f
\colon X \to Y$ if $f(g(y)) = y$ and $g(f(x)) = x$ for all $x \in X$
and $y \in Y$.
\end{defn}

\begin{explain}
When do functions have inverses?  A good candidate for an inverse of
$f\colon X \to Y$ is $g\colon Y \to X$ ``defined by''
\[
g(y) = \text{``the'' $x$ such that $f(x) = y$.}
\]
The scare quotes around ``defined by'' suggest that this is not a
definition.  At least, it is not in general.  For in order for this
definition to specify a function, there has to be one and only one~$x$
such that $f(x) = y$---the output of $g$ has to be uniquely specified.
Moreover, it has to be specified for every $y \in Y$.  If there are
$x_1$ and $x_2 \in X$ with $x_1 \neq x_2$ but $f(x_1) = f(x_2)$, then
$g(y)$ would not be uniquely specified for $y = f(x_1) = f(x_2)$. And
if there is no $x$ at all such that $f(x) = y$, then $g(y)$ is not
specified at all.  In other words, for $g$ to be defined, $f$ has to
be !!{injective} and !!{surjective}.
\end{explain}

\begin{prop}
If $f\colon X \to Y$ is !!{bijective}, $f$ has a unique
inverse~$f^{-1}\colon Y \to X$.
\end{prop}

\begin{proof}
Exercise.
\end{proof}

\begin{prob}
Show that if $f$ is bijective, an inverse $g$ of $f$ exists, i.e.,
define such a $g$, show that it is a function, and show that it is an
inverse of~$f$, i.e., $f(g(y)) = y$ and $g(f(x)) = x$ for all $x \in
X$ and $y \in Y$.
\end{prob}

\begin{prob}
Show that if $f\colon X \to Y$ has an inverse~$g$, then $f$ is
!!{bijective}.
\end{prob}

\begin{prob}
Show that if $g\colon Y \to X$ and $g'\colon Y \to X$ are inverses
of~$f\colon X \to Y$, then $g = g'$, i.e., for all $y \in Y$, $g(y) =
g'(y)$.
\end{prob}

\end{document}
