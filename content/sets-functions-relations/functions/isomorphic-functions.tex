% Part: sets-functions-relations
% Chapter: functions
% Section: isomorphisms

\documentclass[../../../include/open-logic-section]{subfiles}


\begin{document}

\olfileid{sfr}{fun}{iso}
\olsection{Isomorphism}

\begin{explain}
An \emph{isomorphism} is a bijection that preserves the structure of the sets it relates, where structure is a matter of the relationships that obtain between the members of the sets. Consider the following two sets $X=\{1,2,3\}$ and $Y=\{4,5,6\}$. These sets are both structured by the relations successor, less than and greater than. An isomorphism between the two sets is a bijection that preserves those structures. So a function $f \colon X \to Y$ is an isomorphism if, among other things, $i<j$ iff $f(i)<f(j)$, and $j$ is the successor of $i$ iff $f(j)$ is the successor of $f(i)$.
\end{explain}

\begin{defn}
Let U be the pair $\langle X, R\rangle$ and V be the pair $\langle Y, S\rangle$ such that $X$ and $Y$ are sets and $R$ and $S$ are relations on $X$ and $Y$ respectively. A bijection $f$ from $X$ to $Y$ is an \emph{isomorphism}  from U to V iff it preserves the relational structure, that is, for any $x_{1}$ and $x_{2}$ in $X$, $\langle x_{1},x_{2}\rangle\in R$ iff $\langle f(x_{1}),f(x_{2})\rangle\in S$.
\end{defn}

\begin{ex}
Consider the following two sets $X=\{1,2,3\}$ and $Y=\{4,5,6\}$, and the relations successor, less than, and greater than. The function $f\colon X \to Y$ where $f(x) = x+3$ is an isomorphism between $X$ and $Y$.
\end{ex}

\end{document}
