% Part: sets-functions-relations
% Chapter: functions
% Section: functions-relations

\documentclass[../../../include/open-logic-section]{subfiles}


\begin{document}

\olfileid{sfr}{fun}{rel}

\olsection{Functions and Relations}

\begin{explain}
A function which maps !!{element}s of~$X$ to !!{element}s of~$Y$
obviously defines a relation between $X$ and $Y$, namely the relation
which holds between $x$ and $y$ iff $f(x) = y$.  In fact, we might
even---if we are interested in reducing the building blocks of
mathematics for instance---\emph{identify} the function $f$ with this
relation, i.e., with a set of pairs.  This then raises the question:
which relations define functions in this way?
\end{explain}

\begin{defn}
Let $f\colon X \pto Y$ be a partial function. The \emph{graph} of $f$
is the relation $R_f \subseteq X \times Y$ defined by
\[
R_f = \Setabs{\tuple{x,y}}{f(x) = y}.
\]
\end{defn}

\begin{prop}
Suppose $R \subseteq X \times Y$ has the property that whenever $Rxy$
and $Rxy'$ then $y = y'$.  Then $R$ is the graph of the partial
function $f\colon X \pto Y$ defined by: if there is a $y$ such that
$Rxy$, then $f(x) = y$, otherwise $f(x) \undefined$.  If $R$ is also 
\emph{serial}, i.e., for each $x \in X$ there is a $y \in Y$ such that
$Rxy$, then $f$ is total.
\end{prop}

\begin{proof}
Suppose there is a $y$ such that $Rxy$.  If there were another $y'
\neq y$ such that $Rxy'$, the condition on $R$ would be
violated. Hence, if there is a $y$ such that $Rxy$, that $y$ is
unique, and so $f$ is well-defined.  Obviously, $R_f = R$ and $f$ is
total if~$R$ is serial.
\end{proof}

\begin{prob}
Suppose $f \colon X \to Y$ and $g \colon Y \to Z$. Show that the graph
of $(g \circ f)$ is $R_f \mid R_g$.
\end{prob}

\end{document}
