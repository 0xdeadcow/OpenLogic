% Part: sets-functions-relations
% Chapter: functions
% Section: functions-relations

\documentclass[../../../include/open-logic-section]{subfiles}


\begin{document}

\olfileid{sfr}{fun}{rel}

\olsection{Functions as Relations}

\begin{explain} 
A function which maps !!{element}s of~$A$ to !!{element}s of~$B$
obviously defines a relation between $A$ and~$B$, namely the relation
which holds between $x$ and $y$ iff $f(x) = y$.  In fact, we might
even---if we are interested in reducing the building blocks of
mathematics for instance---\emph{identify} the function~$f$ with this
relation, i.e., with a set of pairs.  This then raises the question:
which relations define functions in this way?
\end{explain}

\begin{defn}[Graph of a function] Let $f\colon A \to B$ be a function.
The \emph{graph} of~$f$ is the relation $R_f \subseteq A \times B$
defined by
\[
R_f = \Setabs{\tuple{x,y}}{f(x) = y}.
\]
\end{defn}

\begin{explain}
The graph of a function is uniquely determined, by extensionality.
Moreover, extensionality (on sets) will immediately vindicate the
implicit principle of extensionality for functions,
whereby if $f$ and~$g$ share a domain and codomain then they are
identical if they agree on all values. 

Similarly, if a relation is ``functional'', then it is the graph of a function. 
\end{explain}

\begin{prop}\ollabel{prop:graph-function}
Let $R \subseteq A \times B$ be such that:
\begin{enumerate}
\item If $Rxy$ and $Rxz$ then $y = z$; and 
\item for every $x \in A$ there is some $y \in B$ such that $\tuple{x,
y} \in R$.  
\end{enumerate}
Then $R$ is the graph of the function $f\colon A \to B$ defined by
$f(x) = y$ iff $Rxy$. 
\end{prop}

\begin{proof}
Suppose there is a $y$ such that $Rxy$.  If there were another $z \neq
y$ such that $Rxz$, the condition on~$R$ would be violated. Hence, if
there is a $y$ such that $Rxy$, this $y$ is unique, and so $f$ is
well-defined.  Obviously, $R_f = R$.
\end{proof}

\begin{explain}
Every function $f\colon A \to B$ has a graph, i.e., a relation on $A
\times B$ defined by $f(x) = y$. On the other hand, every relation~$R
\subseteq A \times B$ with the properties given in
\olref{prop:graph-function} is the graph of a function~$f \colon A \to
B$. Because of this close connection between functions and their
graphs, we can think of a function simply as its graph. In other
words, functions can be identified with certain relations, i.e., with
certain sets of tuples. \oliflabeldef{sfr:rel:ref:sec}{Note, though,
that the spirit of this ``identification'' is as in
\olref[sfr][rel][ref]{sec}: it is not a claim about the metaphysics of
functions, but an observation that it is convenient to \emph{treat}
functions as certain sets. One reason that this is so convenient, is
that w}{W}e can now consider performing similar operations on
functions as we performed on relations (see
\olref[sfr][rel][ops]{sec}). In particular:
\end{explain}

\begin{defn}\ollabel{defn:funimage}
Let $f \colon A \to B$ be a function with $C\subseteq A$.

The \emph{restriction} of~$f$ to~$C$ is the
function~$\funrestrictionto{f}{C}\colon C \to B$ defined by
$(\funrestrictionto{f}{C})(x) = f(x)$ for all $x \in C$. In other
words, $\funrestrictionto{f}{C} = \Setabs{\tuple{x, y} \in R_f}{x \in
C}$.

The \emph{application} of~$f$ to~$C$ is $\funimage{f}{C} =
\Setabs{f(x)}{x \in C}$. We also call this the \emph{image} of~$C$
under~$f$.
\end{defn}

\begin{explain}
It follows from these definitions that $\ran{f} =
\funimage{f}{\dom{f}}$, for any function~$f$.
\oliflabeldef{sfr:rel:ops:sec}{These notions are exactly as one would
expect, given the definitions in \olref[sfr][rel][ops]{sec} and our
identification of functions with relations. But two other
operations---inverses and relative products---require a little more
detail. We will provide that in \olref[inv]{sec} and
\olref[cmp]{sec}.}{}
\end{explain}

\end{document}
