% Part: sets-functions-relations
% Chapter: functions
% Section: partial-kinds

\documentclass[../../../include/open-logic-section]{subfiles}


\begin{document}

\olfileid{sfr}{fun}{par}

\olsection{Partial Functions}

\begin{explain}
It is sometimes useful to relax the definition of function so that it
is not required that the output of the function is defined for all
possible inputs. Such mappings are called \emph{partial functions}.
\end{explain}

\begin{defn}
A \emph{partial function} $f \colon X \pto Y$ is a mapping which
assigns to every !!{element} of~$X$ at most one !!{element} of~$Y$.
If $f$ assigns an element of~$Y$ to $x \in X$, we say $f(x)$ is
\emph{defined}, and otherwise \emph{undefined}. If $f(x)$ is defined,
we write $f(x) \fdefined$, otherwise $f(x) \fundefined$. The
\emph{domain} of a partial function~$f$ is the subset of~$X$ where it
is defined, i.e., $\dom{f} = \Setabs{x}{f(x) \fdefined}$.
\end{defn}

\begin{ex}
Every function $f\colon X \to Y$ is also a partial function. Partial
functions that are defined everywhere on $X$---i.e., what we so far
have simply called a function---are also called \emph{total}
functions.
\end{ex}

\begin{ex}
The partial function $f \colon \Real \pto \Real$ given by $f(x) = 1/x$
is undefined for $x = 0$, and defined everywhere else.
\end{ex}

\begin{prob}
Given $f\colon X \pto Y$, define the partial function $g\colon Y \pto
X$ by: for any $y \in Y$, if there is a unique $x \in X$ such that
$f(x) = y$, then $g(y) = x$; otherwise $g(y) \fundefined$.  Show that
if $f$ is injective, then $g(f(x)) = x$ for all $x \in \dom{f}$, and
$f(g(y)) = y$ for all $y \in \ran{f}$.
\end{prob}

\end{document}
