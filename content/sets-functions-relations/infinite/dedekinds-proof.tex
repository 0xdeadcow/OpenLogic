% Part: sets-functions-relations
% Chapter: infinite
% Section: dedekinds-proof
%
\documentclass[../../../include/open-logic-section]{subfiles}

\begin{document}

\olfileid{sfr}{infinite}{dedekindsproof}

\olsection[Dedekind's ``Proof'']{Dedekind's ``Proof'' of the 
Existence of an Infinite Set}

In this chapter, we have offered a set-theoretic treatment of the
natural numbers, in terms of Dedekind algebras. In
\olref[arith][ref]{sec}, we reflected on the philosophical
significance of the arithmetisation of analysis (among other things).
Now we should reflect on the significance of what we have achieved
here.

Throughout \olref[sfr][arith][]{chap}, we took the natural numbers as
given, and used them to construct the integers, rationals, and reals,
explicitly. In this chapter, we have not given an explicit
construction of the natural numbers. We have just shown that,
\emph{given any Dedekind infinite set}, we can define a set which will
behave just like we want~$\Nat$ to behave. 

Obviously, then, we cannot claim to have answered a metaphysical
question, such as \emph{which objects are the natural numbers}. But
that's a good thing. After all, in  \olref[sfr][arith][ref]{sec}, we
emphasized that we would be wrong to think of the definition of
$\Real$ as the set of Dedekind cuts as a \emph{discovery}, rather than
a convenient stipulation. The crucial observation is that the Dedekind
cuts exemplify the same key mathematical properties as the real
numbers. So too here: the crucial observation is that \emph{any}
Dedekind algebra exemplifies the key mathematical properties as the
natural numbers. (Indeed, Dedekind pushed this point home by proving
that all Dedekind algebras are \emph{isomorphic} (\citeyear[Theorems
132--3]{Dedekind1888}). It is no surprise, then, that many
contemporary ``structuralists'' cite Dedekind as a forerunner.)

 Moreover, we have shown how to embed the theory of the natural
 numbers into a na\"ive simple set theory, which itself still remains
 rather informal, but which doesn't (apparently) assume the natural
 numbers as given. So, we may be on the way to realising Dedekind's
 own ambitious project, which he explained thus:
\begin{quote}
	In science nothing capable of proof ought to be believed without
	proof. Though this demand seems reasonable, I cannot regard it as
	having been met even in the most recent methods of laying the
	foundations of the simplest science; viz., that part of logic
	which deals with the theory of numbers. In speaking of arithmetic
	(algebra, analysis) as merely a part of logic I mean to imply that
	I consider the number-concept entirely independent of the notions
	or intuitions of space and time---that I rather consider it an
	immediate product of the pure laws of thought.
	\citep[preface]{Dedekind1888}
\end{quote}
Dedekind's bold idea is this. We have just shown how to build the
natural numbers using (na\"ive) set theory alone. In
\olref[sfr][arith][]{chap}, we saw how to construct the reals given
the natural numbers and some set theory. So, perhaps, ``arithmetic
(algebra, analysis)'' turn out to be ``merely a part of logic'' (in
Dedekind's extended sense of the word ``logic'').

That's the idea. But hold on for a moment. Our construction of a
Dedekind algebra (our surrogate for the natural numbers) is
conditional on the existence of a Dedekind infinite set. (Just look
back to \olref[sfr][infinite][dedekind]{thm:DedekindInfiniteAlgebra}.)
Unless the existence of a Dedekind infinite set can be established via
``logic'' or ``the pure laws of thought'', the project stalls. 

So, \emph{can} the existence of a Dedekind infinite set be established
by ``the pure laws of thought''? Here was Dedekind's effort:
\begin{quote}
  My own realm of thoughts, i.e., the totality $S$ of all things which
  can be objects of my thought, is infinite. For if $s$ signifies an
  element of~$S$, then the thought $s'$ that~$s$ can be an object of
  my thought, is itself an element of~$S$. If we regard this as an
  image $\phi(s)$ of the element~$s$, then \dots~$S$ is [Dedekind]
  infinite, which was to be proved.
	\citep[\S66]{Dedekind1888}
\end{quote}
This is quite an astonishing thing to find in the middle of a book
which largely consists of highly rigorous mathematical proofs. Two
remarks are worth making. 

First: this ``proof'' scarcely has what we would now recognize as a
``mathematical'' character. It speaks of psychological objects
(thoughts), and merely \emph{possible} ones at that.

Second: at least as we have presented Dedekind algebras, this
``proof'' has a straightforward technical shortcoming. If Dedekind's
argument is successful, it establishes only that there are infinitely
many things (specifically, infinitely many thoughts). But Dedekind
also needs to give us a reason to regard~$S$ as a single \emph{set},
with infinitely many !!{element}s, rather than thinking of~$S$ as
\emph{some things} (in the plural). 

The fact that Dedekind did not see a gap here might suggest that his
use of the word ``totality'' does not precisely track \emph{our} use
of the word ``set''.\footnote{Indeed, we have other reasons to think
it did not; see e.g., \citet[p.~23]{Potter2004}.} But this would not be
too surprising. The project we have pursued in the last two
chapters---a ``construction'' of the naturals, and from them a
``construction'' of the integers, reals and rationals---has all been
carried out na\"ively. We have helped ourselves to this set, or that
set, as and when we have needed them, without laying down many general
principles concerning exactly which sets exist, and when. But we know
that we need \emph{some} general principles, for otherwise we will
fall into Russell's Paradox.

 The time has come for us to outgrow our na\"ivety. 

\end{document}