\documentclass[../../../include/open-logic-section]{subfiles}

\begin{document}

\olfileid{sfr}{infinite}{card-sb}

\olsection{Appendix: Proving Schr\"oder-Bernstein}

Before we depart from na\"ive set theory, we have one last na\"ive
(but sophisticated!) proof to consider. This is a proof of
Schr\"oder-Bernstein (\olref[sfr][siz][sb]{thm:schroder-bernstein}): if
$\cardle{A}{B}$ and $\cardle{B}{A}$ then $\cardeq{A}{B}$; i.e., given
!!{injection}s $f \colon A \to B$ and $g \colon B \to A$ there is
!!a{bijection} $h \colon A \to B$. 

In this chapter, we followed Dedekind's notion of \emph{closures}. In
fact, Dedekind provided a lovely proof of Schr\"oder-Bernstein using this notion, and we
will present it here. The proof closely follows
\citet[pp.~157--8]{Potter2004}, if you want a slightly different but
essentially similar treatment. A little googling will also convince
you that this is a theorem---rather like the irrationality of
$\sqrt{2}$---for which \emph{many} interesting and different proofs
exist.

Using similar notation as \olref[sfr][infinite][dedekind]{Closure},
let
\[
\Closureofunder{f}{B} = \bigcap \Setabs{X}{B \subseteq X 
\text{ and $X$ is $f$-closed}}
\]
for each set $B$ and function~$f$. Defined thus,
$\Closureofunder{f}{B}$ is the smallest $f$-closed set containing~$B$,
in that:

\begin{lem}\ollabel{Closureprops}
	For any function $f$, and any $B$:
	\begin{enumerate}
		\item\ollabel{Closurehaselem} $B \subseteq \Closureofunder{f}{B}$; and
		\item\ollabel{Closureclosed} $\Closureofunder{f}{B}$ is $f$-closed; and
		\item\ollabel{Closuresmallest} if $X$ is $f$-closed and $B
		\subseteq X$, then $\Closureofunder{f}{B} \subseteq X$.
	\end{enumerate}
\end{lem}

\begin{proof}
Exactly as in \olref[sfr][infinite][dedekind]{closureproperties}.
\end{proof}

We need one last fact to get to Schr\"oder-Bernstein:

\begin{prop}\ollabel{sbhelper}
If $A \subseteq B \subseteq C$ and $A \approx C$, then $\cardeq{\cardeq{A}{B}}{C}$.
\end{prop}

\begin{proof}
Given !!a{bijection}  $f \colon C \to A$, let $F =
\Closureofunder{f}{C \setminus B}$ and define a function $g$ with
domain $C$ as follows:
\[
	g(x) = 
	\begin{cases}
			f(x) &\text{if $x \in F$}\\
			x & \text{otherwise}
		\end{cases}
\]
We'll show that $g$ is !!a{bijection} from $C \to B$, from which it
will follow that $\comp{f^{-1}}{g} \colon A \to B$ is !!a{bijection},
completing the proof.

First we claim that if $x \in F$ but $y\notin F$ then $g(x) \neq
g(y)$. For reductio suppose otherwise, so that $y = g(y) = g(x) =
f(x)$. Since $x \in F$ and $F$ is $f$-closed by \olref{Closureprops},
we have $y = f(x) \in F$, a contradiction. 

Now suppose $g(x) = g(y)$. So, by the above, $x \in F$ iff $y \in F$.
If $x, y \in F$, then $f(x) = g (x) = g(y) = f(y)$ so that $x = y$
since $f$ is !!a{bijection}. If  $x, y \notin F$, then $x = g(x) =
g(y) = y$. So $g$ is !!a{injection}.

It remains to show that $\ran{g} = B$. So fix $x \in B \subseteq C$.
If $x \notin F$, then $g(x) = x$. If $x \in F$, then $x = f(y)$ for
some $y \in F$, since otherwise $F \setminus \{x\}$ would be $f$-closed and extend $C\setminus B$, which is impossible by \olref{Closureprops}; now $g(y) = f(y) = x$.
\end{proof}

Finally, here is the proof of the main result. Recall that given a
function $h$ and set $D$, we define $\funimage{h}{D} = \Setabs{h(x)}{x
\in D}$. 

\begin{proof}[Proof of Schr\"oder-Bernstein] Let $f \colon A \to B$
and $g \colon B \to A$ be !!{injection}s. Since $\funimage{f}{A}
\subseteq B$ we have that $\funimage{g}{\funimage{f}{A}} \subseteq
g[B] \subseteq A$. Also, $\comp{f}{g} \colon A \to
\funimage{g}{\funimage{f}{A}}$ is an !!{injection} since both $g$ and
$f$ are; and indeed $\comp{f}{g}$ is !!a{bijection}, just by the way
we defined its codomain. So
$\cardeq{\funimage{g}{\funimage{f}{A}}}{A}$, and hence by
\olref{sbhelper} there is !!a{bijection} $h \colon A \to
\funimage{g}{B}$. Moreover, $g^{-1}$ is !!a{bijection}
$\funimage{g}{B} \to B$. So $\comp{h}{g^{-1}} \colon A \to B$ is
!!a{bijection}. 
\end{proof}

\end{document}