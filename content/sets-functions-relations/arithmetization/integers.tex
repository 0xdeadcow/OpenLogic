% Part: sets-functions-relations
% Chapter: arithmetization
% Section: From N to Z

\documentclass[../../../include/open-logic-section]{subfiles}


\begin{document}

\olfileid{sfr}{arith}{int}
\olsection{From $\Nat$ to $\Int$}

Here are two basic realisations:
	\begin{enumerate}
		\item Every integer can be written in the form $n - m$, with $n, m \in\Nat$.
		\item The information encoded in an expression $n - m$ can equally be encoded by an ordered pair $\tuple{n, m}$.
	\end{enumerate}
We already know that the ordered pairs of natural numbers are the !!{element}s of $\Nat^2$. And we are assuming that we understand $\Nat$. So here is a na\"{i}ve suggestion, based on the two realisations we have had: \emph{let's treat integers as ordered pairs of natural numbers}.

In fact, this suggestion is too na\"{i}ve. Obviously we want it to be the case that $0- 2 = 4 - 6$. But evidently $\tuple{0, 2 }\neq \tuple{4, 6}$. So we cannot simply say that $\Nat^2$ is the set of integers. 

Generalising from the preceding problem, what we want is the following:
	$$a - b = c - d \text{ iff }a + d = c + b$$
(It should be obvious that this is how integers are \emph{meant} to behave: just add $b$ and $d$ to both sides.) And the easy way to guarantee this behaviour is just to define an equivalence relation between ordered pairs, $\Intequiv$, as follows:
$$\tuple{ a, b } \Intequiv \tuple{c, d}\text{ iff }a + d = c + b$$  
We now have to show that this is an equivalence relation.
\begin{prop} $\Intequiv$ is an equivalence relation.
\end{prop}
	\begin{proof}
	We must show that $\Intequiv$ is reflexive, symmetric, and transitive. 
	
	\emph{Reflexivity:} Evidently $\tuple{a, b} \Intequiv \tuple{a, b}$, since $a + b = b + a$.
	
	\emph{Symmetry:} Suppose $\tuple{a, b} \Intequiv \tuple{c, d}$, so $a + d = c + b$. Then $c + b = a + d$, so that $\tuple{c, d} \Intequiv \tuple{a, b}$.
	
	\emph{Transitivity:} Suppose $\tuple{a, b} \Intequiv \tuple{c, d}\Intequiv \tuple{m, n}$. So $a + d = c + b$ and $c + n = m + d$. So $a + d + c + n = c + b + m + d$, and so $a + n = m + b$. Hence $\tuple{a, b } \Intequiv \tuple{m, n}$.		
\end{proof}

Now we can use this equivalence relation to take equivalence classes:

\begin{defn}
		The integers are the equivalence classes, under $\Intequiv$, of  ordered pairs of natural numbers; that is, $\Int = \equivclass{\Nat^2}{\Intequiv}$.
\end{defn}

Now, one might have plenty of different \emph{philosophical} reactions to this stipulative definition. Before we consider those reactions, though, it is worth continuing with some of the technicalities. 

Having said what the integers are, we shall need to define basic functions and relations on them. Let's write $\equivrep{m, n}{\Intequiv}$ for the equivalence class under $\Intequiv$ with $\tuple{m, n}$ as !!a{element}.\footnote{Note: using the notation introduced in \olref[sfr][rel][eqv]{def:equivalenceclass}, we would have written $\equivrep{\tuple{m,n}}{\Intequiv}$ for the same thing. But that's just a bit harder to read.} That is: 
	$$\equivrep{m, n}{\Intequiv} = \Setabs{\tuple{a, b} \in \Nat^2}{\tuple{a, b}\Intequiv \tuple{m, n}}$$
So now we offer some definitions:
	\begin{align*}
		\equivrep{a, b}{\Intequiv} + \equivrep{c, d}{\Intequiv} &= \equivrep{a + c, b + d}{\Intequiv}\\
		\equivrep{a, b}{\Intequiv} \times \equivrep{c, d}{\Intequiv} &= \equivrep{a c + b  d, a  d + b c}{\Intequiv}\\
		\equivrep{a, b}{\Intequiv} \leq \equivrep{c, d}{\Intequiv} &\text{ iff }a + d \leq b + c
	\end{align*}	
(As is common, I'm using `$ab$' stand for `$(a \times b)$', just to make the axioms easier to read.) Now, we need to make sure that these definitions behave as they \emph{ought} to. Spelling out what this means, and checking it through, is rather laborious; we relegate the details to \olref[check]{sec}. But the short point is: everything works! 

One final thing remains. We have constructed the integers using
natural numbers. But this will mean that the natural numbers \emph{are
not themselves integers}. We will return to the philosophical
significance of this in \olref[ref]{sec}. On a purely technical front,
though, we will need some way to be able to treat natural numbers
\emph{as} integers. The idea is quite easy: for each $n \in \Nat$, we
just stipulate that $n_\Int = \equivrep{n, 0}{\Intequiv}$. We need to
confirm that this definition is well-behaved, i.e., that for any $m, n
\in \Nat$
\begin{align*}
	(m + n)_\Int &= m_\Int + n_\Int\\
	(m \times n)_\Int &= m_\Int \times n_\Int\\
	m \leq n &\liff m_\Int \leq n_\Int
\end{align*}
But this is all pretty straightforward. For example, to show that the
second of these obtains, we can simply help ourselves to the behaviour
of the natural numbers and reason as follows:
%\begin{proof}
%	Helping myself to the behaviour of the natural numbers, evidently:
	\begin{align*}
%		(m + n)_\Int  = \equivrep{m + n, 0}{\Intequiv} = \equivrep{m + n, 0 + 0}{\Intequiv} = \equivrep{m, 0}{\Intequiv} + \equivrep{n, 0}{\Intequiv} = m_\Int + n_\Int\\
		(m \times n)_\Int  &= \equivrep{m \times n, 0}{\Intequiv} \\
		&= \equivrep{m \times n + 0 \times 0, m \times 0 + 0 \times n}{\Intequiv} \\
		&= \equivrep{m, 0}{\Intequiv} \times \equivrep{n, 0}{\Intequiv} \\
		&= m_\Int \times n_\Int
	\end{align*}
%\end{proof}
We leave it as an exercise to confirm that the other two conditions hold.
\begin{prob}
	Show that $(m + n)_\Int = m_\Int + n_\Int$ and $m \leq n \liff m_\Int \leq n_\Int$, for any $m, n \in \Nat$.
\end{prob}
\end{document}
