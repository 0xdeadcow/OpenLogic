% Part: sets-functions-relations
% Chapter: arithmetisation
% Section: rationals

\documentclass[../../../include/open-logic-section]{subfiles}

\begin{document}

\olfileid{sfr}{arith}{rat}
\olsection{From $\Int$ to $\Rat$}

We just saw how to construct the integers from the natural numbers,
using some na\"{i}ve set theory. We shall now see how to construct the
rationals from the integers in a very similar way. Our initial
realisations are:
\begin{enumerate}
	\item Every rational can be written in the form $\nicefrac{i}{j}$,
	where both $i$ and $j$ are integers but $j$ is non-zero.
	\item The information encoded in an expression $\nicefrac{i}{j}$
	can equally be encoded in an ordered pair $\tuple{ i, j}$.
\end{enumerate}
The obvious approach would be to think of the rationals \emph{as}
ordered pairs drawn from $\Int \times (\Int \setminus \{0_\Int\})$. As
before, though, that would be a bit too na\"ive, since we want
$\nicefrac{3}{2} = \nicefrac{6}{4}$, but $\tuple{ 3, 2}\neq \tuple{ 6,
4}$. More generally, we will want the following:
\[
	\nicefrac{a}{b} = \nicefrac{c}{d} \text{ iff } a \times d = b \times c
\]
To get this, we define an {equivalence relation} on  $\Int \times
(\Int \setminus \{0_\Int\})$ thus:
\[
	\tuple{ a, b }\Ratequiv \tuple{ c, d} \text{ iff }a \times d = b \times c
\]
We must check that this is an equivalence relation. This is very much
like the case of $\Intequiv$, and we will leave it as an exercise. 

\begin{prob}
Show that $\Ratequiv$ is an equivalence relation.
\end{prob}

But this allows us to say:
\begin{defn}
The rationals are the equivalence classes, under $\Ratequiv$, of pairs
of integers (whose second element is non-zero):  $\Rat =
\equivclass{(\Int \times (\Int\setminus \{0_\Int\}))}{\Ratequiv}$.
\end{defn}

As with the integers, we also want to define some basic operations.
Where $\equivrep{i,j}{\Ratequiv}$ is the equivalence class under
$\Ratequiv$ with $\tuple{ i, j}$ as !!a{element}, we say:
\begin{align*}
	\equivrep{a, b}{\Ratequiv} + \equivrep{c, d}{\Ratequiv} &= \equivrep{ad + bc,  bd}{\Ratequiv}\\
	\equivrep{a, b}{\Ratequiv} \times \equivrep{c, d}{\Ratequiv} &= \equivrep{a  c, b d}{\Ratequiv}\\
	\equivrep{a, b}{\Ratequiv} \leq \equivrep{c, d}{\Ratequiv} &\text{ iff }ad \leq bc
\end{align*}
We then need to check that these definitions behave as they
\emph{ought} to; and we relegate this to \olref[check]{sec}. But they
indeed do!{} Finally, we want some way to treat integers \emph{as}
rationals; so for each $i \in \Int$, we stipulate that $i_\Rat =
\equivrep{i, 1_\Int}{\Ratequiv}$. Again, we check that all of this
behaves correctly in \olref[check]{sec}.

\begin{prob}
Show that $(i + j)_\Rat = i_\Rat+ j_\Rat$ and $(i \times j)_\Rat =
i_\Rat \times j_\Rat$ and $i \leq j \liff i_\Rat \leq j_\Rat$, for any
$i, j \in \Int$.
\end{prob}

\end{document}
