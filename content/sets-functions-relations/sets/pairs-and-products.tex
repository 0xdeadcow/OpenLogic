% Part: sets-functions-relations
% Chapter: sets
% Section: pairs-and-products

\documentclass[../../../include/open-logic-section]{subfiles}

\begin{document}

\olfileid{sfr}{set}{pai}
\olsection{Pairs, Tuples, Cartesian Products}

\begin{explain}
Sets have no order to their elements. We just think of them as an
unordered collection. So if we want to represent order, we use
\emph{ordered pairs} $\tuple{x, y}$, or more generally,
\emph{ordered $n$-tuples} $\tuple{x_1, \dots, x_n}$.
\end{explain}

\begin{defn}[Cartesian product]
Given sets $X$ and $Y$, their \emph{Cartesian product} $X \times Y$ is
$\Setabs{\tuple{x, y}}{x \in X \text{ and } y \in Y}$.
\end{defn}

\begin{ex}
If $X = \{0, 1\}$, and $Y = \{1, a, b\}$, then their product is
\[
X \times Y = \{ \tuple{0, 1}, \tuple{0, a}, \tuple{0, b},
    \tuple{1, 1}, \tuple{1, a}, \tuple{1, b} \}.
\]
\end{ex}

\begin{ex}
If $X$ is a set, the product of $X$ with itself, $X \times X$, is also
written~$X^2$. It is the set of \emph{all} pairs $\tuple{x, y}$ with
$x, y \in X$.  The set of all triples $\tuple{x, y, z}$ is $X^3$, and
so on.
\end{ex}

\begin{prob}
List all !!{element}s of $\{1, 2, 3\}^3$.
\end{prob}

\begin{prob}
Show that if $X$ has $n$ !!{element}s, then $X^k$ has $n^k$
!!{element}s.
\end{prob}

\begin{ex}
If $X$ is a set, a \emph{word} over~$X$ is any sequence of
!!{element}s of~$X$.  A sequence can be thought of as an $n$-tuple of
!!{element}s of~$X$. For instance, if $X = \{a, b, c\}$, then the
sequence ``$bac$'' can be thought of as the triple~$\tuple{b, a, c}$.
Words, i.e., sequences of symbols, are of crucial importance in
computer science, of course.  By convention, we count !!{element}s of
$X$ as sequences of length~1, and $\emptyset$ as the sequence of
length~0.  The set of \emph{all} words over~$X$ then is
\[
X^* = \{\emptyset\} \cup X \cup X^2 \cup X^3 \cup \dots
\]
\end{ex}

\end{document}
