% Part: sets-functions-relations
% Chapter: sets
% Section: basics

\documentclass[../../../include/open-logic-section]{subfiles}


\begin{document}

\olfileid{sfr}{set}{bas}
\olsection{Basics}

\begin{explain}
Sets are the most fundamental building blocks of mathematical
objects. In fact, almost every mathematical object can be seen as a
set of some kind. In logic, as in other parts of mathematics, sets
and set theoretical talk is ubiquitous. So it will be important to
discuss what sets are, and introduce the notations necessary to talk
about sets and operations on sets in a standard way.
\end{explain}

\begin{defn}[Set]
A \emph{set} is a collection of objects, considered independently of
the way it is specified, of the order of the objects in the set, or of
their multiplicity. The objects making up the set are called
\emph{elements} or \emph{members} of the set. If $a$ is !!a{element}
of a set~$X$, we write $a \in X$ (otherwise, $a \notin X$). The set
which has no !!{element}s is called the \emph{empty} set and denoted
by the symbol~$\emptyset$.
\end{defn}

\begin{ex}
Whenever you have a bunch of objects, you can collect them together in
a set. The set of Richard's siblings, for instance, is a set that
contains one person, and we could write it as $S=\{\textrm{Ruth}\}$.
In general, when we have some objects $a_{1}$, \dots, $a_{n}$, then
the set consisting of exactly those objects is written $\{
a_{1}, \dots, a_{n}\}$. Frequently we'll specify a set by some
property that its !!{element}s share---as we just did, for instance, by
specifying $S$ as the set of Richard's siblings. We'll use the
following shorthand notation for that: $\Setabs{x}{\ldots x\ldots}$,
where the $\ldots x\ldots$ stands for the property that $x$ has to
have in order to be counted among the !!{element}s of the set. In our
example, we could have specified $S$ also as
\[
S = \Setabs{x}{x \text{ is a sibling of Richard}}.
\]
\end{ex}

\begin{explain}
When we say that sets are independent of the way they are specified,
we mean that the !!{element}s of a set are all that matters. For instance,
it so happens that
\begin{align*}
  & \{\text{Nicole}, \text{Jacob}\},\\
  & \Setabs{x}{\text{is a niece or nephew of Richard}}, \text{ and}\\
  & \Setabs{x}{\text{is a child of Ruth}}
\end{align*}
are three ways of specifying one and the same set.

Saying that sets are considered independently of the order of their
!!{element}s and their multiplicity is a fancy way of saying that
\begin{align*}
  & \{\text{Nicole}, \text{Jacob}\} \text{ and}\\
  & \{\text{Jacob}, \text{Nicole}\}
\end{align*}
are two ways of specifying the same set; and that
\begin{align*}
  & \{\text{Nicole}, \text{Jacob}\} \text{ and}\\
  & \{\text{Jacob}, \text{Nicole}, \text{Nicole}\}
\end{align*}
are also two ways of specifying the same set. In other words, all
that matters is which !!{element}s a set has. The !!{element}s of a
set are not ordered and each !!{element} occurs only once. When we
\emph{specify} or \emph{describe} a set, !!{element}s may occur
multiple times and in different orders, but any descriptions that only
differ in the order of !!{element}s or in how many times !!{element}s
are listed describes the same set.
\end{explain}

\begin{defn}[Extensionality]
  If $X$ and $Y$ are sets, then $X$ and $Y$ are \emph{identical}, $X =
  Y$, iff every !!{element} of~$X$ is also !!a{element} of~$Y$, and
  vice versa.
\end{defn}

\begin{explain}
Extensionality gives us a way for showing that sets are identical: to
show that $X = Y$, show that whenever $x \in X$ then also $x \in Y$,
and whenever $y \in Y$ then also $y \in X$.
\end{explain}

\begin{prob}
Show that there is only one empty set, i.e., show that if $X$ and $Y$
are sets without members, then $X = Y$.
\end{prob}

\end{document}
