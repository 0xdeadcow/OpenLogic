% Part: sets-functions-relations
% Chapter: sets
% Section: russells-paradox

\documentclass[../../../include/open-logic-section]{subfiles}


\begin{document}

\olfileid{sfr}{set}{bas}
\olsection{Russell's Paradox}

We said that one can define sets by specifying a property that its
!!{element}s share, e.g., defining the set of Richard's siblings as
\[
S = \Setabs{x}{x \text{ is a sibling of Richard}}.
\]
In the very general context of mathematics one must be careful,
however: not every property lends itself to \emph{comprehension}. Some
properties do not define sets.  If they did, we would run into
outright contradictions. One example of such a case is Russell's
Paradox.

Sets may be !!{element}s of other sets---for instance, the power set
of a set~$X$ is made up of sets.  And so it makes sense, of course, to
ask or investigate whether a set is !!a{element} of another set.  Can
a set be a member of itself?  Nothing about the idea of a set seems to
rule this out. For instance, surely \emph{all} sets form a collection
of objects, so we should be able to collect them into a single
set---the set of all sets. And it, being a set, would be !!a{element}
of the set of all sets.

Russell's Paradox arises when we consider the property of not having
itself as !!a{element}.  The set of all sets does not have this
property, but all sets we have encountered so far have it. $\Nat$ is
not !!a{element} of~$\Nat$, since it is a set, not a natural
number. $\Pow{X}$ is generally not !!a{element} of~$\Pow{X}$; e.g.,
$\Pow{\Real} \nsubseteq \Pow{\Real}$ since it is a set of sets of real
numbers, not a set of real numbers.  What if we suppose that there is
a set of all sets that do not have themselves as !!a{element}? Does
\[
R = \Setabs{x}{x \notin x}
\]
exist?

If $R$ exists, it makes sense to ask if $R \in R$ or not---it must be
either $\in R$ or $\notin R$. Suppose the former is true, i.e., $R \in
R$. $R$~was defined as the set of all sets that are not !!{element}s
of themselves, and so if $R \in R$, then $R$ does not have this
defining property of~$R$. But only sets that have this property are
in~$R$, hence, $R$ cannot be !!a{element} of~$R$, i.e., $R \notin
R$. But $R$ can't both be and not be !!a{element} of~$R$, so we have a
contradiction.

Since the assumption that $R \in R$ leads to a contradiction, we have
$R \notin R$. But this also leads to a contradiction!{} For if $R
\notin R$, it does have the defining property of~$R$, and so would be
!!a{element} of $R$ just like all the other non-self-containing sets.
And again, it can't both not be and be !!a{element} of~$R$.

\end{document}
