% Part: sets-functions-relations
% Chapter: sets
% Section: unions-and-intersections

\documentclass[../../../include/open-logic-section]{subfiles}

\begin{document}

\olfileid{sfr}{set}{uni}
\olsection{Unions and Intersections}

\begin{explain}
We can define new sets by abstraction, and the property used to define
the new set can mention sets we've already defined. So for instance,
if $X$ and $Y$ are sets, the set $\Setabs{x}{x \in X \lor x \in Y}$
defines a set which consists of all those objects which are
!!{element}s of either $X$ or~$Y$, i.e., it's the set that combines
the !!{element}s of $X$ and $Y$. This operation on sets---combining
them---is very useful and common, and so we give it a name and a symbol.
\end{explain}

\begin{defn}[Union]
The \emph{union} of two sets $X$ and $Y$, written $X \cup Y$, is the
set of all things which are !!{element}s of $X$, $Y$, or both.
\[
X \cup Y = \Setabs{x}{x \in X \lor x\in Y}
\]
\end{defn}

\begin{figure}
  \centerline{\includegraphics[width=\olphotowidth]
    {\olpath/assets/diagrams/union.tikz}}
  \caption{The union $X \cup Y$ of two sets is set of !!{element}s of
    $X$ together with those of~$Y$.}
\end{figure}

\begin{ex}
Since the multiplicity of !!{element}s doesn't matter, the union of two
sets which have !!a{element} in common contains that !!{element} only once,
e.g., $\{ a, b, c\} \cup \{ a, 0, 1\} = \{a, b, c, 0, 1\}$.

The union of a set and one of its subsets is just the bigger set: $\{a,
b, c \} \cup \{a \} = \{a, b, c\}$.

The union of a set with the empty set is identical to the set: $\{a,
b, c \} \cup \emptyset = \{a, b, c \}$.
\end{ex}

\begin{prob}
Prove rigorously that if $X \subseteq Y$, then $X \cup Y = Y$.
\end{prob}

\begin{explain}
The operation that forms the set of all !!{element}s that $X$ and $Y$
have in common is called their \emph{intersection}.
\end{explain}

\begin{defn}[Intersection]
The \emph{intersection} of two sets $X$ and $Y$, written $X \cap Y$, is
the set of all things which are !!{element}s of both $X$ and~$Y$.
\[
X \cap Y = \Setabs{x}{x \in X \land x\in Y}
\]
Two sets are called \emph{disjoint} if their intersection is
empty. This means they have no !!{element}s in common.
\end{defn}

\begin{figure}
  \centerline{\includegraphics[width=\olphotowidth]
    {\olpath/assets/diagrams/intersection.tikz}}
  \caption{The intersection $X \cap Y$ of two sets is the set of
    !!{element}s they have in common.}
\end{figure}

\begin{ex}
If two sets have no !!{element}s in common, their intersection is empty:
$\{ a, b, c\} \cap \{ 0, 1\} = \emptyset$.

If two sets do have !!{element}s in common, their intersection is the set of
all those: $\{a, b, c \} \cap \{a, b, d \} = \{a, b\}$.

The intersection of a set with one of its subsets is just the smaller
set: $\{a, b, c\} \cap \{a, b\} = \{a, b\}$.

The intersection of any set with the empty set is empty: $\{a, b, c \}
\cap \emptyset = \emptyset$.
\end{ex}

\begin{prob}
Prove rigorously that if $X \subseteq Y$, then $X \cap Y = X$.
\end{prob}


\begin{explain}
We can also form the union or intersection of more than two
sets. An elegant way of dealing with this in general is the
following: suppose you collect all the sets you want to form the union
(or intersection) of into a single set. Then we can define the union
of all our original sets as the set of all objects which belong to at
least one !!{element} of the set, and the intersection as the set of
all objects which belong to every !!{element} of the set.
\end{explain}

\begin{defn}
If $Z$ is a set of sets, then $\bigcup Z$ is the set of !!{element}s of
!!{element}s of~$Z$:
\begin{align*}
\bigcup Z & = \Setabs{x}{x \text{ belongs to an !!{element} of } Z},
\text{ i.e.,}\\
\bigcup Z & = \Setabs{x}{\text{there is a } Y \in Z
  \text{ so that } x \in Y}
\end{align*}
\end{defn}

\begin{defn}
If $Z$ is a set of sets, then $\bigcap Z$ is the set of objects which
all elements of~$Z$ have in common:
\begin{align*}
\bigcap Z & = \Setabs{x}{x \text{ belongs to every element of } Z},
\text{ i.e.,}\\
\bigcap Z & = \Setabs{x}{\text{for all } Y \in Z, x \in Y}
\end{align*}
\end{defn}

\begin{ex}
Suppose $Z = \{ \{ a, b \}, \{ a, d, e \}, \{ a, d \} \}$.
Then $\bigcup Z = \{ a, b, d, e \}$ and $\bigcap Z = \{ a \}$.
\end{ex}

We could also do the same for a sequence of sets $X_1$, $X_2$, \dots
\begin{align*}
\bigcup_i X_i & = \Setabs{x}{x \text{ belongs to one of the } X_i}\\
\bigcap_i X_i & = \Setabs{x}{x \text{ belongs to every } X_i}.
\end{align*}

\begin{defn}[Difference]
The \emph{difference}~$X \setminus Y$ is the set of all !!{element}s of
$X$ which are not also !!{element}s of~$Y$, i.e.,
\[
X\setminus Y = \Setabs{x}{x\in X \text{ and } x \notin Y}.
\]
\end{defn}

\begin{figure}
  \centerline{\includegraphics[width=\olphotowidth]
    {\olpath/assets/diagrams/difference.tikz}}
  \caption{The difference $X \setminus Y$ of two sets is the set of
    those !!{element}s of~$X$ which are not also elements of~$Y$.}
\end{figure}


\end{document}
