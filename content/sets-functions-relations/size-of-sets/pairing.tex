% Part:sets-functions-relations
% Chapter: size-of-sets
% Section: pairing

\documentclass[../../../include/open-logic-section]{subfiles}

\begin{document}

\olfileid{sfr}{siz}{pai}
\olsection{Pairing Functions and Codes}

\begin{explain}
Cantor's zig-zag method makes the enumerability of $\Nat^n$ visually
evident. But let us focus on our array depicting $\Nat^2$. Following
the zig-zag line in the array and counting the places, we can check
that $\tuple{1,2}$ is associated with the number~$7$. However, it would
be nice if we could compute this more directly. That is, it would be
nice to have to hand the \emph{inverse} of the zig-zag enumeration,
$g\colon \Nat^2 \to \Nat$, such that
\[
g(\tuple{0,0}) = 0, \;
g(\tuple{0,1}) = 1, \;
g(\tuple{1,0}) = 2, \; \dots,  
g(\tuple{1,2}) = 7, \; \dots
\]
This would enable us to calculate exactly where $\tuple{n, m}$ will occur
in our enumeration. 

In fact, we can define $g$ directly by making two observations. First:
if the $n$th row and $m$th column contains value~$v$, then the
$(n+1)$st row and $(m-1)$st column contains value $v + 1$. Second: the
first row of our enumeration consists of the triangular numbers,
starting with $0$, $1$, $3$, $6$, etc. The $k$th triangular number is
the sum of the natural numbers $< k$, which can be computed as
$k(k+1)/2$. Putting these two observations together, consider this
function:
\[
  g(n,m) = \frac{(n+m+1)(n+m)}{2} + n
\]
We often just write $g(n, m)$ rather that $g(\tuple{n, m})$, since it
is easier on the eyes. This tells you first to determine the
$(n+m)^\text{th}$ triangle number, and then add $n$ to it. And
it populates the array in exactly the way we would like. So in
particular, the pair $\tuple{1, 2}$ is sent to $\frac{4 \times 3}{2} +
1 = 7$. 

This function $g$ is the \emph{inverse} of an enumeration of a set of
pairs. Such functions are called \emph{pairing functions}.
\end{explain}

\begin{defn}[Pairing function] 
  A function $f\colon A \times B \to \Nat$ is an arithmetical
  \emph{pairing function} if $f$ is injective. We also say that $f$
  \emph{encodes} $A \times B$, and that $f(x,y)$ is the
  \emph{code} for $\tuple{x,y}$.
\end{defn}

\begin{explain}
We can use pairing functions to encode, e.g., pairs of natural numbers;
or, in other words, we can represent each \emph{pair} of elements
using a \emph{single} number. Using the inverse of the pairing
function, we can \emph{decode} the number, i.e., find out which
pair it represents.
\end{explain}

\begin{prob}
Give an enumeration of the set of all non-negative rational numbers. 
\end{prob}

\begin{prob}
Show that $\Rat$ is !!{enumerable}. Recall that any rational number 
can be written as a fraction $z/m$ with $z \in \Int$, $m \in \Nat^+$.
\end{prob}

\begin{prob}
Define an enumeration of $\Bin^*$.
\end{prob}

\begin{prob}
Recall from your introductory logic course that each possible truth
table expresses a truth function. In other words, the truth functions
are all functions from $\Bin^k \to \Bin$ for some~$k$. Prove that the
set of all truth functions is enumerable.
\end{prob}

\begin{prob}
Show that the set of all finite subsets of an arbitrary infinite
!!{enumerable} set is !!{enumerable}.
\end{prob}

\begin{prob}
A subset of $\Nat$ is said to be \emph{cofinite} iff it is the
complement of a finite set $\Nat$; that is, $A \subseteq \Nat$ is
cofinite iff $\Nat\setminus A$ is finite. Let $I$ be the set whose
!!{element}s are exactly the finite and cofinite subsets of $\Nat$.
Show that $I$ is !!{enumerable}.
\end{prob}

\begin{prob}
Show that the !!{enumerable} union of !!{enumerable} sets is
!!{enumerable}. That is, whenever $A_1$, $A_2$, \dots{} are sets, and
each $A_i$ is !!{enumerable}, then the union $\bigcup_{i=1}^\infty
A_i$ of all of them is also !!{enumerable}. [NB: this is hard!]
\end{prob}

\begin{prob}
Let $f \colon A \times B \to \Nat$ be an arbitrary pairing function.
Show that the inverse of $f$ is an enumeration of $A \times B$.
\end{prob}

\begin{prob}
Specify a function that encodes $\Nat^3$.
\end{prob}

\end{document}
