% Part: sets-functions-relations
% Chapter: size-of-sets

\documentclass[../../../include/open-logic-chapter]{subfiles}

\begin{document}

\olchapter{sfr}{siz}{The Size of Sets}

\begin{editorial}
This chapter discusses enumerations, countability and uncountability.
Several sections come in two versions: a more elementary one, that
takes enumerations to be lists, or surjections from $\PosInt$; and a
more abstract one that defines enumerations as bijections with $\Nat$.
\end{editorial}

\olimport{introduction}

\olimport{enumerations}

\olimport{zig-zag}

\olimport{pairing}

\olimport{pairing-alt}

\olimport{non-enumerability}

\olimport{reduction}

\olimport{equinumerous-sets}

\olimport{comparing-size}

\olimport{schroder-bernstein}

\begin{editorial}
The following \olref[sfr][siz][enm-alt]{sec},
\olref[sfr][siz][nen-alt]{sec}, \olref[sfr][siz][red-alt]{sec} are
alternative versions of \olref[sfr][siz][enm]{sec},
\olref[sfr][siz][nen]{sec}, \olref[sfr][siz][red]{sec} due to Tim
Button for use in his Open Set Theory text. They are slightly more
advanced and use a difference definition of enumerability more
suitable in a set theory context (i.e., bijection with $\Nat$ or an
initial segment, rather than being listable or being the range of a
surjective function from $\PosInt$).
\end{editorial}

\olimport{enumerations-alt}
\olimport{non-enumerability-alt}
\olimport{reduction-alt}

\OLEndChapterHook

\end{document}
