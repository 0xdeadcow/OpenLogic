% Part:sets-functions-relations
% Chapter: sets
% Section: reduction

\documentclass[../../../include/open-logic-section]{subfiles}

\begin{document}

\olfileid{sfr}{siz}{red}

\olsection{Reduction}

\begin{editorial}
  This section proves non-enumerability by reduction, matching the
  results in \olref[nen]{sec}. An alternative, slightly more condensed
  version matching the results in \olref[nen-alt]{sec} is provided in
  \olref[red-alt]{sec}.
\end{editorial}

We showed $\Pow{\PosInt}$ to be !!{nonenumerable} by a diagonalization
argument. We already had a proof that $\Bin^\omega$, the set of all
infinite sequences of $0$s and $1$s, is !!{nonenumerable}.  Here's
another way we can prove that $\Pow{\PosInt}$ is !!{nonenumerable}:
Show that \emph{if $\Pow{\PosInt}$ is !!{enumerable} then $\Bin^\omega$
  is also !!{enumerable}}.  Since we know $\Bin^\omega$ is not
!!{enumerable}, $\Pow{\PosInt}$ can't be either.  This is called
\emph{reducing} one problem to another---in this case, we reduce the
problem of enumerating $\Bin^\omega$ to the problem of enumerating
$\Pow{\PosInt}$.  A solution to the latter---an enumeration of
$\Pow{\PosInt}$---would yield a solution to the former---an enumeration
of $\Bin^\omega$.

How do we reduce the problem of enumerating a set~$B$ to that of
enumerating a set~$A$?  We provide a way of turning an enumeration
of~$A$ into an enumeration of~$B$.  The easiest way to do that is to
define !!a{surjective} function $f\colon A \to B$.  If $x_1$, $x_2$,
\dots{} enumerates~$A$, then $f(x_1)$, $f(x_2)$, \dots{} would
enumerate~$B$.  In our case, we are looking for a surjective
function $f\colon \Pow{\PosInt} \to \Bin^\omega$.

\begin{prob}
Show that if there is an !!{injective} function $g\colon B \to A$, and
$B$~is !!{nonenumerable}, then so is~$A$. Do this by showing how you
can use~$g$ to turn an enumeration of~$A$ into one of~$B$.
\end{prob}

\begin{proof}[Proof of {\olref[nen]{thm:nonenum-pownat}} by reduction]
Suppose that $\Pow{\PosInt}$ were !!{enumerable}, and thus that
there is an enumeration of it, $Z_{1}$, $Z_{2}$, $Z_{3}$, \dots

Define the function $f \colon \Pow{\PosInt} \to \Bin^\omega$ by letting
$f(Z)$ be the sequence $s_{k}$ such that $s_{k}(n) = 1$ iff $n \in Z$,
and $s_k(n) = 0$ otherwise.  This clearly defines a function, since
whenever $Z \subseteq \PosInt$, any $n \in \PosInt$ either is
!!a{element} of $Z$ or isn't.  For instance, the set $2\PosInt = \{2,
4, 6, \dots\}$ of positive even numbers gets mapped to the sequence
$010101\dots$, the empty set gets mapped to $0000\dots$ and the set
$\PosInt$ itself to $1111\dots$.

It also is !!{surjective}: Every sequence of $0$s and $1$s corresponds
to some set of positive integers, namely the one which has as its
members those integers corresponding to the places where the sequence
has~$1$s. More precisely, suppose $s \in \Bin^\omega$.  Define $Z
\subseteq \PosInt$ by:
\[
Z = \Setabs{n \in \PosInt}{s(n) = 1}
\]
Then $f(Z) = s$, as can be verified by consulting the definition
of~$f$.

Now consider the list
\[
f(Z_1), f(Z_2), f(Z_3), \dots
\]
Since $f$ is !!{surjective}, every member of $\Bin^\omega$ must
appear as a value of~$f$ for some argument, and so must appear on the
list. This list must therefore enumerate all of~$\Bin^\omega$.

So if $\Pow{\PosInt}$ were !!{enumerable}, $\Bin^\omega$ would be
!!{enumerable}.  But $\Bin^\omega$ is !!{nonenumerable}
(\olref[nen]{thm:nonenum-bin-omega}). Hence $\Pow{\PosInt}$ is
!!{nonenumerable}.
\end{proof}

\begin{explain}
It is easy to be confused about the direction the reduction goes in.
For instance, !!a{surjective} function $g \colon \Bin^\omega \to B$
does \emph{not} establish that $B$ is !!{nonenumerable}.  (Consider $g
\colon \Bin^\omega \to \Bin$ defined by $g(s) = s(1)$, the function
that maps a sequence of $0$'s and $1$'s to its first !!{element}.  It
is !!{surjective}, because some sequences start with $0$ and some start
with $1$. But $\Bin$ is finite.)  Note also that the function~$f$ must
be !!{surjective}, or otherwise the argument does not go through:
$f(x_1)$, $f(x_2)$, \dots{} would then not be guaranteed to include
all the !!{element}s of~$B$. For instance, 
\[
h(n) = \underbrace{000\dots0}_{\text{$n$ $0$'s}}
\]
defines a function $h\colon \PosInt \to
\Bin^\omega$, but $\PosInt$ is !!{enumerable}.
\end{explain}

\begin{prob}
Show that the set of all \emph{sets of} pairs of positive integers is
!!{nonenumerable} by a reduction argument.
\end{prob}

\begin{prob}\label{sfr:siz:red:prob:nat-nat}
  Show that the set~$X$ of all functions $f\colon \Nat \to \Nat$ is
  !!{nonenumerable} by a reduction argument (Hint: give a surjective
  function from $X$ to~$\Bin^\omega$.)
\end{prob}

\begin{prob}
Show that $\Nat^\omega$, the set of infinite sequences of
natural numbers, is !!{nonenumerable} by a reduction argument.
\end{prob}

\begin{prob}
Let $P$ be the set of functions from the set of positive
integers to the set $\{0\}$, and let $Q$ be the set of \emph{partial}
functions from the set of positive integers to the set $\{0\}$. Show
that $P$~is !!{enumerable} and $Q$~is not. (Hint: reduce the problem
of enumerating $\Bin^\omega$ to enumerating~$Q$).
\end{prob}

\begin{prob}
Let $S$ be the set of all !!{surjective} functions from the set of
positive integers to the set \{0,1\}, i.e., $S$ consists of all
!!{surjective}~$f\colon \PosInt \to \Bin$.  Show that $S$ is
!!{nonenumerable}.
\end{prob}

\begin{prob}
Show that the set~$\Real$ of all real numbers is !!{nonenumerable}.
\end{prob}

\end{document}
