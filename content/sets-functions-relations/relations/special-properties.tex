% Part: sets-functions-relations
% Chapter: relations
% Section: special-properties

\documentclass[../../../include/open-logic-section]{subfiles}

\begin{document}

\olfileid{sfr}{rel}{prp}
\olsection{Special Properties of Relations}

\begin{intro}
Some kinds of relations turn out to be so common that they have been
given special names.  For instance, $\le$ and $\subseteq$ both relate
their respective domains (say, $\Nat$ in the case of $\le$ and
$\Pow{X}$ in the case of $\subseteq$) in similar ways.  To get at
exactly how these relations are similar, and how they differ, we
categorize them according to some special properties that relations
can have.  It turns out that (combinations of) some of these special
properties are especially important: orders and equivalence relations.
\end{intro}

\begin{defn}[Reflexivity]
A relation $R \subseteq X^2$ is \emph{reflexive} iff, for every $x \in
X$, $Rxx$.
\end{defn}

\begin{defn}[Transitivity]
A relation $R \subseteq X^2$ is \emph{transitive} iff, whenever $Rxy$
and $Ryz$, then also $Rxz$.
\end{defn}

\begin{defn}[Symmetry]
A relation~$R \subseteq X^2$ is \emph{symmetric} iff, whenever
$Rxy$, then also~$Ryx$.
\end{defn}

\begin{defn}[Anti-symmetry]
A relation~$R \subseteq X^2$ is \emph{anti-symmetric} iff, whenever both
$Rxy$ and $Ryx$, then $x=y$ (or, in other words: if $x\neq y$ then
either $\lnot Rxy$ or $\lnot Ryx$).
\end{defn}

\begin{explain}
In a symmetric relation, $Rxy$ and $Ryx$ always hold together, or
neither holds.  In an anti-symmetric relation, the only way for $Rxy$
and $Ryx$ to hold together is if $x = y$.  Note that this does not
\emph{require} that $Rxy$ and $Ryx$ holds when $x = y$, only that it
isn't ruled out.  So an anti-symmetric relation can be reflexive, but
it is not the case that every anti-symmetric relation is
reflexive.  Also note that being anti-symmetric and merely not being
symmetric are different conditions.  In fact, a relation can be both
symmetric and anti-symmetric at the same time (e.g., the identity
relation is).
\end{explain}

\begin{defn}[Connectivity]
A relation $R \subseteq X^2$ is \emph{connected} if for all $x,y\in
X$, if $x \neq y$, then either $Rxy$ or~$Ryx$.
\end{defn}

\begin{defn}[Partial order]
A relation~$R \subseteq X^2$ that is reflexive, transitive, and
anti-symmetric is called a \emph{partial order}. 
\end{defn}

\begin{defn}[Linear order]
A partial order that is also connected is called a \emph{linear order}.
\end{defn}

\begin{defn}[Equivalence relation]
A relation $R \subseteq X^2$ that is reflexive, symmetric, and
transitive is called an \emph{equivalence relation}. $x$ and $y$ are
said to be \emph{R-equivalent} if $Rxy$.
\end{defn}

Moreover, we use \emph{R-equivalence class} to denote a set of
elements that are R-equivalent.
\begin{defn}[Equivalence class]
  The R-equivalence class containing $x$, or $\eqc{x}[R]$, or
  $\eqc{x}$ if $R$ is clear, is defined to
  be the set \Setabs{y}{Rxy}. $x$ is said to be the \emph{representative} of this
  R-equivalence class when we write $\eqc{x}[R]$.
\end{defn}

\begin{explain}
  Note that in the above definition, $x$ is said to be the
  representative only because we use it to denote the class it's in.
  If we write $\eqc{y}$ to denote the same class, then $y$ is the
  representative. It only depends on how we denote the class.
\end{explain}

\begin{prob}
Give examples of relations that are (a) reflexive and symmetric but
not transitive, (b) reflexive and anti-symmetric, (c) anti-symmetric,
transitive, but not reflexive, and (d) reflexive, symmetric, and
transitive.  Do not use relations on numbers or sets.
\end{prob}

\end{document}
