% Part: sets-functions-relations
% Chapter: relations
% Section: operations

\documentclass[../../../include/open-logic-section]{subfiles}

\begin{document}

\olfileid{sfr}{rel}{ops}
\olsection{Operations on Relations}

It is often useful to modify or combine relations. In
\olref[sfr][rel][ord]{prop:stricttopartial}, we considered the \emph{union}
of relations, which is just the union of two relations considered as
sets of pairs. Similarly, in \olref[sfr][rel][ord]{prop:partialtostrict},
we considered the relative difference of relations. Here are some
other operations we can perform on relations.

\begin{defn}\ollabel{relationoperations} 
Let $R$, $S$ be relations, and $A$ be any set. 

The \emph{inverse} of $R$ is $R^{-1} = \Setabs{\tuple{y, x}}{\tuple{x,
    y} \in R}$.

The \emph{relative product} of $R$ and $S$ is $(R \mid S) =
\{\tuple{x, z} : \exists y(Rxy \land Syz)\}$.

The \emph{restriction} of $R$ to $A$ is $\funrestrictionto{R}{A}= R
\cap A^2$.

The \emph{application} of $R$ to $A$ is $\funimage{R}{A} = \{y :
(\exists x \in A)Rxy\}$
\end{defn}

\begin{ex}
Let $S \subseteq \Int^2$ be the successor relation on~$\Int$, i.e.,
$S = \Setabs{\tuple{x, y} \in \Int^2}{x + 1 = y}$, so that $Sxy$ iff $x + 1 = y$.

$S^{-1}$ is the predecessor relation on $\Int$, i.e.,
$\Setabs{\tuple{x,y}\in\Int^2}{x -1 =y}$.

$S\mid S$ is 
$ \Setabs{\tuple{x,y}\in\Int^2}{x + 2 =y}$

$\funrestrictionto{S}{\Nat}$ is the successor relation on~$\Nat$.

$\funimage{S}{\{1,2,3\}}$ is $\{2, 3, 4\}$.
\end{ex}

\begin{defn}[Transitive closure]Let $R \subseteq A^2$ be a binary relation. 
	
The \emph{transitive closure} of~$R$ is $R^+ = \bigcup_{0 < n \in
\Nat} R^n$, where we recursively define $R^1 = R$ and $R^{n+1} = R^n
\mid R$.

The \emph{reflexive transitive closure} of $R$ is $R^* = R^+ \cup
\Id{X}$.
\end{defn}

\begin{ex}
Take the successor relation $S \subseteq \Int^2$. $S^2xy$ iff $x + 2 =
y$, $S^3xy$ iff $x + 3 = y$, etc. So $S^+xy$ iff $x + n = y$ for some
$n > 1$. In other words, $S^+xy$ iff $x < y$, and $S^*xy$ iff $x \le
y$.
\end{ex}

\begin{prob}
Show that the transitive closure of $R$ is in fact transitive.
\end{prob}

\end{document}