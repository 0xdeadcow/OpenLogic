% Part: sets-functions-relations
% Chapter: relations
% Section: orders

\documentclass[../../../include/open-logic-section]{subfiles}

\begin{document}

\olfileid{sfr}{rel}{ord}
\olsection{Orders}

\begin{explain}
Very often we are interested in comparisons between objects, where one
object may be less or equal or greater than another in a certain
respect.  Size is the most obvious example of such a comparative
relation, or \emph{order}. But not all such relations are alike in all
their properties. For instance, some comparative relations require any
two objects to be comparable, others don't. (If they do, we call them
\emph{linear} or \emph{total}.) Some include identity (like~$\le$) and
some exclude it (like~$<$). Let's get some order into all this.
\end{explain}

\begin{defn}[Preorder]
A relation which is both reflexive and transitive is called a
\emph{preorder.}  
\end{defn}

\begin{defn}[Partial order]
A preorder which is also anti-symmetric is called a
\emph{partial order}.
\end{defn}

\begin{defn}[Linear order]
A partial order which is also connected is called a
\emph{total order} or \emph{linear order.}
\end{defn}

\begin{ex}
Every linear order is also a partial order, and every partial order is
also a preorder, but the converses don't hold. For instance, the
identity relation and the full relation on~$X$ are preorders, but they
are not partial orders, because they are not anti-symmetric (if $X$
has more than one element). For a somewhat less silly example,
consider the \emph{no longer than} relation $\preccurlyeq$
on~$\Bin^*$: $x \preccurlyeq y$ iff $\len{x} \le \len{y}$. This is a
preorder, even a connected preorder, but not a partial order.

The relation of \emph{divisibility without remainder} gives us an
example of a partial order which isn't a linear order: for integers
$n$, $m$, we say $n$ (evenly) divides $m$, in symbols: $n\mid m$, if
there is some $k$ so that $m=kn$. On $\Nat$, this is a partial order,
but not a linear order: for instance, $2\nmid3$ and also
$3\nmid2$. Considered as a relation on $\Int$, divisibility is only a
preorder since anti-symmetry fails: $1\mid-1$ and $-1\mid1$ but
$1\neq-1$. Another important partial order is the relation $\subseteq$
on a set of sets.

Notice that the examples $L$ and $G$ from \olref[set]{relations},
although we said there that they were called ``strict orders'' are not
linear orders even though they are connected (they are not
reflexive). But there is a close connection, as we will see
momentarily.
\end{ex}

\begin{defn}[Irreflexivity]
A relation $R$ on $X$ is called \emph{irreflexive} if, for all $x \in
X$, $ \lnot Rxx$. 
\end{defn}

\begin{defn}[Asymmetry]
A relation $R$ on $X$ is called \emph{asymmetric} if for no pair $x,y\in
X$ we have $Rxy$ and $Ryx$. 
\end{defn}

\begin{defn}[Strict order]
A \emph{strict order} is a relation which is irreflexive, asymmetric,
and transitive.
\end{defn}

\begin{defn}[Strict linear order]
A strict order which is also connected is called a
\emph{strict linear order.}
\end{defn}

A strict order on~$X$ can be turned into a partial order
by adding the diagonal $\Id{X}$, i.e., adding all the pairs~$\tuple{x,
  x}$.  (This is called the \emph{reflexive closure} of~$R$.)
Conversely, starting from a partial order, one can get a strict
order by removing~$\Id{X}$.

\begin{prop}
  \ollabel{strict-partial}
  \begin{enumerate}
  \item If $R$ is a strict (linear) order on~$X$, then $R^+ = R
    \cup \Id{X}$ is a partial order (linear order).
  \item If $R$ is a partial order (linear order) on~$X$, then $R^- = R
    \setminus \Id{X}$ is a strict (linear) order.
  \end{enumerate}
\end{prop}

\begin{proof}
  \begin{enumerate}
    \item Suppose $R$ is a strict order, i.e., $R \subseteq
      X^2$ and $R$ is irreflexive, asymmetric, and transitive. Let $R^+
      = R \cup \Id{X}$. We have to show that $R^+$ is reflexive,
      antisymmetric, and transitive.

      $R^+$ is clearly reflexive, since for all $x \in X$, $\tuple{x,
        x} \in \Id{X} \subseteq R^+$.

      To show $R^+$ is antisymmetric, suppose $R^+xy$ and $R^+yx$,
      i.e., $\tuple{x,y}$ and $\tuple{y, x} \in R^+$, and $x \neq y$.
      Since $\tuple{x,y} \in R \cup \Id{X}$, but $\tuple{x, y} \notin
      \Id{X}$, we must have $\tuple{x, y} \in R$, i.e.,
      $Rxy$. Similarly we get that~$Ryx$. But this contradicts the
      assumption that $R$ is asymmetric.

      Now suppose that $R^+xy$ and $R^+yz$. If both $\tuple{x, y} \in
      R$ and $\tuple{y,z} \in R$, it follows that $\tuple{x, z} \in R$
      since $R$~is transitive. Otherwise, either $\tuple{x, y} \in
      \Id{X}$, i.e., $x = y$, or $\tuple{y, z} \in \Id{X}$, i.e., $y =
      z$. In the first case, we have that $R^+yz$ by assumption, $x =
      y$, hence $R^+xz$. Similarly in the second case. In either case,
      $R^+xz$, thus, $R^+$ is also transitive.

      If $R$ is connected, then for all $x \neq y$, either $Rxy$
      or~$Ryx$, i.e., either $\tuple{x, y} \in R$ or $\tuple{y, x} \in
      R$. Since $R \subseteq R^+$, this remains true of $R^+$, so
      $R^+$ is connected as well.
    \item Exercise.
  \end{enumerate}
\end{proof}

\begin{prob}
  Complete the proof of \olref[sfr][rel][ord]{strict-partial}, i.e.,
  prove that if $R$~is a partial order on~$X$, then $R^- =
  R \setminus \Id{X}$ is a strict order. 
\end{prob}


\begin{ex}
$\le$ is the linear order corresponding to the strict linear
  order~$<$. $\subseteq$ is the partial order corresponding to
  the strict order~$\subsetneq$.
\end{ex}

\end{document}
