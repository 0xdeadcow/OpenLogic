% Part: sets-functions-relations
% Chapter: relations
% Section: reflections
%
\documentclass[../../../include/open-logic-section]{subfiles}

\begin{document}

\olfileid{sfr}{rel}{ref}
\olsection{Philosophical Reflections}

In \olref[set]{sec}, we defined relations as certain sets. We should
pause and ask a quick philosophical question: what is such a
definition \emph{doing}? It is extremely doubtful that we should want
to say that we have \emph{discovered} some metaphysical identity
facts; that, for example, the order relation on $\Nat$ \emph{turned
out} to be the set $R=  \Setabs{\tuple{n,m}}{n, m \in \Nat\text{ and }
n < m}$ that we defined in \olref[set]{sec}. Here are three
reasons why. 

First: in \olref[set][pai]{wienerkuratowski}, we defined $\tuple{a, b} =
\{\{a\}, \{a, b\}\}$. Consider instead the definition $\lVert a,
b\rVert = \{\{b\}, \{a, b\}\} = \tuple{b,a}$. When $a \neq b$, we have
that $\tuple{a, b} \neq \lVert a,b\rVert$. But we could equally have
regarded $\lVert a,b\rVert$ as our definition of an ordered pair,
rather than $\tuple{a,b}$. Both definitions would have worked equally
well. So now we have two equally good candidates to ``be'' the order
relation on the natural numbers, namely:
\begin{align*}
		R &= \Setabs{\tuple{n,m}}{n, m \in \Nat \text{ and }n < m}\\
		S &= \Setabs{\lVert n,m\rVert}{n, m \in \Nat \text{ and }n < m}.
\end{align*}
Since $R \neq S$, by extensionality, it is clear that they cannot
\emph{both} be identical to the order relation on~$\Nat$. But it would
just be arbitrary, and hence a bit embarrassing, to claim that $R$
rather than $S$ (or vice versa) \emph{is} the ordering relation, as a
matter of fact. (This is a very simple instance of an argument against
set-theoretic reductionism which Benacerraf made famous in
\citeyear{Benacerraf1965}. We will revisit it several times.)

Second: if we think that \emph{every} relation should be identified
with a set, then the relation of set-membership itself, $\in$, should
be a particular set. Indeed, it would have to be the set
$\Setabs{\tuple{x,y}}{x \in y}$. But does this set exist? Given
Russell's Paradox, it is a non-trivial claim that such a set exists.
In fact, \oliflabeldef{cumul:::part}{the theory of sets which we develop in
\olref[cumul][][]{part} will \emph{deny} the existence of this
set.\footnote{Skipping ahead, here is why. For reductio, suppose $I =
\Setabs{\tuple{x,y}}{x \in y}$ exists. Then $\bigcup \bigcup I$ is the
universal set, contradicting
\olref[sfr][z][sep]{thm:NoUniversalSet}.}}{it is possible to
develop set theory in a rigorous way as an axiomatic theory, and that 
theory will indeed deny the existence of this set.}
So, even if some relations can be treated as sets, the relation of
set-membership will have to be a special case.

Third: when we ``identify'' relations with sets, we said that we would
allow ourselves to write $Rxy$ for $\tuple{x,y} \in R$. This is fine,
provided that the membership relation, ``$\in$'', is treated \emph{as}
a predicate. But if we think that ``$\in$'' stands for a certain kind
of set, then the expression ``$\tuple{x,y} \in R$'' just consists of
three singular terms which stand for sets: ``$\tuple{x,y}$'',
``$\in$'', and ``$R$''. And such a list of names is no more capable of
expressing a proposition than the nonsense string: ``the cup penholder
the table''. Again, even if some relations can be treated as sets, the
relation of set-membership must be a special case. (This rolls
together a simple version of Frege's concept \emph{horse} paradox, and
a famous objection that Wittgenstein once raised against Russell.)

So where does this leave us? Well, there is nothing \emph{wrong} with
our saying that the relations on the numbers are sets. We just have to
understand the spirit in which that remark is made. We are not stating
a metaphysical identity fact. We are simply noting that, in certain
contexts, we can (and will) \emph{treat} (certain) relations as
certain sets.

\end{document}