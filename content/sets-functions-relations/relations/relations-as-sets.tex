% Part: sets-functions-relations
% Chapter: relations
% Section: relatins-as-sets

\documentclass[../../../include/open-logic-section]{subfiles}

\begin{document}

\olfileid{sfr}{rel}{set}
\olsection{Relations as Sets}

\begin{explain}
You will no doubt remember some interesting relations between objects
of some of the sets we've mentioned. For instance, numbers come with
an \emph{order relation}~$<$ and from the theory of whole numbers the
relation of \emph{divisibility without remainder} (usually written
$n\mid m$) may be familar. There is also the relation \emph{is
  identical with} that every object bears to itself and to no other
thing. But there are many more interesting relations that we'll
encounter, and even more possible relations. Before we review them,
we'll just point out that we can look at relations as a special sort
of set. For this, first recall what a \emph{pair} is: if $a$ and $b$
are two objects, we can combine them into the \emph{ordered
  pair}~$\tuple{a, b}$. Note that for ordered pairs the order
\emph{does} matter, e.g, $\tuple{a, b} \neq \tuple{b, a}$, in contrast
to unordered pairs, i.e., 2-element sets, where $\{a, b\}=\{b, a\}$.

If $X$ and $Y$ are sets, then the \emph{Cartesian product} $X \times Y$
of $X$ and $Y$ is the set of all pairs $\tuple{a, b}$ with $a \in X$ and
$b \in Y$. In particular, $X^{2}=X \times X$ is the set of all pairs
from~$X$.

Now consider a relation on a set, e.g., the $<$-relation on the set
$\mathbb{N}$ of natural numbers, and consider the set of all pairs of
numbers $\tuple{n, m}$ where $n<m$, i.e.,
\[
R=\Setabs{\tuple{n, m}}{n, m \in \mathbb{N} \text{ and } n<m}.
\]
Then there is a close connection between the number $n$ being less
than a number $m$ and the corresponding pair $\tuple{n, m}$
being a member of $R$, namely, $n<m$ if and only if $\tuple{n, m}
\in R$. In a sense we can consider the set $R$ to \emph{be} the
$<$-relation on the set $\mathbb{N}$. In the same way we can construct
a subset of $\mathbb{N}^{2}$ for any relation between
numbers. Conversely, given any set of pairs of numbers
$S\subseteq\mathbb{N}^{2}$, there is a corresponding relation between
numbers, namely, the relationship $n$ bears to $m$ if and only if
$\tuple{n, m} \in S$. This justifies the following definition:
\end{explain}

\begin{defn}
A \emph{binary relation} on a set $X$ is a subset of~$X^{2}$. If
$R\subseteq X^{2}$ is a binary relation on~$X$ and $x, y \in X$,
we write $Rxy$ (or $xRy$) for $\tuple{x, y} \in R$.
\end{defn}

\begin{ex}
\ollabel{relations}
The set $\mathbb{N}^{2}$ of
pairs of natural numbers can be listed in a 2-dimensional matrix like
this:
\[
\begin{array}{ccccc}
\mathbf{\tuple{ 0,0 }} & \tuple{ 0,1 } &
  \tuple{ 0,2 } & \tuple{ 0,3 } & \ldots\\
\tuple{ 1,0 } & \mathbf{\tuple{ 1,1 }} &
  \tuple{ 1,2 } & \tuple{ 1,3 } & \ldots\\
\tuple{ 2,0 } & \tuple{ 2,1 } &
  \mathbf{\tuple{ 2,2 }} & \tuple{ 2,3 } & \ldots\\
\tuple{ 3,0 } & \tuple{ 3,1 } & \tuple{ 3,2 } &
  \mathbf{\tuple{ 3,3 }} & \ldots\\
\vdots & \vdots & \vdots & \vdots & \mathbf{\ddots}
\end{array}
\]
The subset consisting of the pairs lying on the diagonal, $\{\tuple{
0,0 }, \tuple{ 1,1 }, \tuple{ 2,2 }, \dots\}$, is the
\emph{identity relation on}~$\mathbb{N}$. (Since the identity
relation is popular, let's define $\Id{X}=\Setabs{\tuple{ x,x }}{x \in
X}$ for any set $X$.) The subset of all pairs lying above the
diagonal, $L = \{\tuple{ 0,1 },\tuple{ 0,2 },\ldots,\tuple{
1,2 }, \tuple{ 1,3 }, \dots, \tuple{ 2,3 }, \tuple{ 2,4
},\ldots\}$ is the \emph{less than} relation, i.e., $Lnm$ iff
$n<m$. The subset of pairs below the diagonal, $G=\{ \tuple{ 1,0
},\tuple{ 2,0 },\tuple{ 2,1 }, \tuple{ 3,0
},\tuple{ 3,1 },\tuple{ 3,2 }, \dots\}$ is the
\emph{greater than} relation, i.e., $Gnm$ iff $n>m$. The union of $L$
with $I$, $K=L\cup I$, is the \emph{less than or equal to} relation:
$Knm$ iff $n \le m$. Similarly, $H=G \cup I$ is the \emph{greater than
  or equal to relation.} $L$, $G$, $K$, and $H$ are special kinds of
relations called \emph{orders}. $L$ and $G$ have the property that no
number bears $L$ or $G$ to itself (i.e., for all $n$, neither $Lnn$
nor $Gnn$). Relations with this property are called
\emph{antireflexive}, and, if they also happen to be orders, they are
called \emph{strict orders.}
\end{ex}

\begin{explain}
Although orders and identity are important and natural relations, it
should be emphasized that according to our definition \emph{any}
subset of $X^{2}$ is a relation on~$X$, regardless of how unnatural or
contrived it seems. In particular, $\emptyset$ is a relation on any set
(the \emph{empty relation}, which no pair of elements bears), and
$X^{2}$~itself is a relation on $X$ as well (one which every pair
bears). But also something like $E=\Setabs{\tuple{n, m}}{n>5 \text{ or
  } m \times n \ge 34}$ counts as a relation.
\end{explain}

\begin{prob}
List the !!{element}s of the relation $\subseteq$ on the set
$\Pow{\{a, b, c\}}$.
\end{prob}

\end{document}
