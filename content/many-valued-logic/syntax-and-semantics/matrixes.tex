% Part: many-valued-logic
% Chapter: syntax-and-semantics
% Section: matrixes

\documentclass[../../../include/open-logic-section]{subfiles}

\begin{document}

\olfileid{mvl}{syn}{mat}

\olsection{Matrixes}

A many-valued logic is defined by its language, its set of truth
values~$V$, a subset of designated truth values, and truth functions
for its connective.  Together, these elements are called a \emph{matrix}.  

\begin{defn}[Matrix]
\ollabel{defn:matrix}
A \emph{matrix} for the logic~$\Log L$ consists of:
\begin{enumerate}
\item a set of connectives making up a language~$\Lang L$;
\item a set $V \neq \emptyset$ of truth values;
\item a set $V^+ \subseteq V$ of designated truth values;
\item for each $n$-place connective $\star$ in $\Lang L$, a truth
function~$\tf{\star} : V^n \to V$. If $n = 0$, then $\tf{\star}$ is
just an element of~$V$.
\end{enumerate}
\end{defn}

\begin{ex}
The matrix for classical logic~\LogCL{} consists of:
\begin{enumerate}
  \item The standard propositional language $\Lang L_0$ with
  $\lfalse$, $\lnot$, $\land$, $\lor$, $\lif$.
  \item The set of truth values $V = \{\True, \False\}$.
  \item $\True$ is the only designated value, i.e., $V^+ = \{\True\}$.
  \item For $\lfalse$, we have $\tf{\lfalse} = \False$. The other
  truth functions are given by the usual truth tables (see
  \olref{fig:tf-CL}).
\end{enumerate}
\begin{figure}
  \begin{center}
      \begin{tabular}{c|c} 
        $\tf{\lnot}$ & \\ 
        \hline  
        $\True$ & $\False$ \\ 
        $\False$ & $\True$ 
      \end{tabular}
      \quad
      \begin{tabular}{c|cc} 
        $\tf{\land}$ & $\True$ & $\False$ \\ 
        \hline 
        $\True$ & $\True$ & $\False$ \\ 
        $\False$ & $\False$ & $\False$ 
      \end{tabular}
      \quad
      \begin{tabular}{c|cc} 
        $\tf{\lor}$ & $\True$ & $\False$ \\ 
        \hline 
        $\True$ & $\True$ & $\True$ \\ 
        $\False$ & $\True$ & $\False$ 
      \end{tabular}
      \quad
      \begin{tabular}{c|cc} 
        $\tf{\lif}$ & $\True$ & $\False$ \\ 
        \hline 
        $\True$ & $\True$ & $\False$ \\ 
        $\False$ & $\True$ & $\True$ 
      \end{tabular}
    \end{center} 
    \caption{Truth functions for classical logic~$\LogCL$.}
    \ollabel{fig:tf-CL}
  \end{figure}
  \end{ex}

\end{document}
