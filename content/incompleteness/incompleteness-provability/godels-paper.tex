% Part: incompleteness
% Chapter: incompleteness-provability
% Section: godels-paper

\documentclass[../../../include/open-logic-section]{subfiles}

\begin{document}

\olfileid{inc}{inp}{gop}

\olsection{Comparison with G\"odel's Original Paper}

It is worthwhile to spend some time with G\"odel's 1931
paper. The introduction sketches the ideas we have just discussed.
Even if you just skim through the paper, it is easy to see what is
going on at each stage: first G\"odel describes the formal system $P$
(syntax, axioms, proof rules); then he defines the primitive recursive
functions and relations; then he shows that $x B y$ is primitive
recursive, and argues that the primitive recursive functions and
relations are represented in $\Th{P}$. He then goes on to prove the
incompleteness theorem, as above. In section 3, he shows that one can
take the unprovable assertion to be a sentence in the language of
arithmetic. This is the origin of the $\beta$-lemma, which is what we
also used to handle sequences in showing that the recursive functions
are representable in $\Th{Q}$. G\"odel doesn't go so far to isolate a
minimal set of axioms that suffice, but we now know that $\Th{Q}$ will do
the trick.  Finally, in Section 4, he sketches a proof of the second
incompleteness theorem.

\end{document}
