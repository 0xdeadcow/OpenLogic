% Part: incompleteness
% Chapter: incompleteness-provability
% Section: first-incompleteness-thm

\documentclass[../../../include/open-logic-section]{subfiles}

\begin{document}

\olfileid{inc}{inp}{1in}

\olsection{The First Incompleteness Theorem}

We can now describe G\"odel's original proof of the first
incompleteness theorem. Let $\Th{T}$ be any computably axiomatized theory
in a language extending the language of arithmetic, such that $\Th{T}$
includes the axioms of $\Th{Q}$. This means that, in particular, $\Th{T}$
represents computable functions and relations.

We have argued that, given a reasonable coding of formulas and proofs
as numbers, the relation $\Prf[T](x,y)$ is computable, where
$\Prf[T](x,y)$ holds if and only if $x$ is the G\"odel number of
!!a{derivation} of the !!{formula} with G\"odel number~$y$
in~$\Th{T}$. In fact, for the particular theory that G\"odel had in
mind, G\"odel was able to show that this relation is primitive
recursive, using the list of 45 functions and relations in his
paper. The 45th relation, $x B y$, is just $\Prf[T](x,y)$ for his
particular choice of~$\Th{T}$. Remember that where G\"odel uses the
word ``recursive'' in his paper, we would now use the phrase
``primitive recursive.''

Since $\Prf[T](x,y)$ is computable, it is representable in $\Th{T}$. We
will use $\OPrf[T](x,y)$ to refer to the formula that represents
it. Let $\OProv[T](y)$ be the formula
$\lexists[x][\OPrf[T](x,y)]$. This describes the 46th relation,
$\fn{Bew}(y)$, on G\"odel's list. As G\"odel notes, this is the only
relation that ``cannot be asserted to be recursive.''  What he
probably meant is this: from the definition, it is not clear that it
is computable; and later developments, in fact, show that it isn't.

Let $\Th{T}$ be an !!{axiomatizable} theory containing~$\Th{Q}$. Then
$\Prf[T](x, y)$ is decidable, hence representable in~$\Th{Q}$ by
!!a{formula}~$\OPrf[T](x, y)$. Let $\OProv[T](y)$ be the formula we
described above. By the fixed-point lemma, there is a formula
$!G_\Th{T}$ such that $\Th{Q}$ (and hence $\Th{T}$) !!{derive}s
\begin{equation}
\ollabel{eqn:qpf}
!G_\Th{T} \liff \lnot \OProv[T](\gn{!G_\Th{T}}).
\end{equation}
Note that $!G_\Th{T}$ says, in essence, ``$!G_\Th{T}$ is not
!!{derivable} in~$\Th{T}$.''

\begin{lem}\ollabel{lem:cons-G-unprov}
If $\Th{T}$ is a consistent, !!{axiomatizable} theory
extending~$\Th{Q}$, then $\Th{T} \Proves/ !G_\Th{T}$.
\end{lem}

\begin{proof}
Suppose $\Th{T}$ !!{derive}s $!G_\Th{T}$. Then there \emph{is}
!!a{derivation}, and so, for some number $m$, the relation $\Prf[T](m,
\Gn{!G_\Th{T}})$ holds. But then $\Th{Q}$ !!{derive}s the sentence
$\OPrf[T](\num m, \gn{!G_\Th{T}})$. So $\Th{Q}$ !!{derive}s
$\lexists[x][\OPrf[T](x,\gn{!G_\Th{T}})]$, which is, by definition,
$\OProv[T](\gn{!G_\Th{T}})$. By \olref{eqn:qpf}, $\Th{Q}$ !!{derive}s
$\lnot !G_\Th{T}$, and since $\Th{T}$ extends $\Th{Q}$, so
does~$\Th{T}$. We have shown that if $\Th{T}$ !!{derive}s $!G_\Th{T}$, then
it also !!{derive}s $\lnot !G_\Th{T}$, and hence it would be inconsistent.
\end{proof}

\begin{defn}
\ollabel{thm:oconsis-q}
A theory $\Th{T}$ is \emph{$\omega$-consistent} if the following holds: if
$\lexists[x][!A(x)]$ is any sentence and $\Th{T}$ !!{derive}s $\lnot
!A(\num 0)$, $\lnot !A(\num 1)$, $\lnot !A(\num 2)$, \dots then $\Th{T}$
does not prove $\lexists[x][!A(x)]$.
\end{defn}

Note that every $\omega$-consistent theory is also consistent. This
follows simply from the fact that if $\Th{T}$ is inconsistent, then
$\Th{T} \Proves !A$ for every~$!A$. In particular, if $\Th{T}$ is
inconsistent, it !!{derive}s both $\lnot !A(\num n)$ for every~$n$ and
also !!{derive}s~$\lexists[x][!A(x)]$. So, if $\Th{T}$ is
inconsistent, it is $\omega$-inconsistent. By contraposition, if
$\Th{T}$ is $\omega$-consistent, it must be consistent.

\begin{lem}\ollabel{lem:omega-cons-G-unref}
If $\Th{T}$ is an $\omega$-consistent, !!{axiomatizable} theory
extending~$\Th{Q}$, then $\Th{T} \Proves/ !G_\Th{T}$.
\end{lem}

\begin{proof}
We show that if $\Th{T}$ !!{derive}s $\lnot !G_\Th{T}$, then it is
$\omega$-inconsistent. Suppose $\Th{T}$ !!{derive}s $\lnot !G_\Th{T}$. If
$\Th{T}$ is inconsistent, it is $\omega$-inconsistent, and we are
done. Otherwise, $\Th{T}$ is consistent, so it does not !!{derive}
$!G_\Th{T}$ by \olref{lem:cons-G-unprov}. Since there is no
!!{derivation} of $!G_\Th{T}$ in $\Th{T}$, $\Th{Q}$ !!{derive}s
\[
\lnot \OPrf[T](\num 0, \gn{!G_\Th{T}}), \lnot \OPrf[T](\num 1,
\gn{!G_\Th{T}}), \lnot \OPrf[T](\num 2, \gn{!G_\Th{T}}), \dots
\]
and so does~$\Th{T}$.  On the other hand, by \olref{eqn:qpf}, $\lnot
!G_\Th{T}$ is equivalent to
$\lexists[x][\OPrf[T](x,\gn{!G_\Th{T}})]$. So $\Th{T}$ is
$\omega$-inconsistent.
\end{proof}

\begin{prob}
  Every $\omega$-consistent theory is consistent. Show that the
  converse does not hold, i.e., that there are consistent but
  $\omega$-inconsistent theories. Do this by showing that $\Th{Q} \cup
  \{\lnot !G_\Th{Q}\}$ is consistent but $\omega$-inconsistent.
\end{prob}

\begin{thm}
\ollabel{thm:first-incompleteness} Let $\Th{T}$ be any
$\omega$-consistent, !!{axiomatizable} theory extending~$\Th{Q}$. Then
$\Th{T}$ is not complete.
\end{thm}

\begin{proof}
  If $\Th{T}$ is $\omega$-consistent, it is consistent, so $\Th{T}
  \Proves/ !G_\Th{T}$ by \olref{lem:cons-G-unprov}.  By
  \olref{lem:omega-cons-G-unref}, $\Th{T} \Proves/ \lnot !G_\Th{T}$.
  This means that $\Th{T}$ is incomplete, since it !!{derive}s neither
  $!G_\Th{T}$ nor $\lnot !G_\Th{T}$.
\end{proof}

\end{document}
