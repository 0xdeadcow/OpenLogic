% Part: incompleteness
% Chapter: incompleteness-provability
% Section: fixed-point-lemma

\documentclass[../../../include/open-logic-section]{subfiles}

\begin{document}

\olfileid{inc}{inp}{fix}

\olsection{The Fixed-Point Lemma}

\begin{explain}
The fixed-point lemma says that for any !!{formula}~$!B(x)$, there is
!!a{sentence}~$!A$ such that $\Th{T} \Proves !A \liff !B(\gn{!A})$,
provided $\Th{T}$ extends~$\Th{Q}$.  In the case of the liar sentence,
we'd want $!A$ to be equivalent (provably in~$\Th{T}$) to~``$\gn{!A}$
is false,'' i.e., the statement that $\Gn{!A}$ is the G\"odel number
of a false !!{sentence}. To understand the idea of the proof, it will
be useful to compare it with Quine's informal gloss of~$!A$ as,
``{}`yields a falsehood when preceded by its own quotation' yields a
falsehood when preceded by its own quotation.''  The operation of
taking an expression, and then forming a sentence by preceding this
expression by its own quotation may be called \emph{diagonalizing} the
expression, and the result its diagonalization. So, the
diagonalization of `yields a falsehood when preceded by its own
quotation' is ``{}`yields a falsehood when preceded by its own
quotation' yields a falsehood when preceded by its own quotation.''
Now note that Quine's liar sentence is not the diagonalization of
`yields a falsehood' but of `yields a falsehood when preceded by its
own quotation.' So the property being diagonalized to yield the liar
sentence itself involves diagonalization!{}

In the language of arithmetic, we form quotations of !!a{formula} with
one free variable by computing its G\"odel numbers and then
substituting the standard numeral for that G\"odel number into the
free variable. The diagonalization of~$!E(x)$ is $!E(\num{n})$, where
$n = \Gn{!E(x)}$. (From now on, let's abbreviate $\num{\Gn{!E(x)}}$ as
$\gn{!E(x)}$.)  So if $!B(x)$ is ``is a falsehood,'' then ``yields a
falsehood if preceded by its own quotation,'' would be ``yields a
falsehood when applied to the G\"odel number of its diagonalization.''
If we had a symbol~$\Obj{diag}$ for the function $\fn{diag}(n)$ which
computes the G\"odel number of the diagonalization of the !!{formula}
with G\"odel number~$n$, we could write $!E(x)$ as
$!B(\Obj{diag}(x))$. And Quine's version of the liar sentence would
then be the diagonalization of it, i.e., $!E(\gn{!E})$ or
$!B(\Obj{diag}(\gn{!B(\Obj{diag}(x))}))$.  Of course, $!B(x)$ could now
be any other property, and the same construction would work. For the
incompleteness theorem, we'll take $!B(x)$ to be ``$x$ is not !!{derivable}
in~$\Th{T}$.'' Then $!E(x)$ would be ``yields !!a{sentence} not !!{derivable}
in~$\Th{T}$ when applied to the G\"odel number of its
diagonalization.''

To formalize this in~$\Th{T}$, we have to find a way to formalize
$\fn{diag}$. The function $\fn{diag}(n)$ is computable, in fact, it is
primitive recursive: if $n$ is the G\"odel number of a
formula~$!E(x)$, $\fn{diag}(n)$ returns the G\"odel number
of~$!E(\gn{!E(x)})$. (Recall, $\gn{!E(x)}$ is the standard numeral of
the G\"odel number of~$!E(x)$, i.e., $\num{\Gn{!E(x)}}$). If
$\Obj{diag}$ were a function symbol in $\Th{T}$ representing the
function $\fn{diag}$, we could take $!A$ to be the formula
$!B(\Obj{diag}(\gn{!B(\Obj{diag}(x))}))$. Notice that
\begin{align*}
\fn{diag}(\Gn{!B(\Obj{diag}(x))}) & = 
\Gn{!B(\Obj{diag}(\gn{!B(\Obj{diag}(x))}))} \\
& = \Gn{!A}.
\end{align*}
Assuming $\Th{T}$ can !!{derive}
\[
\Obj{diag}(\gn{!B(\Obj{diag}(x))}) = \gn{!A},
\]
it can !!{derive} $!B(\Obj{diag}(\gn{!B(\Obj{diag}(x))}))
\liff !B(\gn{!A})$. But the left hand side is, by
definition,~$!A$.

Of course, $\Obj{diag}$ will in general not be a function symbol of
$\Th{T}$, and certainly is not one of~$\Th{Q}$. But, since $\fn{diag}$
is computable, it is \emph{representable} in~$\Th{Q}$ by some formula
$!D_{\fn{diag}}(x,y)$. So instead of writing $!B(\Obj{diag}(x))$ we
can write $\lexists[y][(!D_{\fn{diag}}(x,y) \land !B(y))]$. Otherwise,
the proof sketched above goes through, and in fact, it goes through
already in~$\Th{Q}$.
\end{explain}

\begin{lem}
\ollabel{lem:fixed-point} Let $!B(x)$ be any formula with one free
variable~$x$. Then there is a sentence~$!A$ such that $\Th{Q} \Proves
!A \liff !B(\gn{!A})$.
\end{lem}


\begin{proof}
Given $!B(x)$, let $!E(x)$ be the formula
$\lexists[y][(!D_{\fn{diag}}(x,y) \land !B(y))]$ and let $!A$~be its
diagonalization, i.e., the formula $!E(\gn{!E(x)})$.

Since $!D_{\fn{diag}}$ represents $\fn{diag}$, and
$\fn{diag}(\Gn{!E(x)}) = \Gn{!A}$, $\Th{Q}$ can !!{derive}
\begin{align}
  & !D_{\fn{diag}}(\gn{!E(x)}, \gn{!A}) \ollabel{repdiag1} \\
  & \lforall[y][(!D_{\fn{diag}}(\gn{!E(x)},y) \lif
  \eq[y][\gn{!A}])]. \ollabel{repdiag2}
\end{align}
Now we show that $\Th{Q} \Proves !A \liff !B(\gn{!A})$. We argue
informally, using just logic and facts !!{derivable} in~$\Th{Q}$.

First, suppose~$!A$, i.e., $!E(\gn{!E(x)})$. Going back to the
definition of $!E(x)$, we see that $!E(\gn{!E(x)})$ just is
\[
\lexists[y][(!D_{\fn{diag}}(\gn{!E(x)},y) \land !B(y))].
\]
Consider such a~$y$. Since $!D_{\fn{diag}}(\gn{!E(x)},y)$, by
\olref{repdiag2}, $y = \gn{!A}$. So, from $!B(y)$ we have
$!B(\gn{!A})$.

Now suppose $!B(\gn{!A})$. By \olref{repdiag1}, we have
$!D_{\fn{diag}}(\gn{!E(x)}, \gn{!A}) \land !B(\gn{!A})$. It follows
that $\lexists[y][(!D_{\fn{diag}}(\gn{!E(x)},y) \land !B(y))]$. But
that's just $!E(\gn{!E})$, i.e., $!A$.
\end{proof}

\begin{digress}
You should compare this to the proof of the fixed-point lemma in
computability theory. The difference is that here we want to define a
\emph{statement} in terms of itself, whereas there we wanted to define
a \emph{function} in terms of itself; this difference aside, it is
really the same idea.
\end{digress}

\end{document}
