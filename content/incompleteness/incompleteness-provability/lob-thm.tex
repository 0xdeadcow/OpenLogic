% Part: incompleteness
% Chapter: incompleteness-provability
% Section: lob-thm

\documentclass[../../../include/open-logic-section]{subfiles}

\begin{document}

\olfileid{inc}{inp}{lob}

\olsection{L\"ob's Theorem}

The G\"odel sentence for a theory~$\Th{T}$ is a fixed point of $\lnot
\OProv[T](x)$, i.e., !!a{sentence}~$!G$ such that
\[
\Th{T} \Proves \lnot \OProv[T](\gn{!G}) \liff !G.
\]
It is not !!{derivable}, because if $\Th{T} \Proves !G$, (a) by !!{derivability}
condition~(1), $\Th{T} \Proves \OProv[T](\gn{!G})$, and (b) $\Th{T}
\Proves !G$ together with $\Th{T} \Proves \lnot \OProv[T](\gn{!G})
\liff !G$ gives $\Th{T} \Proves \lnot \OProv[T](\gn{!G})$, and so
$\Th{T}$ would be inconsistent.  Now it is natural to ask about the
status of a fixed point of $\OProv[T](x)$, i.e., !!a{sentence}~$!H$
such that
\[
\Th{T} \Proves \OProv[T](\gn{!H}) \liff !H.
\]
If it were !!{derivable}, $\Th{T} \Proves \OProv[T](\gn{!H})$ by
condition~(1), but the same conclusion follows if we apply modus
ponens to the equivalence above. Hence, we don't get that $\Th{T}$ is
inconsistent, at least not by the same argument as in the case of the
G\"odel sentence. This of course does not show that $\Th{T}$
\emph{does} !!{derive}~$!H$.

We can make headway on this question if we generalize it a bit. The
left-to-right direction of the fixed point equivalence,
$\OProv[T](\gn{!H}) \lif !H$, is an instance of a general schema
called a \emph{reflection principle}: $\OProv[T](\gn{!A}) \lif !A$.
It is called that because it expresses, in a sense, that $\Th{T}$ can
``reflect'' about what it can !!{derive}; basically it says, ``If $\Th{T}$
can !!{derive}~$!A$, then~$!A$ is true,'' for any~$!A$.  This is true for
sound theories only, of course, and this suggests that theories will
in general not !!{derive} every instance of it.  So which instances can a
theory (strong enough, and satisfying the !!{derivability} conditions)
!!{derive}?  Certainly all those where $!A$ itself is !!{derivable}. And that's
it, as the next result shows.

\begin{thm}
Let $\Th{T}$ be !!a{axiomatizable} theory extending $\Th{Q}$, and
suppose $\OProv[T](y)$ is a formula satisfying conditions P1--P3 from
\olref[2in]{sec}. If $\Th{T}$ !!{derive}s $\OProv[T](\gn{!A}) \lif !A$,
then in fact $\Th{T}$ !!{derive}s $!A$.
\end{thm}

Put differently, if $\Th{T} \Proves/ !A$, then $\Th{T} \Proves/
\OProv[T](\gn{!A}) \lif !A$. This result is known as L\"ob's
theorem.

\begin{explain}
The heuristic for the proof of L\"ob's theorem is a clever proof that
Santa Claus exists. (If you don't like that conclusion, you are free
to substitute any other conclusion you would like.) Here it is:
\begin{enumerate}
\item Let $X$ be the sentence, ``If $X$ is true, then Santa Claus
  exists.''
\item Suppose $X$ is true.
\item Then what it says holds; i.e., we have: if $X$ is true, then
  Santa Claus exists.
\item Since we are assuming $X$ is true, we can conclude that
  Santa Claus exists, by modus ponens from (2) and~(3).
\item We have succeeded in deriving (4), ``Santa Claus exists,'' from
  the assumption~(2), ``$X$ is true.'' By conditional proof, we have
  shown: ``If $X$ is true, then Santa Claus exists.''
\item But this is just the sentence~$X$. So we have shown that $X$ is
  true.
\item But then, by the argument (2)--(4) above, Santa Claus exists.
\end{enumerate}
A formalization of this idea, replacing ``is true'' with ``is
!!{derivable},'' and ``Santa Claus exists'' with~$!A$, yields the proof of
L\"ob's theorem. The trick is to apply the fixed-point lemma to the
!!{formula}~$\OProv[T](y) \lif !A$. The fixed point of that
corresponds to the sentence~$X$ in the preceding sketch.
\end{explain}

\begin{proof}
Suppose $!A$ is !!a{sentence} such that $\Th{T}$ !!{derive}s
$\OProv[T](\gn{!A}) \lif !A$. Let $!B(y)$ be the !!{formula}~$\OProv[T](y)
\lif !A$, and use the fixed-point lemma to find !!a{sentence}~$!D$
such that $\Th{T}$ !!{derive}s $!D \liff !B(\gn{!D})$. Then each of the
following is !!{derivable} in $\Th{T}$:
\begin{align}
  & !D \liff (\OProv[T](\gn{!D}) \lif !A) \ollabel{L-1}\\
  & \qquad \text{$!D$ is a fixed point of~$!B(y)$}\notag \\
  & !D \lif (\OProv[T](\gn{!D}) \lif !A) \ollabel{L-2}\\
  & \qquad\text{from \olref{L-1}}\notag\\
  & \OProv[T](\gn{!D \lif (\OProv[T](\gn{!D}) \lif !A)}) \ollabel{L-3}\\
  & \qquad \text{from \olref{L-2} by condition P1}\notag \\
  & \OProv[T](\gn{!D}) \lif \OProv[T](\gn{\OProv[T](\gn{!D}) \lif !A})
  \ollabel{L-4}\\
  &\qquad \text{from \olref{L-3} using condition P2}\notag \\
  & \OProv[T](\gn{!D}) \lif (\OProv[T](\gn{\OProv[T](\gn{!D})}) \lif \OProv[T](\gn{!A})) \ollabel{L-5}\\
  &\qquad \text{from \olref{L-4} using P2 again} \notag\\
& \OProv[T](\gn{!D}) \lif \OProv[T](\gn{\OProv[T](\gn{!D})}) \ollabel{L-6}\\
  & \qquad\text{by !!{derivability} condition P3} \notag\\
  & \OProv[T](\gn{!D}) \lif \OProv[T](\gn{!A}) \ollabel{L-7} \\
  &\qquad\text{from \olref{L-5} and \olref{L-6}}\notag\\
  & \OProv[T](\gn{!A}) \lif !A \ollabel{L-8}\\
  &\qquad\text{by assumption of the theorem} \notag\\
  & \OProv[T](\gn{!D}) \lif !A \ollabel{L-9}\\
  &\qquad\text{from \olref{L-7} and \olref{L-8}}\notag\\
  & (\OProv[T](\gn{!D}) \lif !A) \lif !D \ollabel{L-10}\\
  & \qquad \text{from \olref{L-1}}\notag \\
  & !D \ollabel{L-11}\\
  & \qquad\text{from \olref{L-9} and \olref{L-10}}\notag \\
  & \OProv[T](\gn{!D}) \ollabel{L-12}\\
  & \qquad\text{from \olref{L-11} by condition~P1}\notag \\
  & !A \qquad\qquad\text{from \olref{L-8} and \olref{L-12}}\notag
\end{align}
\end{proof}

With L\"ob's theorem in hand, there is a short proof of the first
incompleteness theorem (for theories having a !!{derivability} predicate
satisfying conditions P1--P3: if $\Th{T} \Proves
\OProv[T](\gn{\lfalse}) \lif \lfalse$, then $\Th{T} \Proves \lfalse$.
If $\Th{T}$ is consistent, $\Th{T} \Proves/ \lfalse$. So, $\Th{T}
\Proves/ \OProv[T](\gn{\lfalse}) \lif \lfalse$, i.e., $\Th{T} \Proves/
\OCon[\Th{T}]$.  We can also apply it to show that~$!H$, the fixed
point of $\OProv[T](x)$, is !!{derivable}. For since
\begin{align*}
  \Th{T} & \Proves \OProv[T](\gn{!H}) \liff !H\\
  \intertext{in particular}
    \Th{T} & \Proves \OProv[T](\gn{!H}) \lif !H
\end{align*}
and so by L\"ob's theorem, $\Th{T} \Proves !H$.

% Going in the other direction, for homework I
% may ask you to work through a short proof of L\"ob's theorem, using
% the second incompleteness theorem instead of the fixed-point lemma.

\begin{prob}
Let $\Th{T}$ be a computably axiomatized theory, and
let $\OProv[T]$ be !!a{derivability} predicate for $\Th{T}$. Consider the
following four statements:
\begin{enumerate}
\item If $T \Proves !A$, then $T \Proves \OProv[T](\gn{!A})$.
\item $T \Proves !A \lif \OProv[T](\gn{!A})$.
\item If $T \Proves \OProv[T](\gn{!A})$, then $T \Proves !A$.
\item $T \Proves \OProv[T](\gn{!A}) \lif !A$
\end{enumerate}
Under what conditions are each of these statements true?
\end{prob}

\end{document}
