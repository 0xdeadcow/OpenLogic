% Part: incompleteness
% Chapter: theories-computability
% Section: complete-decidable

\documentclass[../../../include/open-logic-section]{subfiles}

\begin{document}

\olfileid{inc}{tcp}{cdc}

\olsection{\printtoken{S}{axiomatizable} Complete Theories are Decidable}

A theory is said to be {\em complete} if for every sentence $!A$,
either $!A$ or $\lnot !A$ is provable.

\begin{lem}
Suppose a theory $\Th{T}$ is complete and !!{axiomatizable}. Then
$\Th{T}$ is decidable.
\end{lem}

\begin{proof}
Suppose $\Th{T}$ is complete and $A$ is a computable set of axioms.
If $\Th{T}$ is inconsistent, it is clearly computable. (Algorithm: ``just
say yes.'') So we can assume that $\Th{T}$ is also consistent.

To decide whether or not a sentence $!A$ is in $\Th{T}$, simultaneously
search for a proof of $!A$ from $A$ and a proof of $\lnot !A$. Since
$\Th{T}$ is complete, you are bound to find one or another; and since $\Th{T}$
is consistent, if you find a proof of $\lnot !A$, there is no proof
of $!A$.

Put in different terms, we already know that $\Th{T}$ is !!{c.e.}; so
by a theorem we proved before, it suffices to show that the complement
of $\Th{T}$ is !!{c.e.} also. But a formula $!A$ is in $\Th{\bar T}$ if and
only if $\lnot !A$ is in $\Th{T}$; so $\Th{\bar T} \leq_m \Th{T}$.
\end{proof}

\end{document}
