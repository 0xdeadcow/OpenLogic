% Part: incompleteness
% Chapter: theories-computability
% Section: introduction

\documentclass[../../../include/open-logic-section]{subfiles}

\begin{document}

\olfileid{inc}{tcp}{int}

\olsection{Introduction}

\begin{editorial}
This section should be rewritten.  
\end{editorial}

We have the following:
\begin{enumerate}
\item A definition of what it means for a function to be representable
  in $\Th{Q}$ (\olref[req][int]{defn:representable-fn})
\item a definition of what it means for a relation to be representable
  in $\Th{Q}$ (\olref[req][rel]{defn:representing-relations})
\item a theorem asserting that the representable functions of $\Th{Q}$
  are exactly the computable ones
  (\olref[req][int]{thm:representable-iff-comp})
\item a theorem asserting that the representable relations of $\Th{Q}$
  are exactly the computable ones
  \olref[req][rel]{thm:representing-rels})
\end{enumerate}

A {\em theory} is a set of sentences that is deductively closed, that
is, with the property that whenever $T$ proves $!A$ then $!A$ is in
$T$. It is probably best to think of a theory as being a collection of
sentences, together with all the things that these sentences imply.
From now on, I will use $\Th{Q}$ to refer to the {\em theory}
consisting of the set of sentences derivable from the eight axioms in
\olref[req][int]{sec}.  Remember that we can code formula of $\Th{Q}$
as numbers; if $!A$ is such a formula, let $\Gn{!A}$ denote the number
coding $!A$. Modulo this coding, we can now ask whether various sets
of formulas are computable or not.

\end{document}
