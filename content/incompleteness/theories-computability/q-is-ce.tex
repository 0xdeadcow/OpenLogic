% Part: incompleteness
% Chapter: theories-computability
% Section: q-is-ce

\documentclass[../../../include/open-logic-section]{subfiles}

\begin{document}

\olfileid{inc}{tcp}{qce}

\olsection{$\Th{Q}$ is \printtoken{S}{c.e.}-Complete}


\begin{thm}
$\Th{Q}$ is !!{c.e.} but not decidable. In fact, it is a complete
  !!{c.e.} set.
\end{thm}

\begin{proof}
It is not hard to see that $\Th{Q}$ is !!{c.e.}, since it is the set of
(codes for) sentences $y$ such that there is a proof $x$ of $y$ in
$\Th{Q}$:
\[
Q = \Setabs{y}{\lexists[x][\Prf[\Th{Q}](x,y)]}.
\]
But we know that $\Prf[\Th{Q}](x,y)$ is computable (in fact, primitive
recursive), and any set that can be written in the above form is c.e.

Saying that it is a complete c.e.\ set is equivalent to saying that $K
\leq_m Q$, where $K = \Setabs{x}{!A_x(x) \downarrow}$. So let us show
that $K$ is reducible to $\Th{Q}$. Since Kleene's predicate $T(e,x,s)$ is
primitive recursive, it is representable in $\Th{Q}$, say, by $!A_T$. Then
for every $x$, we have
\begin{align*}
x \in K & \lif \lexists[s][T(x,x,s)] \\
& \lif \lexists[s][(\Th{Q} \Proves !A_T(\num x,\num x, \num s))]
  \\
& \lif \Th{Q} \Proves \lexists[s][!A_T(\num x, \num x, s)].
\end{align*}
Conversely, if $\Th{Q} \Proves \lexists[s][!A_T(\num x, \num x, s)]$,
then, in fact, for some natural number $n$ the formula $!A_T(\num x,
\num x, \num n)$ must be true.  Now, if $T(x,x,n)$ were false,
$\Th{Q}$ would prove $\lnot !A_T(\num x, \num x, \num n)$, since
$!A_T$ represents $T$.  But then $\Th{Q}$ proves a false formula,
which is a contradiction. So $T(x,x,n)$ must be true, which implies
$!A_x(x) \downarrow$.

In short, we have that for every $x$, $x$ is in $K$ if and only if
$\Th{Q}$ proves $\lexists[s][T(\num x,\num x,s)]$. So the function $f$
which takes $x$ to (a code for) the sentence $\lexists[s][T(\num x,
  \num x, s)]$ is a reduction of $K$ to $\Th{Q}$.
\end{proof}

\end{document}
