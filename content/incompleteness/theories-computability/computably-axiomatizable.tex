% Part: incompleteness
% Chapter: theories-computability
% Section: computably-axiomatizable

\documentclass[../../../include/open-logic-section]{subfiles}

\begin{document}

\olfileid{inc}{tcp}{cax}

\olsection{\printtoken{S}{axiomatizable} Theories}

A theory $\Th{T}$ is said to be \emph{!!{axiomatizable}} if it has a
computable set of axioms~$A$. (Saying that $A$ is a set of axioms for
$\Th{T}$ means $T = \Setabs{!A}{A \Proves !A}$.) Any ``reasonable''
axiomatization of the natural numbers will have this property. In
particular, any theory with a finite set of axioms is
!!{axiomatizable}.

\begin{lem}
Suppose $\Th{T}$ is !!{axiomatizable}. Then $\Th{T}$ is !!{computably
enumerable}.
\end{lem}

\begin{proof}
Suppose $A$ is a computable set of axioms for $\Th{T}$. To determine
if $!A \in T$, just search for !!a{derivation} of $!A$ from the axioms.

Put slightly differently, $!A$ is in $\Th{T}$ if and only if there is
a finite list of axioms $!B_1$, \dots, $!B_k$ in $A$ and !!a{derivation} of
$(!B_1 \land \dots \land !B_k) \lif !A$ in first-order logic.  But
we already know that any set with a definition of the form ``there
exists \dots such that \dots'' is !!{c.e.}, provided the second ``\dots''
is computable.
\end{proof}

\end{document}
