% Part: incompleteness
% Chapter: introduction
% Section: definitions

\documentclass[../../../include/open-logic-section]{subfiles}

\begin{document}

\olfileid{inc}{int}{def}

\olsection{Definitions}

In order to carry out Hilbert's project of formalizing mathematics and
showing that such a formalization is consistent and complete, the
first order of business would be that of picking a language, logical
framework, and a system of axioms.  For our purposes, let us suppose
that mathematics can be formalized in a first-order language, i.e.,
that there is some set of !!{constant}s, !!{function}s, and
!!{predicate}s which, together with the connectives and quatifiers of
first-order logic, allow us to express the claims of mathematics.
Most people agree that such a language exists: the language of set
theory, in which $\in$ is the only non-logical symbol.  That such a
simple language is so expressive is of course a very implausible claim
at first sight, and it took a lot of work to establish that
practically of all mathematics can be expressed in this very austere
vocabulary.  To keep things simple, for now, let's restrict our
discussion to arithmetic, so the part of mathematics that just deals
with the natural numbers~$\Nat$.  The natural language in which to
express facts of arithmetic is~$\Lang L_A$. $\Lang L_A$ contains a
single two-place !!{predicate}~$<$, a single !!{constant}~$\Obj 0$,
one one-place !!{function}~$\prime$, and two two-place
!!{function}s~$+$ and~$\times$.

\begin{defn}
A set of !!{sentence}s~$\Gamma$ is a \emph{theory} if it is closed
under entailment, i.e., if $\Gamma = \Setabs{!A}{\Gamma \Entails
!A}$.
\end{defn}

There are two easy ways to specify theories. One is as the set of
!!{sentence}s true in some !!{structure}.  For instance, consider the
!!{structure} for $\Lang L_A$ in which the !!{domain} is~$\Nat$ and
all non-logical symbols are interpreted as you would expect.

\begin{defn}\ollabel{def:standard-model}
The \emph{standard model of arithmetic} is the !!{structure}~$\Struct
N$ defined as follows:
\begin{enumerate}
\item $\Domain N = \Nat$
\item $\Assign{\Obj 0}{N} = 0$
\item $\Assign{\Obj \prime}{N}(n) = n + 1$ for all $n \in \Nat$
\item $\Assign{\Obj +}{N}(n, m) = n + m$ for all $n, m \in \Nat$
\item $\Assign{\Obj \times}{N}(n, m) = n\cdot m$ for all $n, m \in \Nat$
\item $\Assign{\Obj <}{N} = \Setabs{\tuple{n, m}}{n \in \Nat, m \in
  \Nat, n < m}$
\end{enumerate}
\end{defn}

Note the difference between $\times$ and $\cdot$: $\times$ is a symbol
in the language of arithmetic. Of course, we've chosen it to remind us
of multiplication, but $\times$ is not the multiplication operation
but a two-place function symbol (officially, $\Obj f^2_1$). By
contrast, $\cdot$ \emph{is} the ordinary multiplication function. When
you see something like $n \cdot m$, we mean the product of the numbers
$n$ and $m$; when you see something like $x \times y$ we are talking
about a term in the language of arithmetic. In the standard model, the
function symbol times is interpreted as the function $\cdot$ on the
natural numbers. For addition, we use $+$ as both the function symbol
of the language of arithmetic, and the addition function on the
natural numbers. Here you have to use the context to determine what is
meant.

\begin{defn}
The theory of \emph{true arithmetic} is the set of !!{sentence}s
satisfied in the standard model of arithmetic, i.e.,
\[
\Th{TA} = \Setabs{!A}{\Sat{N}{!A}}.
\]
\end{defn}

$\Th{TA}$ is a theory, for whenever $\Th{TA} \Entails !A$, $!A$ is
satisfied in every !!{structure} which satisfies $\Th{TA}$. Since
$\Sat{M}{\Th{TA}}$, $\Sat{M}{!A}$, and so $!A \in \Th{TA}$.

The other way to specify a theory~$\Gamma$ is as the set of
!!{sentence}s entailed by some set of sentences~$\Gamma_0$. In that
case, $\Gamma$ is the ``closure'' of $\Gamma_0$ under entailment.
Specifying a theory this way is only interesting if $\Gamma_0$ is
explicitly specified, e.g., if the !!{element}s of~$\Gamma_0$ are
listed. At the very least, $\Gamma_0$ has to be decidable, i.e., there
has to be a computable test for when !!a{sentence} counts as an
element of~$\Gamma_0$ or not. We call the !!{sentence}s
in~$\Gamma_0$ \emph{axioms} for~$\Gamma$, and $\Gamma$
\emph{axiomatized} by~$\Gamma_0$.

\begin{defn}
A theory~$\Gamma$ is \emph{axiomatized} by~$\Gamma_0$ iff
\[
\Gamma = \Setabs{!A}{\Gamma_0 \Entails !A}
\]
\end{defn}

\begin{defn}
The theory $\Th{Q}$ axiomatized by the following sentences is known
as ``Robinson's $\Th{Q}$'' and is a very simple theory of arithmetic.
\begin{align*}
& \lforall[x][\lforall[y][(\eq[x'][y'] \lif \eq[x][y])]] \tag{$!Q_1$}\\
& \lforall[x][\eq/[\Obj 0][x']] \tag{$!Q_2$}\\
& \lforall[x][(\eq[x][\Obj 0] \lor \lexists[y][\eq[x][y']])] \tag{$!Q_3$}\\
& \lforall[x][\eq[(x + \Obj 0)][x]] \tag{$!Q_4$}\\
& \lforall[x][\lforall[y][\eq[(x + y')][(x + y)']]] \tag{$!Q_5$}\\
& \lforall[x][\eq[(x \times \Obj 0)][\Obj 0]] \tag{$!Q_6$}\\
& \lforall[x][\lforall[y][\eq[(x \times y')][((x \times y) + x)]]] \tag{$!Q_7$}\\
& \lforall[x][\lforall[y][(x < y \liff \lexists[z][\eq[(z' + x)][y]])]] \tag{$!Q_8$}
\end{align*}
The set of !!{sentence}s $\{!Q_1, \dots, !Q_8\}$ are the axioms of
$\Th{Q}$, so $\Th{Q}$ consists of all !!{sentence}s entailed by them:
\[
\Th{Q} = \Setabs{!A}{\{!Q_1, \dots, !Q_8\} \Entails !A}.
\]
\end{defn}

\begin{defn}
Suppose $!A(x)$ is !!a{formula} in $\Lang L_A$ with free variables~$x$
and $y_1$, \dots, $y_n$. Then any !!{sentence} of the form
\[
\lforall[y_1][\dots\lforall[y_n][((!A(\Obj 0) \land \lforall[x][(!A(x)
\lif !A(x'))]) \lif \lforall[x][!A(x)])]]
\]
is an instance of the \emph{induction schema}.

\emph{Peano arithmetic}~$\Th{PA}$ is the theory axiomatized by the
axioms of $\Th{Q}$ together with all instances of the induction
schema.
\end{defn}

\begin{explain}
Every instance of the induction schema is true in~$\Struct{N}$. This
is easiest to see if the !!{formula}~$!A$ only has one free
!!{variable}~$x$.  Then $!A(x)$ defines a subset~$X_{!A}$ of~$\Nat$
in~$\Struct{N}$.  $X_{!A}$ is the set of all~$n \in \Nat$ such that
$\Sat{N}{!A(x)}[s]$ when $s(x) = n$.  The corresponding instance of
the induction schema is
\[
((!A(\Obj 0) \land \lforall[x][(!A(x) \lif !A(x'))]) \lif 
  \lforall[x][!A(x)]).
\]
If its antecedent is true in~$\Struct{N}$, then $0 \in X_{!A}$ and,
whenever $n \in X_{!A}$, so is $n+1$.  Since $0 \in X_{!A}$, we get
$1 \in X_{!A}$. With $1 \in X_{!A}$ we get $2 \in X_{!A}$. And so on.
So for every $n \in \Nat$, $n \in X_{!A}$. But this means that
$\lforall[x][!A(x)]$ is satisfied in~$\Struct{N}$.
\end{explain}

Both $\Th{Q}$ and $\Th{PA}$ are axiomatized theories.  The big
question is, how strong are they? For instance, can $\Th{PA}$ prove
all the truths about~$\Nat$ that can be expressed in~$\Lang L_A$?
Specifically, do the axioms of $\Th{PA}$ settle all the questions
that can be formulated in~$\Lang L_A$?

Another way to put this is to ask: Is $\Th{PA} = \Th{TA}$?
$\Th{TA}$ obviously does prove (i.e., it includes) all the truths
about~$\Nat$, and it settles all the questions that can be formulated
in~$\Lang L_A$, since if $!A$ is !!a{sentence} in $\Lang L_A$, then
either $\Sat{N}{!A}$ or $\Sat{N}{\lnot !A}$, and so either $\Th{TA}
\Entails !A$ or $\Th{TA} \Entails \lnot !A$.  Call such a theory
\emph{!!{complete}}.

\begin{defn}
A theory $\Gamma$ is \emph{!!{complete}} iff for every
!!{sentence}~$!A$ in its language, either $\Gamma \Entails !A$ or
$\Gamma \Entails \lnot !A$.
\end{defn}

\begin{explain}
By the Completeness Theorem, $\Gamma \Entails !A$ iff $\Gamma \Proves
!A$, so $\Gamma$ is complete iff for every !!{sentence}~$!A$ in its
language, either $\Gamma \Proves !A$ or $\Gamma \Proves \lnot !A$.
\end{explain}

Another question we are led to ask is this: Is there a computational
procedure we can use to test if !!a{sentence} is in~$\Th{TA}$, in
$\Th{PA}$, or even just in~$\Th{Q}$?  We can make this more precise
by defining when a set (e.g., a set of !!{sentence}s) is
!!{decidable}.

\begin{defn}
A set $X$ is \emph{!!{decidable}} iff there is a computational procedure
which on input~$x$ returns~$1$ if $x \in X$ and $0$ otherwise.
\end{defn}

So our question becomes: Is $\Th{TA}$ ($\Th{PA}$, $\Th{Q}$) !!{decidable}?

The answer to all these questions will be: no. None of these theories
are decidable. However, this phenomenon is not specific to these
particular theories. In fact, \emph{any} theory that satisfies certain
conditions is subject to the same results. One of these conditions,
which $\Th{Q}$ and $\Th{PA}$ satisfy, is that they are axiomatized by
!!a{decidable} set of axioms.

\begin{defn}
A theory is \emph{!!{axiomatizable}} if it is axiomatized
by !!a{decidable} set of axioms.
\end{defn}

\begin{ex}
Any theory axiomatized by a finite set of !!{sentence}s is
!!{axiomatizable}, since any finite set is !!{decidable}.
Thus, $\Th{Q}$, for instance, is !!{axiomatizable}.

Schematically axiomatized theories like~$\Th{PA}$ are also
!!{axiomatizable}. For to test if $!B$ is among the axioms
of~$\Th{PA}$, i.e., to compute the function $\Char{X}$ where
$\Char{X}(!B) = 1$ if $!B$ is an axiom of~$\Th{PA}$ and $= 0$
otherwise, we can do the following: First, check if $!B$ is one of the
axioms of~$\Th{Q}$. If it is, the answer is ``yes'' and the value of
$\Char{X}(!B) = 1$. If not, test if it is an instance of the induction
schema.  This can be done systematically; in this case, perhaps it's
easiest to see that it can be done as follows: Any instance of the
induction schema begins with a number of universal quantifiers, and
then a sub-!!{formula} that is a conditional. The consequent of that
conditional is $\lforall[x][!A(x, y_1, \dots, y_n)]$ where $x$ and
$y_1$, \dots, $y_n$ are all the free variables of~$!A$ and the initial
quantifiers of~$!B$ bind the variables~$y_1$, \dots,~$y_n$.  Once we
have extracted this~$!A$ and checked that its free variables match the
variables bound by the universal quantifiers at the front
and~$\lforall[x]$, we go on to check that the antecedent of the
conditional matches
\[
!A(\Obj 0, y_1, \dots, y_n) \land \lforall[x][(!A(x, y_1, \dots, y_n)
\lif !A(x', y_1, \dots, y_n))]
\]
Again, if it does, $!B$ is an instance of the induction schema, and if
it doesn't, $!B$ isn't.
\end{ex}

In answering this question---and the more general question of which
theories are complete or decidable---it will be useful to consider
also the following definition. Recall that a set $X$ is !!{enumerable}
iff it is empty or if there is !!a{surjective} function~$f \colon \Nat
\to X$. Such a function is called an enumeration of~$X$.

\begin{defn}
A set $X$ is called \emph{!!{computably enumerable}} (!!{c.e.} for
short) iff it is empty or it has a computable enumeration.
\end{defn}

In addition to !!{axiomatizability}, another condition on theories to
which the incompleteness theorems apply will be that they are strong
enough to prove basic facts about computable functions and
!!{decidable} relations. By ``basic facts,'' we mean !!{sentence}s
which express what the values of computable functions are for each of
their arguments.  And by ``strong enough'' we mean that the theories
in question count these sentences among its theorems. For instance,
consider a prototypical computable function: addition.  The value of
$+$ for arguments $2$ and $3$ is~$5$, i.e., $2+3 = 5$. A sentence in
the language of arithmetic that expresses that the value of $+$ for
arguments $2$ and $3$ is~$5$ is: $(\num{2} + \num{3}) = \num{5}$.
And, e.g., $\Th{Q}$ proves this sentence.  More generally, we would
like there to be, for each computable function $f(x_1, x_2)$
!!a{formula}~$!A_f(x_1, x_2, y)$ in~$\Lang{L_A}$ such that $\Th{Q}
\Proves !A_f(\num{n_1}, \num{n_2}, \num{m})$ whenever $f(n_1, n_2) =
m$. In this way, $\Th{Q}$ proves that the value of~$f$ for arguments
$n_1$, $n_2$ is~$m$. In fact, we require that it proves a bit more,
namely that no other number is the value of~$f$ for arguments
$n_1$,~$n_2$. And the same goes for !!{decidable} relations. This is
made precise in the following two definitions.

\begin{defn}
A !!{formula}~$!A(x_1, \dots, x_k, y)$ \emph{!!{represents}} the
function $f\colon \Nat^k \to \Nat$ in~$\Gamma$ iff whenever $f(n_1,
\dots, n_k) = m$, then
\begin{enumerate}
\item $\Gamma \Proves !A(\num{n_1}, \dots, \num{n_k}, \num{m})$, and
\item $\Gamma \Proves \lforall[y](!A(\num{n_1}, \dots, \num{n_k},
y) \lif y = \num{m})$.
\end{enumerate}
\end{defn}

\begin{defn}
A !!{formula}~$!A(x_1, \dots, x_k)$ \emph{!!{represents}} the
relation $R \subseteq \Nat^k$ iff,
\begin{enumerate}
\item whenever $R(n_1, \dots, n_k)$, $\Gamma \Proves !A(\num{n_1},
\dots, \num{n_k})$, and
\item whenever not $R(n_1, \dots, n_k)$, $\Gamma \Proves \lnot
!A(\num{n_1}, \dots, \num{n_k})$.
\end{enumerate}
\end{defn}

A theory is ``strong enough'' for the incompleteness theorems to
apply if it !!{represents} all computable functions and all
!!{decidable} relations. $\Th{Q}$~and its extensions satisfy this
condition, but it will take us a while to establish this---it's a
non-trivial fact about the kinds of things $\Th{Q}$ can prove, and
it's hard to show because $\Th{Q}$ has only a few axioms from which
we'll have to prove all these facts. However, $\Th{Q}$ is a very weak
theory. So although it's hard to prove that $\Th{Q}$ represents all
computable functions, most interesting theories are stronger
than~$\Th{Q}$, i.e., prove more than $\Th{Q}$ does. And if $\Th{Q}$
proves something, any stronger theory does; since $\Th{Q}$ represents
all computable functions, every stronger theory does. This means that
many interesting theories meet this condition of the incompleteness
theorems. So our hard work will pay off, since it shows that the
incompleteness theorems apply to a wide range of theories. Certainly,
any theory aiming to formalize ``all of mathematics'' must prove
everything that $\Th{Q}$ proves, since it should at the very least be
able to capture the results of elementary computations.  So any theory
that is a candidate for a theory of ``all of mathematics'' will be one
to which the incompleteness theorems apply.


\end{document}
