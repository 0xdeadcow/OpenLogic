% Part: incompleteness
% Chapter: arithmetization-syntax
% Section: proofs-in-ax

\documentclass[../../../include/open-logic-section]{subfiles}

\begin{document}

\olfileid{inc}{art}{pax}
\olsection{Axiomatic \usetoken{P}{derivation}}

\begin{explain}
In order to arithmetize axiomatic !!{derivation}s, we must represent
!!{derivation}s as numbers. Since !!{derivation}s are simply sequences
of !!{formula}s, the obvious approach is to code every !!{derivation}
as the code of the sequence of codes of !!{formula}s in it.
\end{explain}

\begin{defn}
If $\delta$ is an axiomatic !!{derivation} consisting of !!{formula}s
$!A_1$, \dots,~$!A_n$, then $\Gn{\delta}$ is
\[
\tuple{\Gn{!A_1}, \dots, \Gn{!A_n}}.
\]
\end{defn}

\begin{ex}
  Consider the very simple !!{derivation}
  \begin{derivation}
  1. & $!B \lif (!B \lor !A)$ \\
  2. & $(!B \lif (!B \lor !A)) \lif (!A  \lif (!B \lif (!B \lor !A)))$\\
  3. & $!A  \lif (!B \lif (!B \lor !A))$
  \end{derivation}
  The G\"odel number of this derivation would simply be
  \[
  \tuple{\Gn{!B \lif (!B \lor !A)}, 
  \Gn{(!B \lif (!B \lor !A)) \lif (!A  \lif (!B \lif (!B \lor !A)))},
  \Gn{!A  \lif (!B \lif (!B \lor !A))}}.
  \]
\end{ex}

\begin{explain}
Having settled on a representation of !!{derivation}s, we must also
show that we can manipulate such !!{derivation}s primitive recursively, and
express their essential properties and relations so.  Some operations
are simple: e.g., given a G\"odel number~$d$ of a !!{derivation},
$(d)_{\len{d}-1}$ gives us the G\"odel number of its end-!!{formula}.  Some are
much harder.  We'll at least sketch how to do this.  The goal is to
show that the relation ``$\delta$ is !!a{derivation} of~$!A$
from~$\Gamma$'' is primitive recursive on the G\"odel numbers of
$\delta$ and~$!A$.
\end{explain}

\begin{prop}
\ollabel{prop:followsby}
The following relations are primitive recursive:
\begin{enumerate}
\item $!A$ is an axiom.
\item The $i$th line in $\delta$ is justified by modus ponens
\item The $i$th line in $\delta$ is justified by \QR.
\item $\delta$ is a correct !!{derivation}.
\end{enumerate}
\end{prop}

\begin{proof}
We have to show that the corresponding relations between G\"odel
numbers of !!{formula}s and  G\"odel numbers of !!{derivation}s 
are primitive recursive.
\begin{enumerate}
\item We have a given list of axiom schemas, and $!A$ is an axiom if
  it is of the form given by one of these schemas.  Since the list of
  schemas is finite, it suffices to show that we can test primitive
  recursively, for each axiom schema, if $!A$ is of that form. For
  instance, consider the axiom schema
  \[
  !B \lif (!C \lif !B).
  \]
  $!A$ is an instance of this axiom schema if there are !!{formula}s
  $!B$ and $!C$ such that we obtain $!A$ when we concatenate $($ with
  $!B$ with $\lif$ with $($ with $!C$ with $\lif$ with $!B$ and with
  $))$. We can test the corresponding property of the G\"odel number
  $n$ of~$!A$, since concatenation of sequences is primitive
  recursive, and the G\"odel numbers of $!B$ and $C$ must be smaller
  than the G\"odel number of~$!A$, since when the relation holds, both
  $!B$ and $!C$ are sub-!!{formula}s of~$!A$. Hence, we can define
  \begin{multline*}
  \fn{IsAx}_{!B \lif (!C \lif !B)}(n) \defiff \bexists{b< n}{
    \bexists{c < n}{(\fn{Sent}(b) \land \fn{Sent}(c) \land {}}}\\ n =
  \Gn{(} \concat b \concat \Gn{\lif} \concat \Gn{(} \concat c \concat
  \Gn{\lif} \concat b \concat \Gn{))}).
  \end{multline*}
  If we have such a definition for each axiom schema, their
  disjunction defines the property $\fn{IsAx}(n)$, ``$n$ is the
  G\"odel number of an axiom.''
\item The $i$th line in $\delta$ is justified by modus ponens iff there
  are lines $j$ and $k < i$ where the !!{sentence} on line $j$ is some
  formula $!A$, the sentence on line~$k$ is $!A \lif !B$, and the
  sentence on line~$i$ is~$!B$.
  \begin{multline*}
    \fn{MP}(d, i) \defiff \bexists{j < i}{\bexists{k < i}{}}\\
    (d)_k = \Gn{(} \concat (d)_j
      \concat \Gn{\lif} \concat (d)_i \concat \Gn{)}
  \end{multline*}
  Since bounded quantification, concatenation, and $=$ are primitive
  recursive, this defines a primitive recursive relation. 
\item A line in $\delta$ is justified by \QR{} if it is of the form
  $!B \lif \lforall[x][!A(x)]$, a preceding line is $!B \lif !A(c)$
  for some !!{constant}~$c$, and $c$ does on occur in~$!B$. This is
  the case iff
  \begin{enumerate}
  \item there is !!a{sentence}~$!B$ and
  \item !!a{formula}~$!A(x)$ with a single variable~$x$ free so that
  \item line $i$ contains $!B \lif \lforall[x][!A(x)]$
  \item some line $j < i$ contains $!B \lif \Subst{!A}{c}{x}$ for a constant~$c$
  \item which does not occur in~$!B$.
  \end{enumerate}
  All of these can be tested primitive recursively, since the G\"odel
  numbers of $!B$, $!A(x)$, and $x$ are less than the G\"odel number of the
  formula on line~$i$, and that of $a$ less than the G\"odel number of the
  formula on line~$j$:
  \begin{multline*}
    \fn{QR}_1(d, i) \defiff \bexists{b < (d)_i}{\bexists{x <
        (d)_i}{\bexists{a < (d)_i}{\bexists{c < (d)_j}{(}}}}
    \\ \fn{Var}(x) \land \fn{Const}(c) \land {} \\
    (d)_i = \Gn{(} \concat
    b \concat \Gn{\lif} \concat \Gn{\lforall} \concat x \concat a
    \concat \Gn{)} \land {}\\
    (d)_j = \Gn{(} \concat b \concat
    \Gn{\lif} \concat \fn{Subst}(a,c,x) \concat \Gn{)} \land {}\\ \fn{Sent}(b)
    \land \fn{Sent}(\fn{Subst}(a,c,x)) \land {} \bforall{k <
      \len{b}}{(b)_k \neq (c)_0})
  \end{multline*}
  Here we assume that $c$ and $x$ are the G\"odel numbers of the
  variable and constant considered as terms (i.e., not their symbol
  codes).  We test that $x$ is the only free variable of~$!A(x)$ by
  testing if $\Subst{!A(x)}{c}{x}$ is !!a{sentence}, and ensure that
  $c$ does not occur in $!B$ by requiring that every symbol of~$!B$ is
  different from~$c$.

  We leave the other version of \QR{} as an exercise.
\item $d$ is the G\"odel number of a correct !!{derivation} iff every
  line in it is an axiom, or justified by modus ponens or \QR. Hence:
  \[
  \fn{Deriv}(d) \defiff \bforall{i < \len{d}}{(\fn{IsAx}((d)_i) \lor
    \fn{MP}(d,i) \lor \fn{QR}(d, i))}
  \]
\end{enumerate}
\end{proof}

\begin{prob}
Define the following relations as in
\olref[inc][art][pax]{prop:followsby}:
\begin{enumerate}
\item $\fn{IsAx}_{!A \lif (!B \lif (!A \land !B))}(n)$,
\item $\fn{IsAx}_{\lforall[x][!A(x)] \lif !A(t)}(n)$,
\item $\fn{QR}_{2}(d, i)$ (for the other version of \QR).
\end{enumerate}
\end{prob}

\begin{prop}
Suppose $\Gamma$ is a primitive recursive set of !!{sentence}s.  Then
the relation $\Prf[\Gamma](x, y)$ expressing ``$x$ is the code of
!!a{derivation}~$\delta$ of $!A$ from~$\Gamma$ and $y$ is the G\"odel
number of~$!A$'' is primitive recursive.
\end{prop}

\begin{proof}
Suppose ``$y \in \Gamma$'' is given by the primitive recursive
predicate~$R_\Gamma(y)$.  We have to show that the relation
$\Prf[\Gamma](x, y)$ is primitive recursive, where $\Prf[\Gamma](x,
y)$ holds iff $y$ is the G\"odel number of !!a{sentence}~$!A$ and
$x$~is the code of !!a{derivation} of~$!A$ from $\Gamma$.

By the previous proposition, the property $\fn{Deriv}(x)$ which holds
iff $x$ is the code of a correct !!{derivation}~$\delta$ is primitive
recursive. However, that definition did not take into account the set
$\Gamma$ as an additional way to justify lines in the derivation. Our
primitive recursive test of whether a line is justified by \QR{} also
left out of consideration the requirement that the constant~$c$ is not
allowed to occur in~$\Gamma$. It is possible to amend our definition so
that it takes into account $\Gamma$ directly, but it is easier to use
$\fn{Deriv}$ and the deduction theorem. $\Gamma \Proves !A$ iff
there is some finite list of !!{sentence}s $!B_1$, \dots, $!B_n \in
\Gamma$ such that $\{!B_1, \dots, !B_n\} \Proves !A$. And by the
deduction theorem, this is the case if $\Proves (!B_1 \lif (!B_2 \lif
\cdots (!B_n \lif !A)\cdots))$. Whether !!a{sentence} with G\"odel
number~$z$ is of this form can be tested primitive recursively.  So,
instead of considering $x$ as the G\"odel number of !!a{derivation} of
the !!{sentence} with G\"odel number~$y$ \emph{from $\Gamma$}, we consider
$x$ as the G\"odel number of !!a{derivation} of a nested conditional
of the above form from~$\emptyset$.

First, if we have a sequence of !!{sentence}s, we can primitive
recursively form the conditional with all these sentences as
antecedents and given !!{sentence} as consequent:
\begin{align*}
  \fn{hCond}(s, y, 0) & = y \\
  \fn{hCond}(s, y, n+1) & = \Gn{(} \concat (s)_{n} \concat \Gn{\lif}
  \concat \fn{Cond}(s, y, n) \concat \Gn{)}\\
  \fn{Cond}(s, y) & = \fn{hCond}(s, y, \len{s})\\
  \intertext{So we can define $\Prf[\Gamma](x, y)$ by}
  \Prf[\Gamma](x, y) & \defiff \bexists{s < \fn{sequenceBound}(x,x)}{(} \\
  &\qquad (x)_{\len{x}-1} = \fn{Cond}(s,y) \land {} \\
  &\qquad\bforall{i<\len{s}}{(s)_i \in \Gamma} \land {} \\
  &\qquad\fn{Deriv}(x)).
\end{align*}
The bound on~$s$ is given by considering that each $(s)_i$ is the
G\"odel number of a subformula of the last line of the derivation,
i.e., is less than $(x)_{\len{x}-1}$. The number of antecedents $!B
\in \Gamma$, i.e., the length of~$s$, is less than the length of the
last line of~$x$.
\end{proof}

\end{document}
