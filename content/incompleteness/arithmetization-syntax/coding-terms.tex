% Part: incompleteness
% Chapter: arithmetization-syntax
% Section: coding-terms

\documentclass[../../../include/open-logic-section]{subfiles}

\begin{document}

\olfileid{inc}{art}{trm}
\olsection{Coding Terms}

\begin{explain}
A term is simply a certain kind of sequence of symbols: it is built up
inductively from constants and variables according to the formation
rules for terms.  Since sequences of symbols can be coded as
numbers---using a coding scheme for the symbols plus a way to code
sequences of numbers---assigning G\"odel numbers to terms is not
difficult.  The challenge is rather to show that the property a number
has if it is the G\"odel number of a correctly formed term is
computable, or in fact primitive recursive.
\end{explain}

\begin{prop}
\ollabel{prop:term-primrec}
The relations $\fn{Term}(x)$ and $\fn{ClTerm}(x)$ which hold iff $x$
is the G\"odel number of a term or a closed term, respectively, are
primitive recursive.
\end{prop}

\begin{proof}
A sequence of symbols~$s$ is a term iff there is a sequence~$s_0$,
\dots, $s_{k-1} = s$ of terms which records how the term~$s$ was formed
from !!{constant}s and !!{variable}s according to the formation rules
for terms. To express that such a putative formation sequence follows
the formation rules it has to be the case that, for each $i < k$, either
\begin{enumerate}
\item $s_i$ is a variable~$\Obj v_j$, or
\item $s_i$ is a !!{constant}~$\Obj c_j$, or
\item $s_i$ is built from $n$ terms $t_1$, \dots, $t_n$ occurring
  prior to place~$i$ using an $n$-place !!{function}~$\Obj f^n_j$.
\end{enumerate}
To show that the corresponding relation on G\"odel numbers is
primitive recursive, we have to express this condition primitive
recursively, i.e., using primitive recursive functions, relations, and
bounded quantification.

Suppose $y$ is the number that codes the sequence $s_0$, \dots, $s_{k-1}$,
i.e., $y = \tuple{\Gn{s_0}, \dots, \Gn{s_k}}$.  It codes a formation
sequence for the term with G\"odel number~$x$ iff for all $i < k$:
\begin{enumerate}
\item there is a $j$ such that $(y)_i = \Gn{\Obj v_j}$, or
\item there is a $j$ such that $(y)_i = \Gn{\Obj c_j}$, or
\item there is an $n$ and a number~$z = \tuple{z_1, \dots, z_n}$ such
  that each $z_l$ is equal to some $(y)_{i'}$ for $i' < i$ and
\[
(y)_i = \Gn{\Obj f^n_j(} \concat \fn{flatten}(z) \concat \Gn{)},
\]
\end{enumerate}
and moreover $(y)_{k-1} = x$.  The function $\fn{flatten}(z)$ turns
the sequence $\tuple{\Gn{t_1}, \dots, \Gn{t_n}}$ into $\Gn{t_1, \dots,
  t_n}$ and is primitive recursive.

The indices $j$, $n$, the G\"odel numbers $z_l$ of the terms $t_l$,
and the code~$z$ of the sequence~$\tuple{z_1, \dots, z_n}$, in (3) are
all less than~$y$.  We can replace $k$ above with $\len{y}$.  Hence we
can express ``$y$ is the code of a formation sequence of the term with
G\"odel number~$x$'' in a way that shows that this relation is
primitive recursive.

We now just have to convince ourselves that there is a primitive
recursive bound on~$y$.  But if $x$ is the G\"odel number of a term,
it must have a formation sequence with at most $\len{x}$ terms (since
every term in the formation sequence of~$s$ must start at some place
in~$s$, and no two subterms can start at the same place).  The G\"odel
number of each subterm of~$s$ is of course $\le x$.  Hence, there
always is a formation sequence with code $\le x^{\len{x}}$.

For $\fn{ClTerm}$, simply leave out the clause for !!{variable}s.
\end{proof}

\begin{proof}[Alternative proof of \olref{prop:term-primrec}]
The inductive definition says that !!{constant}s and !!{variable}s are
terms, and if $t_1$, \dots,~$t_n$ are terms, then so is $\Obj
f^n_j(t_1, \dots, t_n)$, for any $n$ and $j$. So terms are formed in
stages: !!{constant}s and !!{variable}s at stage~$0$, terms involving
one !!{function} at stage~$1$, those involving at least two nested
!!{function}s at stage~$2$, etc.  Let's say that a sequence of
symbols~$s$ is a term of level~$l$ iff $s$ can be formed by applying
the inductive definition of terms $l$ (or fewer) times, i.e., it
``becomes'' a term by stage~$l$ or before.  So $s$ is a term of
level~$l+1$ iff
\begin{enumerate}
\item $s$ is a variable~$\Obj v_j$, or
\item $s$ is !!a{constant}~$\Obj c_j$, or
\item $s$ is built from $n$ terms $t_1$, \dots, $t_n$ of level~$l$
  and an $n$-place !!{function}~$\Obj f^n_j$.
\end{enumerate}
To show that the corresponding relation on G\"odel numbers is
primitive recursive, we have to express this condition primitive
recursively, i.e., using primitive recursive functions, relations, and
bounded quantification.

The number $x$ is the G\"odel number of a term $s$ of level~$l+1$ iff
\begin{enumerate}
\item there is a $j$ such that $x = \Gn{\Obj v_j}$, or
\item there is a $j$ such that $x = \Gn{\Obj c_j}$, or
\item there is an $n$, a $j$, and a number~$z = \tuple{z_1, \dots, z_n}$ such
  that each $z_i$ is the G\"odel number of a term of level~$l$ and
\[
x = \Gn{\Obj f^n_j(} \concat \fn{flatten}(z) \concat \Gn{)},
\]
\end{enumerate}
and moreover $(y)_{k-1} = x$.

The indices $j$, $n$, the G\"odel numbers $z_i$ of the terms $t_i$,
and the code~$z$ of the sequence~$\tuple{z_1, \dots, z_n}$, in (3) are
all less than~$x$. So we get a primitive recursive definition by:
\begin{align*}
\fn{lTerm}(x, 0) = {}& \fn{Var}(x) \lor \fn{Const}(x)\\
\fn{lTerm}(x, l+1) = {}&\fn{Var}(x) \lor \fn{Const}(x) \lor {}\\
& \exists z<x(\forall i<\len{z}\, \fn{lTerm}((z)_i, l) \land \\
& \exists j<x \,x = \Gn{\Obj f^{\len{z}}_j(} \concat
\fn{flatten}(z) \concat \Gn{)})
\end{align*}
We can now define $\fn{Term}(x)$ by $\fn{lTerm}(x, x)$, since the level
of a term is always less than the G\"odel number of the term.
\end{proof}

\begin{prob}
Show that the function $\fn{flatten}(z)$, which turns the sequence
$\tuple{\Gn{t_1}, \dots, \Gn{t_n}}$ into $\Gn{t_1, \dots, t_n}$, is
primitive recursive.
\end{prob}

\end{document}
