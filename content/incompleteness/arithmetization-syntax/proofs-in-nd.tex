% Part: incompleteness
% Chapter: arithmetization-syntax
% Section: proofs-in-nd

\documentclass[../../../include/open-logic-section]{subfiles}

\begin{document}

\olfileid{inc}{art}{pnd}
\olsection{\usetoken{P}{derivation} in Natural Deduction}

\begin{explain}
In order to arithmetize !!{derivation}s, we must represent
!!{derivation}s as numbers. Since !!{derivation}s are trees of
!!{formula}s where each inference carries one or two labels, a
recursive representation is the most obvious approach: we represent a
!!{derivation} as a tuple, the components of which are the number of
immediate sub-!!{derivation}s leading to the premises of the last
inference, the representations of these sub-!!{derivation}s, and the
end-!!{formula}, the discharge label of the last inference, and a
number indicating the type of the last inference.
\end{explain}

\begin{defn}
  If $\delta$ is !!a{derivation} in natural deduction, then $\Gn{\delta}$ is
  defined inductively as follows:
  \begin{enumerate}
  \item 
    If $\delta$ consists only of the assumption~$!A$, then $\Gn{\delta}$
    is $\tuple{0, \Gn{!A}, n}$. The number~$n$ is~$0$ if it is an
    !!{undischarged} assumption, and the numerical label otherwise.
  \item 
    If $\delta$ ends in an inference with one, two, or three premises,
    then $\Gn{\delta}$ is
    \begin{align*}
      & \tuple{1, \Gn{\delta_1}, \Gn{!A}, n, k}, \\
      & \tuple{2, \Gn{\delta_1}, \Gn{\delta_2}, \Gn{!A}, n, k}, \text{ or}\\
      & \tuple{3, \Gn{\delta_1}, \Gn{\delta_2}, \Gn{\delta_3}, \Gn{!A}, n,
        k},
    \end{align*}
    respectively.  Here $\delta_1$, $\delta_2$, $\delta_3$ are the
    sub-!!{derivation}s ending in the premise(s) of the last inference
    in $\delta$, $!A$ is the conclusion of the last inference
    in~$\delta$, $n$ is the discharge label of the last inference ($0$
    if the inference does not discharge any assumptions), and $k$~is
    given by the following table according to which rule was used in the
    last inference.
  
    \begin{tabular}{lccccccc}
      \text{Rule:} & \Intro{\land} & \Elim{\land} & \Intro{\lor} & \Elim{\lor} \\
      $k$: & 1 & 2 & 3 & 4  \\[1.5ex]
      \text{Rule:} &  \Intro{\lif} & \Elim{\lif} & \Intro{\lnot} & \Elim{\lnot} \\
      $k$: & 5 & 6 & 7 & 8 \\[1.5ex]
      \text{Rule:}  &  \FalseInt & \FalseCl & \Intro{\lforall} & \Elim{\lforall} \\
      $k$: & 9 & 10 & 11 & 12 \\[1.5ex]
      \text{Rule:} & \Intro{\lexists} & \Elim{\lexists} & \Intro{\eq} & \Elim{\eq} \\
      $k$: & 13 & 14 & 15 & 16 
    \end{tabular}
  \end{enumerate}
\end{defn}

\begin{ex}
  Consider the very simple !!{derivation}
  \begin{prooftree}
    \AxiomC{$\Discharge{!A \land !B}{1}$}
    \RightLabel{\Elim{\land}}
    \UnaryInfC{$!A$}
    \DischargeRule{\Intro{\lif}}{1}
    \UnaryInfC{$(!A \land !B) \lif !A$}
  \end{prooftree}
  The G\"odel number of the assumption would be $d_0 = \tuple{0,
    \Gn{!A \land !B}, 1}$.  The G\"odel number of the !!{derivation}
  ending in the conclusion of~$\Elim{\land}$ would be $d_1 = \tuple{1,
     d_0, \Gn{!A}, 0, 2}$ ($1$ since $\Elim{\land}$ has one premise,
  the G\"odel number of conclusion~$!A$, $0$ because no assumption is
  discharged, and $2$ is the number coding $\Elim{\land}$). The G\"odel
  number of the entire !!{derivation} then is
  $\tuple{1, d_1, \Gn{((!A \land !B) \lif
      !A)}, 1, 5}$, i.e.,
  \[
  \tuple{1, \tuple{1, \tuple{0, \Gn{(!A \land !B)}, 1}, \Gn{!A}, 0, 2},
    \Gn{((!A \land !B) \lif !A)}, 1, 5}.
  \]
\end{ex}

\begin{explain}
Having settled on a representation of !!{derivation}s, we must also
show that we can manipulate G\"odel numbers of such !!{derivation}s
primitive recursively, and express their essential properties and
relations.  Some operations are simple: e.g., given a G\"odel
number~$d$ of !!a{derivation}, $\fn{EndFmla}(d) = (d)_{(d)_0+1}$
gives us the G\"odel number of its end-!!{formula},
$\fn{DischargeLabel}(d) = (d)_{(d)_0 + 2}$ gives us the discharge
label and $\fn{LastRule}(d) = (d)_{(d)_0 + 3}$ the number indicating
the type of the last inference.  Some are much harder.  We'll at least
sketch how to do this.  The goal is to show that the relation
``$\delta$ is !!a{derivation} of~$!A$ from~$\Gamma$'' is a primitive
recursive relation of the G\"odel numbers of $\delta$ and~$!A$.
\end{explain}

\begin{prop}
  The following relations are primitive recursive:
  \begin{enumerate}
  \item $!A$ occurs as an assumption in~$\delta$ with label~$n$.
  \item All assumptions in $\delta$ with label~$n$ are of the form~$!A$
    (i.e., we can !!{discharge} the assumption~$!A$ using label~$n$
    in~$\delta$).
  \end{enumerate}
\end{prop}

\begin{proof}
  We have to show that the corresponding relations between G\"odel
  numbers of !!{formula}s and G\"odel numbers of !!{derivation}s are
  primitive recursive.
  \begin{enumerate}
  \item We want to show that $\fn{Assum}(x, d, n)$, which holds if $x$
    is the G\"odel number of an assumption of the !!{derivation} with
    G\"odel number~$d$ labelled~$n$, is primitive recursive.  This is
    the case if the !!{derivation} with G\"odel number~$\tuple{0, x,
      n}$ is a sub-!!{derivation} of~$d$. Note that the way we code
    !!{derivation}s is a special case of the coding of trees introduced in
    \olref[cmp][rec][tre]{sec}, so the primitive recursive function
    $\fn{SubtreeSeq}(d)$ gives a sequence of G\"odel numbers of all
    sub-!!{derivation}s of~$d$ (of length a most $d$). So we can
    define
    \[
    \fn{Assum}(x, d, n) \defiff \bexists{i<d}{(\fn{SubtreeSeq}(d))_i =
      \tuple{0, x, n}}.
    \]
  \item We want to show that $\fn{Discharge}(x, d, n)$, which holds if
    all assumptions with label~$n$ in the !!{derivation} with G\"odel
    number~$d$ all are the !!{formula} with G\"odel number~$x$.  But
    this relation holds iff $\bforall{y<d}{(\fn{Assum}(y, d, n) \lif y
      = x)}$.
  \end{enumerate}
\end{proof}

\begin{prop}
  \ollabel{prop:followsby}
  The property $\fn{Correct}(d)$ which holds iff the last inference in
  the !!{derivation}~$\delta$ with G\"odel number~$d$ is correct, is
  primitive recursive.
\end{prop}

\begin{proof}
  Here we have to show that for each rule of inference~$R$ the
  relation $\fn{FollowsBy}_R(d)$ is primitive recursive, where
  $\fn{FollowsBy}_R(d)$ holds iff $d$ is the G\"odel number of
  !!{derivation}~$\delta$, and the end-!!{formula} of~$\delta$ follows
  by a correct application of~$R$ from the immediate
  sub-!!{derivation}s of~$\delta$.

  A simple case is that of the \Intro{\land} rule. If $\delta$ ends in
  a correct $\Intro{\land}$ inference, it looks like this:
  \begin{prooftree}
    \AxiomC{}
    \RightLabel{$\delta_1$}
    \DeduceC{$!A$}
    
    \AxiomC{}
    \RightLabel{$\delta_2$}
    \DeduceC{$!B$}

    \RightLabel{\Intro\land}
    \BinaryInfC{$!A \land !B$}
  \end{prooftree}
  Then the G\"odel number~$d$ of~$\delta$ is $\tuple{2, d_1, d_2,
    \Gn{(!A \land !B)}, 0, k}$ where $\fn{EndFmla}(d_1) = \Gn{!A}$,
  $\fn{EndFmla}(d_2) = \Gn{B}$, $n=0$, and $k=1$. So we can define
  $\fn{FollowsBy}_{\Intro\land}(d)$ as
  \begin{multline*}
    (d)_0 = 2 \land \fn{DischargeLabel}(d) = 0 \land \fn{LastRule}(d) = 1 \land {}\\
    \fn{EndFmla}(d) = \Gn{(} \concat \fn{EndFmla}((d)_1) \concat \Gn{\land}
    \concat \fn{EndFmla}((d)_2) \concat \Gn{)}.
  \end{multline*}

  Another simple example if the $\Intro\eq$ rule. Here the premise is
  an empty !!{derivation}, i.e., $(d)_1 = 0$, and no discharge label,
  i.e., $n=0$.  However, $!A$ must be of the form $\eq[t][t]$, for a
  closed term~$t$. Here, a primitive recursive definition is
  \begin{multline*}
    (d)_0 = 1 \land (d)_1 = 0 \land \fn{DischargeLabel}(d) = 0 \land {}\\
    \bexists{t<d}{(\fn{ClTerm}(t) \land 
      \fn{EndFmla}(d) = 
      \Gn{{\eq}(} \concat t \concat \Gn{,} \concat t \concat \Gn{)})}
  \end{multline*}

  For a more complicated example, $\fn{FollowsBy}_{\Intro{\lif}}(d)$
  holds iff the end-!!{formula} of~$\delta$ is of the form $(!A \lif
  !B)$, where the end-!!{formula} of $\delta_1$ is~$!B$, and any
  assumption in~$\delta$ labelled~$n$ is of the form~$!A$.  We can
  express this primitive recursively by
  \begin{multline*}
    (d)_0 = 1 \land {}\\
    \bexists{a<d}{(\fn{Discharge}(a, (d)_1, \fn{DischargeLabel}(d)) \land {}}\\
      \fn{EndFmla}(d) = (\Gn{(} \concat a \concat \Gn{\lif}
      \concat \fn{EndFmla}((d)_1) \concat \Gn{)}))
  \end{multline*}
  (Think of $a$ as the G\"odel number of~$!A$).

  For another example, consider \Intro{\lexists}.  Here, the last
  inference in~$\delta$ is correct iff there is !!a{formula}~$!A$, a
  closed term~$t$ and !!a{variable}~$x$ such that $\Subst{!A}{t}{x}$
  is the end-!!{formula} of the !!{derivation}~$\delta_1$ and
  $\lexists[x][!A]$ is the conclusion of the last inference.  So,
  $\fn{FollowsBy}_{\Intro{\lexists}}(d)$ holds iff
  \begin{multline*}
    (d)_0 = 1 \land \fn{DischargeLabel}(d) = 0 \land {} \\
    \bexists{a < d}{\bexists{x<d}{\bexists{t<d}{
          (\fn{ClTerm}(t) \land \fn{Var}(x)  \land {}}}}\\
    \fn{Subst}(a,t,x) = \fn{EndFmla}((d)_1) \land
    \fn{EndFmla}(d) = (\Gn{\lexists} \concat x \concat a)).
  \end{multline*}

  We then define $\fn{Correct}(d)$ as
  \begin{multline*}
    \fn{Sent}(\fn{EndFmla}(d)) \land {}\\
    (\fn{LastRule}(d) = 1 \land
    \fn{FollowsBy}_{\Intro\land}(d)) \lor \dots \lor {}\\
    (\fn{LastRule}(d) = 16 \land \fn{FollowsBy}_{\Elim\eq}(d)) \lor {}\\
    \bexists{n<d}{\bexists{x<d}{(d = \tuple{0, x, n})}}.
  \end{multline*}
  The first line ensures that the end-!!{formula} of~$d$ is a
  sentence. The last line covers the case where $d$ is just an
  assumption.
\end{proof}

\begin{prob}
  Define the following properties as in
  \olref[inc][art][pnd]{prop:followsby}:
  \begin{enumerate}
  \item $\fn{FollowsBy}_{\Elim{\lif}}(d)$,
  \item $\fn{FollowsBy}_{\Elim{\eq}}(d)$,
  \item $\fn{FollowsBy}_{\Elim{\lor}}(d)$,
  \item $\fn{FollowsBy}_{\Intro{\lforall}}(d)$.
  \end{enumerate}
  For the last one, you will have to also show that you can test
  primitive recursively if the last inference of the
  !!{derivation} with G\"odel number~$d$ satisfies the eigenvariable
  condition, i.e., the eigenvariable~$a$ of the $\Intro{\lforall}$
  inference occurs neither in the end-!!{formula} of~$d$ nor in an
  open assumption of~$d$. You may use the primitive recursive
  predicate $\fn{OpenAssum}$ from
  \olref[inc][art][pnd]{prop:openassum} for this.
\end{prob}

\begin{prop}
  \ollabel{prop:deriv}
  The relation $\fn{Deriv}(d)$ which holds if $d$ is the G\"odel
  number of a correct !!{derivation}~$\delta$, is primitive recursive.
\end{prop}

\begin{proof}
  !!^a{derivation}~$\delta$ is correct if every one of its inferences
  is a correct application of a rule, i.e., if every one of its
  sub-!!{derivation}s ends in a correct inference. So, $\fn{Deriv}(d)$
  iff
  \[
  \bforall{i<\len{\fn{SubtreeSeq}(d)}}{\fn{Correct}((\fn{SubtreeSeq}(d))_i)}
  \qedhere
  \]
\end{proof}

\begin{prop}
  \ollabel{prop:openassum} The relation $\fn{OpenAssum}(z, d)$ that
  holds if $z$ is the G\"odel number of !!a{undischarged} assumption~$!A$
  of the !!{derivation}~$\delta$ with G\"odel number~$d$, is primitive
  recursive.
\end{prop}

\begin{proof}
  An occurrence of an assumption is !!{discharged} if it occurs with
  label~$n$ in a sub-!!{derivation} of~$\delta$ that ends in a rule
  with discharge label~$n$. So $!A$ is !!a{undischarged} assumption
  of~$\delta$ if at least one of its occurrences is not !!{discharged}
  in~$\delta$. We must be careful: $\delta$ may contain both
  !!{discharged} and !!{undischarged} occurrences of~$!A$.

  Consider a sequence $\delta_0$, \dots, $\delta_k$ where $\delta_0 =
  d$, $\delta_k$ is the assumption $\Discharge{!A}{n}$ (for some~$n$),
  and $\delta_{i}$ is an immediate sub-!!{derivation}
  of~$\delta_{i+1}$. If such a sequence exists in which no $\delta_i$
  ends in an inference with discharge label~$n$, then $!A$ is
  !!a{undischarged} assumption of~$\delta$.

  The primitive recursive function $\fn{SubtreeSeq}(d)$ provides us
  with a sequence of G\"odel numbers of all sub-!!{derivation}s
  of~$\delta$. Any sequence of G\"odel numbers of sub-!!{derivation}s
  of~$\delta$ is a subsequence of it. Being a subsequence of is a
  primitive recursive relation: $\fn{Subseq}(s, s')$ holds iff
  $\bforall{i<\len{s}}{\lexists[j<\len{s'}][(s)_i = (s)_j]}$. Being an
  immediate sub-!!{derivation} is as well: $\fn{Subderiv}(d, d')$ iff
  $\bexists{j<(d')_0}{d = (d')_j}$. So we can define
  $\fn{OpenAssum}(z, d)$ by
  \begin{multline*}
    \bexists{s<\fn{SubtreeSeq}(d)}{(\fn{Subseq}(s, \fn{SubtreeSeq}(d))
      \land (s)_0 = d \land {}} \\
    \bexists{n<d}{((s)_{\len{s} \tsub 1} = \tuple{0, z, n} \land {}}\\
      \bforall{i<(\len{s} \tsub 1)}{(\fn{Subderiv}((s)_i, (s)_{i+1})] \land {}}\\
      \fn{DischargeLabel}((s)_{i+1}) \neq n))).\qedhere
  \end{multline*}
\end{proof}

\begin{prop}
  Suppose $\Gamma$ is a primitive recursive set of !!{sentence}s.
  Then the relation $\Prf[\Gamma](x, y)$ expressing ``$x$ is the code
  of !!a{derivation}~$\delta$ of $!A$ from !!{undischarged}
  assumptions in~$\Gamma$ and $y$ is the G\"odel number of~$!A$'' is
  primitive recursive.
\end{prop}

\begin{proof}
  Suppose ``$y \in \Gamma$'' is given by the primitive recursive
  predicate~$R_\Gamma(y)$.  We have to show that $\Prf[\Gamma](x, y)$
  which holds iff $y$ is the G\"odel number of a sentence~$!A$ and
  $x$~is the code of a natural deduction !!{derivation} with end
  !!{formula}~$!A$ and all !!{undischarged} assumptions in~$\Gamma$ is
  primitive recursive.

  By \olref{prop:deriv}, the property $\fn{Deriv}(x)$ which holds iff
  $x$ is the G\"odel number of a correct derivation~$\delta$ in
  natural deduction is primitive recursive. Thus we can define
  $\Prf[\Gamma](x, y)$ by
  \begin{align*}
    \Prf[\Gamma](x, y) \defiff {}
    & \fn{Deriv}(x) \land \fn{EndFmla}(x) = y \land {} \\
    & \bforall{z < x}{(\fn{OpenAssum}(z, x) \lif R_\Gamma(z))}.
    \qedhere
  \end{align*}
\end{proof}

\end{document}
