% Part: incompleteness
% Chapter: arithmetization-syntax
% Section: proofs-in-nd

\documentclass[../../../include/open-logic-section]{subfiles}

\begin{document}

\olfileid{inc}{art}{pnd}
\olsection{\usetoken{P}{derivation} in Natural Deduction}

\begin{explain}
In order to arithmetize !!{derivation}s, we must represent
!!{derivation}s as numbers. Since !!{derivation}s are trees of
!!{formula}s where each inference carries one or two labels, a
recursive representation is the most obvious approach: we represent a
!!{derivation} as a tuple, the components of which are the
end-!!{formula}, the labels, and the representations of the
sub-!!{derivation}s leading to the premises of the last inference.
\end{explain}

\begin{defn}
If $\delta$ is a !!{derivation} in natural deduction, then $\Gn{\delta}$ is
\begin{enumerate}
\item $\tuple{0, \Gn{!A}, n}$ if $\delta$ consists only of the
  assumption~$!A$. The number~$n$ is~$0$ if it is an !!{undischarged}
  assumption, and the numerical label otherwise.
\item $\tuple{1, \Gn{!A}, n, k, \Gn{\delta_1}}$ if $\delta$
  ends in an inference with one premise, $k$~is given by the following
  table according to which rule was used in the last inference, and
  $\delta_1$ is the immediate subproof ending in the premise of the last
  inference. $n$~is the label of the inference, or~$0$
  if the inference does not !!{discharge} any assumptions.

\begin{tabular}{lccccccc}
\text{Rule:} & \Elim{\land} & \Intro{\lor} & \Intro{\lif} & \Intro{\lnot} & \FalseInt \\
$k$: & 1 & 2 & 3 & 4 & 5  \\[2ex]
\text{Rule:} & \FalseCl  & \Intro{\lforall} &
   \Elim{\lforall} & \Intro{\lexists} & \Intro{\eq} \\
$k$: & 6 & 7 & 8 & 9 & 10 
\end{tabular}
\item $\tuple{2, \Gn{!A}, n, k, \Gn{\delta_1}, \Gn{\delta_2}}$ if
  $\delta$ ends in an inference with two premises, $k$~is given by the
  following table according to which rule was used in the last
  inference, and $\delta_1$, $\delta_2$ are the immediate
  sub!!{derivation}s ending in the left and right premise of the last
  inference, respectively. $n$~is the label of the inference, or $0$
  if the inference does not !!{discharge} any assumptions.

\begin{tabular}{lcccc}
\text{Rule:}  & \Intro{\land} & \Elim{\lif} & \Elim{\lnot} \\
$k$: & 1 & 2 & 3 
\end{tabular}
\item $\tuple{3, \Gn{!A}, n, \Gn{\delta_1}, \Gn{\delta_2},
  \Gn{\delta_3}}$ if $\delta$ ends in an \Elim{\lor} inference.
  $\delta_1$, $\delta_2$, $\delta_3$ are the immediate
  sub!!{derivation}s ending in the left, middle, and right premise of
  the last inference, respectively, and $n$ is the label of the
  inference.
\end{enumerate}
\end{defn}

\begin{ex}
  Consider the very simple derivation
  \begin{prooftree}
    \AxiomC{$\Discharge{(!A \land !B)}{1}$}
    \RightLabel{\Elim{\land}}
    \UnaryInfC{$!A!$}
    \DischargeRule{\Intro{\lif}}{1}
    \UnaryInfC{$(!A \lif !B)$}
  \end{prooftree}
  The G\"odel number of the assumption would be $d_0 = \tuple{0,
    \Gn{(!A \land !B)}, 1}$.  The G\"odel number of the derivation
  ending in the conclusion of~$\Elim{\land}$ would be $d_1 = \tuple{1,
    \Gn{!A}, 0, 1, d_0}$ ($1$ since $\Elim{\land}$ has one premise,
  G\"odel number of conclusion~$!A$, $0$ because no assumption is
  discharged, $1$ is the number coding $\Elim{\land}$). The G\"odel
  number of the entire derivation then is $\tuple{1, \Gn{(!A \lif
      !B)}, 1, 3, d_1}$, i.e.,
  \[
  2^2\cdot 3^{\Gn{(!A \lif !B)}+1}\cdot 5^2\cdot 7^4\cdot
  11^{(2^2\cdot 3^{\Gn{!A}+1}\cdot 5^1 \cdot 7^2 \cdot
    11^{(2^1\cdot 3^{\Gn{(!A \land !B)}+1} \cdot 5^2)})}.
  \]
\end{ex}

\begin{explain}
Having settled on a representation of !!{derivation}s, we must also
show that we can manipulate such !!{derivation}s primite recursively, and
express their essential properties and relations so.  Some operations
are simple: e.g., given a G\"odel number~$d$ of a !!{derivation},
$(d)_1$ gives us the G\"odel number of its end-!!{formula}.  Some are
much harder.  We'll at least sketch how to do this.  The goal is to
show that the relation ``$\delta$ is !!a{derivation} of~$!A$
from~$\Gamma$'' is primitive recursive on the G\"odel numbers of
$\delta$ and~$!A$.
\end{explain}

\begin{prop}
\ollabel{prop:followsby}
The following relations are primitive recursive:
\begin{enumerate}
\item $!A$ occurs as an assumption in~$\delta$ with label~$n$.
\item All assumption in $\delta$ with label~$n$ are of the form~$!A$
  (i.e., we can !!{discharge} the assumption~$!A$ using label~$n$
  in~$\delta$).
\item $!A$ is !!a{undischarged} assumption of~$\delta$.
\item An inference with conclusion~$!A$, upper !!{derivation}s
  $\delta_1$ (and $\delta_2$, $\delta_3$), and discharge label~$n$ is
  correct.
\item $\delta$ is a correct natural deduction !!{derivation}.
\end{enumerate}
\end{prop}

\begin{proof}
We have to show that the corresponding relations between G\"odel
numbers of !!{formula}s, sequences of G\"odel numbers of !!{formula}s
(which code sets of !!{formula}s), and G\"odel numbers of !!{derivation}s 
are primitive recursive.
\begin{enumerate}
\item We want to show that $\fn{Assum}(x, d, n)$, which holds if $x$
  is the G\"odel number of an assumption of the !!{derivation} with
  G\"odel number~$d$ labelled~$n$, is primitive recursive. For this we
  need a helper relation $\fn{hAssum}(x, d, n, i)$ which holds if
  the !!{formula}~$!A$ with G\"odel number~$x$ occurs as an initial
  !!{formula} with label~$n$ in the !!{derivation} with G\"odel
  number~$d$ within~$i$ inferences up from the end-!!{formula}.
  \begin{multline*}
    \begin{aligned}
    \fn{hAssum}(x, d, n, 0) \defiff {} & \True\\   
    \fn{hAssum}(x, d, n, i+1) \defiff {} &
    \end{aligned}\\
    \begin{aligned}
      &  \fn{Sent}(x) \land (d = \tuple{0, x, n} \lor {}\\
      & ((d)_0 = 1 \land \fn{hAssum}(x, (d)_4, n, i)) \lor {}\\
      & ((d)_0 = 2 \land (\fn{hAssum}(x, (d)_4, n, i) \lor {}\\
      & \qquad\fn{hAssum}(x, (d)_5, n, i))) \lor {}\\
      & ((d)_0 = 3 \land (\fn{hAssum}(x, (d)_3, n, i) \lor {}\\
      & \qquad \fn{hAssum}(x, (d)_2, n, i)) \lor
      \fn{hAssum}(x, (d)_3, n, i))
    \end{aligned}
  \end{multline*}
  If the number~$i$ is large enough, e.g., larger than the maximum
  number of inferences between an initial !!{formula} and the
  end-!!{formula} of~$\delta$, it holds of $x$, $d$, $n$, and~$i$ iff
  $!A$ is an initial formula in $\delta$ labelled~$n$.  The number~$d$
  itself is larger than that maximum number of inferences.  So we can
  define
   \[
   \fn{Assum}(x, d, n) = \fn{hAssum}(x, d, n, d).
   \]
\item We want to show that $\fn{Discharge}(x, d, n)$, which holds if
  all assumptions with label~$n$ in the !!{derivation} with G\"odel
  number~$d$ all are the !!{formula} with G\"odel number~$x$.  But
  this relation holds iff $\bforall{y<d}{(\fn{Assum}(y, d, n) \lif y =
  x)}$.
\item An occurrence of an assumption is not open if it occurs
  with label~$n$ in a subderivation that ends in a rule with discharge
  label~$n$. Define the helper relation $\fn{hNotOpen}(x, d, n, i)$
  as
  \begin{multline*}
    \begin{aligned}
    \fn{hNotOpen}(x, d, n, 0) \defiff {} & \True\\   
    \fn{hNotOpen}(x, d, n, i+1) \defiff {} 
    \end{aligned}\\
    \begin{aligned}
      & (d)_2 = n \lor {}\\
      & ((d)_0 = 1 \land \fn{hNotOpen}(x, (d)_4, n, i)) \lor {}\\
      & ((d)_0 = 2 \land \fn{hNotOpen}(x, (d)_4, n, i) \land {}\\
      & \qquad\fn{hNotOpen}(x, (d)_5, n, i))) \lor {}\\
      & ((d)_0 = 3 \land \fn{hNotOpen}(x, (d)_3, n, i) \land {}\\
      & \qquad \fn{hNotOpen}(x, (d)_4, n, i) \land
      \fn{hNotOpen}(x, (d)_5, n, i))\\
    \end{aligned}
  \end{multline*}
  Note that all assumptions of the form~$!A$ labelled~$n$ are
  !!{discharged} in~$\delta$ iff either the last inference of~$\delta$
  !!{discharge}s them (i.e., the last inference has label~$n$), or if
  it is !!{discharged} in all of the immediate subderivations.

  A !!{formula}~$!A$ is an open assumption of~$\delta$ iff it is an
  initial formula of~$\delta$ (with label~$n$) and is not
  !!{discharged} in~$\delta$ (by a rule with lable~$n$). We can then
  define $\fn{OpenAssum}(x, d)$ as
  \[
  \bexists{n<d}{(\fn{Assum}(x,d,n,d)
  \land \lnot\fn{hNotOpen}(x, d, n, d))}.
  \]
\item Here we have to show that for each rule of inference~$R$ the
  relation $\fn{FollowsBy}_R(x, d_1, n)$ which holds if $x$ is the
  G\"odel number of the conclusion and $d_1$~is the G\"odel number of a
  derivation ending in the premise of a correct application of~$R$
  with label~$n$ is primitive recursive, and similarly for rules with
  two or three premises.

  The simplest case is that of the \Intro{\eq} rule. Here there is no
  premise, i.e., $d_1 = 0$.  However, $!A$ must be of the form
  $\eq[t][t]$, for a closed term~$t$. Here, a primitive recursive
  definition is
  \[
  \bexists{t<x}{(\fn{ClTerm}(t) \land x = 
    (\Gn{{\eq}(} \concat t \concat \Gn{,} \concat t \concat \Gn{)}))}
  \land d_1 = 0).
  \]

  For a more complicated example, $\fn{FollowsBy}_{\Intro{\lif}}(x, d_1,
    n)$ holds iff $!A$ is of the form $(!B \lif !C)$, the
  end-!!{formula} of $\delta$ is~$!C$, and any initial formula
  in~$\delta$ labelled~$n$ is of the form~$!B$.  We can express this
  primitive recursively by
  \begin{multline*}
    \bexists{y<x}{(\fn{Sent}(y) \land \fn{Discharge}(y, d_1) \land {}}\\
      \bexists{z<x}{(\fn{Sent}(y)  \land  (d)_1 = z) \land {}}\\
      x = (\Gn{(} \concat y \concat \Gn{\lif}
  \concat z\concat \Gn{)}))
  \end{multline*}
  (Think of $y$ as the G\"odel number of~$!B$ and $z$ as that
  of~$!C$.)

  For another example, consider \Intro{\lexists}.  Here, $!A$ is the
  conclusion of a correct inference with one upper derivation iff
  there is !!a{formula}~$!B$, a closed term~$t$ and
  !!a{variable}~$x$ such that $\Subst{!B}{t}{x}$ is the
  end-!!{formula} of the upper derivation and $\lexists[x][!B]$
  is the conclusion~$!A$, i.e., the formula with G\"odel number~$x$.
  So $\fn{FollowsBy}_{\Intro{\lexists}}(x, d_1, n)$ holds iff
\begin{multline*}
\fn{Sent}(x) \land \bexists{y < x}{\bexists{v<x}{\bexists{t<d}{
(\fn{Frm}(y) \land \fn{Term}(t) \land \fn{Var}(v)  \land {}}}}\\
\fn{FreeFor}(y, t, v) \land \fn{Subst}(y,t,v) = (d_1)_1 \land
x = (\Gn{\lexists} \concat v \concat z))
\end{multline*}
\item We first define a helper relation $\fn{hDeriv}(d, i)$ which
  holds if $d$ codes a correct derivation at least to $i$ inferences
  up from the end sequent.  $\fn{hDeriv}(d, 0)$ holds always.
  Otherwise, $\fn{hDeriv}(d, i+1)$ iff either $d$ just codes an
  assumption or $d$ ends in a correct inference and the codes of the
  immediate sub-!!{derivation}s satisfy $\fn{hDeriv}(d', i)$.
\begin{multline*}
  \begin{aligned}
\fn{hDeriv}(d, 0) \defiff {} & \True \\
\fn{hDeriv}(d, i+1) \defiff {} &
  \end{aligned}\\
  \begin{aligned}
& \bexists{x<d}{\bexists{n<d}{(\fn{Sent}(x) \land d = \tuple{0, x, n}) \lor {}}}\\
& ((d)_0 = 1 \land {}\\
& \quad ((d)_3 = 1 \land
\fn{FollowsBy}_{\Elim{\land}}((d)_1, (d)_4, (d)_2) \lor{}\\
& \qquad \vdots\\
& \quad ((d)_3 = 10 \land
\fn{FollowsBy}_{\Intro{\eq}}((d)_1, (d)_4, (d)_2)) \land {}\\
& \quad \fn{nDeriv}((d)_4, i)) \lor {}\\
& ((d)_0 = 2 \land {}\\
& \quad ((d)_3 = 1 \land 
\fn{FollowsBy}_{\Intro{\land}}((d)_1, (d)_4, (d)_5, (d)_2)) \lor{}\\
& \qquad \vdots\\
& \quad ((d)_3 = 3 \land
\fn{FollowsBy}_{\Elim{\lnot}}((d)_1, (d)_4, (d)_5, (d)_2)) \land {}\\
& \quad \fn{hDeriv}((d)_4, i) \land \fn{hDeriv}((d)_5, i)) \lor {}\\
& ((d)_0 = 3 \land {}\\
& \quad \fn{FollowsBy}_{\Elim{\lor}}((d)_1, (d)_3, (d)_4, (d)_5, (d)_2) \land{}\\
& \quad \fn{hDeriv}((d)_3, i) \land \fn{hDeriv}((d)_4, i)) \land \fn{hDeriv}((d)_5, i)  
  \end{aligned}
  \end{multline*}
This is a primitive recursive definition. Again we can define
$\fn{Deriv}(d)$ as $\fn{hDeriv}(d, d)$.
\end{enumerate}
\end{proof}

\begin{prob}
Define the following relations as in
\olref[inc][art][pnd]{prop:followsby}:
\begin{enumerate}
\item $\fn{FollowsBy}_{\Elim{\lif}}(x, d_1, d_2, n)$,
\item $\fn{FollowsBy}_{\Elim{\eq}}(x, d_1, d_2, n)$,
\item $\fn{FollowsBy}_{\Elim{\lor}}(x, d_1, d_2, d_3, n)$,
\item $\fn{FollowsBy}_{\Intro{\lforall}}(x, d_1, n)$.
\end{enumerate}
For the last one, you will have to also show that you can test
primitive recursively if the formula with G\"odel number~$x$ and the
!!{derivation} with G\"odel number~$d$ satisfy the eigenvariable
condition, i.e., the eigenvariable~$a$ of the $\Intro{\lforall}$
inference occurs neither in~$x$ nor in an open assumption of~$d$.
\end{prob}

\begin{prop}
Suppose $\Gamma$ is a primitive recursive set of !!{sentence}s.  Then
the relation $\Prf[\Gamma](x, y)$ expressing ``$x$ is the code of
!!a{derivation}~$\delta$ of $!A$ from !!{undischarged} assumptions
in~$\Gamma$ and $y$ is the G\"odel number of~$!A$'' is primitive
recursive.
\end{prop}

\begin{proof}
Suppose ``$y \in \Gamma$'' is given by the primitive recursive
predicate~$R_\Gamma(y)$.  We have to show that $\Prf[\Gamma](x, y)$
which holds iff $y$ is the G\"odel number of a sentence~$!A$ and
$x$~is the code of a natural deduction !!{derivation} with end
!!{formula}~$!A$ and all !!{undischarged} assumptions in~$\Gamma$ is
primitive recursive.

By the previous proposition, the property $\fn{Deriv}(x)$ which holds
iff $x$ is the code of a correct derivation~$\delta$ in natural
deduction is primitive recursive. If $x$ is such a code, then $(x)_1$
is the code of the end-!!{formula} of~$\delta$. Thus we can define
$\Prf[\Gamma](x, y)$ by
\begin{align*}
\Prf[\Gamma](x, y) \defiff {} &
\fn{Deriv}(x) \land (x)_1 = y \land {} \\
& \bforall{z < x}{(\fn{OpenAssum}(z, x) \lif R_\Gamma(z))}
\end{align*}
\end{proof}

\end{document}
