% Part: incompleteness
% Chapter: arithmetization-syntax
% Section: proofs-in-lk

\documentclass[../../../include/open-logic-section]{subfiles}

\begin{document}

\olfileid{inc}{art}{plk}
\olsection{\usetoken{P}{derivation} in $\Log{LK}$}

\begin{explain}
In order to arithmetize !!{derivation}s, we must represent
!!{derivation}s as numbers. Since !!{derivation}s are trees of sequents
where each inference carries also a label, a recursive
representation is the most obvious approach: we represent a
!!{derivation} as a tuple, the components of which are the
end-sequent, the label, and the representations of the
sub-!!{derivation}s leading to the premises of the last inference.
\end{explain}

\begin{defn}
If $\Gamma$ is a finite sequence of !!{sentence}s, $\Gamma =
\tuple{!A_1, \dots, !A_n}$, then $\Gn{\Gamma} = \tuple{\Gn{!A_1},
  \dots, \Gn{!A_n}}$.

If $\Gamma \Sequent \Delta$ is a sequent, then a G\"odel number of
$\Gamma \Sequent \Delta$ is
\[
\Gn{\Gamma \Sequent \Delta} = \tuple{\Gn{\Gamma}, \Gn{\Delta}}
\]

If $\pi$ is a !!{derivation} in $\Log{LK}$, then $\Gn{\pi}$ is
\begin{enumerate}
\item $\tuple{0, \Gn{\Gamma \Sequent \Delta}}$ if $\pi$ consists only
  of the initial sequent $\Gamma \Sequent \Delta$.
\item $\tuple{1, \Gn{\Gamma \Sequent \Delta}, k, \Gn{\pi'}}$ if $\pi$
  ends in an inference with one premise, $k$ is given by the following
  table according to which rule was used in the last inference, and
  $\pi'$ is the immediate subproof ending in the premise of the last
  inference.

\begin{tabular}{lccccccc}
  \text{Rule:} & \LeftR{\Weakening} & \RightR{\Weakening} &
  \LeftR{\Contraction} & \RightR{\Contraction} &
  \LeftR{\Exchange} & \RightR{\Exchange} \\
  $k$: & 1 & 2 & 3 & 4 & 5 & 6 \\[2ex]
  \text{Rule:} &   \LeftR{\lnot} & \RightR{\lnot} &
  \LeftR{\land} & 
  \RightR{\lor} &
  \RightR{\lif} \\
$k$: & 7 & 8 & 9 & 10 & 11 \\[2ex]
\text{Rule:} & \LeftR{\lforall} & \RightR{\lforall} &
   \LeftR{\lexists} & \RightR{\lexists} & = \\
$k$: & 12 & 13 & 14 & 15 & 16
\end{tabular}
\item $\tuple{2, \Gn{\Gamma \Sequent \Delta}, k, \Gn{\pi'},
  \Gn{\pi''}}$ if $\pi$ ends in an inference with two premises, $k$ is
  given by the following table according to which rule was used in the
  last inference, and $\pi'$, $\pi''$ are the immediate subproof
  ending in the left and right premise of the last inference,
  respectively.

\begin{tabular}{lcccc}
\text{Rule:} & \Cut & \RightR{\land} & \LeftR{\lor} & \LeftR{\lif} \\
$k$: & 1 & 2 & 3 & 4
\end{tabular}
\end{enumerate}
\end{defn}

\begin{explain}
Having settled on a representation of !!{derivation}s, we must also
show that we can manipulate such derivations primitive recursively, and
express their essential properties and relations so.  Some operations
are simple: e.g., given a G\"odel number~$d$ of a !!{derivation},
$(s)_1$ gives us the G\"odel number of its end-sequent.  Some are much
harder.  We'll at least sketch how to do this.  The goal is to show
that the relation ``$\pi$ is !!a{derivation} of~$!A$ from~$\Gamma$''
is a primitive recursive relation of the G\"odel numbers of $\pi$
and~$!A$.
\end{explain}

\begin{prop}
\ollabel{prop:followsby}
The following relations are primitive recursive:
\begin{enumerate}
\item $\Gamma \Sequent \Delta$ is an initial sequent.
\item $\Gamma \Sequent \Delta$ follows from $\Gamma' \Sequent \Delta'$
  (and $\Gamma'' \Sequent \Delta''$) by a rule of $\Log{LK}$.
\item $\pi$ is a correct $\Log{LK}$-!!{derivation}.
\end{enumerate}
\end{prop}

\begin{proof}
We have to show that the corresponding relations between G\"odel
numbers of !!{formula}s, sequences of G\"odel numbers of !!{formula}s
(which code sequences of !!{formula}s), and G\"odel numbers of sequents,
are primitive recursive.
\begin{enumerate}
\item $\Gamma \Sequent \Delta$ is an initial sequent if either there
  is !!a{sentence}~$!A$ such that $\Gamma \Sequent \Delta$ is $!A
  \Sequent !A$, or there is a term~$t$ such that $\Gamma \Sequent
  \Delta$ is $\emptyset \Sequent \eq[t][t]$.  In terms of G\"odel
  numbers, $\fn{InitSeq}(s)$ holds iff
\begin{align*}
\bexists{x < s}{} (\fn{Sent}(x) & \land
s = \tuple{\tuple{x},\tuple{x}})
\lor {}\\
\bexists{t<s}{} (\fn{Term}(t) & \land
s = \tuple{0, \tuple{\Gn{{\eq}(} \concat t \concat \Gn{,} \concat t \concat \Gn{)}}}).
\end{align*}
\item Here we have to show that for each rule of inference~$R$ the
  relation $\fn{FollowsBy}_R(s, s')$ which holds if $s$ and $s'$ are
  the G\"odel numbers of conclusion and premise of a correct
  application of~$R$ is primitive recursive.  If $R$ has two premises,
  $\fn{FollowsBy}_R$ of course has three arguments.

For instance, $\Gamma \Sequent \Delta$ follows correctly from $\Gamma'
\Sequent \Delta'$ by \RightR{\lexists} iff $\Gamma = \Gamma'$ and
there is a sequence of !!{formula}s~$\Delta''$, !!a{formula}~$!A$, a
variable~$x$ and a closed term~$t$ such that $\Delta' = \Delta'',
\Subst{!A}{t}{x}$ and $\Delta = \Delta'', \lexists[x][!A]$. We just
have to translate this into G\"odel numbers.  If $s = \Gn{\Gamma
  \Sequent \Delta}$ then $(s)_0 = \Gn{\Gamma}$ and $(s)_1 =
\Gn{\Delta}$.  So, $\fn{FollowsBy}_{\RightR{\lexists}}(s, s')$
holds iff
\begin{align*}
& (s)_0 = (s')_0 \land {}\\
&
  \bexists{d<s}{\bexists{f<s}{\bexists{x<s}{\bexists{t<s'}{
          (\fn{Frm}(f) \land \fn{Var}(y) \land \fn{Term}(t) \land {}}}}} \\
  & \qquad (s')_1 = d \concat \tuple{\fn{Subst}(f,t,x)} \land {} \\
  & \qquad (s)_1 = d \concat \tuple{\#(\lexists) \concat y \concat f}
\end{align*}
The individual lines express, respectively, ``$\Gamma = \Gamma$,''
``there is a sequence~($\Delta''$) with G\"odel number~$d$,
!!a{formula}~($!A$) with G\"odel number~$f$, a variable with G\"odel
number~$x$, and a term with G\"odel number~$t$,'' ``$\Delta' =
\Delta'', \Subst{!A}{t}{x}$,'' and ``$\Delta = \Delta'',
\lexists[x][!A]$''. (Remember that $\Gn{\Delta}$ is the number of a
sequence of G\"odel numbers of !!{formula}s in $\Delta$.)

\item We first define a helper relation $\fn{hDeriv}(s,n)$ which holds
  if $s$ codes a correct derivation to at least~$n$ inferences up from
  the end sequent.  If $n=0$ we let the relation be satisfied by
  default.  Otherwise, $\fn{hDeriv}(s, n+1)$ iff either $s$ consists
  just of an initial sequent, or it ends in a correct inference and
  the codes of the immediate sub!!{derivation}s satisfy
  $\fn{hDeriv}(s, n)$.
\begin{multline*}
\begin{aligned}
\fn{hDeriv}(s, 0) \defiff {} & \text{true}\\
\fn{hDeriv}(s, n+1) \defiff {}
\end{aligned}\\
\begin{aligned}
& ((s)_0 = 0 \land \fn{InitialSeq}((s)_1)) \lor{}\\
& ((s)_0 = 1 \land {}\\
& \quad ((s)_2 = 1 \land \fn{FollowsBy}_{\LeftR{\Contraction}}((s)_1, ((s)_3)_1)) \lor{}\\
& \qquad \vdots\\
& \quad ((s)_2 = 16 \land \fn{FollowsBy}_{=}((s)_1, ((s)_3)_1)) \land {}\\
& \quad \fn{hDeriv}((s)_3, n)) \lor {}\\
& ((s)_0 = 2 \land {}\\
& \quad ((s)_2 = 1 \land \fn{FollowsBy}_{\Cut}((s)_1, ((s)_3)_1), ((s)_4)_1)) \lor{}\\
& \qquad \vdots\\
& \quad ((s)_2 = 4 \land \fn{FollowsBy}_{\LeftR{\lif}}((s)_1, ((s)_3)_1), ((s)_4)_1)) \land {}\\
& \quad \fn{hDeriv}((s)_3, n) \land \fn{hDeriv}((s)_4, n))
\end{aligned}
\end{multline*}
This is a primitive recursive definition.  If the number~$n$ is large
enough, e.g., larger than the maximum number of inferences between an
initial sequent and the end sequent in~$s$, it holds of $s$ iff $s$ is
the G\"odel number of a correct !!{derivation}.  The number $s$ itself
is larger than that maximum number of inferences.  So we can now define
$\fn{Deriv}(s)$ by $\fn{hDeriv}(s,s)$.
\end{enumerate}
\end{proof}

\begin{prob}
Define the following relations as in
\olref[inc][art][plk]{prop:followsby}:
\begin{enumerate}
\item $\fn{FollowsBy}_{\RightR{\land}}(s, s', s'')$,
\item $\fn{FollowsBy}_{\eq}(s, s')$,
\item $\fn{FollowsBy}_{\RightR{\lforall}}(s, s')$.
\end{enumerate}
\end{prob}

\begin{prop}
Suppose $\Gamma$ is a primitive recursive set of !!{sentence}s.  Then
the relation $\Prf[\Gamma](x, y)$ expressing ``$x$ is the code of
!!a{derivation}~$\pi$ of $\Gamma_0 \Sequent !A$ for some finite
$\Gamma_0 \subseteq \Gamma$ and $x$ is the G\"odel number of~$!A$'' is
primitive recursive.
\end{prop}

\begin{proof}
Suppose ``$y \in \Gamma$'' is given by the primitive recursive
predicate~$R_\Gamma(y)$.  We have to show that $\Prf[\Gamma](x, y)$
which holds iff $y$ is the G\"odel number of a sentence~$!A$ and
$x$~is the code of an $\Log{LK}$-!!{derivation} with end sequent
$\Gamma_0 \Sequent !A$ is primitive recursive.

By the previous proposition, the property $\fn{Deriv}(x)$ which holds
iff $x$ is the code of a correct derivation~$\pi$ in $\Log{LK}$ is
primitive recursive.  If $x$ is such a code, then $(x)_1$ is the code
of the end sequent of~$\pi$, and so $((x)_1)_0$ is the code of the
left side of the end sequent and $((x)_1)_1$ the right side.  So we can
express ``the right side of the end sequent of~$\pi$ is~$!A$'' as
$\len{((x)_1)_1} = 1 \land (((x)_1)_1)_0 = x$.  The left side of the
end sequent of $\pi$ is of course automatically finite, we just have
to express that every sentence in it is in~$\Gamma$.  Thus we can
define $\Prf[\Gamma](x, y)$ by
\begin{align*}
\Prf[\Gamma](x, y) \defiff {}&
\fn{Sent}(y) \land \fn{Deriv}(x) \land {} \\
& \bforall{i <
  \len{((x)_1)_0}}{R_\Gamma((((x)_1)_0)_i)} \land {}\\
& \len{((x)_1)_1} = 1 \land (((x)_1)_1)_0 = x
\end{align*}
\end{proof}

\end{document}
