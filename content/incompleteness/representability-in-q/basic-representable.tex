% Part: incompleteness
% Chapter: representability-in-q
% Section: basic-representable

\documentclass[../../../include/open-logic-section]{subfiles}

\begin{document}

\olfileid{inc}{req}{bre}
\olsection{Basic Functions are Representable in $\Th{Q}$}

First we have to show that all the basic functions are representable
in~$\Th{Q}$. In the end, we need to show how to assign to each $k$-ary
basic function $f(x_0,\dots,x_{k-1})$ !!a{formula}
$!A_f(x_0,\dots,x_{k-1},y)$ that represents it.

We will be able to represent zero, successor, plus, times, the
characteristic function for equality, and projections. In each case,
the appropriate representing function is entirely straightforward; for
example, zero is represented by the formula $y = \Obj 0$, successor is
represented by the !!{formula} $x_0' = y$, and addition is represented
by the !!{formula} $(x_0 + x_1) = y$. The work involves showing that
$\Th{Q}$ can prove the relevant !!{sentence}s; for example, saying
that addition is represented by the !!{formula} above involves showing
that for every pair of natural numbers $m$ and $n$, $\Th{Q}$ proves
\begin{align*}
& \eq[\num n + \num m][\num {n+m}] \text{ and}\\
& \lforall[y][(\eq[(\num n + \num m)][y] \lif \eq[y][\num{n+m}])].
\end{align*}

\begin{prop}
\ollabel{prop:rep-zero}
The zero function $\Zero(x) = 0$ is represented in~$\Th{Q}$ by
$\eq[y][\Obj 0]$.
\end{prop}

\begin{prop}
\ollabel{prop:rep-succ}
The successor function $\Succ(x) = x+1$ is represented in~$\Th{Q}$ by
$\eq[y][x']$.  
\end{prop}

\begin{prop}
\ollabel{prop:rep-proj}
The projection function $\Proj{n}{i}(x_0, \dots, x_{n-1}) = x_i$ is
represented in~$\Th{Q}$ by $\eq[y][x_i]$.
\end{prop}

\begin{prob}
Prove that $\eq[y][\Obj 0]$, $\eq[y][x']$, and $\eq[y][x_i]$ represent
$\Zero$, $\Succ$, and $\Proj{n}{i}$, respectively.
\end{prob}

\begin{prop}
\ollabel{prop:rep-id}
The characteristic function of~$=$,
\[
\Char{=}(x_0, x_1) =
\begin{cases}
  1 & \text{if } x_0 =x_1\\
  0 & otherwise
\end{cases}
\]
is represented in~$\Th{Q}$ by
\[
(\eq[x_0][x_1] \land \eq[y][\num{1}]) \lor (\eq/[x_0][x_1] \land
\eq[y][\num{0}]).
\]
\end{prop}

The proof requires the following lemma.

\begin{lem}
\ollabel{lem:q-proves-neq} Given natural numbers $n$ and $m$, if $n
\neq m$, then $\Th{Q} \Proves \eq/[\num n][\num m]$.
\end{lem}

\begin{proof}
Use induction on $n$ to show that for every $m$, if $n \neq m$, then
$Q \Proves \eq/[\num n][\num m]$.

In the base case, $n = 0$. If $m$ is not equal to $0$, then $m = k +
1$ for some natural number $k$. We have an axiom that says
$\lforall[x][\eq/[0][x']]$. By a quantifier axiom, replacing $x$ by
  $\num k$, we can conclude $\eq/[0][\num k']$. But $\num k'$ is just
  $\num m$.

In the induction step, we can assume the claim is true for $n$, and
consider $n+1$. Let $m$ be any natural number. There are two
possibilities: either $m = 0$ or for some $k$ we have $m = k+1$. The
first case is handled as above. In the second case, suppose $n+1 \neq
k+1$. Then $n \neq k$. By the induction hypothesis for $n$ we have
$\Th{Q} \Proves \eq/[\num n][\num k]$. We have an axiom that says
$\lforall[x][\lforall[y][\eq[x'][y'] \lif \eq[x][y]]]$. Using a
quantifier axiom, we have $\eq[\num n'][\num k'] \lif \eq[\num n][\num
  k]$. Using propositional logic, we can conclude, in $\Th{Q}$,
$\eq/[\num n][\num k] \lif \eq/[\num n'][\num k']$. Using modus
ponens, we can conclude $\eq/[\num n'][\num k']$, which is what we want,
since $\num k'$ is $\num m$.
\end{proof}

\begin{explain}
Note that the lemma does not say much: in essence it says that $\Th{Q}$ can
prove that different numerals denote different objects. For example,
$\Th{Q}$ proves $0'' \neq 0'''$. But showing that this holds in general
requires some care. Note also that although we are using induction, it
is induction \emph{outside} of $\Th{Q}$.
\end{explain}

\begin{proof}[Proof of \olref{prop:rep-id}]
If $n = m$, then $\num{n}$ and $\num{m}$ are the same term, and
$\Char{=}(n, m) = 1$. But $\Th{Q} \Proves (\eq[\num{n}][\num{m}] \land
\eq[\num{1}][\num{1}])$, so it proves $!A_=(\num{n}, \num{m},
\num{1})$.  If $n \neq m$, then $\Char=(n, m) = 0$. By
\olref{lem:q-proves-neq}, $\Th{Q} \Proves \eq/[\num{n}][\num{m}]$ and
so also $(\eq/[\num{n}][\num{m}] \land \Obj 0 = \Obj 0)$. Thus $\Th{Q}
\Proves !A_=(\num{n}, \num{m}, \num{0})$.

For the second part, we also have two cases. If $n = m$, we have to
show that that $\Th{Q} \Proves \lforall[(!A_=(\num{n}, \num{m}, y)
  \lif \eq[y][\num{1}])]$.  Arguing informally, suppose $!A_=(\num{n},
\num{m}, y)$, i.e.,
\[
(\eq[\num{n}][\num{n}] \land \eq[y][\num{1}]) \lor
(\eq/[\num{n}][\num{n}] \land \eq[y][\num{0}])
\]
The left disjunct implies $\eq[y][\num{1}]$ by logic; the right
contradicts $\eq[\num{n}][\num{n}]$ which is provable by logic.

Suppose, on the other hand, that $n \neq m$. Then $!A_=(\num{n},
\num{m}, y)$ is
\[
(\eq[\num{n}][\num{m}] \land \eq[y][\num{1}]) \lor
(\eq/[\num{n}][\num{m}] \land \eq[y][\num{0}])
\]
Here, the left disjunct contradicts $\eq/[\num{n}][\num{m}]$, which is
provable in $\Th{Q}$ by \olref{lem:q-proves-neq}; the right disjunct
entails $\eq[y][\num{0}]$.
\end{proof}

\begin{prop}
\ollabel{prop:rep-add}
The addition function $\Add(x_0, x_1) = x_0+x_1$ is is represented
in~$\Th{Q}$ by
\[
y = (x_0 + x_1).
\]
\end{prop}

\begin{lem}
\ollabel{lem:q-proves-add}
$\Th{Q} \Proves \eq[(\num{n} + \num{m})][\num{n+m}]$
\end{lem}

\begin{proof}
We prove this by induction on~$m$. If $m = 0$, the claim is that
$\Th{Q} \Proves \eq[(\num{n} + \Obj 0)][\num{n}]$. This follows by
axiom~$!Q_4$.  Now suppose the claim for $m$; let's prove the claim
for $m+1$, i.e., prove that $\Th{Q} \Proves \eq[(\num{n} +
  \num{m+1})][\num{n+m+1}]$. Note that $\num{m+1}$ is just $\num{m}'$,
and $\num{n+m+1}$ is just $\num{n+m}'$.  By axiom $!Q_5$, $\Th{Q}
\Proves \eq[(\num{n} + \num{m}')][(\num{n}+\num{m})']$. By induction
hypothesis, $\Th{Q} \Proves \eq[(\num{n} + \num{m})][\num{n+m}]$. So
$\Th{Q} \Proves \eq[(\num{n} + \num{m}')][\num{n+m}']$.
\end{proof}

\begin{proof}[Proof of \olref{prop:rep-add}]
The !!{formula}~$!A_\Add(x_0, x_1, y)$ representing $\Add$ is
$\eq[y][(x_0 + x_1)]$. First we show that if $\Add(n, m) = k$, then
$\Th{Q} \Proves !A_\Add(\num{n}, \num{m}, \num{k})$, i.e., $\Th{Q}
\Proves \eq[\num{k}][(\num{n} + \num{m})]$. But since $k = n + m$,
$\num{k}$ just is $\num{n+m}$, and we've shown in
\olref{lem:q-proves-add} that $\Th{Q} \Proves \eq[(\num{n} +
  \num{m})][\num{n+m}]$.

We also have to show that if $\Add(n, m) = k$, then $\Th{Q} \Proves
\lforall[y][(!A_\Add(\num{n}, \num{m}, y) \lif \eq[y][\num{k}])]$.
But if we have $\eq[(\num{n} + \num{m}][y]$, since $\Th{Q} \Proves
\eq[(\num{n}+\num{m})][\num{n+m}]$, we also have $\eq[\num{n+m}][y]$,
for arbitrary~$y$.
\end{proof}

\begin{prop}
\ollabel{prop:rep-mult}
The multiplication function $\Mult(x_0, x_1) = x_0 \cdot x_1$ is represented
in~$\Th{Q}$ by
\[
y = (x_0 \times x_1).
\]
\end{prop}

\begin{proof} Exercise. \end{proof}

\begin{lem}
\ollabel{lem:q-proves-mult}
$\Th{Q} \Proves \eq[(\num{n} \times \num{m})][\num{n \cdot m}]$
\end{lem}

\begin{proof} Exercise. \end{proof}

\begin{prob}
Prove \olref[inc][req][bre]{lem:q-proves-mult}.
\end{prob}

\begin{prob}
Use \olref[inc][req][bre]{lem:q-proves-mult} to prove
\olref[inc][req][bre]{prop:rep-mult}.
\end{prob}


\end{document}
