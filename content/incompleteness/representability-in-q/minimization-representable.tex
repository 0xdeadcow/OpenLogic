% Part: incompleteness
% Chapter: representability-in-q
% Section: minimization-representable

\documentclass[../../../include/open-logic-section]{subfiles}

\begin{document}

\olfileid{inc}{req}{min}
\olsection{Regular Minimization is Representable in $\Th{Q}$}

Let's consider unbounded search. Suppose $g(x, z)$ is regular and
representable in $\Th{Q}$, say by the !!{formula}~$!A_g(x, z, y)$. Let
$f$ be defined by $f(z) = \umin{x}{[g(x, z) = 0]}$. We would like to find
!!a{formula}~$!A_f(z, y)$ representing~$f$.  The value of~$f(z)$ is
that number~$x$ which (a) satisfies $g(x, z) = 0$ and (b) is the least
such, i.e., for any $w < x$, $g(w, z) \neq 0$.  So the following is a
natural choice:
\[
!A_f(z,y) \ident !A_g(y, z, 0) \land \lforall[w][(w < y \lif \lnot
  !A_g(w, z, 0))].
\]
In the general case, of course, we would have to replace $z$ with
$z_0$, \dots, $z_k$.

The proof, again, will involve some lemmas about things $\Th{Q}$ is
strong enough to prove.

\begin{lem}
\ollabel{lem:succ} For every !!{constant}~$a$ and every natural
number~$n$,
\[
\Th{Q} \Proves \eq[(a' + \num n)][(a + \num n)'].
\]
\end{lem}

\begin{proof}
The proof is, as usual, by induction on $n$. In the base case, $n =
0$, we need to show that $\Th{Q}$ proves $\eq[(a'+0)][(a+0)']$. But we have:
\begin{align}
  \Th{Q} & \Proves \eq[(a' + 0)][a'] \quad \text{by axiom $Q_4$}
  \ollabel{step1}\\
  \Th{Q} & \Proves  \eq[(a + 0)][a] \quad \text{by axiom $Q_4$}
  \ollabel{step2} \\
  \Th{Q} & \Proves \eq[(a + 0)'][a'] \quad \text{by \olref{step2}}
  \ollabel{step3} \\
  \Th{Q} & \Proves \eq[(a' + 0)][(a + 0)'] \quad
  \text{by \olref{step1} and \olref{step3}}\notag
\end{align}
In the induction step, we can assume that we have shown that $\Th{Q}
\Proves \eq[(a' + \num n)][(a + \num n)']$. Since
$\num{n+1}$ is $\num{n}'$, we need to show that $\Th{Q}$ proves
$\eq[(a' + \num{n}')][(a + \num{n}')']$. We have:
\begin{align}
  \Th{Q} & \Proves \eq[(a' + \num n')][(a' + \num n)'] \quad
  \text{by axiom $Q_5$} \ollabel{step5}\\
  \Th{Q} & \Proves \eq[(a' + \num n')][(a + \num n')'] \quad
  \text{inductive hypothesis} \ollabel{step6}\\
  \Th{Q} & \Proves \eq[(a' + \num n)'][(a + \num n')'] \quad
  \text{by \olref{step5} and \olref{step6}.} \notag
\end{align}
\end{proof}

It is again worth mentioning that this is weaker than saying that
$\Th{Q}$ proves $\lforall[x][\lforall[y][(x' + y) = (x + y)']]$.
Although this !!{sentence} is true in~$\Struct{N}$, $\Th{Q}$ does not
prove it.

\begin{lem}
\ollabel{lem:less}
\begin{enumerate}
\item $\Th{Q} \Proves \lforall[x][\lnot x < \Obj 0]$.
\item For every natural number~$n$, 
  \[
  \Th{Q} \Proves
  \lforall[x][(x < \num {n+1} \lif (\eq[x][\Obj 0] \lor \dots \lor
    \eq[x][\num n]))].
  \]
\end{enumerate}
\end{lem}

\begin{proof}
Let us do 1 and part of 2, informally (i.e., only giving hints as to
how to construct the formal derivation).

For part 1, by the definition of $<$, we need to prove $\lnot
\lexists[y][\eq[(y'+a)][\Obj 0]]$ in $\Th{Q}$, which is equivalent
(using the axioms and rules of first-order logic) to
$\lforall[b][\eq/[(y' + a)][0]]$. Here is the idea: suppose
$\eq[(y'+b)][\Obj 0]$. If $\eq[a][\Obj 0]$, we have $\eq[(y' + \Obj
  0)][\Obj 0]$. But by axiom~$Q_4$ of $\Th{Q}$, we have $\eq[(b' +
  \Obj 0)][b']$, and by axiom~$Q_2$ we have $\eq/[b'][\Obj 0]$, a
contradiction. So $\lforall[y][\eq/[(y' +a)][\Obj 0]]$. If
$\eq/[a][\Obj 0]$, by axiom~$Q_3$, there is a $c$ such that
$\eq[a][c']$. But then we have $\eq[(b' + c')][0]$. By axiom~$Q_5$, we
have $\eq[(b' + c)'][\Obj 0]$, again contradicting axiom~$Q_2$.

For part 2, use induction on~$n$. Let us consider the base case, when
$n = 0$. In that case, we need to show $a < \num 1 \lif \eq[a][\Obj
  0]$. Suppose $a < \num 1$. Then by the defining axiom for $<$, we
have $\lexists[y][\eq[(y'+a)][\Obj 0']]$.

Suppose $b$ has that property, i.e., we have $\eq[b'+a][\Obj 0']$. We
need to show $\eq[a][\Obj 0]$. By axiom~$Q_3$, if $\eq/[a][\Obj 0]$,
we get $\eq[a][c']$ for some~$z$. Then we have $\eq[(b' + c')][\Obj
  0']$. By axiom~$Q_5$ of $\Th{Q}$, we have $\eq[(b'+c)'][\Obj
  0']$. By axiom~$Q_1$, we have $\eq[(b' + c)][\Obj 0]$. But this
means, by definition, $z < \Obj 0$, contradicting part~1.
\end{proof}

\begin{lem}
  \ollabel{lem:trichotomy} For every $m \in \Nat$,
  \[
  \Th{Q} \Proves
  \lforall[y][((y < \num{m} \lor \num{m} < y) \lor \eq[y][\num{m}])].
  \]
\end{lem}

\begin{proof}
By induction on~$m$. First, consider the case $m=0$. $\Th{Q} \Proves
\lforall[y][(\eq/[y][\Obj 0] \lif \lexists[z][\eq[y][z']])]$ by~$!Q_3$.
But if $\eq[b][c']$, then $\eq[(c' + \Obj 0)][(b + \Obj 0)]$ by the
logic of~$\eq$. By $!Q_4$, $\eq[(b + \Obj 0)][b]$, so we have $\eq[(c'
  + \Obj 0)][b]$, and hence $\lexists[z][\eq[(z' + \Obj 0)][b]]$. By
the definition of $<$ in $!Q_8$, $\Obj 0 < b$.  If $\Obj 0 < b$, then also
$\Obj 0 < b \lor b < \Obj 0$. We obtain: $\eq/[b][\Obj 0] \lif (\Obj 0
< b \lor b < \Obj 0)$, which is equivalent to $(\Obj 0 < b \lor b <
\Obj 0) \lor \eq[b][\Obj 0]$.

Now suppose we have
\begin{align*}
  \Th{Q} & \Proves \lforall[y][((y < \num{m} \lor \num{m} < y) \lor
    \eq[y][\num{m}])]
  \intertext{and we want to show}
  \Th{Q} & \Proves \lforall[y][((y < \num{m+1} \lor \num{m+1} < y) \lor
    \eq[y][\num{m+1}])]
\end{align*}
The first disjunct $b < \num{m}$ is equivalent (by $!Q_8$) to
$\lexists[z][\eq[(z' + b)][\num{m}]]$. Suppose $c$ has this
property. If $\eq[(c' + b)][\num{m}]$, then also $\eq[(c' +
  b)'][\num{m}']$. By $Q_4$, $\eq[(c' + b)'][(c'' + b)]$. Hence,
$\eq[(c'' + b)][\num{m}']$. We get $\lexists[u][\eq[(u' +
    b)][\num{m+1}]]$ by existentially generalizing on~$c'$ and keeping
in mind that $\num{m}'$ is $\num{m+1}$. Hence, if $b < \num{m}$ then
$b < \num{m+1}$.

Now suppose $\num{m} < b$, i.e., $\lexists[z][\eq[(z' +
    \num{m})][b]]$.  Suppose $c$ is such a~$z$. By $Q_3$ and some
logic, we have $\eq[c][\Obj 0] \lor \lexists[u][\eq[c][u']]$.  If
$\eq[c][\Obj 0]$, we have $\eq[(\Obj 0' + \num{m})][b]$. Since $\Th{Q}
\Proves \eq[(\Obj 0'+\num{m})][\num{m+1}]$, we have
$\eq[b][\num{m+1}]$. Now suppose $\lexists[u][\eq[c][u']]$. Let $d$ be
such a~$u$. Then:
\begin{align*}
  b & \eq (c' + \num{m}) \quad \text{by assumption}\\
  (c' + \num{m}) & \eq (d'' + \num{m}) \quad \text{from $\eq[c][d']$}\\
  (d'' + \num{m}) & \eq (d' + \num{m})' \quad \text{by \olref{lem:succ}}\\
  (d' + \num{m})' & \eq (d' + \num{m}') \quad \text{by $Q_5$, so}\\
  b & = (d' + \num{m+1}) 
\end{align*}
By existential generalization, $\lexists[u][\eq[(u' + \num{m+1})][b]]$,
  i.e., $\num{m+1} < b$. So, if $\num{m} < b$, then $\num{m+1} < b
  \lor \eq[b][\num{m+1}]$.

Finally, assume $\eq[b][\num{m}]$. Then, since $\Th{Q} \Proves
\eq[(\Obj 0' + \num{m})][\num{m+1}]$, $\eq[(\Obj 0' +
  b)][\num{m+1}]$. From this we get $\lexists[z][\eq[(z' +
    b)][\num{m+1}]]$, or $b < \num{m+1}$.

Hence, from each disjunct of the case for~$m$, we can obtain the case
for~$m+1$.
\end{proof}
  
\begin{prop}
\ollabel{prop:rep-minimization}
If $!A_g(x, z, y)$ represents $g(x, y)$ in~$\Th{Q}$, then
\[
!A_f(z,y) \ident !A_g(y, z, \Obj 0) \land \lforall[w][(w < y \lif \lnot
  !A_g(w, z, \Obj 0))].
\]
represents $f(z) = \umin{x}{[g(x, z) = 0]}$.
\end{prop}

\begin{proof}
First we show that if $f(n) = m$, then $\Th{Q} \Proves !A_f(\num n, \num m)$,
i.e.,  
\begin{align}
  \Th{Q} & \Proves !A_g(\num{m}, \num{n}, \Obj 0) \land \lforall[w][(w <
    \num{m} \lif \lnot !A_g(w, \num{n}, \Obj 0))]. \notag
  \intertext{Since $!A_g(x, z, y)$ represents $g(x, z)$ and $g(m, n) =
    0$ if $f(n) = m$, we have}
\Th{Q} & \Proves !A_g(\num{m}, \num{n}, \Obj 0). \notag
\intertext{If $f(n) = m$, then for every $k < m$, $g(k, n) \neq 0$. So}
\Th{Q} & \Proves \lnot !A_g(\num{k}, \num{n}, \Obj 0). \notag
\intertext{We get that}
\Th{Q} & \Proves \lforall[w][(w < \num{m} \lif \lnot
  !A_g(w, \num{n}, \Obj 0))]. \ollabel{rep-less}
\end{align}
by \olref{lem:less} (by (1) in case $m = 0$ and by (2) otherwise).

Now let's show that if $f(n) = m$, then $\Th{Q} \Proves
\lforall[y][(!A_f(\num{n}, y) \lif \eq[y][\num{m}])]$.  We again
sketch the argument informally, leaving the formalization to the
reader.

Suppose $!A_f(\num{n}, b)$. From this we get (a) $!A_g(b, \num{n},
\Obj 0)$ and (b) $\lforall[w][(w < b \lif \lnot !A_g(w, \num{n}, \Obj
  0))]$.  By \olref{lem:trichotomy}, $(b < \num{m} \lor \num{m} < b)
\lor \eq[b][\num{m}]$. We'll show that both $b < \num{m}$ and $\num{m}
< b$ leads to a contradiction.

If $\num{m} < b$, then $\lnot !A_g(\num{m}, \num{n}, \Obj 0)$
from~(b). But $m = f(n)$, so $g(m, n) = 0$, and so $\Th{Q} \Proves
!A_g(\num{m}, \num{n}, \Obj 0)$ since $!A_g$ represents~$g$. So we
have a contradiction.

Now suppose $b < \num{m}$. Then since $\Th{Q} \Proves \lforall[w][(w <
  \num{m} \lif \lnot !A_g(w, \num{n}, \Obj 0))]$ by \olref{rep-less}, we
get $\lnot !A_g(b, \num{n}, \Obj 0)$. This again contradicts~(a).
\end{proof}

\end{document}
