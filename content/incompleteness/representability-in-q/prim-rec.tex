% Part: incompleteness
% Chapter: representability-in-q
% Section: prim-rec

\documentclass[../../../include/open-logic-section]{subfiles}

\begin{document}

\olfileid{inc}{req}{pri}
\olsection{Simulating Primitive Recursion}

Now we can show that definition by primitive recursion can be
``simulated'' by regular minimization using the beta function. Suppose
we have $f(\vec z)$ and $g(u,v,\vec z)$. Then the function~$h(x,\vec
z)$ defined from $f$ and~$g$ by primitive recursion is
\begin{eqnarray*}
h(0,\vec z) & = & f(\vec z) \\
h(x+1,\vec z) & = & g(x,h(x,\vec z),\vec z).
\end{eqnarray*}
We need to show that $h$ can be defined from $f$ and~$g$ using just
composition and regular minimization, using the basic functions and
functions defined from them using composition and regular minimization
(such as~$\beta$).

\begin{lem}
\ollabel{lem:prim-rec}
If $h$ can be defined from $f$ and $g$ using primitive recursion, it
can be defined from $f$, $g$, the functions $\Zero$, $\Succ$,
$\Proj{n}{i}$, $\Add$, $\Mult$, $\Char{=}$, using composition and
regular minimization.
\end{lem}

\begin{proof}
First, define an auxiliary function $\hat h(x,\vec z)$ which returns
the least number~$d$ such that $d$ codes a sequence which satisfies
\begin{enumerate}
\item $(d)_0 = f(\vec z)$, and
\item for each $i < x$, $(d)_{i+1} = g(i,(d)_i,\vec z)$,
\end{enumerate}
where now $(d)_i$ is short for $\beta(d,i)$. In other words, $\hat h$
returns the sequence $\tuple{h(0,\vec z), h(1,\vec z), \dots, h(x,\vec
  z)}$. We can write $\hat h$ as
\[
\hat h(x,z) = \umin{d}{(\beta(d,0) = f(\vec z) \land \lforall[i <
  x][\beta(d,i+1) = g(i,\beta(d,i),\vec z)])}.
\]
Note: no primitive recursion is needed here, just minimization. The
function we minimize is regular because of the beta function lemma
\olref[bet]{lem:beta}.

But now we have
\[
h(x,\vec z) = \beta(\hat h(x,\vec z),x),
\]
so $h$ can be defined from the basic functions using just composition
and regular minimization.
\end{proof}

\end{document}
