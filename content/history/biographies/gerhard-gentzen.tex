% Part: history
% Chapter: biographies
% Section: gerhard-gentzen

\documentclass[../../../include/open-logic-section]{subfiles}

\begin{document}

\olfileid{his}{bio}{gen}

\olsection{Gerhard Gentzen}

\olphoto{gerhard-gentzen}{Gerhard Gentzen}

Gerhard Gentzen is known primarily as the creator of structural proof
theory, and specifically the creation of the natural deduction and
sequent calclus proof systems. He was born on November 24, 1909 in
Greifswald, Germany. Gerhard was homeschooled for three years before
attending prepratory school, where he was behind most of his
classmates in terms of education. Despite this, he was a brilliant
student and showed a strong aptitude for mathematics. His interests
were varied, and he, for instance, also write poems for his mother and plays
for the school theatre.

Gentzen began his university studies at the University of Greifswald,
but moved around to G\"{o}ttingen, Munich, and Berlin. He recieved his
doctorate in 1933 from the University of G\"{o}ttingen under Hermann
Weyl.  (Paul Bernays supervised most of his work, but was dismissed
from the university by the Nazis.)  In 1934, Gentzen began work as an
assistant to David Hilbert. That same year he developed the sequent
calculus and natural deduction proof systems, in his papers
\emph{Untersuchungen \"{u}ber das logische Schlie\ss en I--II
  [Investigations Into Logical Deduction I--II]}. He proved the
consistency of the Peano axioms in 1936.

Gentzen's relationship with the Nazis is complicated.  At the same
time his mentor Bernays was forced to leave Germany, Gentzen joined
the university branch of the SA, the Nazi paramilitary
organization. Like many Germans, he was a member of the Nazi
party. During the war, he served as a telecommunications officer for
the air intelligence unit. However, in 1942 he was released from duty
due to a nervous breakdown. It is unclear whether or not Gentzen's
loyalties lay with the Nazi party, or whether he joined the party in
order to ensure academic success.

In 1943, Gentzen was offered an academic position at the Mathematical
Institute of the German University of Prague, which he
accepted. However, in 1945 the citizens of Prague revolted against
German occupation. Soviet forces arrived in the city and arrested all
the professors at the univeristy.  Because of his membership in Nazi
organizations, Gentzen was taken to a forced labour camp. He died of
malnutrition while in his cell on August 4, 1945 at the age of 35.

\begin{reading}
For a full biography of Gentzen, see \citet{Menzler-Trott2007}.  An
interesting read about mathematicians under Nazi rule, which gives a
brief note about Gentzen's life, is given by \citet{Segal2014}.
Gentzen's papers on logical deduction are available in the original
german \citep{Gentzen1935a,Gentzen1935b}.  English translations of
Gentzen's papers have been collected in a single volume by
\citet{Gentzen1969}, which also includes a biographical sketch.
\end{reading}

\end{document}
