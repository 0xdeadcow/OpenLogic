% Part: history 
% Chapter: biographies 
% Section: julia-robinson
\documentclass[../../../include/open-logic-section]{subfiles}

\begin{document}

\olfileid{his}{bio}{rob}

\olsection{Julia Robinson}

\olphoto{julia-robinson}{Julia Robinson}

Julia Robinson was an American mathematician. She is known mainly for her
work on decision problems, and most famously for her contributions to
Hilbert's tenth problem. Robinson was born in St. Louis, Missouri on
December 8, 1919. At a young age Robinson recalls being intrigued by
numbers \citep[4]{Reid1986}. At age nine Robinson contracted scarlet fever
and suffered from several recurrent bouts of rheumatic fever. This forced
her to spend much of her time in bed, putting her behind in her education.
Although she was able to catch up with the help of private tutors, the
physical effects of her illness would have a lasting impact on her life.

Despite her childhood struggles, Robinson graduated high school with
several awards in mathematics and the sciences. She started her university
career at San Diego State College, and transferred to the University of
California at Berkeley as a senior. She was highly influenced by
mathematician Raphael Robinson and they quickly became good friends. They
married in 1941. As she was married to a faculty member, Robinson was
barred from teaching in the mathematics department at Berkeley. Although
she continued to audit mathematics classes, she hoped to leave university
and start a family. Not long after her marriage Robinson contracted
pneumonia. She was told that there was substantial scar tissue build up on
her heart due to the rheumatic fever she suffered as a child. Due to the
severity of the scar tissue, the doctor predicted that she would not live
past forty and she was advised not to have children \citep[13]{Reid1986}.

Robinson was depressed for a long time, but decided to continue studying
mathematics. She returned to Berkeley and completed her PhD in 1948 under
the supervision of Alfred Tarski. It was during this time that Robsinson
became interested in decision problems, and she attempted to find a
solution Hilbert's tenth problem. This problem was one of 23 mathematical
problems posed by David Hilbert in 1900; the tenth problem asks whether
there is an algorithm that will answer, in a finite amount of time, whether
or not a Diophantine equation (a type of polynomial equation) has an answer
in integers. In the early 1950s Robinson proposed a hypothesis, later
called J.R., that would end up being a sufficient condition for the
unsolvability of the tenth problem. Robsinson continued to work on the
problem but was never able to give a complete solution. However, in 1970 a
young Russian mathematician named Yuri Matijasevich managed to prove the
J.R. hypothesis and used it to solve Hilbert's tenth problem in the
negative. Although Robinson wasn't able to solve the problem herself, her
work was essential to the solution. Matijasevich and Robinson became
friends collaborated on several papers. In a letter to Matijasevich,
Robinson once wrote that "actually I am very pleased that working together
(thousands of miles apart) we are obviously making more progress than
either one of us could alone" \citep[45]{Matijasevich1992}.

Robinson was the first female president of the American Mathematical
society, and the first woman to become a member of the National Academy of
Science. Although her career marked many achievements for women in
mathematics, she wished to be remembered only for her mathematical
contributions. Robinson passed away on July 30, 1985 at the age of 65 after
being diagnosed with Leukaemia.


\begin{reading} Robinson's collected works can be found in an edited
volume, see \citet{Robinson1996}. Robinson's older sister published an
Autobiography of Julia, based off of personal interviews. See
\citet{Reid1986}. A short documentary film about Robinson and Hilbert's
tenth problem was directed by George Csicsery. For more information on this
film and other mathematics documentaries, see \citet{Csicsery2016}. For a
brief memoir about Yuri Matijasevich's collaborations with Robsinon, and
her influence on his work, see \citet{Matijasevich1992}. \end{reading}

\end{document}
