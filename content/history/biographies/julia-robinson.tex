% Part: history 
% Chapter: biographies 
% Section: julia-robinson
\documentclass[../../../include/open-logic-section]{subfiles}

\begin{document}

\olfileid{his}{bio}{rob}

\olsection{Julia Robinson}

\olphoto{julia-robinson}{Julia Robinson}

Julia Robinson was born in St. Louis, Missouri on December 8, 1919. As a
child she had already developed a liking of the natural numbers (which she
would later describe as "the only real thing" \citep[xix]{Feferman1996}),
despite the fact that she was slow to talk. When she was nine she suffered
from scarlett fever, and several recurrent bouts of rheumatic fever. This
forced her to spend much of her time in bed, and she was forced to catch up
in school with the help of private tutors. The physical effects of the
fever would have a lasting impact on her life.

Despite her childhood struggles, Robinson graduated highschool with several
awards in the mathematics and sciences. She started her university career
at San Diego State College, and transferred to the Univeristy of California
at Berkeley as a senior. She was highly influenced by her instructors
Raphael Robinson and Alfred Tarski; she married the former in 1941. Being
married to a faculty member, Robinson was barred from teaching in the 
mathematics department at Berkeley. She instead wanted to start a family;
this dream was cut short when she contracted pneumonia, and found out
that there was substantial scar tissue build up on her heart. Due to the scar
tissue, the doctor predicted that she would not live past forty. She was advised
not to have children.

Robinson returned to mathematics at the urging of her husband. She completed
her PhD in 1948 under the supervision of Alfred Tarski. Robinson was the first 
female president of the American Mathematical society, and the first woman 
to become a member of the National Academy of Science. 
Although her career marked many achievements for women in
mathematics, she wished to be remembered only for her mathematical
contributions. Most famously, Robinson is known for aiding in a solution to
Hilbert's tenth problem. Although she was not able to come up with a solution
herself, she was extremely close - and her work inspired a young mathematician
to solve the problem. Throughout her career, Robinson focused on the areas of
algorithmic solvability and decision problems. Robinson passed way on July
30, 1985 at the age of 65 after being diagnosed with Lukemia.


\begin{reading} 
Robinson's works can be found in an edited volume, see
\citet{Robinson1996}. Robinson's older sister has published an
autobiography of Robinson, based off of personal interviews. See
\citet{Reid1986}. A short documentary film about Robinson and 
Hilbert's tenth problem was directed by George Csicsery. For more
information on this film and other mathematics documentaries, see
\citet{Csicsery2016}.
\end{reading}

\end{document}
