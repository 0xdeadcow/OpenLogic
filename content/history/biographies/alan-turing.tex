% Part: history
% Chapter: biographies 
% Section: alan-turing 

\documentclass[../../../include/open-logic-section]{subfiles}

\begin{document}

\olfileid{his}{bio}{tur} 

\olsection{Alan Turing}

Alan Turing was born in Mailda Vale, London, on June 23, 1912. He is
considered the father of theoretical computer science. Turing's
interest in the physical sciences and mathematics started at a young
age. However, as a boy his interests were not represented well in his
schools, where emphasis was placed on literature and
classics. Consequently, he did poorly in school and was reprimanded by
many of his teachers.

\olphoto{turing-alan}{Alan Turing}

Turing attended King's College, Cambridge as an undergraduate, where
he studied mathematics. In 1936 Turing developed (what is now called)
the Turing machine as an attempt to precisely define the notion of a
computable function and to prove the undecidability of the decision
problem. He was beaten to the result by Alonzo Church, who proved the
result via his own lambda calculus. Turing's paper was still published
with reference to Church's result. Church invited Turing to Princeton,
where he spent 1936--1938, and obtained a doctorate under Church.

Despite his interest in logic, Turing's earlier interests in physical
sciences remained prevalent. His practical skills were put to work
during his service with the British cryptanalytic department at
Bletchley Park during World War~II. Turing was a central figure in
cracking the cypher used by German Naval communications---the Enigma
code.  Turing's expertise in statistics and cryptography, together
with the introduction of electronic machinery, gave the team the
ability to crack the code by creating a de-crypting machine called a
``bombe.'' His ideas also helped in the creation of the world's first
programmable electronic computer, the Colossus, also used at Bletchley
park to break the German Lorenz cypher.

Turing was gay. Nevertheless, in 1942 he proposed to Joan Clarke, one
of his teammates at Bletchley Park, but later broke off the engagement
and confessed to her that he was homosexual. He had several lovers
throughout his lifetime, although homosexual acts were then criminal
offences in the UK. In 1952, Turing's house was burgled by a friend of
his lover at the time, and when filing a police report, Turing
admitted to having a homosexual relationship, under the impression
that the government was on their way to legalizing homosexual
acts. This was not true, and he was charged with gross
indecency. Instead of going to prison, Turing opted for a hormone
treatment that reduced libido.  Turing was found dead on June 8, 1954,
of a cyanide overdose---most likely suicide. He was given a royal
pardon by Queen Elizabeth~II in 2013.

\begin{reading}
For a comprehensive biography of Alan Turing, see \citet{Hodges2014}.
Turing's life and work inspired a play, \emph{Breaking the Code},
which was produced in 1996 for TV starring Derek Jacobi as
Turing. \emph{The Imitation Game}, an Academy Award nominated film
starring Bendict Cumberbatch and Kiera Knightley, is also loosely
based on Alan Turing's life and time at Bletchley Park
\citep{Imitation2014}.

\citet{Radiolab2012} has several podcasts on Turing's life and work.
BBC Horizon's documentary \emph{The Strange Life and Death of
  Dr.~Turing} is available to watch online \citep{Sykes1992}.
\citep{Theelen2012} is a short video of a working LEGO Turing
Machine---made to honour Turing's centenary in 2012.

Turing's original paper on Turing machines and the decision problem is
\citet{Turing1937}.
\end{reading}

\end{document}
