% Part: history
% Chapter: biographies 
% Section: alonzo-church

\documentclass[../../../include/open-logic-section]{subfiles}

\begin{document}

\olfileid{his}{bio}{chu}

\olsection{Alonzo Church}

\olphoto{church-alonzo}{Alonzo Church}

Alonzo Church was born in Washington, DC on June 14, 1903.  In early
childhood, an air gun incident left Church blind in one eye. He
finished preparatory school in Connecticut in 1920 and began his
university education at Princeton that same year. He completed his
doctoral studies in 1927. After a couple years abroad, Church returned
to Princeton. Church was known exceedingly polite and careful. His
blackboard writing was immaculate, and he would preserve important
papers by carefully covering them in Duco cement. Outside of his
academic pursuits, he enjoyed reading science fiction magazines and
was not afraid to write to the editors if he spotted any inaccuracies
in the writing.

Church's academic achievements were great.  Together with his students
Stephen Kleene and Barkley Rosser, he developed a theory of effective
calculability, the lambda calculus, independently of Alan Turing's
development of the Turing machine. The two definitions of
computability are equivalent, and give rise to what is now known as
the \emph{Church-Turing Thesis}, that a function of the natural
numbers is effectively computable if and only if it is computable via
Turing machine (or lambda calculus). He also proved what is now known
as \emph{Church's Theorem}: The decision problem for the validity of
first-order formulas is unsolvable.

Church continued his work into old age. In 1967 he left Princeton for
UCLA, where he was professor until his retirement in 1990. Church
passed away on August 1, 1995 at the age of 92.

\begin{reading} 
For a brief biography of Church, see \citet{EndertonND}.  Church's
original writings on the lambda calculus and the Entscheidungsproblem
(Church's Thesis) are \citet{Church1936,Church1936a}.
\citet{Aspray1984} records an interview with Church about the
Princeton mathematics community in the 1930s.
Church wrote a series of book reviews of the \emph{Journal of
Symbolic Logic} from 1936 until 1979. They are all archived on John
MacFarlane's website \citep{MacFarlane2015}.
\end{reading} 
\end{document}
