% Part: history
% Chapter: biographies
% Section: ernst-zermelo

\documentclass[../../../include/open-logic-section]{subfiles}

\begin{document}

\olfileid{his}{bio}{zer}

\olsection{Ernst Zermelo}

\olphoto{zermelo-ernst}{Ernst Zermelo}

Ernst Zermelo was born on July 27, 1871 in Berlin, Germany. He had
five sisters, though his family suffered from poor health and only
three made it to adulthood. His parents also passed away when he was
young, leaving him and his siblings orphans when he was seventeen.
Zermelo had a deep interest in the arts, and especially in poetry. He
was known for being sharp, witty, and critical. His most celebrated
mathematical achievements include the introduction of the axiom of
choice (in 1904), and his axiomatization of set theory (in 1908).

Zermelo's interests at university were varied. He took courses in
physics, mathematics, and philosophy. Under the supervision of Hermann
Schwarz, Zermelo completed his dissertation \emph{Investigations in
  the Calculus of Variations} in 1894 at the University of Berlin. In
1897, he decided to pursue more studies at the University of
G\"{o}ttigen, where he was heavily influenced by the foundational work
of David Hilbert. In 1899 he became eligible for professorship, but
did not get one until eleven years later---possibly due to his strange
demeanour and ``nervous haste.''

Zermelo finally received a paid professorship at the University of
Zurich in 1910, but was forced to retire in 1916 due to
tuberculosis. After his recovery, he was given an honourary
professorship at the University of Freiburg in 1921. During this time
he worked on foundational mathematics.  He became irritated with the
works of Thoralf Skolem and Kurt G\"{o}del, and publicly criticized
their approaches in his papers.  He was dismissed from his position at
Freiburg in 1935, due to his unpopularity and his opposition to
Hitler's rise to power in Germany.
 
The later years of Zermelo's life were marked by isolation. After his
dismissal in 1935, he abandoned mathematics. He moved to the country
where he lived modestly. He married in 1944, and became completely
dependent on his wife as he was going blind. Zermelo lost his sight
completely by 1951. He passed away in G\"{u}nterstal, Germany, on May
21, 1953.

\begin{reading}
For a full biography of Zermelo, see \citet{Ebbinghaus2015}.
Zermelo's seminal 1904 and 1908 papers are available to read in the
original German \citep{Zermelo1904,Zermelo1908}.  Zermelo's collected
works, including his writing on physics, are available in English
translation in \citet{Ebbinghaus2010,Ebbinghaus2013}.
\end{reading}

\end{document}
