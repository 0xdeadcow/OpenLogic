% Part: history
% Chapter: biographies
% Section: bertrand-russell

\documentclass[../../../include/open-logic-section]{subfiles}

\begin{document}

\olfileid{his}{bio}{rus} 

\olsection{Bertrand Russell}

\olphoto{russell-bertrand}{Bertrand Russell}

Bertrand Russell is hailed as one of the founders of modern analytic
philosophy. Born May 18, 1872, Russell was not only known for his work
in philosophy and logic, but wrote many popular books in various
subject areas. He was also an ardent political activist throughout his
life.

Russell was born in Trellech, Monmouthshire, Wales. His parents were
members of the British nobility. They were free-thinkers, and even
made friends with the radicals in Boston at the time.  Unfortunately,
Russell's parents died when he was young, and Russell was sent to live
with his grandparents. There, he was given a religious upbringing
(something his parents had wanted to avoid at all costs). His
grandmother was very strict in all matters of morality. During
adolescence he was mostly homeschooled by private tutors.

Russell's influence in analytic philosophy, and especially logic, is
tremendous. He studied mathematics and philosophy at Trinity College,
Cambridge, where he was influenced by the mathematician and
philosopher Alfred North Whitehead.  In 1910, Russell and Whitehead
published the first volume of \emph{Principia Mathematica}, where they
championed the view that mathematics is reducible to logic. He went on
to publish hundreds of books, essays and political pamphlets. In 1950,
he won the Nobel Prize for literature.

Russell's was deeply entrenched in politics and social
activism. During World War~I he was arrested and sent to prison for
six months due to pacifist activities and protest. While in prison, he
was able to write and read, and claims to have found the experience
``quite agreeable.''. He remained a pacifist throughout his life, and
was again incarcerated for attending a nuclear disarmament rally in
1961. He also survived a plane crash in 1948, where the only survivors
were those sitting in the smoking section. As such, Russell claimed
that he owed his life to smoking. Russell was married four times, but
had a reputation for carrying on extra-marital affairs.  He died on
February 2, 1970 at the age of 97 in Penrhyndeudraeth, Wales.

\begin{reading}
Russell wrote an autobiography in three parts, spanning his life from
1872--1967 \citep{Russell1967,Russell1968,Russell1969}.  The Bertrand
Russell Research Centre at McMaster University is home of the Bertrand
Russell archives. See their website at \citet{Duncan2015}, for
information on the volumes of his collected works (including
searchable indexes), and archival projects.  Russell's paper \emph{On
  Denoting} \citep{Russell1905} is a classic of 20th century analytic
philosophy.

The Stanford Encyclopedia of Philosophy entry on Russell
\citep{Irvine2015} has sound clips of Russell speaking on Desire and
Political theory. Many video interviews with Russell are available
online. To see him talk about smoking and being involved in a plane
crash, e.g., see \citet{RussellND}. Some of Russell's works, including
his \emph{Introduction to Mathematical Philosophy} are available as
free audiobooks on \citet{LibriVoxND}.
\end{reading}

\end{document}
