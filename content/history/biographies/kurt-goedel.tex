% Part: history
% Chapter: biographies 
% Section: kurt-goedel

\documentclass[../../../include/open-logic-section]{subfiles}

\begin{document}

\olfileid{his}{bio}{god}

\olsection{Kurt G\"odel}

\olphoto{kurt-goedel}{Kurt G{\"o}del}

Kurt G{\"o}del (\textsc{ger}-dle) was born on April 28, 1906 in
Br{\"u}nn in the Austro-Hungarian empire (now Brno in the Czech
Republic). Due to his inquisitive and bright nature, young Kurtele was
often called ``Der kleine Herr Warum'' (Little Mr.~Why) by his
family. He excelled in academics from primary school onward, where he
got less than the highest grade only in mathematics. G{\"o}del was
often absent from school due to poor health and was exempt from
physical education. G{\"o}del was diagnosed with rheumatic fever
during his childhood. Throughout his life, he believed this
permanently affected his heart despite medical assessment saying
otherwise.

G{\"o}del began studying at the University of Vienna in 1920 and
completed his doctoral studies in 1929. He first intended to study
physics, but his interests soon moved to mathematics and especially
logic, in part due to the influence of the philosopher Rudolf
Carnap. His dissertation, written under the supervision of Hans Hahn,
proved the completeness theorem of first-order predicate logic with
identity. Only a couple years later, his most famous results were
published---the first and second incompleteness
theorems~\citep{Godel1931}. During his time in Vienna, G{\"o}del was
also involved with the Vienna Circle, a group of scientifically-minded
philosophers.

In 1938, G\"odel married Adele Nimbursky. His parents were not
pleased: not only was she six years older than him and already
divorced, but she worked as a dancer in a nightclub. 
Social pressures did not affect G{\"o}del, however,
and they remained happily married until his death.

After Nazi Germany annexed Austria in 1938, G{\"o}del and Adele
emmigrated to the United States, where he took up a position at the
Institute for Advanced Study in Princeton, New Jersey. Despite his
introversion and eccentric nature, G{\"o}del's time at Princeton was
collaborative and fruitful.  He published essays in set theory,
philosophy and physics. Notably, he struck up a particularly strong
friendship with his colleague at the IAS, Albert Einstein.

In his later years, G{\"o}del's mental health deteriorated. His wife's
hospitalization in 1977 meant she was no longer able to cook his meals
for him. Succumbing to both paranoia and anorexia, and deathly afraid
of being poisoned, G{\"o}del refused to eat. He died of starvation on
January 14, 1978 in Princeton.

\begin{reading}
For a complete biography of G{\"o}del's life is available, see
\citet{Dawson1997}. For further biographical pieces, as well as essays
about G{\"o}del's contributions to logic and philosophy, see
\citet{Wang1990}, \citet{Baaz2011}, \citet{Takeuti2003}, and
\citet{Sigmund2007}.

G{\"o}del's PhD thesis is available in the original German
\citep{Godel1929}.  The original text of the incompleteness theorems
is \citep{Godel1931}. All of G\"odel's published and unpublished
writings, as well as a selection of correspondence, are available in
English in his \emph{Collected Papers} \citet{Godel1986,Godel1990}.

For a detailed treatment of G{\"o}del's incompleteness theorems, see
\citet{Smith2013}. For an informal, philosophical discussion of
G{\"o}del's theorems, see Mark Linsenmayer's podcast
\citep{Linsenmayer2014}.

The Kurt G{\"o}del society keeps G{\"o}del's memory alive by promoting
research in logic and other areas influenced by his works
\citet{Society2015}.
\end{reading}

\end{document}
