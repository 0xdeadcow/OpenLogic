% Part: history % Chapter: biographies % Section: rozsa-Peter
\documentclass[../../../include/open-logic-section]{subfiles}

\begin{document}

\olfileid{his}{bio}{pet} 

\olsection{R\'{o}zsa P\'{e}ter}

\olphoto{rozsa-Peter}{R\'{o}zsa P\'{e}ter}
 
R\'{o}zsa P\'{e}ter was born in Budapest, Hungary on February 17, 1905. She
is best known for her work on recursive functions, which was essential for
the creation of the field of recursion theory. P\'{e}ter was raised during
harsh political times, but was able to attend the affluent Maria Terezia
Girls' School in Budapest. She graduated in 1922 before attending
university.

P\'{e}ter attended P\'{a}zm\'{a}ny P\'{e}ter University (later renamed
Lor\'{a}nd E\"{o}tv\"{o}s University) in Budapest. She began studying
chemistry at the insistence of her father, but later switched to
mathematics. She graduated in 1927. Although she had the credentials to
teach mathematics, the economic situation at the time was dire as the Great
Depression affected the world economy. During this time, P\'{e}ter took odd
jobs as a tutor and private teacher of mathematics. She eventually returned
to university, taking up graduate studies at the University of Budapest.
She had originally planned work in number theory, but after finding out
that her results had already been proven, almost gave up on mathematics
altogether \citep{Oconnor2014}. She was encouraged to work on G\"{o}del's
incompleteness theorems, and unknowingly proved several of his results in
different ways \citep{Oconnor2014}. This restored her confidence, and
P\'{e}ter went on to write her first papers on recursion theory, inspired
by David Hilbert's foundational program. Her papers from this time are
credited as founding the field of recursive function theory. She achieved
her Ph.D. in 1935, and in 1937 she became an editor for the \emph{Journal of
Symbolic Logic}.

Despite the importance and influence of her work, P\'{e}ter did not gain a
full-time teaching position until 1945. During the Nazi occupation of
Hungary during World War II, P\'{e}ter was not allowed to teach due to
anti-Semitic laws. In 1944 the government created a Jewish ghetto in
Budapest; the ghetto was cut off from the rest of the city and attended by
armed guards. P\'{e}ter was forced to live in the ghetto until 1945 when it
was liberated. She went on to teach at the Budapest Teachers Training
College, and E\"{o}tv\"{o}s Lor\'{a}nd University. She was the first female
Hungarian mathematician to become an Academic Doctor of Mathematics.

P\'{e}ter was known as a passionate teacher of mathematics, who preferred
to explore the nature and beauty of mathematical problems with her students
rather than teach. As a result, she was affectionately called "Aunt Rosa"
by her students. P\'{e}ter passed away in 1977 at the age of 71.

\begin{reading} For more biographical reading, see \citet{Oconnor2014} and
\citet{Andrasfai1986}. See \citet{Tamassy1994} for a brief interview with
R\'{o}zsa. For a fun read about mathematics, see P\'{e}ter's book
\emph{Playing With Infinity} \citep{Peter2010}. \end{reading}

\end{document}
