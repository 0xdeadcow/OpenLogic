% Part: history
% Chapter: biographies
% Section: rozsa-peter

\documentclass[../../../include/open-logic-section]{subfiles}

\begin{document}

\olfileid{his}{bio}{pet} 

\olsection{R\'ozsa P\'eter}

\olphoto{peter-rozsa}{R\'ozsa P\'eter}
 
R\'ozsa P\'eter was born R\'osza Politzer, in Budapest, Hungary, on
February 17, 1905. She is best known for her work on recursive
functions, which was essential for the creation of the field of
recursion theory.

P\'eter was raised during harsh political times---WWI raged when she
was a teenager---but was able to attend the affluent Maria Terezia
Girls' School in Budapest, from where she graduated in 1922.  She then
studied at P\'azm\'any P\'eter University (later renamed Lor\'and
E\"otv\"os University) in Budapest. She began studying chemistry at
the insistence of her father, but later switched to mathematics, and
graduated in 1927. Although she had the credentials to teach high
school mathematics, the economic situation at the time was dire as the
Great Depression affected the world economy. During this time, P\'eter
took odd jobs as a tutor and private teacher of mathematics. She
eventually returned to university to take up graduate studies in
mathematics.  She had originally planned to work in number theory, but
after finding out that her results had already been proven, she almost
gave up on mathematics altogether. She was encouraged to work on
G\"odel's incompleteness theorems, and unknowingly proved several of
his results in different ways. This restored her confidence, and
P\'eter went on to write her first papers on recursion theory,
inspired by David Hilbert's foundational program. She received her PhD
in 1935, and in 1937 she became an editor for the \emph{Journal of
  Symbolic Logic}.

P\'eter's early papers are widely credited as founding contributions
to the field of recursive function theory. In \cite{Peter1935a}, she
investigated the relationship between different kinds of recursion.
In \cite{Peter1935b}, she showed that a certain recursively defined
function is not primitive recursive. This simplified an earlier result
due to Wilhelm Ackermann. P\'eter's simplified function is what's now
often called the Ackermann function---and sometimes, more properly,
the Ackermann-P\'eter function. She wrote the first book on recursive
function theory \citep{Peter1951}.

Despite the importance and influence of her work, P\'eter did not
obtain a full-time teaching position until 1945. During the Nazi
occupation of Hungary during World War II, P\'eter was not allowed to
teach due to anti-Semitic laws. In 1944 the government created a
Jewish ghetto in Budapest; the ghetto was cut off from the rest of the
city and attended by armed guards. P\'eter was forced to live in the
ghetto until 1945 when it was liberated. She then went on to teach at
the Budapest Teachers Training College, and from 1955 onward at E\"otv\"os
Lor\'and University. She was the first female Hungarian mathematician
to become an Academic Doctor of Mathematics, and the first woman to be
elected to the Hungarian Academy of Sciences.

P\'{e}ter was known as a passionate teacher of mathematics, who
preferred to explore the nature and beauty of mathematical problems
with her students rather than to merely lecture. As a result, she was
affectionately called ``Aunt Rosa'' by her students. P\'eter died
in 1977 at the age of~71.

\begin{reading}
For more biographical reading, see \citep{Oconnor2014} and
\citep{Andrasfai1986}. \citet{Tamassy1994} conducted a brief interview
with P\'eter. For a fun read about mathematics, see P\'eter's book
\emph{Playing With Infinity} \citep{Peter2010}.
\end{reading}

\end{document}
