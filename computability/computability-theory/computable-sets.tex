% Part: computability
% Chapter: computability-theory
% Section: computable-sets

\documentclass[../../include/open-logic-section]{subfiles}

\begin{document}

\olfileid{cmp}{thy}{cps}
\olsection{Computable Sets}

We can extend the notion of computability from computable functions to
computable sets:

\begin{defn}
  Let $S$ be a set of natural numbers. Then $S$ is \emph{computable}
  iff its characteristic function is. In other words, $S$ is
  computable iff the function
\[
\Char{S}(x) = 
\begin{cases}
1 & \text{if $x \in S$} \\
0 & \text{otherwise}
\end{cases}
\]
is computable. Similarly, a relation $\Atom{R}{x_0, \dots, x_{k-1}}$ is
computable if and only if its characteristic function is.
\end{defn}

\begin{explain}
Computable sets are also called \emph{decidable}.

Notice that we now have a number of notions of computability: for
partial functions, for functions, and for sets. Do not get them
confused!{} \iftag{TMs}{The Turing machine computing a partial function
  returns the output of the function, for input values at which the
  function is defined; the Turing machine computing a set returns
  either 1 or 0, after deciding whether or not the input value is in
  the set or not.}{}
\end{explain}

\end{document}
