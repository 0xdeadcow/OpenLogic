% Part: computability
% Chapter: computability-theory
% Section: normal-form

\documentclass[../../include/open-logic-section]{subfiles}

\begin{document}

\olfileid{cmp}{thy}{nfm}
\olsection{The Normal Form Theorem}

\begin{thm}[Kleene's Normal Form Theorem]
\ollabel{thm:normal-form} 
There are a primitive recursive relation $T(k, x, s)$ and a primitive
recursive function $U(s)$, with the following property: if $f$ is any
partial computable function, then for some~$k$,
\[
f(x) \simeq U(\umin{s}{T(k, x, s)})
\]
for every $x$.
\end{thm}

\begin{proof}[Proof Sketch]
For any model of computation one can rigorously define a description
of the computable function~$f$ and code such description using a
natural number~$k$.  One can also rigorously define a notion of
``computation sequence'' which records the process of computing the
function with index~$k$ for input~$x$. These computation sequences can
likewise be coded as numbers~$s$. This can be done in such a way that
(a) it is decidable whether a number $s$ codes the computation
sequence of the function with index~$k$ on input~$x$ and (b) what the
end result of the computation sequence coded by~$s$ is.  In fact, the
relation in (a) and the function in (b) are primitive recursive.
\end{proof}

\begin{explain}
In order to give a rigorous proof of the Normal Form Theorem, we would
have to fix a model of computation and carry out the coding of
descriptions of computable functions and of computation sequences in
detail, and verify that the relation~$T$ and function~$U$ are
primitive recursive.  For most applications, it suffices that $T$
and~$U$ are computable and that $U$~is total.

It is probably best to remember the proof of the normal form theorem
in slogan form: $\umin{s}{T(k, x, s)}$ searches for a computation
sequence of the function with index~$k$ on input $x$, and $U$ returns
the output of the computation sequence if one can be found.
\end{explain}

$T$ and $U$ can be used to define the enumeration $\cfind{0}$,
$\cfind{1}$, $\cfind{2}$, \dots.  From now on, we will assume that we
have fixed a suitable choice of $T$ and $U$, and take the equation
\[
\cfind{e}(x) \simeq U(\umin{s}{T(e,x,s)})
\]
to be the \emph{definition} of $\cfind{e}$.

Here is another useful fact:

\begin{thm}
Every partial computable function has infinitely many indices.
\end{thm}

Again, this is intuitively clear. Given any (description of) a
computable function, one can come up with a different description
which computes the same function (input-output pair) but does so,
e.g., by first doing something that has no effect on the computation
(say, test if $0 = 0$, or count to $5$, etc.).  The index of the
altered description will always be different from the original
index. Both are indices of the same function, just computed slightly
differently.

\end{document}
