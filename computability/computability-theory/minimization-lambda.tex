% Part: computability
% Chapter: computability-theory
% Section: minimization-lambda

\documentclass[../../include/open-logic-section]{subfiles}

\begin{document}

\olfileid{cmp}{thy}{mla}
\olsection{Minimization with Lambda Terms}

When it comes to the lambda calculus, we've shown the following:
\begin{enumerate}
\item Every primitive recursive function is represented by a lambda term.
\item There is a lambda term $Y$ such that for any lambda term~$G$,
  $YG \red G(YG)$.
\end{enumerate}
To show that every partial computable function is represented by some
lambda term, we only need to show the following.

\begin{lem}
Suppose $f(x,y)$ is primitive recursive. Let $g$ be defined by
\[
g(x) \simeq \umin{y}{f(x,y) = 0}.
\]
Then $g$ is represented by a lambda term.
\end{lem}

\begin{proof}
The idea is roughly as follows. Given $x$, we will use the
fixed-point lambda term $Y$ to define a function $h_x(n)$ which searches
for a $y$ starting at~$n$; then $g(x)$ is just
$h_x(0)$. The function $h_x$ can be expressed as the solution of a
fixed-point equation:
\[
h_x(n) \simeq
\begin{cases}
n & \text{if $f(x,n) = 0$} \\
h_x(n+1) & \text{otherwise.}
\end{cases}
\]

Here are the details. Since $f$ is primitive recursive, it is
represented by some term $F$. Remember that we also have a lambda
term~$D$ such that $D(M,N,\num{0}) \red M$ and $D(M,N,\num{1}) \red N$.
Fixing $x$ for the moment, to represent $h_x$ we want to find a term
$H$ (depending on~$x$) satisfying
\[
H(\num{n}) \equiv D(\num{n},H(S(\num{n})),F(x, \num{n})).
\]
We can do this using the fixed-point term~$Y$. First, let $U$ be the
term
\[
\lambd[h][\lambd[z][D(z, (h(Sz)), F(x,z))]],
\]
and then let $H$ be the term $YU$. Notice that the only free variable
in $H$ is $x$. Let us show that $H$ satisfies the equation above.

By the definition of $Y$, we have
\[
H = YU \equiv U(YU) = U(H).
\]
In particular, for each natural number~$n$, we have
\begin{align*}
H(\num{n}) & \equiv U(H,\num{n}) \\
& \red D(\num{n}, H(S(\num{n})), F(x,\num{n})),
\end{align*}
as required. Notice that if you substitute a numeral $\num{m}$ for $x$
in the last line, the expression reduces to $\num{n}$ if $F(\num{m},
\num{n})$ reduces to $\num{0}$, and it reduces to $H(S(\num{n}))$ if
$F(\num{m},\num{n})$ reduces to any other numeral.

To finish off the proof, let $G$ be $\lambd[x][H(\num{0})]$. Then $G$
represents~$g$; in other words, for every $m$, $G(\num{m})$ reduces to
reduces to $\num{g(m)}$, if $g(m)$ is defined, and has no
normal form otherwise.
\end{proof}

\end{document}
