% Part: computability
% Chapter: computability-theory
% Section: def-functions-self-reference

\documentclass[../../include/open-logic-section]{subfiles}

\begin{document}

\olfileid{cmp}{thy}{slf}
\olsection{Defining Functions using Self-Reference}

It is generally useful to be able to define functions in terms of
themselves. For example, given computable functions $k$, $l$, and~$m$,
the fixed-point lemma tells us that there is a partial computable
function~$f$ satisfying the following equation for every~$y$:
\[
f(y) \simeq
\begin{cases}
k(y) & \text{if $l(y) = 0$} \\
f(m(y)) & \text{otherwise.}
\end{cases}
\]
Again, more specifically, $f$~is obtained by letting
\[
g(x,y) \simeq
\begin{cases}
k(y) & \text{if $l(y) = 0$} \\
\cfind{x}(m(y)) & \text{otherwise}
\end{cases}
\]
and then using the fixed-point lemma to find an index~$e$ such that
$\cfind{e}(y) = g(e,y)$.

For a concrete example, the ``greatest common divisor'' function
$\fn{gcd}(u,v)$ can be defined by
\[
\fn{gcd}(u,v) \simeq
\begin{cases}
v & \text{if $0 = 0$} \\
\fn{gcd}(\fn{mod}(v, u), u) & \text{otherwise}
\end{cases}
\]
where $\fn{mod}(v, u)$ denotes the remainder of dividing $v$
by~$u$. An appeal to the fixed-point lemma shows that $\fn{gcd}$ is
partial computable. (In fact, this can be put in the format above,
letting $y$ code the pair $\tuple{u, v}$.) A subsequent induction
on~$u$ then shows that, in fact, $\fn{gcd}$ is total.

Of course, one can cook up self-referential definitions that are much
fancier than the examples just discussed. Most programming languages
support definitions of functions in terms of themselves, one way or
another. Note that this is a little bit less dramatic than being able
to define a function in terms of an {\em index} for an algorithm
computing the functions, which is what, in full generality, the
fixed-point theorem lets you do.

\end{document}
