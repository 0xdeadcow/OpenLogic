% Part: computability
% Chapter: recursive-functions
% Section: general-recursive-functions

\documentclass[../../include/open-logic-section]{subfiles}

\begin{document}

\olfileid{cmp}{rec}{gen}
\olsection{General Recursive Functions}

There is another way to obtain a set of total functions. Say a total
function $f(x,\vec z)$ is \emph{regular} if for every sequence of
natural numbers $\vec z$, there is an $x$ such that $f(x,\vec z) = 0$.
In other words, the regular functions are exactly those functions to
which one can apply unbounded search, and end up with a total
function. One can, conservatively, restrict unbounded search to
regular functions:

\begin{defn}
\ollabel{defn:general-recursive}
The set of \emph{general recursive functions} is the smallest set of
functions from the natural numbers to the natural numbers (of various
arities) containing zero, successor, and projections, and closed under
composition, primitive recursion, and unbounded search applied to
\emph{regular} functions.
\end{defn}

Clearly every general recursive function is total. The difference
between \olref{defn:general-recursive} and
\olref[par]{defn:recursive-fn} is that in the latter one is allowed to
use partial recursive functions along the way; the only requirement is
that the function you end up with at the end is total. So the word
``general,'' a historic relic, is a misnomer; on the surface,
\olref{defn:general-recursive} is \emph{less} general than
\olref[par]{defn:recursive-fn}. But, fortunately, the difference is
illusory; though the definitions are different, the set of general
recursive functions and the set of recursive functions are one and the
same.

\end{document}

