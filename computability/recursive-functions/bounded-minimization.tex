% Part: computability
% Chapter: recursive-functions
% Section: bounded-minimization

\documentclass[../../include/open-logic-section]{subfiles}

\begin{document}

\olfileid{cmp}{rec}{bmi}
\olsection{Bounded Minimization}


\begin{prop}
If $R(x, \vec z)$ is primitive recursive, so is the function $m_R(y,
\vec{z})$ which returns the least~$x$ less than $y$ such that
$R(x,\vec{z})$ holds, if there is one, and 0 otherwise.  We will write
the function~$m_R$ as
\[
\bmin{x < y}{R(x, \vec{z})},
\]
\end{prop}

\begin{proof}
Note than there can be no~$x < y$ such that $R(x, \vec{z})$ since
there is no $x < y$ at all.  So $m_R(x, 0) = 0$.

In case the bound is $y + 1$ we have three cases: (a) There is an $x <
y$ such that $R(x, \vec{z})$, in which case $m_R(y+1, \vec{z}) =
m_R(y, \vec{z})$. (b) There is no such~$x$ but $R(y, \vec{z})$ holds, then
$m_R(y+1, \vec{z}) = y$. (c) There is no $x < y+1$ such that $R(x,
\vec{z})$, then $m_R(y+1,\vec{z}) = 0$. So,
\begin{align*}
m_R(0, \vec{z}) & = 0\\
m_R(y+1, \vec{z}) & = 
\begin{cases}
m_R(y, \vec{z}) & \text{if $\lexists[x < y][R(x, \vec{z})]$} \\
y & \text{otherwise, provided $R(y, \vec{z})$}\\
0 & \text{otherwise.}
\end{cases}
\end{align*}
\end{proof}

\begin{explain}
The choice of ``$0$ otherwise'' is somewhat arbitrary. It is in fact
even easier to recursively define the function~$m'_R$ which returns
the least $x$ less than $y$ such that $R(x,\vec z)$ holds, and $y+1$
otherwise.  When we use $\fn{min}$, however, we will always know that
the least $x$ such that $R(x, \vec z)$ exists and is less
than~$y$. Thus, in practice, we will not have to worry about the
possibility that if $\bmin{x<y}{R(x, \vec{z})} = 0$ we do not know if that
value indicates that $R(0, \vec z)$ or that for no $x < y$, $R(x, \vec
z)$. As with bounded quantification, $\bmin{x \leq y}{\dots}$
can be understood as $\bmin{x < y + 1}{\dots}$.
\end{explain}

All this provides us with a good deal of machinery to show that
natural functions and relations are primitive recursive. For example,
the following are all primitive recursive:
\begin{enumerate}
\item The relation ``$x$ divides $y$'', written $x \mid y$, defined by
\[
x \mid y \defiff \lexists[z \leq y][(x \cdot z) = y].
\]

\item The relation $\fn{Prime}(x)$, which holds iff $x$ is prime,
  defined by
\[
\fn{Prime}(x) \defiff (x \geq 2 \land \lforall[y \leq x][(y \mid x \lif y
  = 1 \lor y = x))].
\]
\item The function $\fn{nextPrime(x)}$, which returns the first prime
  number larger than $x$, defined by
\[
  \fn{nextPrime(x)} = 
  \bmin{y \leq x\fac+1}{(y > x \land \fn{Prime}(y))}
\]
Here we are relying on Euclid's proof of the fact that there is always
a prime number between $x$ and $x\fac+1$.

\item The function $p(x)$, returning the $x$th prime, defined by $p(0)
  = 2, p(x+1) = \fn{nextPrime}(p(x))$. For convenience we will write
  this as $p_x$ (starting with 0; i.e.\ $p_0=2$).
\end{enumerate}

\begin{prob}
Define integer division $d(x, y)$ using bounded minimization.
\end{prob}

\end{document}
