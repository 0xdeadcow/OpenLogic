% Part: computability
% Chapter: recursive-functions
% Section: examples

\documentclass[../../include/open-logic-section]{subfiles}

\begin{document}

\olfileid{cmp}{rec}{exa}
\olsection{Examples of Primitive Recursive Functions}


Here are some examples of primitive recursive functions:
\begin{enumerate}
\item Constants: for each natural number $n$, $n$ is a 0-ary primitive
  recursive function, since it is equal to $S(S(\dots S(0)))$.

\item The identity function: $\fn{id}(x) = x$, i.e.\ $\Proj{1}{0}$

\item Addition, $x+y$

\item Multiplication, $x \cdot y$

\item Exponentiation, $x^y$ (with $0^0$ defined to be $1$)

\item Factorial, $x\fac$

\item The predecessor function, $\fn{pred}(x)$, defined by
\[
\fn{pred}(0) = 0, \quad \fn{pred}(x+1) = x
\]

\item Truncated subtraction, $x \tsub y$, defined by 
\[
x \tsub 0 = x, \quad x \tsub (y+1) = \fn{pred}(x \tsub y)
\]

\item Maximum, $\fn{max}(x,y)$, defined by 
\[
\fn{max}(x,y) \defis x + (y \tsub x)
\]

\item Minimum, $\fn{min}(x,y)$

\item Distance between $x$ and $y$, $\left|x-y\right|$
\end{enumerate}

The set of primitive recursive functions is further closed under the
following two operations:
\begin{enumerate}
\item Finite sums: if $f(x,\vec z)$ is primitive recursive, then so is the
function 
\[
g(y,\vec z) \defis \Sigma_{x = 0}^y f(x,\vec z).
\]
\item Finite products: if $f(x,\vec z)$ is primitive recursive, then so is the
function 
\[
h(y,\vec z) \defis \Pi_{x = 0}^y f(x,\vec z).
\]
\end{enumerate}
For example, finite sums are defined recursively by the equations
\[
g(0,\vec z) = f(0,\vec z), \quad g(y+1,\vec z) = g(y,\vec z) +
f(y+1,\vec z).
\]
We can also define boolean operations, where $1$ stands for true, and
$0$ for false:
\begin{enumerate}
\item Negation, $\fn{not}(x) \defis 1 \tsub x$
\item Conjunction, $\fn{and}(x,y) \defis x \cdot y$
\end{enumerate}
Other classical boolean operations like $\fn{or}(x,y)$ and
$\fn{ifthen}(x,y)$ can be defined from these in the usual way.

\begin{defn}
A relation $R(\vec x)$ is said to be primitive recursive if its characteristic
function,
\[
\Char{R}(\vec x) = \left\{
  \begin{array}{ll}
  1 & \mbox{if $R(\vec x)$} \\
  0 & \mbox{otherwise} 
  \end{array}
\right.
\]
is primitive recursive. 
\end{defn}

In other words, when one speaks of a primitive
recursive relation $R(\vec x)$, one is referring to a relation of the
form $\Char{R}(\vec x) = 1$, where $\Char{R}$ is a primitive recursive
function which, on any input, returns either 1 or 0. For example, the
relation $\fn{Zero}(x)$, which holds if and only if $x = 0$,
corresponds to the function $\Char{\fn{Zero}}$, defined using primitive
recursion by 
\[
\Char{\fn{Zero}}(0) = 1, \quad \Char{\fn{Zero}}(x+1) = 0.
\]

It should be clear that one can compose relations with other primitive
recursive functions. So the following are also primitive recursive:
\begin{enumerate}
\item The equality relation, $x = y$, defined by $\fn{Zero}(\left|x -
  y\right|)$
\item The less-than relation, $x \leq y$, defined by $\fn{Zero}(x
  \tsub y)$
\end{enumerate}
Furthermore, the set of primitive recursive relations is closed under
boolean operations:
\begin{enumerate}
\item Negation, $\lnot P$
\item Conjunction, $P \land Q$
\item Disjunction, $P \lor Q$
\item Implication $P \lif Q$
\end{enumerate}
are all primitive recursive, if $P$ and $Q$ are.

One can also define relations using bounded quantification:
\begin{enumerate}
\item Bounded universal quantification: if $R(x, \vec z)$ is a
primitive recursive relation, then so is the relation 
\[
\lforall[x < y][R(x, \vec z)]
\]
which holds if and only if $R(x,\vec z)$ holds for every $x$ less than
$y$.
\item Bounded existential quantification: if $R(x, \vec z)$ is a
primitive recursive relation, then so is 
\[
\lexists[x < y][R(x, \vec z)].
\]
\end{enumerate}
By convention, we take expressions of the form $\lforall[x < 0][R(x,\vec
z)]$ to be true (for the trivial reason that there \emph{are} no $x$
less than $0$) and $\lexists[x < 0][R(x,\vec z)]$ to be false. A universal
quantifier functions just like a finite product; it can also be
defined directly by
\[
g(0,\vec z) = 1, \quad g(y+1,\vec z) = \Char{\fn{and}}(g(y,\vec
z),\Char{R}(y,\vec z)).
\]
Bounded existential quantification can similarly be defined using
$\fn{or}$. Alternatively, it can be defined from bounded universal
quantification, using the equivalence, $\lexists[x < y][!A(x)] \liff \lnot
\lforall[x < y][\lnot !A(x)]$. Note that, for example, a bounded quantifier
of the form $\lexists[x \leq y]$ is equivalent to $\lexists[x < y+1]$.

Another useful primitive recursive function is:
\begin{enumerate}
\item The conditional function, $\fn{cond}(x,y,z)$, defined by
\[
\fn{cond}(x,y,z) = \left\{ \begin{array}{ll}
  y & \mbox{if $x$ = 0} \\
  z & \mbox{otherwise}
\end{array}\right.
\]
\end{enumerate}
This is defined recursively by
\[
\fn{cond}(0,y,z) = y, \quad \fn{cond}(x+1,y,z) = z.
\]
One can use this to justify:
\begin{enumerate}
\item Definition by cases: if $g_0(\vec x), \dots, g_m(\vec x)$ are
functions, and $R_1(\vec x), \dots, R_{m-1}(\vec x)$ are relations, then
the function $f$ defined by
\[
f(\vec x) = \left\{\begin{array}{ll}
    g_0(\vec x) & \mbox{if $R_0(\vec{x})$} \\
    g_1(\vec x) & \mbox{if $R_1(\vec{x})$ and not $R_0(\vec{x})$} \\
    \vdots & \\
    g_{m-1}(\vec x) & \mbox{if $R_{m-1}(\vec{x})$ and none of the
      previous hold}
    \\
    g_m(\vec x) & \mbox{otherwise}
\end{array}\right.
\]
is also primitive recursive.
\end{enumerate}
When $m = 1$, this is just the the function defined by 
\[
f(\vec x) = \fn{cond}(\Char{\lnot R_0}(\vec x),g_0(\vec x),g_1(\vec
 x)).
\]
For $m$ greater than $1$, one can just compose definitions of this
form. 

We will also make good use of bounded minimization:
\begin{enumerate}
\item Bounded minimization: if $R(x, \vec z)$ is primitive recursive,
  so is the function $f(y, \vec z)$, written 
\[
\fn{min} \; x < y \; R(x, \vec z),
\]
which returns the least $x$ less than $y$ such that $R(x,\vec z)$
holds, if there is one, and 0 otherwise.
\end{enumerate}

\begin{explain}
The choice of ``$0$ otherwise'' is somewhat arbitrary. It is easier to
recursively define a function that returns the least $x$ less than $y$
such that $R(x,\vec z)$ holds, and $y$ otherwise, and then define
$\fn{min}$ from that. As with bounded quantification, $\fn{min} \; x
\leq y \; \dots$ can be understood as $\fn{min} \; x < y + 1 \;
\dots$.
\end{explain}

All this provides us with a good deal of machinery to show that
natural functions and relations are primitive recursive. For example,
the following are all primitive recursive:
\begin{enumerate}
\item The relation ``$x$ divides $y$'', written $x \mid y$, defined by
\[
x \mid y \defiff \lexists[z \leq y][(x \cdot z) = y].
\]

\item The relation $\fn{Prime}(x)$, which holds iff $x$ is prime,
  defined by
\[
\fn{Prime}(x) \defiff (x \geq 2 \land \lforall[y \leq x][(y \mid x \lif y
  = 1 \lor y = x))].
\]
\item The function $\fn{nextPrime(x)}$, which returns the first prime
  number larger than $x$, defined by
\[
  \fn{nextPrime(x)} = \fn{min} \; {y \leq x\fac+1} \\ (y > x \land \fn{Prime}(y))
\]
Here we are relying on Euclid's proof of the fact that there is always
a prime number between $x$ and $x\fac+1$.

\item The function $p(x)$, returning the $x$th prime, defined by $p(0)
  = 2, p(x+1) = \fn{nextPrime}(p(x))$. For convenience we will write
  this as $p_x$ (starting with 0; i.e.\ $p_0=2$).
\end{enumerate}

\end{document}
