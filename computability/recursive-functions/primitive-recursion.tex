% Part: computability
% Chapter: recursive-functions
% Section: primitive-recursion

\documentclass[../../include/open-logic-section]{subfiles}

\begin{document}

\olfileid{cmp}{rec}{pre}
\olsection{Primitive Recursion}

\begin{explain}
Suppose we specify that a certain function $l$ from $\Nat$ to $\Nat$
satisfies the following two clauses:
\begin{eqnarray*}
l(0) & = & 1 \\
l(x+1) & = & 2 \cdot l(x).
\end{eqnarray*}
It is pretty clear that there is only one function, $l$, that meets
these two criteria. This is an instance of a \emph{definition by
  primitive recursion}. We can define even more fundamental functions
like addition and multiplication by
\begin{eqnarray*}
f(x,0) & = & x \\
f(x,y+1) & = & f(x,y)+1
\end{eqnarray*}
and
\begin{eqnarray*}
g(x,0) & = & 0 \\
g(x,y+1) & = & f(g(x,y),x).
\end{eqnarray*}
Exponentiation can also be defined recursively, by
\begin{eqnarray*}
h(x,0) & = & 1 \\
h(x,y+1) & = & g(h(x,y),x).
\end{eqnarray*}
We can also compose functions to build more complex ones; for example,
\begin{eqnarray*}
k(x) & = & x^x + (x + 3) \cdot x \\
& = & f(h(x,x),g(f(x,3),x)).
\end{eqnarray*}

Remember that the \emph{arity} of a function is the number of
arguments. For convenience, we will consider a constant, like 7, to be
a 0-ary function. (Send it zero arguments, and it returns 7.)  The set
of \emph{primitive recursive functions} is the set of functions from
$\Nat$ to $\Nat$ that you get if you start with $0$ and the successor
function, $S(x) = x+1$, and iterate the two operations above,
primitive recursion and composition. The idea is that primitive
recursive functions are defined in a very straightforward and explicit
way, so that it is intuitively clear that each one can be computed
using finite means.
\end{explain}

\begin{defn}
If $f$ is a $k$-ary function and $g_0$, \dots,~$g_{k-1}$ are $l$-ary
functions on the natural numbers, the \emph{composition} of $f$ with
$g_0$, \dots,~$g_{k-1}$ is the $l$-ary function $h$ defined by
\[
h(x_0,\dots,x_{l-1}) =
f(g_0(x_0,\dots,x_{l-1}),\dots,g_{k-1}(x_0,\dots,x_{l-1})) .
\]
\end{defn}

\begin{defn}
If $f(z_0,\dots,z_{k-1})$ is a $k$-ary function and
$g(x,y,z_0,\dots,z_{k-1})$ is a $k+2$-ary function, then the function
defined by \emph{primitive recursion from $f$ and $g$} is the
$k+1$-ary function $h$, defined by the equations
\begin{eqnarray*}
h(0,z_0,\dots,z_{k-1}) & = & f(z_0,\dots,z_{k-1}) \\
h(x+1,z_0,\dots,z_{k-1}) & = & g(x,h(x,z_0,\dots,z_{k-1}),z_0,\dots,z_{k-1})
\end{eqnarray*}
\end{defn}

In addition to the constant, 0, and the successor function, $S(x)$, we
will include among primitive recursive functions the projection
functions,
\[
\Proj{n}{i}(x_0,\dots,x_{n-1}) = x_i,
\]
for each natural number $n$ and $i < n$. In the end, we have the
following:

\begin{defn}
  The set of primitive recursive functions is the set of functions of
  various arities from the set of natural numbers to the set of
  natural numbers, defined inductively by the following clauses:
\begin{enumerate}
\item The constant, $0$, is primitive recursive.
\item The successor function, $S$, is primitive recursive.
\item Each projection function $\Proj{n}{i}$ is primitive recursive.
\item If $f$ is a $k$-ary primitive recursive function and
  $g_0$, \dots,~$g_{k-1}$ are $l$-ary primitive recursive functions, then
  the composition of $f$ with $g_0$, \dots,~$g_{k-1}$ is primitive
  recursive.
\item If $f$ is a $k$-ary primitive recursive function and $g$ is a
  $k+2$-ary primitive recursive function, then the function defined by
  primitive recursion from $f$ and $g$ is primitive recursive.
\end{enumerate}
\end{defn}

\begin{explain}
Put more concisely, the set of primitive recursive functions is the
smallest set containing the constant 0, the successor function,
and projection functions, and closed under composition and primitive
recursion. 

Another way of describing the set of primitive recursive functions
keeps track of the ``stage'' at which a function enters the set. Let
$S_0$ denote the set of starting functions: zero, successor, and the
projections. Once $S_i$ has been defined, let $S_{i+1}$ be the set of
all functions you get by applying a single instance of composition or
primitive recursion to functions in $S_i$. Then
\[
S = \bigcup_{i \in \Nat} S_i
\]
is the set of primitive recursive functions
\end{explain}

Our definition of composition may seem too rigid, since
$g_0$, \dots,~$g_{k-1}$ are all required to have the same arity, $l$. But
adding the projection functions provides the desired flexibility. For
example, suppose $f$ and $g$ are ternary functions and $h$ is the
binary function defined by
\[
h(x,y) = f(x,g(x,x,y),y).
\]
Then the definition of $h$ can be rewritten with the projection
functions, as
\[
h(x,y) = f(\Proj{2}{0}(x,y), g(\Proj{2}{0}(x,y), \Proj{2}{0}(x,y),
\Proj{2}{1}(x,y)), \Proj{2}{1}(x,y)).
\]
Then $h$ is the composition of $f$ with $\Proj{2}{0},l,\Proj{2}{1}$, where
\[
l(x,y) = g(\Proj{2}{0}(x,y),\Proj{2}{0}(x,y),\Proj{2}{1}(x,y)),
\]
i.e., $l$ is the composition of $g$ with $\Proj{2}{0},\Proj{2}{0},\Proj{2}{1}$.

For another example, let us consider one of the informal examples
given above, namely, addition. This is
described recursively by the following two equations:
\begin{eqnarray*}
x + 0 & = & x \\
x + (y+1) & = & S(x+y).
\end{eqnarray*}
In other words, addition is the function $g$ defined recursively by
the equations
\begin{eqnarray*}
g(0,x) & = & x \\
g(y+1,x) & = & S(g(y,x)).
\end{eqnarray*}
But even this is not a strict primitive recursive definition; we need
to put it in the form
\begin{eqnarray*}
g(0,x) & = & k(x) \\
g(y+1,x) & = & h(y,g(y,x),x)
\end{eqnarray*}
for some 1-ary primitive recursive function $k$ and some 3-ary
primitive recursive function $h$. We can take $k$ to be $\Proj{1}{0}$, and
we can define $h$ using composition,
\[
h(y,w,x) = S(\Proj{3}{1}(y,w,x)).
\]
The function $h$, being the composition of basic primitive recursive
functions, is primitive recursive; and hence so is $g$. (Note that,
strictly speaking, we have defined the function $g(y,x)$ meeting the
recursive specification of $x + y$; in other words, the variables are
in a different order. Luckily, addition is commutative, so here the
difference is not important; otherwise, we could define the function
$g'$ by
\[
g'(x,y) = g(\Proj{2}{1}(y,x)),\Proj{2}{0}(y,x)) = g(y,x),
\]
using composition.)


\begin{explain}
One advantage to having the precise description of the primitive
recursive functions is that we can be systematic in describing them.
For example, we can assign a ``notation'' to each such function, as
follows. Use symbols $0$, $S$, and $\Proj{n}{i}$ for zero, successor, and
the projections. Now suppose $f$ is defined by composition from a
$k$-ary function $h$ and $l$-ary functions $g_0$, \dots,~$g_{k-1}$, and
we have assigned notations $H,G_0$, \dots,~$G_{k-1}$ to the latter
functions. Then, using a new symbol $\fn{Comp}_{k,l}$, we can denote
the function $f$ by $\fn{Comp}_{k,l}[H,G_0,\dots,G_{k-1}]$. For the
functions defined by primitive recursion, we can use analogous
notations of the form $\fn{Rec}_k[G,H]$, where $k$ denotes that arity
of the function being defined. With this setup, we can denote the
addition function by
\[
\fn{Rec}_2[\Proj{1}{0},\fn{Comp}_{1,3}[S,\Proj{3}{1}]].
\]
Having these notations sometimes proves useful.
\end{explain}

\end{document}
