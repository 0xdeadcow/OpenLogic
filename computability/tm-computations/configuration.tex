% Part: computability
% Chapter: tm-computations
% Section: turing-machines

\documentclass[../../include/open-logic-section]{subfiles}

\begin{document}

\olfileid{cmp}{tur}{con}
\olsection{Configurations and Computations}

\begin{wordy}
The imaginary mechanism consisting of tape, read/write head, and
Turing machine program is really just in intuitive way of visualizing
what a Turing machine computation is.  Formally, we can define the
computation of a Turing machine on a given input as a sequence of
\emph{configurations}---and a configuration in turn is a sequence of
symbols (corresponding to the contents of the tape at a given point in
the compuation), a number indicating the position of the read/write
head, and a state.  Using these, we can define what the Turing machine
$M$ computes on a given input.
\end{wordy}

\begin{defn}
A emph{configuration} of Turing machine $M = \langle Q, \Sigma, s,
I\rangle$ is a triple $\langle C, n, q\rangle$ where
\begin{enumerate}
\item $C \in \Sigma^*$ is a finite sequence of symbols from $\Sigma$,
\item $n \in \Nat$ is a number $\le \len{C}$, and
\item $q \in Q$ 
\end{enumerate}
\end{defn}

\begin{wordy}
The potential input for a Turing machine is a sequence of symbols,
usually a sequence that encodes a number in some form.  The initial
configuration of the Turing machne is that configuration in which we
start the Turing machine to work on that input: the tape contains the
tape end marker immediately followed by the input written on the
squares to the right, the read/write head is scanning the leftmost
square of the tape (i.e., the left end marker), and the mechanism is
in the designated start state~$s$.
\end{wordy}

\begin{defn}
The \emph{initial configuration} of $M$ for input $I \in \Sigma^*$ is
\[
\langle \TMendtape \frown I, 0, s\rangle
\]
\end{defn}

\begin{defn}
We say that a configuration $\langle C, n, q\rangle$ \emph{yields
  $\langle C', n', q'\rangle$ in one step} (according to $M$), iff
\begin{enumerate}
\item the $n$-th symbol of $C$ is $\sigma$,
\item the instruction set of $M$ contains a tuple $\langle q, \sigma,
  q', \sigma', D\rangle$, 
\item the $n$-th symbol of $C'$ is $sigma'$,
\item 
\begin{enumerate}
\item $D = L$ and $n' = n -1$, or
\item $D = R$ and $n' = n$, or
\item $D = N$ and $n' = n$, 
\end{enumerate}
\item for all $i \neq n$, $C'(i) = C(i)$,
\item if $n' > \len{C}$, then $\len{C'} = \len{C} + 1$ and the $n'$-th
  symbol of $C'$ is $\TMblank$.
\end{enumerate}
\end{defn}

\begin{defn}
A \emph{run of $M$ on input~$I$} is a sequence $C_i$ of configurations
of $M$, where $C_0$ is the initial confiuration of $M$ for input $I$,
and each $C_i$ yields $C_{i+1}$ in one step.  

We say that $M$ \emph{halts on input $I$ after $k$ steps} if $C_k =
\langle \TMendtape \frown O, n, h\rangle$. In that case the
\emph{output} of $M$ for input $I$ is $O$.
\end{defn}



\end{document}
