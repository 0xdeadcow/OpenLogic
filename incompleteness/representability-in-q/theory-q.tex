% Part: incompleteness
% Chapter: representability-in-q
% Section: theory-q

\documentclass[../../include/open-logic-section]{subfiles}

\begin{document}

\olfileid{inc}{req}{thq}
\olsection{The Theory $\Th{Q}$}

The language of $\Th{Q}$ is the language of
arithmetic; $\Th{Q}$ consists of the following axioms
(to be used in conjunction with the other axioms and rules of
first-order logic with !!{identity}):
\begin{enumerate}
\item $x' = y' \lif x = y$
\item $\eq/[0][x']$
\item $\eq/[x][0] \lif \lexists[y][\eq[x][y']]$
\item $x+0 = x$
\item $x+ y' = (x+y)'$
\item $x \times 0 = 0$
\item $x \times y' = x \times y + x$
\item $x < y \liff \lexists[z][(z' + x) = y]$
\end{enumerate}
For each natural number $n$, define the numeral $\bar n$ to be the
term $0^{''\ldots'}$ where there are $n$ tick marks in all.

As a theory of arithmetic, $Q$ is \emph{extremely} weak; for example,
you can't even prove very simple facts like $\lforall[x][\eq/[x][x']]$
or $\lforall[x][\lforall[y][(x+ y) = (y+x)]]$.  But we will see that
much of the reason that $Q$ is so interesting is \emph{because} it is
so weak, in fact, just barely strong enough for the incompleteness
theorem to hold; and also because it has a \emph{finite} set of
axioms.

\end{document}

