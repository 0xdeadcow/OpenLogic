% Part: incompleteness
% Chapter: representability-in-q
% Section: prim-rec

\documentclass[../../include/open-logic-section]{subfiles}

\begin{document}

\olfileid{inc}{req}{pri}
\olsection{Primitive Recursion in $C$}

Now we can show that $C$ is closed under primitive recursion. Suppose
$f(\vec z)$ and $g(u,v,\vec z)$ are both in $C$. Let $h(x,\vec z)$ be
the function defined by
\begin{eqnarray*}
h(0,\vec z) & = & f(\vec z) \\
h(x+1,\vec z) & = & g(x,h(x,\vec z),\vec z).
\end{eqnarray*}
We need to show that $h$ is in $C$.

First, define an auxiliary function $\hat h(x,\vec z)$ which returns
the least number $d$ such that $d$ codes a sequence satisfying
\begin{enumerate}
\item $(d)_0 = f(\vec z)$, and
\item for each $i < x$, $(d)_{i+1} = g(i,(d)_i,\vec z)$,
\end{enumerate}
where now $(d)_i$ is short for $\beta(d,i)$. In other words, $\hat h$
returns a sequence that begins $\langle h(0,\vec z), h(1,\vec z), \dots,
h(x,\vec z)\rangle$. $\hat h$ is in $C$, because we can write it as
\[
\hat h(x,z) = \mu d \; (\beta(d,0) = f(\vec z) \land \lforall[i <
  x][\beta(d,i+1) = g(i,\beta(d,i),\vec z)]).
\]
But then we have
\[
h(x,\vec z) = \beta(\hat h(x,\vec z),x),
\]
so $h$ is in $C$ as well.

\end{document}
