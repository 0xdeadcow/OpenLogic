% Part: incompleteness
% Chapter: representability-in-q
% Section: c-representable

\documentclass[../../include/open-logic-section]{subfiles}

\begin{document}

\olfileid{inc}{req}{cre}
\olsection{Functions in $C$ are Representable in $\Th{Q}$}

We have to show that every function in $C$ is representable in
$\Th{Q}$.  In the end, we need to show how to assign to each $k$-ary
function $f(x_0,\dots,x_{k-1})$ in $C$ a formula
$!A_f(x_0,\dots,x_{k-1},y)$ that represents it.

\begin{lem}
Given natural numbers $n$ and $m$, if $n \neq m$, then $Q \Proves \num
n \neq \num m$.
\end{lem}

\begin{proof}
Use induction on $n$ to show that for every $m$, if $n \neq m$, then
$Q \Proves \eq/[\num n][\num m]$.

In the base case, $n = 0$. If $m$ is not equal to $0$, then $m = k +
1$ for some natural number $k$. We have an axiom that says
$\lforall[x][\eq/[0][x']]$. By a quantifier axiom, replacing $x$ by
  $\num k$, we can conclude $\eq/[0][\num k']$. But $\num k'$ is just
  $\num m$.

In the induction step, we can assume the claim is true for $n$, and
consider $n+1$. Let $m$ be any natural number. There are two
possibilities: either $m = 0$ or for some $k$ we have $m = k+1$.  The
first case is handled as above. In the second case, suppose $n+1 \neq
k+1$. Then $n \neq k$. By the induction hypothesis for $n$ we have
$\Th{Q} \Proves \eq/[\num n][\num k]$. We have an axiom that says
$\lforall[x][\lforall[y][\eq[x'][y'] \lif \eq[x][y]]]$. Using a
quantifier axiom, we have $\eq[\num n'][\num k'] \lif \eq[\num n][\num
  k]$. Using propositional logic, we can conclude, in $\Th{Q}$,
$\eq/[\num n][\num k] \lif \eq/[\num n'][\num k']$.  Using modus
ponens, we can conclude $\eq/[\num n'][\num k']$, which is what we want,
since $\num k'$ is $\num m$.
\end{proof}

Note that the lemma does not say much: in essence it says that $\Th{Q}$ can
prove that different numerals denote different objects. For example,
$\Th{Q}$ proves $0'' \neq 0'''$. But showing that this holds in general
requires some care. Note also that although we are using induction, it
is induction \emph{outside} of $\Th{Q}$.

We will be able to represent zero, successor, plus, times, the
characteristic function for equality, and projections. In each case,
the appropriate representing function is entirely straightforward; for
example, zero is represented by the formula
\[
y = 0,
\]
successor is represented by the formula
\[
x_0' = y,
\]
and plus is represented by the formula
\[
x_0 + x_1 = y.
\]
The work involves showing that $\Th{Q}$ can prove the relevant statements;
for example, saying that plus is represented by the formula above
involves showing that for every pair of natural numbers $m$ and $n$,
$\Th{Q}$ proves
\[
\eq[\num n + \num m][\num {n+m}]
\]
and
\[
\lforall[y][(\eq[(\num n + \num m)][y] \lif \eq[y][\num {n+m}])].
\]

What about composition? Suppose $h$ is defined by
\[
h(x_0,\dots,x_{l-1}) = f(g_0(x_0,\dots,x_{l-1}), \dots,
g_{k-1}(x_0,\dots,x_{l-1})).
\]
where we have already found formulas $!A_f, !A_{g_0}, \dots,
!A_{g_{k-1}}$ representing the functions $f,g_0,\dots,g_{k-1}$,
respectively. Then we can define a formula $!A_h$ representing $h$,
by defining $!A_h(x_0,\dots,x_{l-1},y)$ to be
\begin{multline*}
  \lexists[z_0,\dots][\lexists[z_{k-1}][(!A_{g_0}(x_0,\dots,x_{l-1},z_0) \land
  \dots \land !A_{g_{k-1}}(x_0,\dots,x_{l-1},z_{k-1}) \land]]\\
  !A_f(z_0,\dots,z_{k-1},y)).
\end{multline*}

Finally, let us consider unbounded search. Suppose $g(x,\vec z)$ is
regular and representable in $\Th{Q}$, say by the formula $!A_g(x,\vec
z,y)$. Let $f$ be defined by $f(\vec z) = \mu x \; g(x,\vec z)$. We would
like to find a formula $!A_f(\vec z,y)$ representing $f$. Here is
a natural choice:
\[
!A_f(\vec z,y) \ident !A_g(y,\vec z,0) \land \lforall[w][(w < y
\lif \lnot !A_g(w,\vec z,0))].
\]

It can be shown that this works using some additional lemmas, e.g.,

\begin{lem}
  For every variable $x$ and every natural number $n$, $\Th{Q}$ proves $\eq[(x'
  + \num n)][(x + \num n)']$.
\end{lem}

It is again worth mentioning that this is weaker than saying that $\Th{Q}$
proves $\lforall[x][\lforall[y][(x' + y) = (x +y)']]$ (which is false). 

\begin{proof}
The proof is, as usual, by induction on $n$. In the base case, $n =
0$, we need to show that $\Th{Q}$ proves $\eq[(x'+0)][(x+0)']$. But we have:
\begin{eqnarray*}
& & \eq[(x' + 0)][x'] \quad \text{from axiom 4} \\
& & \eq[(x + 0)][x] \quad \text{from axiom 4} \\
& & \eq[(x+0)'][x'] \quad \text{by line 2} \\
& & \eq[(x' + 0)][(x + 0)'] \quad \text{lines 1 and 3}
\end{eqnarray*}
In the induction step, we can assume that we have derived $\eq[(x' +
  \num n)][(x + \num n)']$ in $\Th{Q}$. Since $\num{n+1}$ is $\num
n'$, we need to show that $\Th{Q}$ proves $\eq[(x' + \num n')][(x + \num
n')']$. The following chain of equalities can be derived in $\Th{Q}$:
\begin{eqnarray*}
(x' + \num n') & \eq & (x' + \num n)' \quad \text{axiom 5} \\
& \eq & (x + \num n')' \quad \text{from the inductive hypothesis}
\end{eqnarray*}
\end{proof}

\begin{lem}
\begin{enumerate}
\item $\Th{Q}$ proves $\lnot (x < \num 0)$.
\item For every natural number $n$, $\Th{Q}$ proves
\[
x < \num {n+1} \lif (\eq[x][\num 0] \lor \dots \lor \eq[x][\num n]).
\]
\end{enumerate}
\end{lem}

\begin{proof} 
Let us do 1 and part of 2, informally (i.e., only giving hints as to
how to construct the formal derivation).

For part 1, by the definition of $<$, we need to prove $\lnot
\lexists[y][\eq[(y'+x)][0]]$ in $\Th{Q}$, which is equivalent (using
the axioms and rules of first-order logic) to $\lforall[y][\eq/[(y' +
    x)][0]]$. Here is the idea: suppose $\eq[(y'+x)][0]$. If $x$ is 0,
we have $\eq[(y' + 0)][0]$. But by axiom 4 of $\Th{Q}$, we have
$\eq[(y' + 0)][y']$, and by axiom 2 we have $\eq/[y'][0]$, a
contradiction. So $\lforall[y][\eq/[(y' +x)][0]]$. If $x$ is not $0$,
by axiom 3 there is a $z$ such that $\eq[x][z']$. But then we have
$\eq[(y' + z')][0]$. By axiom~5, we have $\eq[(y' + z)'][0]$, again
contradicting axiom~2.

For part 2, use induction on $n$. Let us consider the base case, when
$n =0$. In that case, we need to show $x < \num 1 \lif \eq[x][\num
  0]$. Suppose $x < \num 1$. Then by the defining axiom for $<$, we
have $\lexists[y][\eq[(y'+x)][0']]$. Suppose $y$ has that property; so
we have $\eq[y'+x][0']$.

We need to show $\eq[x][0]$. By axiom 3, if $x$ is not 0, it is equal
to $z'$ for some $z$. Then we have $\eq[(y' + z')][0']$. By axiom~5 of
  $\Th{Q}$, we have $\eq[(y'+z)'][0']$. By axiom 1, we have $\eq[(y' +
    z)][0]$. But this means, by definition, $z < 0$, contradicting
  part~1.
\end{proof}

\begin{explain}
We have shown that the set of computable functions can be
characterized as the set of functions representable in $\Th{Q}$. In fact,
the proof is more general. From the definition of representability, it
is not hard to see that any theory extending $\Th{Q}$ (or in which one can
interpret $\Th{Q}$) can represent the computable functions; but,
conversely, in any proof system in which the notion of proof is
computable, every representable function is computable. So, for
example, the set of computable functions can be characterized as the
set of functions represented in Peano arithmetic, or even Zermelo
Fraenkel set theory. As G\"odel noted, this is somewhat surprising.
We will see that when it comes to provability, questions are very
sensitive to which theory you consider; roughly, the stronger the
axioms, the more you can prove. But across a wide range of axiomatic
theories, the representable functions are exactly the computable ones.
\end{explain}

\end{document}
