% Part: incompleteness
% Chapter: representability-in-q
% Section: representable-comp

\documentclass[../../include/open-logic-section]{subfiles}

\begin{document}

\olfileid{inc}{req}{rpc}
\olsection{Functions Representable in $\Th{Q}$ are Computable}

\begin{lem}
Every function that is representable in $\Th{Q}$ is computable.
\end{lem}

\begin{proof}
All we need to know is that we can code terms, formulas, and proofs in
such a way that the relation ``$d$ is a proof of $!A$ in the theory
$\Th{Q}$'' is computable, as well as the function
$\fn{SubNumeral}(!A,n,v)$ which returns (a numerical code of) the
result of substituting the numeral corresponding to $n$ for the
variable (coded by) $v$ in the formula (coded by) $!A$. Assuming this,
suppose the function $f$ is represented by
$!A_f(x_0,\dots,x_{k},y)$. Then the algorithm for computing $f$ is as
follows: on input $n_0,\dots,n_{k}$, search for a number $m$ and a
proof of the formula $!A_f(\bar n_0,\dots,\bar n_{k},\bar m)$; when
you find one, output $m$. In other words,
\[
f(n_0,\dots,n_{k}) = (\mu s (\mbox{``$(s)_0$ is a proof of $!A(\bar
  n_0,\dots,\bar n_{k},\bar{(s)_1})$ in $\Th{Q}$''}))_1.
\]
This completes the proof, modulo the (involved but routine) details of
coding and defining the function and relation above.
\end{proof}

\end{document}
