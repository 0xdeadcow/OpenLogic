% Part: incompleteness
% Chapter: representability-in-q
% Section: introduction

\documentclass[../../include/open-logic-section]{subfiles}

\begin{document}

\olfileid{inc}{req}{int}
\olsection{Introduction}

We will describe a very minimal such theory called ``$\Th{Q}$'' (or,
sometimes, ``Robinson's $Q$,'' after Raphael Robinson). We will say
what it means for a function to be \emph{representable} in $\Th{Q}$, and
then we will prove the following:
\begin{quote}
  A function is representable in $\Th{Q}$ if and only if it is computable.
\end{quote}
For one thing, this provides us with another model of
computability. But we will also use it to show that the set
$\Setabs{!A}{\Th{Q} \Proves !A}$ is not decidable, by reducing the
halting problem to it. By the time we are done, we will have proved
much stronger things than this.

The language of $\Th{Q}$ is the language of
arithmetic; $\Th{Q}$ consists of the following axioms
(to be used in conjunction with the other axioms and rules of
first-order logic with !!{identity}):
\begin{enumerate}
\item $\lforall[x][\lforall[y][x' = y' \lif x = y]]$
\item $\lforall[x][\eq/[0][x']]$
\item $\lforall[x][\eq/[x][0] \lif \lexists[y][\eq[x][y']]]$
\item $\lforall[x][(x+0) = x]$
\item $\lforall[x][\lforall[y][(x+ y') = (x+y)']]$
\item $\lforall[x][(x \times 0) = 0]$
\item $\lforall[x][\lforall[y][(x \times y') = ((x \times y) + x)]]$
\item $\lforall[x][\lforall[y][x < y \liff \lexists[z][(z' + x) = y]]]$
\end{enumerate}
For each natural number $n$, define the numeral $\bar n$ to be the
term $0^{''\ldots'}$ where there are $n$ tick marks in all.

As a theory of arithmetic, $Q$ is \emph{extremely} weak; for example,
you can't even prove very simple facts like $\lforall[x][\eq/[x][x']]$
or $\lforall[x][\lforall[y][(x+ y) = (y+x)]]$. But we will see that
much of the reason that $Q$ is so interesting is \emph{because} it is
so weak, in fact, just barely strong enough for the incompleteness
theorem to hold; and also because it has a \emph{finite} set of
axioms.

A stronger theory than $\Th{Q}$ called \emph{Peano arithmetic} $\Th{PA}$,
is obtained by adding a schema of induction to~$\Th{Q}$:
\[
(!A(0) \land \lforall[x][(!A(x) \lif !A(x'))]) \lif \lforall[x][!A(x)]
\]
where $!A(x)$ is any formula, possibly with free variables other than
$x$. Using induction, one can do much better; in fact, it takes a
good deal of work to find ``natural'' statements about the natural
numbers that can't be proved in Peano arithmetic!

\begin{defn}
\ollabel{defn:representable-fn}
  A function $f(x_0,\ldots,x_k)$ from the natural numbers to
  the natural numbers is said to be {\em representable in $\Th{Q}$} if
  there is a formula $!A_f(x_0,\dots,x_k,y)$ such that whenever
  $f(n_0,\dots,n_k) = m$, $\Th{Q}$ proves
\begin{enumerate}
\item $!A_f(\bar n_0, \dots, \bar n_k,\bar m)$
\item $\lforall[y][(!A_f(\bar n_0, \dots, \bar n_k,y) \lif \bar m = y)]$.
\end{enumerate}
\end{defn}

There are other ways of stating the definition; for example, we could
equivalently require that $\Th{Q}$ proves $\lforall[y][(!A_f(\bar n_0, \dots,
\bar n_k,y) \liff \bar m = y)]$.

\begin{thm}
\ollabel{thm:representable-iff-comp}
A function is representable in $\Th{Q}$ if and only if it is computable.
\end{thm}

There are two directions to proving the theorem. One of them is fairly
straightforward once arithmetization of syntax is in place. The other
direction requires more work.

\end{document}
