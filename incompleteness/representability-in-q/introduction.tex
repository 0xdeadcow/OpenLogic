% Part: incompleteness
% Chapter: representability-in-q
% Section: introduction

\documentclass[../../include/open-logic-section]{subfiles}

\begin{document}

\olfileid{inc}{req}{int}
\olsection{Introduction}

We will describe a very minimal such theory called ``$\Th{Q}$'' (or,
sometimes, ``Robinson's $Q$,'' after Raphael Robinson). We will say
what it means for a function to be \emph{representable} in $\Th{Q}$, and
then we will prove the following:
\begin{quote}
  A function is representable in $\Th{Q}$ if and only if it is computable.
\end{quote}
For one thing, this provides us with another model of
computability. But we will also use it to show that the set
$\Setabs{!A}{\Th{Q} \Proves !A}$ is not decidable, by reducing the
halting problem to it. By the time we are done, we will have proved
much stronger things than this.

\end{document}
