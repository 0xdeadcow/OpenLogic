% Part: incompleteness
% Chapter: incompleteness-provability
% Section: second-incompleteness-thm

\documentclass[../../include/open-logic-section]{subfiles}

\begin{document}

\olfileid{inc}{inp}{2in}

\olsection{The Second Incompleteness Theorem}

How can we express the assertion that $\Th{PA}$ doesn't prove its own
consistency? Saying $\Th{PA}$ is inconsistent amounts to saying that
$\Th{PA}$ proves $\eq[0][1]$. So we can take $\OCon[PA]$ to be the
formula $\lnot \OProv[PA](\gn{\eq[0][1]})$, and then the following
theorem does the job:

\begin{thm}
\ollabel{thm:second-incompleteness} 
Assuming $\Th{PA}$ is consistent, then $\Th{PA}$ does not prove
$\OCon[PA]$.
\end{thm}

It is important to note that the theorem depends on the particular
representation of $\OCon[PA]$ (i.e., the particular
representation of $\OProv[PA](y)$). All we will use is that the
representation of $\OProv[PA](y)$ has the three properties above, so the
theorem generalizes to any theory with a provability predicate having
these properties.

It is informative to read G\"odel's sketch of an argument, since the
theorem follows like a good punch line. It goes like this. Let
$!G_\Th{PA}$ be the G\"odel sentence that we constructed in the proof
of \olref[1in]{thm:first-incompleteness}. We have shown ``If $\Th{PA}$
is consistent, then $\Th{PA}$ does not prove $!G_\Th{PA}$.'' If we
formalize this {\em in} $\Th{PA}$, we have a proof of
\[
\OCon[PA] \lif \lnot \OProv[PA](\gn{!G_\Th{PA}}).
\]
Now suppose $\Th{PA}$ proves $\OCon[PA]$. Then it proves $\lnot
\Prov[PA](\gn{!G_\Th{PA}})$. But since $!G_\Th{PA}$ is a G\"odel
sentence, this is equivalent to $!G_\Th{PA}$. So $\Th{PA}$ proves
$!G_\Th{PA}$.

But: we know that if $\Th{PA}$ is consistent, it doesn't prove
$!G_\Th{PA}$!{}  So if $\Th{PA}$ is consistent, it can't prove
$\OCon[PA]$.

To make the argument more precise, we will let $!G_\Th{PA}$ be the
G\"odel sentence for~$\Th{PA}$ and use properties 1--3 above to show
that $\Th{PA}$ proves $\OCon[PA] \lif !G_\Th{PA}$. This will show that
$\Th{PA}$ doesn't prove $\OCon[PA]$. Here is a sketch of the proof,
in~$\Th{PA}$:
\begin{align*}
& !G_\Th{PA} \lif \lnot \OProv[PA](\gn{!G_\Th{PA}}) &\ & 
  \text{since $!G_\Th{PA}$ is a G\"odel  sentence} \\
& \OProv[PA](\gn{!G_\Th{PA} \lif \lnot \OProv[PA](\gn{!G_\Th{PA}})}) && 
   \text{by 1} \\
& \OProv[PA](\gn{!G_\Th{PA}}) \lif \\
& \qquad \OProv[PA](\gn{\lnot \OProv[PA](\gn{!G_\Th{PA}})}) &&
   \text{by 2} \\
& \OProv[PA](\gn{!G_\Th{PA}}) \lif \\
& \qquad \OProv[PA](\gn{\OProv[PA](\gn{!G_\Th{PA}}) \lif 0 = 1}) &&
   \text{by 1 and 2} \\
& \OProv[PA](\gn{!G_\Th{PA}}) \lif \\
& \qquad \OProv[PA](\gn{\OProv[PA](\gn{!G_\Th{PA}})}) &&
   \text{by 3} \\
& \OProv[PA](\gn{!G_\Th{PA}}) \lif \OProv[PA](\gn{0=1}) &&
   \text{using 1 and 2} \\
& \OCon[PA] \lif \lnot \OProv[PA](\gn{!G_\Th{PA}}) &&
   \text{by contraposition} \\
& \OCon[PA] \lif !G_\Th{PA} &&
   \text{since $!G_\Th{PA}$ is a G\"odel sentence}
\end{align*}
The move from the third to the fourth line uses the fact that $\lnot
\OProv[PA](\gn{!G_\Th{PA}})$ is equivalent to
$\OProv[PA](\gn{!G_\Th{PA}}) \lif 0 = 1$ in $\Th{PA}$. The more
abstract version of the incompleteness theorem is as follows:

\begin{thm}
  \ollabel{thm:second-incompleteness-gen}
  Let $\Th{T}$ be any theory extending $\Th{Q}$ and let
  $\OProv[T](y)$ be any formula satisfying 1--3 for $\Th{T}$. Then
  if $\Th{T}$ is consistent, then $\Th{T}$ does not prove~$\OCon[T]$.
\end{thm}

\begin{prob}
Show that $\Th{PA}$ proves $!G_{\Th{PA}} \lif \OCon[PA]$.
\end{prob}

\begin{digress}
The moral of the story is that no ``reasonable'' consistent theory for
mathematics can prove its own consistency. Suppose $\Th{T}$ is a
theory of mathematics that includes $\Th{Q}$ and Hilbert's
``finitary'' reasoning (whatever that may be). Then, the whole of
$\Th{T}$ cannot prove the consistency of $\Th{T}$, and so, a fortiori,
the finitary fragment can't prove the consistency of $\Th{T}$
either. In that sense, there cannot be a finitary consistency proof
for ``all of mathematics.''

There is some leeway in interpreting the term finitary, and G\"odel, in
the 1931 paper, grants the possibility that something we may consider
``finitary'' may lie outside the kinds of mathematics Hilbert wanted
to formalize. But G\"odel was being charitable; today, it is hard to
see how we might find something that can reasonably be called finitary
but is not formalizable in, say, $\Th{ZFC}$.
\end{digress}

\end{document}
