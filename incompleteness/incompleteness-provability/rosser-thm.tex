% Part: incompleteness
% Chapter: incompleteness-provability
% Section: rosser-thm

\documentclass[../../include/open-logic-section]{subfiles}

\begin{document}

\olfileid{inc}{inp}{ros}

\olsection{Rosser's Theorem}

Can we modify G\"odel's proof to get a stronger result, replacing
``$\omega$-consistent'' with simply ``consistent''? The answer is
``yes,'' using a trick discovered by Rosser. Let $\fn{not}(x)$ be the
primitive recursive function which does the following: if $x$ is the
code of a formula $!A$, $\fn{not}(x)$ is a code of $\lnot !A$.  To
simplify matters, assume $\Th{T}$ has a function symbol $\Obj{not}$
such that for any formula $!A$, $\Th{T}$ proves $\Obj{not}(\gn{!A}) =
\gn{\lnot !A}$. This is not a major assumption; since $\fn{not}(x)$ is
computable, it is represented in $\Th{T}$ by some formula
$!D_{\fn{not}}(x,y)$, and we could eliminate the reference to the
function symbol in the same way that we avoided using a function
symbol $\Obj{diag}$ in the proof of the fixed-point lemma.

Rosser's trick is to use a ``modified'' provability predicate
$\ORProv_T(y)$, defined to be
\[
\lexists[x][(\OPrf[T](x,y) \land \lforall[z][(z < x \lif \lnot
  \OPrf[T](z,{\Obj{not}}(y)))])].
\]
Roughly, $\ORProv[T](y)$ says ``there is a proof of $y$ in $\Th{T}$,
and there is no shorter proof of the negation of $y$.'' (You might
find it convenient to read $\ORProv[T](y)$ as ``$y$ is shmovable.'')
Assuming $\Th{T}$ is consistent, $\ORProv[T](y)$ is true of the same
numbers as $\OProv[T](y)$; but from the point of view of {\em
  provability} in $\Th{T}$ (and we now know that there is a difference
between truth and provability!) the two have different properties.

By the fixed-point lemma, there is a formula $!R_\Th{T}$ such that $\Th{T}$
proves
\[
!R_\Th{T} \liff \lnot \ORProv[T](\gn{!R_\Th{T}}).
\]
In contrast to the proof above, here we claim that if $\Th{T}$ is
consistent, $\Th{T}$ doesn't prove $!R_\Th{T}$, and $\Th{T}$ also
doesn't prove $\lnot !R_\Th{T}$. (In other words, we don't need the
assumption of $\omega$-consistency.)


\end{document}

% Following digression should only be included if
% incompleteness/theories-computability/first-incompleteness.tex also
% compiled.

\begin{digress}
% This comment doesn't make much sense, what are you trying to get at?
By comparison to the proof of
\olref[inc][tcp][inc]{thm:first-incompleteness}, the proofs of
\olref[inc][inp][1in]{thm:first-incompleteness} and its improvement by Rosser
explicitly exhibit a statement $!A$ that is independent of $\Th{T}$.
In the former, you have to dig to extract it from the argument. The
G\"odel-Rosser methods therefore have the advantage of making the
independent statement perfectly clear.
\end{digress}
