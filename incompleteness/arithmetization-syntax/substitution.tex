% Part: incompleteness
% Chapter: arithmetization-syntax
% Section: substitution

\documentclass[../../include/open-logic-section]{subfiles}

\begin{document}

\olfileid{inc}{art}{sub}
\olsection{Substitution}

\begin{prop}
There is a primitive recursive function $\fn{Subst}(x, y, z)$ with
\[
\fn{Subst}(\Gn{!A}, \Gn{t}, \Gn{x}) = \Gn{\Subst{!A}{t}{x}}
\]
\end{prop}

\begin{proof}
Let us suppose that the predicate $\fn{FreeOcc}(x, z, i)$, which holds
if the $i$-th symbols of the formula with G\"odel number~$x$ is a free
occurrence of the variable with G\"odel number~$z$, is primitive
recursive.  We can then define a function $\fn{Subst}'$ by primitive
recursion as follows:

\begin{align*}
\fn{Subst}'(0, x, y, z) & = \emptyseq \\
\fn{Subst}'(i + 1, x, y, z) & =
\begin{cases}
\fn{Subst}'(i, x, y, z) \concat y & \text{if $\fn{FreeOcc}(x, z, i+1)$} \\
\fn{append}(\fn{Subst}'(i, x, y, z), (x)_{i+1}) & \text{otherwise.}
\end{cases}
\end{align*}
$\fn{Subst}(x, y, z)$ can now be defined as $\fn{Subst}'(\len{x}, x, y, z)$.
\end{proof}

\begin{prob}
Show that the relation $\fn{FreeOcc}(x, z, i)$, which holds if the
$i$-th symbols of the formula with G\"odel number~$x$ is a free
occurrence of the variable with G\"odel number~$z$, is primitive
recursive.
\end{prob}

\begin{prob}
Show that $\fn{FreeFor}(x, y, z)$, which holds iff the term with
G\"odel number~$y$ is !!{free for} the variable with G\"odel
number~$z$ in the the formula with G\"odel number~$x$, is primitive
recursive.
\end{prob}

\end{document}
