% Part: incompleteness
% Chapter: arithmetization-syntax
% Section: proofs-in-lk

\documentclass[../../include/open-logic-section]{subfiles}

\begin{document}

\olfileid{inc}{art}{plk}
\olsection{Proofs in $\Log{LK}$}

\begin{defn}
If $\Gamma$ is a finite set of !!{sentence}s, $\Gamma = \{!A_1, \dots,
!A_n\}$, then $\Gn{\Gamma} = \tuple{\Gn{!A_1}, \dots, \Gn{!A_n}}$.

If $\Gamma \Sequent \Delta$ is a sequent, then a G\"odel number of
$\Gamma \Sequent \Delta$ is
\[
\Gn{\Gamma \Sequent \Delta} = \tuple{\Gn{\Gamma}, \Gn{\Delta}}
\]

If $\Pi$ is a !!{derivation} in $\Log{LK}$, then $\Gn{\Pi}$ is
\begin{enumerate}
\item $\tuple{0, \Gn{\Gamma \Sequent \Delta}}$ if $\Pi$ consists only
  of the initial sequent $\Gamma \Sequent \Delta$.
\item $\tuple{1, \Gn{\Gamma \Sequent \Delta}, k, \Gn{\Pi'}}$ if $\Pi$
  ends in an inference with one premise, $k$ is given by the following
  table according to which rule was used in the last inference, and
  $\Pi'$ is the immediate subproof ending in the premise of the last
  inference.

\begin{tabular}{lccccccc}
\text{Rule:} & \text{Contr} & $\lnot$ left & $\lnot$ right & 
   $\land$ left  & $\lor$ right & $\lif$ right \\
$k$: & 1 & 2 & 3 & 4 & 5 & 6 \\[2ex]
\text{Rule:} & $\lforall$ left & 
   $\lforall$ right & $\lexists$ left & $\lexists$ right & = \\
$k$: & 7 & 8 & 9 & 10 & 11  
\end{tabular}
\item $\tuple{2, \Gn{\Gamma \Sequent \Delta}, k, \Gn{\Pi'},
  \Gn{\Pi''}}$ if $\Pi$ ends in an inference with two premises, $k$ is
  given by the following table according to which rule was used in the
  last inference, and $\Pi'$, $\Pi''$ are the the immediate subproof
  ending in the laft and right premise of the last inference,
  respectively.

\begin{tabular}{lcccc}
\text{Rule:} & \text{Cut} & $\land$ right & $\lor$ left & $\lif$ left \\
$k$: & 1 & 2 & 3 & 4 
\end{tabular}
\end{enumerate}
\end{defn}

\begin{prop}
\ollabel{prop:followsby}
The following relations are primitive recursive:
\begin{enumerate}
\item $!A \in \Gamma$.
\item $\Gamma \Sequent \Delta$ is an initial sequent.
\item $\Gamma \Sequent \Delta$ follows from $\Gamma' \Sequent \Delta'$
  (and $\Gamma'' \Sequent \Delta''$) by a rule of $\Log{LK}$.
\item $\Pi$ is a correct $\Log{LK}$-!!{derivation}.
\end{enumerate}
\end{prop}

\begin{proof}
\begin{enumerate}
\item $\fn{IsIn}(x, g) = \lexists[i < \len{g}][(g)_i = x]$.
\item $\fn{InitSeq}(s) = \lexists[x < s][(\fn{Sent}(x) \land s =
  \tuple{\tuple{x},\tuple{x}})] \lor \lexists[t<s][(\fn{Term}(t) \land
  s = \tuple{0, t\concat \eq \concat t})]$.
\item Here we have to show that for each rule of inference~$R$ the
  relation $\fn{FollowsBy}_R(s, s')$ which holds if $s$ and $s'$ are
  the G\"odel numbers of conclusion and premise of a correct
  application of~$R$ is primitive recursive.  If $R$ has two premises,
  $\fn{FollowsBy}_R$ of course has three arguments.

For instance, $\Gamma \Sequent \Delta$ follows from $\Gamma' \Sequent
\Delta'$ by $\lexists$right iff there is a formula~$!A$, a
variable~$x$ and a closed term~$t$ such that $\Gamma = \Gamma'$,
$\Subst{!A}{t}{x} \in \Delta'$, $\lexists[x][!A] \in \Delta$, and for
every $!B \in \Delta$, either $!B = \lexists[x][!A]$ or $!B \in
\Gamma'$.  We just have to translate into G\"odel numbers.  If $s =
\Gn{\Gamma \Sequent \Delta}$ then $(s)_0 = \Gn{\Gamma}$ and $(s)_1 =
\Gn{\Gamma}$. So:
\begin{align*}
\fn{FollowsBy}_{\lexists \text{right}}(s, s') \defiff {} & 
\lexists[f<s][\lexists[x<s][\lexists[t<s'][(\fn{Frm}(f) \land \fn{Var}(x) \land \fn{Term}(t) \land {}]]] \\
& \fn{Subst}(f,t,x) \in (s')_1 \land \#(\lexists) \concat x \concat f \in (s)_1 \land {}\\
& \lforall[i < \len{(s)_1}][(((s)_1)_i = \#(\lexists) \concat x \concat f \lor ((s)_1)_i \in (s')_1)])
\end{align*}
\item We first define a helper relation $\fn{hDeriv}(s,n)$ which holds
  if $s$ codes a correct derivation at least to $n$ inferences up from
  the end sequent.  If $n=0$ we let the relation be satisfied by
  default.  Otherwise, $\fn{hDeriv}(s, n+1)$ iff either $s$ consists
  just of an initial sequent, or it ends in a correct inference and
  the codes of the immediate sub!!{derivation}s satisfy
  $\fn{nDeriv}(s, n)$.
\begin{align*}
\fn{nDeriv}(s, 0) \defiff {} & \True\\
\fn{nDeriv}(s, n+1) \defiff {} & ((s)_0 = 0 \land \fn{InitialSeq}((s)_1)) \lor{}\\
& ((s)_0 = 1 \land {}\\
& \quad ((s)_2 = 1 \land \fn{FollowsBy}_{\text{Contr}}((s)_1, ((s)_3)_1)) \lor{}\\
& \qquad \vdots\\
& \quad ((s)_2 = 11 \land \fn{FollowsBy}_{=}((s)_1, ((s)_3)_1)) \land {}\\
& \quad \fn{nDeriv}((s)_3, n)) \lor {}\\
& ((s)_0 = 2 \land {}\\
& \quad ((s)_2 = 1 \land \fn{FollowsBy}_{\text{Cut}}((s)_1, ((s)_3)_1), ((s)_4)_1)) \lor{}\\
& \qquad \vdots\\
& \quad ((s)_2 = 4 \land \fn{FollowsBy}_{\lif\text{left}}((s)_1, ((s)_3)_1), ((s)_4)_1)) \land {}\\
& \quad \fn{nDeriv}((s)_3, n) \land \fn{nDeriv}((s)_4, n))
\end{align*}
This is a primitive recursive definition, if the number~$n$ is large
enough, e.g., larger than the maximum number of inferences between an
initial sequent and the end sequent in~$s$, it holds of $s$ iff $s$ is
the G\"odel number of a correct !!{derivation}.  So we can now define
$\fn{Deriv}(s)$ by $\fn{nDeriv}(s,s)$.
\end{enumerate}
\end{proof}

\begin{prob}
Define the following relations as in
\olref[inc][art][plk]{prop:followsby}:
\begin{enumerate}
\item $\fn{FollowsBy}_{\land\text{right}}(s, s', s'')$,
\item $\fn{FollowsBy}_{\eq}(s, s')$,
\item $\fn{FollowsBy}_{\lforall\text{right}}(s, s', s'')$.
\end{enumerate}
\end{prob}

\begin{prop}
Suppose $\Gamma$ is a primitive recursive set of !!{sentence}s.  Then
the relation $\fn{Pr}_\Gamma(x, y)$ expressing ``$x$ is the G\"odel
number of a sentence~$!A$ and $y$ is the code of !!a{derivation}~$\Pi$
of $\Gamma_0 \Sequent !A$ for some finite $\Gamma_0 \subseteq
\Gamma$'' is primitive recursive.
\end{prop}

\begin{proof}
Suppose ``$x \in \Gamma$'' is given by the primitive recursive
predicate~$R_\Gamma(x)$. We have to show that $\fn{Pr}_\Gamma(x, y)$
which holds iff $x$ is the G\"odel number of a sentence~$!A$ and $y$
is the code of an $\Log{LK}$-!!{derivation} with end sequent
$\Gamma_0 \Sequent !A$ is primitive recursive.

By the previous proposition, the property $\fn{Deriv}(y)$ which holds
iff $y$`is the code of a correct derivation~$\Pi$ in $\Log{LK}$ is
primitive recursive.  If $y$ is such a code, then $(y)_1$ is the code
of the end sequent of`$\Pi$, and so $((y)_1)_0$ is the code of the
left side of the end sequent and $((y)_1)_1$ the right side. So we can
express ``the right side of the end sequent of~$\Pi$ is~$!A$'' as
$\len{((y)_1)_1} = 1 \land (((y)_1)_1)_0 = x$.  The left side of the
end sequent of $\Pi$ is of course automatically finite, we just have
to express that every sentence in it is in~$\Gamma$.  Thus we can
define $\fn{Pr}_\Gamma(x, y)$ by
\begin{align*}
\fn{Pr}_\Gamma(x, y) \defiff {}&
\fn{Sent}(x) \land \fn{Deriv}(y) \land {} \\
& \lforall[i <
  \len{((y)_1)_0}][(((y)_1)_0)_i \in \Gamma] \land {}\\
& \len{((y)_1)_1} = 1 \land (((y)_1)_1)_0 = x
\end{align*}
\end{proof}

\end{document}
