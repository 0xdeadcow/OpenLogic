% Part: incompleteness
% Chapter: arithmetization-syntax
% Section: proofs-in-nd

\documentclass[../../include/open-logic-section]{subfiles}

\begin{document}

\olfileid{inc}{art}{pnd}
\olsection{\usetoken{P}{derivation} in Natural Deduction}

\begin{explain}
In order to arithmetize !!{derivation}s, we must represent
!!{derivation}s as numbers. Since !!{derivation}s are trees of formula
where each inference carries one or two labels, a recursive
representation is the most obvious approach: we represent a
!!{derivation} as a tuple, the components of which are the
end-!!{formula}, the labels, and the representations of the
sub-!!{derivation}s leading to the premises of the last inference.
\end{explain}

\begin{defn}
If $\Gamma$ is a finite set of !!{sentence}s, $\Gamma = \{!A_1, \dots,
!A_n\}$, then $\Gn{\Gamma} = \tuple{\Gn{!A_1}, \dots, \Gn{!A_n}}$.

If $\delta$ is a !!{derivation} in natural deduction, then $\Gn{\delta}$ is
\begin{enumerate}
\item $\tuple{0, \Gn{!A}, n}$ if $\delta$ consists only of the initial
  formula~$!A$. The number $n$ is $0$ if it is an !!{undischarged}
  assumption, and the numerical label otherwise.
\item $\tuple{1, \Gn{!A}, n, k, \Gn{\delta'}}$ if $\delta$
  ends in an inference with one premise, $k$ is given by the following
  table according to which rule was used in the last inference, and
  $\delta'$ is the immediate subproof ending in the premise of the last
  inference. $n$ is the label of the inference, or $0$
  if the inference does not !!{discharge} any assumptions.

\begin{tabular}{lccccccc}
\text{Rule:} & \Elim{\lfalse} & \Intro{\lnot} & \Elim{\lnot} &
   \Elim{\land} & \Intro{\lor} & \Intro{\lif} \\
$k$: & 1 & 2 & 3 & 4 & 5 & 6 \\[2ex]
\text{Rule:} & \Intro{\lforall} &
   \Elim{\lforall} & \Intro{\lexists} & \Intro{\eq} \\
$k$: & 7 & 8 & 9 & 10 
\end{tabular}
\item $\tuple{2, \Gn{!A}, n, k, \Gn{\delta'}, \Gn{\delta''}}$ if $\delta$
  ends in an inference with two premises, $k$ is given by the
  following table according to which rule was used in the last
  inference, and $\delta'$, $\delta''$ are the immediate
  sub!!{derivation}s ending in the left and right premise of the last
  inference, respectively. $n$ is the label of the inference, or $0$
  if the inference does not !!{discharge} any assumptions.

\begin{tabular}{lcccc}
\text{Rule:} & \Intro{\lfalse} & \Intro{\land} & \Elim{\lif} \\
$k$: & 1 & 2 & 3 
\end{tabular}
\item $\tuple{3, \Gn{!A}, n, \Gn{\delta'}, \Gn{\delta''}, \Gn{\delta'''}}$ if $\delta$ ends in an \Elim{\lor} inference.  $\delta'$,
  $\delta''$, $\delta'''$ are the immediate sub!!{derivation}s ending
  in the left, middle, and right premise of the last inference,
  respectively, and $n$ is the label of the inference.
\end{enumerate}
\end{defn}

\begin{explain}
Having settled on a representation of !!{derivation}s, we must also
show that we can manipulate such derivations primite recursively, and
express their essential properties and relations so.  Some operations
are simple: e.g., given a G\"odel number~$d$ of a !!{derivation},
$(d)_1$ gives us the G\"odel number of its end-!!{formula}.  Some are
much harder.  We'll at least sketch how to do this.  The goal is to
show that the relation ``$\delta$ is !!a{derivation} of~$!A$
from~$\Gamma$'' is primitive recursive on the G\"odel numbers of
$\delta$ and~$!A$.
\end{explain}

\begin{prop}
\ollabel{prop:followsby}
The following relations are primitive recursive:
\begin{enumerate}
\item $!A$ is an initial formula in $\delta$ with label $n$.
\item $!A$ is an !!{undischarged} assumption of~$\delta$.
\item An inference with conclusion~$!A$ and upper !!{derivation}s
  $\delta$ (and $\delta'$, $\delta''$) labelled $n$ is correct.
\item $\delta$ is a correct natural deduction !!{derivation}.
\end{enumerate}
\end{prop}

\begin{proof}
We have to show that the corresponding relations between G\"odel
numbers of !!{formula}s, sequences of G\"odel numbers of !!{formula}s
(which code sets of !!{formula}s), and G\"odel numbers of !!{derivation}s 
are primitive recursive.
\begin{enumerate}
\item For this we need a helper relation $\fn{hInitFrm}(x, d, n, i)$
  which holds if the !!{formula}~$!A$ with G\"odel number~$x$ occurs
  as an initial !!{formula} with label~$n$ in the !!{derivation} with
  G\"odel number~$d$ within~$i$ inferences up from the
  end-!!{formula}.
  \begin{multline*}
    \begin{aligned}
    \fn{hInitFrm}(x, d, n, 0) \defiff {} & \True\\   
    \fn{hInitFrm}(x, d, n, i+1) \defiff {} &
    \end{aligned}\\
    \begin{aligned}
&  d = \tuple{0, \Gn{!A}, n} \lor {}\\
      & ((d)_0 = 1 \land \fn{hInitFrm}(x, (d)_4, n, i)) \lor {}\\
      & ((d)_0 = 2 \land (\fn{hInitFrm}(x, (d)_4, n, i) \lor {}\\
      & \qquad\fn{hInitFrm}(x, (d)_5, n, i))) \lor {}\\
      & ((d)_0 = 3 \land (\fn{hInitFrm}(x, (d)_3, n, i) \lor {}\\
      & \qquad \fn{hInitFrm}(x, (d)_2, n, i)) \lor
      \fn{hInitFrm}(x, (d)_3, n, i))
\end{aligned}
  \end{multline*}
   If the number~$i$ is large enough, e.g., larger than the maximum
   number of inferences between an initial !!{formula} and the
   end-!!{formula} of~$\delta$, it holds of $x$, $d$, $n$, and $i$ iff
   $!A$ is an initial formula in $\delta$ labelled~$n$.  The number
   $d$ itself is larger than that maximum number of inferences.  So we
   can define $\fn{InitFrm}(x, d, n)$ as $\fn{InitFrm}(x,
   d, n, d)$.
\item For this we proceed similarly: Define the helper relation
  $\fn{hOpenAssum}(x, d, n, i)$ as
  \begin{multline*}
    \begin{aligned}
    \fn{hOpenAssum}(x, d, n, 0) \defiff {} & \True\\   
    \fn{hOpenAssum}(x, d, n, i+1) \defiff {} &
    \end{aligned}\\
    \begin{aligned}
&  d = \tuple{0, \Gn{!A}, n} \lor {}\\
      &  ((d)_2 \neq n \land {}\\
      & \quad((d)_0 = 1 \land \fn{hOpenAssum}(x, (d)_4, n, i)) \lor {}\\
      & \quad((d)_0 = 2 \land (\fn{hOpenAssum}(x, (d)_4, n, i) \lor {}\\
      & \qquad\fn{hOpenAssum}(x, (d)_5, n, i))) \lor {}\\
      & ((d)_0 = 3 \land (\fn{hOpenAssum}(x, (d)_3, n, i) \lor {}\\
      & \qquad \fn{hOpenAssum}(x, (d)_2, n, i)) \lor
      \fn{hOpenAssum}(x, (d)_3, n, i))
\end{aligned}
  \end{multline*}
  Here the main difference is that an assumption is !!{undischarged}
  not only if it is !!{undischarged} in one of the immediate
  subderivations, but it must also not be !!{discharged} by the last
  inference, i.e., the label must be different from the label of the
  inference, $(d)_3$.  We can then define $\fn{OpenAssum}(x, d)$ as
  $\forall n<d \fn{InitFrm}(x,d,n,d)$.
\item Here we have to show that for each rule of inference~$R$ the
  relation $\fn{FollowsBy}_R(x, d, n)$ which holds if $x$ is the
  G\"odel number of the conclusion and $d$ is the G\"odel number of a
  derivation ending in the premise of a correct application of~$R$
  with label~$n$ is primitive recursive, and similarly for rules with
  two or three premises.

  The simplest case is that of the \Intro{\eq} rule. Here there is no
  premise, i.e., $d = 0$.  However, $!A$ must be of the form
  $\eq[t][t]$, for a closed term~$t$. Here, a primitive recursive
  definition is
  \[\lexists[t<x][(\fn{Term}(t) \land x = t\concat
    \Gn{\eq} \concat t)] \land d = 0).\]

  For a more complicated example, $\fn{FollowsBy}_{\Intro{\lif}}(x, d,
    n)$ holds iff $!A$ is of the form $!B \lif !C$, the
  end-!!{formula} of $\delta$ is $!C$, and any initial formula
  in~$\delta$ labelled~$n$ is of the form~$!B$.  We can express this
  primitive recursively by
  \begin{align*}
  \exists y<x\exists z<x &(x = \Gn{(} \concat y \concat \Gn{\lif}
  \concat z\concat \Gn{)} \land (d)_1 = z \land {}\\
&  \forall
  u<d((\fn{Sent}(u) \land \fn{InitFrm}(u, d, n)) \lif u = y)
  \end{align*}

  For another example, consider \Intro{\lexists}.  Here, $!A$ is the
  conclusion of a correct inference with one upper derivation iff
  there is !!a{formula}~$!B$, a closed term~$t$ and
  !!a{variable}~$x$ such that $\Subst{!B}{t}{x}$ is the
  end-!!{formula} of the upper derivation and $\lexists[x][!B]$
  is the conclusion~$!A$, i.e., the formula with G\"odel number~$x$.
\begin{multline*}
\fn{FollowsBy}_{\Intro{\lexists}}(x, d, n) \defiff {} \\
\exists y < x\exists v<x\exists t<d(\fn{Frm}(y) \land \fn{Term}(t) \land \fn{Var}(v)  \land {}\\
\fn{Subst}(y,t,v) = (d)_1 \land \#(\lexists) \concat v \concat z = x)
\end{multline*}
\item We first define a helper relation $\fn{hDeriv}(d, i)$ which
  holds if $d$ codes a correct derivation at least to $i$ inferences
  up from the end sequent.  $\fn{hDeriv}(d, 0)$ holds always.
  Otherwise, $\fn{hDeriv}(d, i+1)$ iff either $d$ just codes an
  initial !!{formula} or $d$ it ends in a correct inference with
  label~$n$ and the codes of the immediate sub-!!{derivation}s satisfy
  $\fn{hDeriv}(d', i)$.
\begin{multline*}
  \begin{aligned}
\fn{hDeriv}(d, 0) \defiff {} & \True \\
\fn{hDeriv}(d, i+1) \defiff {} &
  \end{aligned}\\
  \begin{aligned}
& \exists x<d\exists n<d(\fn{Frm}(x) \land d = \tuple{0, x, n}) \lor {}\\
& ((d)_0 = 1 \land {}\\
& \quad ((s)_3 = 1 \land
\fn{FollowsBy}_{\Elim{\lfalse}}((d)_1, (d)_4, (d)_2) \lor{}\\
& \qquad \vdots\\
& \quad ((s)_3 = 10 \land
\fn{FollowsBy}_{\Intro{\eq}}((d)_1, (d)_4, (d)_2)) \land {}\\
& \quad \fn{nDeriv}((d)_4, i)) \lor {}\\
& ((s)_0 = 2 \land {}\\
& \quad ((s)_3 = 1 \land 
\fn{FollowsBy}_{\Intro{\lfalse}}((d)_1, (d)_4, (d)_5, (d)_2)) \lor{}\\
& \qquad \vdots\\
& \quad ((s)_3 = 3 \land
\fn{FollowsBy}_{\Elim{\lif}}((d)_1, (d)_4, (d)_5, (d)_2)) \land {}\\
& \quad \fn{hDeriv}((d)_4, i) \land \fn{hDeriv}((d)_5, i)) \lor {}\\
& ((s)_0 = 3 \land {}\\
& \quad \fn{FollowsBy}_{\Elim{\lor}}((d)_1, (d)_3, (d)_4, (d)_5, (d)_2) \land{}\\
& \quad \fn{hDeriv}((d)_3, i) \land \fn{hDeriv}((d)_4, i)) \land \fn{hDeriv}((d)_5, i)  
  \end{aligned}
  \end{multline*}
This is a primitive recursive definition. Again we can define
$\fn{Deriv}(d)$ as $\fn{hDeriv}(d, d)$.
\end{enumerate}
\end{proof}

\begin{prob}
Define the following relations as in
\olref[inc][art][pnd]{prop:followsby}:
\begin{enumerate}
\item $\fn{FollowsBy}_{\Elim{\lif}}(x, d', d'', n)$,
\item $\fn{FollowsBy}_{\Elim{\eq}}(x, d, d', n)$,
\item $\fn{FollowsBy}_{\Elim{\lor}}(x, d, d', d'', n)$,
\item $\fn{FollowsBy}_{\Intro{\lforall}}(x, d, n)$.
\end{enumerate}
\end{prob}

\begin{prop}
Suppose $\Gamma$ is a primitive recursive set of !!{sentence}s.  Then
the relation $\fn{Pr}_\Gamma(x, y)$ expressing ``$x$ is the code of
!!a{derivation}~$\delta$ of $!A$ from !!{undischarged} assumptions in $\Gamma$
and $y$ is the G\"odel number of~$!A$'' is
primitive recursive.
\end{prop}

\begin{proof}
Suppose ``$y \in \Gamma$'' is given by the primitive recursive
predicate~$R_\Gamma(y)$.  We have to show that $\fn{Pr}_\Gamma(x, y)$
which holds iff $y$ is the G\"odel number of a sentence~$!A$ and
$x$~is the code of a natural deduction !!{derivation} with end
!!{formula} $!A$ and all !!{undischarged} assumptions in~$\Gamma$ is
primitive recursive.

By the previous proposition, the property $\fn{Deriv}(x)$ which holds
iff $x$ is the code of a correct derivation~$\delta$ in natural deduction is
primitive recursive.  If $x$ is such a code, then $(x)_1$ is the code
of the end !!{formula} of~$\delta$. Thus we can
define $\fn{Pr}_\Gamma(x, y)$ by
\begin{align*}
\fn{Pr}_\Gamma(x, y) \defiff {}&
\fn{Sent}(y) \land \fn{Deriv}(x) \land (x)_1=y \land {} \\
& \lforall[z < x]((\fn{Sent}(z) \land \fn{OpenAssum}(z, x)) \lif R_\Gamma(z))
\end{align*}
\end{proof}

\end{document}
