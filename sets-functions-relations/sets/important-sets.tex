% Part: sets-functions-relations
% Chapter: sets
% Section: important-sets

\documentclass[open-logic-section]{subfiles}

\begin{document}

\olfileid{sfr}{set}{set}
\olsection{Some Important Sets}

\begin{ex}
Mostly we'll be dealing with sets that have mathematical objects as
members. You will remember the various sets of numbers: $\Nat$
is the set of \emph{natural} numbers $0$, $1$, $2$, $3$, \dots{};
$\Int$ the set of \emph{integers} \ldots{}, $-3$, $-2$,
$-1$, $0$, $1$, $2$, $3$, \ldots{}; $\Rat$ the set of
\emph{rationals} ($\Rat = \Setabs{p/q}{p,q\in \Int}$); and
$\Real$ the set of \emph{real} numbers. These are all \emph{infinite}
sets, that is, they each have infinitely many elements. As it turns
out, $\Nat$, $\Int$, $\Rat$ have the same number
of elements, while $\Real$ has a whole bunch more---$\Nat$,
$\Int$, $\Rat$ are ``countably infinite'' whereas
$\Real$ is ``uncountable''.
\end{ex}

\begin{ex}[Strings]
Another interesting set is the set $A^{*}$of
\emph{strings} over an alphabet $A$: any sequence of elements of
$A$ is a string over $A$. We include the \emph{empty string $\Lambda$}
among the strings over $A$, for every alphabet $A$. For instance,
if $A=\{0,1\}$, then 
\[
\BinStr = A^{*}
=\{\Lambda,0,1,00,01,10,11,000,001,010,011,100,101,110,111,0000,\ldots\}.
\]
 If $x=x_{1}\ldots x_{n}\in A^{*}$is a string consisting of $n$
``letters'' from $A$, then we say \emph{length} of the string is~$n$
and write $\len{x}=n$.
\end{ex}

\end{document}
