% Part: sets-functions-relations
% Chapter: sets
% Section: subsets

\documentclass[../../include/open-logic-section]{subfiles}

\begin{document}

\olfileid{sfr}{set}{sub}
\olsection{Subsets}

\begin{explain}
Sets are made up of their elements, and every element of a set is a
part of that set. But there is also a sense that some of the elements
of a set \emph{taken together} are a ``part of'' that set. For
instance, the number~$2$ is part of the set of integers, but the set
of even numbers is also a part of the set of integers. It's important
to keep those two senses of being part of a set separate.
\end{explain}

\begin{defn}
If every element of a set $X$ is also an element of
    $Y$, then we say that $X$ is a \emph{subset} of $Y$, and write $X
    \subseteq Y$.
\end{defn}

\begin{ex}
First of all, every set is a subset of itself, and $\emptyset$ is a
subset of every set. The set of even numbers is a subset of the set of
natural numbers. Also, $\{ a, b \} \subseteq \{ a, b, c \}$.

But $\{ a, b, e \}$ is not a subset of $\{ a, b, c \}$.
\end{ex}

\begin{explain}
Note that a set may contain other sets!{} In particular, a set may
happen to \emph{both} be an !!{element} and a subset of another, e.g.,
$\{0\} \in \{0, \{0\}\}$ and also $\{0\} \subseteq \{0, \{0\}\}$.
\end{explain}


\begin{defn}
The set consisting of all subsets of a set~$X$ is called the
\emph{power set of}~$X$, written $\Pow{X}$.
    \[\Pow{X} = \Setabs{x}{x \subseteq X} \]
\end{defn}

\begin{ex}
What are all the possible subsets of $\{ a, b, c \}$? They are:
$\emptyset$, $\{a \}$, $\{b\}$, $\{c\}$, $\{a, b\}$, $\{a, c\}$, $\{b,
c\}$, $\{a, b, c\}$. The set of all these subsets is
$\Pow{\{a,b,c\}}$:
\[
\Pow{\{ a, b, c \}} = \{\emptyset, \{a \}, \{b\}, \{c\}, \{a, b\},
\{b, c\}, \{a, c\}, \{a, b, c\}\}
\]
\end{ex}

\begin{prob}
List all subsets of $\{a, b, c, d\}$.
\end{prob}

\end{document}
