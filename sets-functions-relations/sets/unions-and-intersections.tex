% Part: sets-functions-relations
% Chapter: sets
% Section: unions-and-intersections

\documentclass[../../include/open-logic-section]{subfiles}

\begin{document}

\olfileid{sfr}{set}{uni}
\olsection{Unions and Intersections}

\begin{defn} 
The \emph{union} of two sets $X$ and $Y$, written $A \cup B$, is the
set of all things which are members of $X$, $Y$, or both.
\[
X \cup Y = \Setabs{x}{x \in X \lor x\in Y}
\]
\end{defn}

\begin{ex}
Since the multiplicity of elements doesn't matter, the union of two
sets which have an element in common contains that element only once,
e.g., $\{ a, b, c\} \cup \{ a, 0, 1\} = \{a, b, c, 0, 1\}$.

The union of a set and one of its subsets isjust the bigger set: $\{a,
b, c \} \cup \{a \} = \{a, b, c\}$.

The union of a set with the empty set is identical to the set: $\{a,
b, c \} \cup \emptyset = \{a, b, c \}$.
\end{ex}

\begin{defn}
The emph{intersection} of two sets$X$ and $Y$, written $X \cap Y$, is
the set of all things which are members of both $X$ and~$Y$. 
\[
X \cap Y = \Setabs{x}{x \in X \land x\in Y}
\]
Two sets are called \emph{disjoint} if their intersection is
empty. This means they have no elements in common.
\end{defn}

\begin{ex}
$\{ a, b, c\} \cap \{ 0, 1\} = \emptyset$.

$\{a, b, c \} \cap \{a, 0, 1 \} = \{a\}$.

$\{a, b, c \} \cap \emptyset = \emptyset$.
\end{ex}

\begin{explain}
We can obviously also form the union or intersection of more than two
sets.  An elegant way of dealing with this in general is the
following: suppose you collect all the sets you want to form the union
(or intersection) of into a single set. Then we can define the union
or intersection of all our original sets as the set of all objects
which belong to at least one, respectively, to all members of the set.
\end{explain}

\begin{defn}
If $C$ is a set of sets, then $\bigcup C$ is the set of elements of
elements of $C$:
\[
\bigcup C = \Setabs{x}{x \text{ belongs to an element of } C}
\]
\end{defn}

\begin{defn}
If $C$ is a set of sets, then $\bigcap C$ is the set of objects which
all elements of $C$ have in common:
\[
\bigcap C = \Setabs{x}{x \text{ belongs to every element of } C}
\]
\end{defn}

\begin{ex}
Suppose $C = \{ \{ a, b \}, \{ a, d, e \}, \{ a, d \} \}$.
Then $\bigcup C = \{ a, b, d, e \}$ and $\bigcap C = \{ a \}$.
\end{ex}

We could also do the same for a sequence of sets $A_1, A_2, ... $

$\bigcup_i A_i = \{ x : x \mbox{ belongs to one of the } A_i \}$.

$\bigcap_i A_i = \{ x : x \mbox{ belongs to every } A_i \}$.

\begin{defn}
The \emph{difference}~$X \setminus Y$ is the set of all elements of
$X$ which are not also elements of $Y$, i.e.,
\[
X\setminus Y = \Setabs{x}{x\in X \text{ and } x \notin Y}.
\]
\end{defn}


\end{document}
