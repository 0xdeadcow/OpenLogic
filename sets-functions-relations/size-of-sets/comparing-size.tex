% Part:sets-functions-relations
% Chapter: sets
% Section: comparing-sizes

\documentclass[../../include/open-logic-section]{subfiles}

\begin{document}

\olfileid{sfr}{set}{car}

\olsection{Comparing Sizes of Sets}

\begin{explain}
Just like we were able to make precise when two sets have the same
size in a way that also accounts for the size of infinite sets, we can
also compare the sizes of sets in a precise way. Our definition of
``is smaller than (or equinumerous)'' will require, instead of a
!!{bijection} between the sets, a total !!{injective} function from the first
set to the second. If such a function exists, the size of the first
set is less than or equal to the size of the second.  Intuitively, 
!!a{injective} function from one set to another guarantees that the range of
the function has at least as many elements as the domain, since no two
!!{element}s of the domain map to the same !!{element} of the range.
\end{explain}

\begin{defn}
$\card{X} \leq \card{Y}$ if and only if there is an !!{injective}
  function~$f \colon X \to Y$.
\end{defn}

\begin{thm}[Schr\"oder-Bernstein]
Let $X$ and $Y$ be sets. If $\card{X} \leq \card{Y}$ and $\card{Y}
\leq \card{X}$, then $\card{X} = \card{Y}$.
\end{thm}

\begin{explain}
In other words, if there is a total !!{injective} function from $X$ to
$Y$, and if there is a total !!{injective} function from $Y$ back to $X$,
then there is a total !!{bijection} from $X$ to $Y$. Sometimes, it can be
difficult to think of a !!{bijection} between two equinumerous sets, so
the Schr\"oder-Bernstein theorem allows us to break the comparison
down into cases so we only have to think of an !!{injection} from the
first to the second, and vice-versa. The Schr\"oder-Bernstein theorem,
apart from being convenient, justifies the act of discussing the
``sizes'' of sets, for it tells us that set cardinalities have the
familiar anti-symmetric property that numbers have.
\end{explain}

\begin{defn}
$\card{X} < \card{Y}$ if and only if there is !!a{injective}
  function~$f\colon X \to Y$ but no !!{bijective}~$g\colon X \to Y$.
\end{defn}

\begin{thm}[Cantor]
For all $X$, $\card{X} < \card{\Pow{X}}$.
\end{thm}

\begin{proof}
The function~$f \colon X \to \Pow{X}$ that maps any $x \in X$ to its
singleton~$\{x\}$ is !!{injective}, since if $x \neq y$ then also $f(x) =
\{x\} \neq \{y\} = f(y)$.

There cannot be !!a{surjective} function~$g\colon X \to \Pow{X}$, let
alone a !!{bijective} one. For assume that a surjective $g\colon X \to
\Pow{X}$ exists.  Then let $Y = \Setabs{x \in X}{x \notin g(x)}$. If
$g(x)$ is defined for all $x \in X$, then $Y$ is clearly a
well-defined subset of~$X$.  If $g$ is !!{surjective}, $Y$ must be the
value of~$g$ for some $x_0 \in X$, i.e., $Y = g(x_0)$.  Now consider
$x_0$: it cannot be !!a{element} of $Y$, since if $x_0 \in Y$ then
$x_0 \in g(x_0)$, and the definition of~$Y$ then would have $x_0
\notin Y$.  On the other hand, it must be !!a{element} of~$Y$, since
if it were not, then $x_0 \notin Y = g(x_0)$.  But then $x_0$
satisfies the defining condition of~$Y$, and so $x_0 \in Y$. In either
case, we have a contradiction.
\end{proof}


\end{document}
