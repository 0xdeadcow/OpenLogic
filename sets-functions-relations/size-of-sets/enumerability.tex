% Part:sets-functions-relations
% Chapter: sets
% Section: enumerability

\documentclass[../../include/open-logic-section]{subfiles}

\begin{document}

\olfileid{sfr}{siz}{enm}

\olsection{!S{enumerable} Sets}

\begin{defn}
Informally, an \emph{enumeration} of a set $X$ is a list (possibly
infinite) such that every !!{element} of $X$ appears some finite number of
places into the list. If $X$ has an enumeration, then $X$ is said to
be \emph{!!{enumerable}}.
\end{defn}

\begin{explain}
A couple of points about enumerations:

\begin{enumerate}
\item The order of !p{element} of $X$ in the enumeration does not
  matter, as long as every !!{element} appears: $4$, $1$, $2$5,
  $16$,~$9$ enumerates the (set of the) first five square numbers just
  as well as $1$, $4$, $9$, $16$,~$25$ does.
\item Redundant enumerations are still enumerations: $1$, $1$, $2$,
  $2$, $3$, $3$,~\dots{} enumerates the same set as $1$, $2$,
  $3$,~\dots{} does.
\item Order and redundancy \emph{do} matter when we specify an
  enumeration: we can enumerate the natural numbers beginning with
  $1$, $2$, $3$, $1$, \dots{}, but the pattern is easier to see when
  enumerated in the standard way as $1$, $2$, $3$, $4$,~\dots
\item Enumerations must have a beginning: \dots, $3$, $2$, $1$ is not
  an enumeration of the natural numbers because it has no first
  !!{element}. To see how this follows from the informal definition,
  ask yourself, ``at what place in the list does the number 76
  appear?''
\item The following is not an enumeration of the natural numbers: $1$,
  $3$, $5$, \dots, $2$, $4$, $6$, \dots\@ The problem is that the even
  numbers occur at places $\infty + 1$, $\infty + 2$, $\infty + 3$,
  rather than at finite positions.
\item Lists may be gappy: $2$, $-$, $4$, $-$, $6$, $-$, \dots{}
  enumerates the even natural numbers.
\item The empty set is enumerable: it is enumerated by the empty list!
\end{enumerate}
\end{explain}

The following provides a more formal definition of an 
enumeration:

\begin{defn}
An \emph{enumeration} of a set $X$ is any !!{surjective} function $f: 
\Nat \rightarrow X$.
\end{defn}

\begin{explain}
Let's convince ourselves that the formal definition and the informal
definition using an possibly gappy, possibly infinite list are
equivalent. !A{surjective} function (partial or total) from $\Nat$ to
a set $X$ enumerates~$X$. Such a function determines an enumeration as
defined informally above. Then an enumeration for $X$ is the list
$f(1)$, $f(2)$, $f(3)$, \dots. Since $f$ is !!{surjective}, every
!!{element} of $X$ is guaranteed to be the value of $f(n)$ for some $n
\in \Nat$.  Hence, every !!{element} of $X$ appears at some finite
place in the list. Since the function may be partial or one-to-one,
the list may be gappy or redundant, but that is acceptable (as noted
above). On the other hand, given a list that enumerates all
!p{element} of~$X$, we can define an !!{surjective} function $f\colon
\Nat \to X$ by letting $f(n)$ be the $(n-1)$st member of the list, or
undefined if the list has a gap in the $(n-1)$st spot.
\end{explain}

\begin{ex}
A function enumerating the natural numbers ($\Nat$) is 
simply the identity function given by $f(n) = n$.
\end{ex}

\begin{ex}
The functions $f\colon \Nat \to \Nat$ and $g \colon \Nat \to \Nat$ given by
\begin{align}
f(n) & = 2n \tex{ and}\\
g(n) = 2n+1
\end{align}
enumerate the even natural numbers and the odd natural numbers, 
respectively. However, neither function is an enumeration of 
$\Nat$, since neither is !!{surjective}.
\end{ex}

\begin{ex}
The function $f(n) = \lceil \frac{(-1)^n n}{2}\rceil$ (where $\lceil x
\rceil$ denotes the \emph{ceiling} function, which rounds $x$ up to
the nearest integer) enumerates the set of integers~$\Int$. Notice
how $f$ generates the values of $\Int$ by ``hopping'' back and forth
between positive and negative integers: 
\[
\begin{array}{c c c c c c c}
f(1) & f(2) & f(3) & f(4) & f(5) & f(6) & \dots \\ \\
\lceil - \tfrac{1}{2}\rceil & \lceil \tfrac{2}{2} \rceil & \lceil - 
\tfrac{3}{2} \rceil & \lceil \tfrac{4}{2} \rceil  & \lceil -\tfrac{5}{2} 
\rceil & \lceil \tfrac{6}{2} \rceil & \dots \\ \\
0 & 1 & -1 & 2 & -2 & 3 & \dots
\end{array}
\]
\end{ex}

\begin{explain}
That is fine for ``easy'' sets. What about the set of, say, pairs of 
natural numbers?
\[ 
\Nat^2 = \Nat \times \Nat = \Setabs{\tuple{n,m}}{n,m \in \Nat}
\]
Another method we can use to enumerate sets is to organize them 
in an \emph{array}, such as the following:
\[
\begin{array}{ c | c | c | c | c | c}
& \textbf 1 & \textbf 2 & \textbf 3 & \textbf 4 & \dots \\
\hline
\textbf 1 & \tuple{1,1} & \tuple{1,2} & \tuple{1,3} & \tuple{1,4} & \dots \\
\hline
\textbf 2 & \tuple{2,1} & \tuple{2,2} & \tuple{2,3} & \tuple{2,4} & \dots \\
\hline
\textbf 3 & \tuple{3,1} & \tuple{3,2} & \tuple{3,3} & \tuple{3,4} & \dots \\
\hline
\textbf 4 & \tuple{4,1} & \tuple{4,2} & \tuple{4,3} & \tuple{4,4} & \dots \\
\hline
\vdots & \vdots & \vdots & \vdots & \vdots & \ddots\\
\end{array}
\]

Clearly, every ordered pair in $\Nat^2$ will appear at least 
once in the array. In particular, $\tuple{n,m}$ will appear in the $n$th 
column and $m$th row. But how do we organize the elements of an 
array into a list? The pattern in the array below demonstrates one 
way to do this:
\[
\begin{array}{ c | c | c | c | c | c }
& & & & & \\
\hline
& 1 & 2 & 4 & 7 & \dots \\
\hline
& 3 & 5 & 8 & \dots & \dots \\
\hline
& 6 & 9 & \dots & \dots & \dots \\
\hline
& 10 & \dots & \dots & \dots & \dots \\
\hline
& \vdots & \vdots & \vdots & \vdots & \ddots\\ 
\end{array}
\]
This pattern is called \emph{Cantor's zig-zag method}. Other 
patterns are perfectly permissible, as long as they ``zig-zag'' 
through every cell of the array. By Cantor's zig-zag method, the 
enumeration for $\Nat^2$ according to this scheme would be:
\[
\tuple{1,1}, \tuple{1,2}, \tuple{2,1}, \tuple{1,3}, \tuple{2,2}, \tuple{3,1}, \tuple{1,4}, \tuple{2,3}, \tuple{3,2}, \tuple{4,1}, \dots
\]

What ought we do about enumerating, say, the set of ordered triples 
of natural numbers?
\[ 
\Nat^3 = \Nat \times \Nat \times \Nat = \{ (n,m,k) : n,m,k \in \Nat \} 
\]
We can think of $\Nat^3$ as the Cartesian product of
$\Nat$ and $\Int^2$, that is, 
\[ 
\Nat^3 = \Nat^2 \times \Nat = \Setabs{(\vec a, 
k)}{\vec a \in \Nat^2, k \in \Nat } 
\]
and thus we can enumerate $\Nat^3$ with an array by 
labelling one axis with the enumeration of $\Nat$, and the 
other axis with the enumeration of $\Nat^2$:
\[
\begin{array}{ c | c | c | c | c | c}
& \textbf 1 & \textbf 2 & \textbf 3 & \textbf 4 & \dots \\
\hline
\mathbf{\tuple{1,1}} & \tuple{1,1,1} & \tuple{1,1,2} & \tuple{1,1,3} & \tuple{1,1,4} & \dots \\
\hline
\mathbf{\tuple{1,2}} & \tuple{1,2,1} & \tuple{1,2,2} & \tuple{1,2,3} & \tuple{1,2,4} & \dots \\
\hline
\mathbf{\tuple{2,1}} & \tuple{2,1,1} & \tuple{2,1,2} & \tuple{2,1,3} & \tuple{2,1,4} & \dots \\
\hline
\mathbf{\tuple{1,3}} & \tuple{1,3,1} & \tuple{1,3,2} & \tuple{1,3,3} & \tuple{1,3,4} & \dots\\
\hline
\vdots & \vdots & \vdots & \vdots & \vdots & \ddots \\
\end{array}
\]
Thus, by using a method like Cantor's zig-zag method, we may 
similarly obtain an enumeration of $\Nat^3$. 
\end{explain}

\end{document}
