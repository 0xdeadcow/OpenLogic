% Part: sets-functions-relations
% Chapter: relations
% Section: orders

\documentclass[../../include/open-logic-section]{subfiles}

\begin{document}

\olfileid{sfr}{rel}{ord}
\olsection{Orders}

\begin{defn}
A relation which is both reflexive and transitive is called a
\emph{preorder.}  A preorder which is also anti-symmetric is called a
\emph{partial order}. A partial order which is also total is called a
\emph{total order} or \emph{linear order.} (If we want to emphasize
that the order is reflexive, we add the adjective ``weak''---see
below).
\end{defn}

\begin{ex}
Every linear order is also a partial order, and every partial order is
also a preorder, but the converses don't hold. For instance, the
identity relation and the full relation on~$X$ are preorders, but they
are not partial orders, because they are not anti-symmetric (if $X$
has more than one element). For a somewhat less silly example,
consider the \emph{no longer than} relation $\preccurlyeq$
on~$\Bin^*$: $x \preccurlyeq y$ iff $\len(x) \le \len(y)$. This is a
preorder, even a total preorder, but not a partial order.

The relation of \emph{divisibility without remainder} gives us an
example of a partial order which isn't a total order: for integers
$n$, $m$, we say $n$ (evenly) divides $m$, in symbols: $n\mid m$, if
there is some $k$ so that $m=kn$.  On $\Nat$, this is a partial order,
but not a linear order: for instance, $2\nmid3$ and also
$3\nmid2$. Considered as a relation on $\Int$, divisibility is only a
preorder since anti-symmetry fails: $1\mid-1$ and $-1\mid1$ but
$1\neq-1$. Another important partial order is the relation $\subseteq$
on a set of sets.

Notice that the examples $L$ and $G$ from \olref[set]{relations},
although we said there that they were called ``strict orders'' are not
total orders even though they are total. But there is a close
connection, as we will see momentarily.
\end{ex}

\begin{defn}
A relation $R$ on $X$is called \emph{irreflexive} if, for all $x\in
X$, $x\not Rx$. $R$ is called \emph{asymmetric} if for no pair $x,y\in
X$ we have $xRy$ and $yRx$. A \emph{strict partial order} is a
relation which is irreflexive, asymmetric, and transitive. A strict
partial order which is also linear is called a \emph{strict linear
  order.}
\end{defn}

A strict partial order $R$ on $X$ can be turned into a weak partial
order $R'$by adding the identity relation on $X$: $R' = R \cup \Id{X}$.
Conversely, starting from a weak partial order, one can get a strict
partial order by removing~$\Id{X}$

\begin{prop}
$R$ is a strict partial (linear) order on $X$ iff $R'$is a weak
  partial (linear) order. Moreover, $xRy$ iff $xR'y$ for all $x \neq
  y$.
\end{prop}

\begin{ex}
$\le$ is the weak linear order corresponding to the strict linear
  order $<$. $\subseteq$is the weak partial order corresponding to the
  strict partial order $\subsetneq$.
\end{ex}

\begin{prob}
Show that if $R$ is a weak partial order on $X$, then
$R^{-}=R\setminus \Id{X}$ is a strict partial order and $xRy$ iff
$xR^{-}y$ for all $x\neq y$.
\end{prob}

\end{document}
