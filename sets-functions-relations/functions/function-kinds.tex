% Part: sets-functions-relations
% Chapter: functions
% Section: kinds

\documentclass[../../include/open-logic-section]{subfiles}


\begin{document}

\olfileid{sfr}{fun}{kind}
\olsection{Kinds of functions}

\begin{defn}
A function $f \colon X \rightarrow Y$ is \emph{!!{surjective}} iff $Y$ is also
the range of~$f$.
\end{defn}

\begin{explain}
If you want to show that a function is !!{surjective}, then you need
to show that every object in the codomain is the output of the
function given some input or other.
\end{explain}

\begin{defn}
A function $f \colon X \rightarrow Y$ is \emph{!!{injective}} iff for
each $x \in X$ there is at most one $y$ in $Y$ such $f(x) = y$.
\end{defn}

\begin{explain}
Any function pairs each input with a unique output. !!^a{injective} function has a unique input for each possible output. If you want to show that a function $f$ is !!{injective}, you need to show that for any element $y$ of the codomain, if $f(x)=y$ and $f(w)=y$, then $x=w$.

A function which is neither !!{injective}, nor !!{surjective}, is the constant function $f\colon \Nat \rightarrow \Nat$ where $f(x) = 1$.
    
A function which is both !!{injective} and !!{surjective} is the identity function $f\colon \Nat \rightarrow \Nat$ where $f(x) = x$.

The successor function $f \colon \Nat \rightarrow \Nat$ where $f(x) = x+1$ is !!{injective}, but not !!{surjective}.

The function 
\[
f(x) = 
\begin{cases}
  \frac{x}{2} & \text{if $x$ is even} \\
  \frac{x+1}{2} & \text{if $x$ is odd.}
\end{cases}
\]
is !!{surjective}, but not !!{injective}.

\end{explain}

\begin{defn}
\Article{bijection} \emph{!!{bijection}} from $X$ to $Y$ is a
function $f \colon X \rightarrow Y$ which is both !!{surjective} and
!!{injective}.
\end{defn}

\begin{prob}
Show that if $f \colon X \to Y$ and $g \colon Y \to Z$ are both
!!{injective}, then $g \circ f \colon X \to Z$ is !!{injective}.
\end{prob}

\begin{prob}
Show that if $f \colon X \to Y$ and $g \colon Y \to Z$ are both
!!{surjective}, then $g \circ f \colon X \to Z$ is !!{surjective}.
\end{prob}

\begin{prob}
A function $g \colon Y \to X$ is the \emph{inverse} of a function $f
\colon Y \to X$ if $f(g(y)) = y$ and $g(f(x)) = x$ for all $x \in X$
and $y \in Y$.

Show that if $f$ is bijective, such a function~$g$ exists, i.e.,
define such a function and show that it is a function. Then show that
it is also bijective.
\end{prob}

\end{document}
