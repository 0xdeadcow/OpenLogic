% Part: sets-functions-relations
% Chapter: functions
% Section: kinds

\documentclass[../../include/open-logic-section]{subfiles}


\begin{document}

\olfileid{sfr}{fun}{kind}
\olsection{Kinds of functions}

\begin{defn}
A function $f \colon X \rightarrow Y$ is \emph{onto} iff $Y$ is also the range of $f$.
\end{defn}

\begin{explain}
If you want to show that a function is onto, then you need to show that every object in the codomain can be the output of the function given some input or other.
\end{explain}

\begin{defn}
A function $f \colon X \rightarrow Y$ is \emph{one-to-one} iff for each $x$ in $X$ there is at most one $y$ in $Y$ such $f(x)=y$.
\end{defn}

\begin{explain}
Any function pairs each input with a unique output. A one-to-one function has a unique input for each possible output. If a function is one-to-one then its inverse is also a function, whereas the inverse of a many-to-one function is a relation but not a function. If you want to show that a function $f$ is one-to-one, you need to show that for any element $y$ of the codomain, if $f(x)=y$ and $f(w)=y$, then $x=w$.

A function which is neither one-to-one, nor onto, is the constant function $f\colon \Nat \rightarrow \Nat$ where $f(x) = 1$.
    
A function which is both one-to-one and onto is the identity function $f\colon \Nat \rightarrow \Nat$ where $f(x) = x$.

The successor function $f \colon \Nat \rightarrow \Nat$ where $f(x) = x+1$ is one-to-one, but not onto.

The function 
\[
f(x) = 
\begin{cases}
  \frac{x}{2} & \text{if $x$ is even} \\
  \frac{x+1}{2} & \text{if $x$ is odd.}
\end{cases}
\]
is onto, but not one-to-one.

\end{explain}

\begin{defn}
A \emph{bijection} between $X$ to $Y$ is a function $f \colon X \rightarrow Y$ which is both onto and one-to-one.
\end{defn}

\begin{prob}
Show that the inverse of any bijection is also a bijection.
\end{prob}

\end{document}
