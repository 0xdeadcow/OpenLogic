% Part: sets-functions-relations
% Chapter: functions
% Section: basics

\documentclass[../../include/open-logic-section]{subfiles}


\begin{document}

\olfileid{sfr}{fun}{bas}
\olsection{Basics}

\begin{explain}
A function is a relation in which each object is related to a unique
partner. Many functions are familiar to us from basic arithmetic. For
instance, addition and multiplication are functions. They take in two
numbers and return a third. A function, more generally, is something
that takes one or more things as input and returns some kind of
output. A function is a \emph{black box}: what matters is only what
output is paired with what input, not the method for calculating the
output.
\end{explain}

\begin{defn}
A \emph{function} $f \colon X \to Y$ is a mapping of each !!{element}
of~$X$ to an !!{element} of~$Y$. We call $X$ the \emph{domain} of $f$
and $Y$ the \emph{codomain} of $f$. The \emph{range} of $f$ is the
subset of the codomain that is actually output by $f$ for some input.
\end{defn}

\begin{ex}
Multiplication goes from $\Nat \times \Nat$ (the domain) to $\Nat$
(the codomain). As it turns out, the range is also $\Nat$, since every
$n \in \Nat$ is $n \times 1$.
\end{ex}

\begin{explain}
Multiplication is a function because it pairs each input---each pair
of natural numbers---with a single output: $\times \colon \Nat^2 \to
\Nat$. In contrast, the square root operation applied to the domain
$\Nat$ is not functional, since each positive integer $n$ has two
square roots: $\sqrt{n}$ and $-\sqrt{n}$. We can make it functional by
only returning the positive square root: $\sqrt{\phantom{X}} \colon
\Nat \to \Real$. The relation that pairs each student in a class with
their final grade is a function---no student can get two different
final grades in the same class. The relation that pairs each student
in a class with their parents is not a function---generally each
student will have at least two parents.
\end{explain}

\begin{ex}
Let $f \colon \Nat \to \Nat$ be defined such that $f(x) = x+1$. This
tells us that $f$ is a function which takes in natural numbers and
outputs natural numbers. It then tells us that, given a natural
number, $f$ will output its successor.
\end{ex}

\begin{explain}
In this case, the codomain $\Nat$ is not the range of $f$, since the
natural number $0$ is not the successor of any natural number. The
range of~$f$ is the set of all positive integers, $\Int^{+}$.
\end{explain}

\begin{ex}
Let $g \colon \Nat \to \Nat$ be defined such that $g(x) = x-1+2$. This
tells us that $g$ is a function which takes in natural numbers and
outputs natural numbers. It then tells us that, given a natural
number, $g$ will output the successor of the successor of its
predecessor. Despite their different definitions, $g$ and $f$ are the
same function.
\end{ex}

\begin{explain}
Functions $f$ and $g$ defined above are the same because for any
natural number $x$, $x-1+2 = x+1$. $f$ and $g$ pair each natural
number with the same output. Functions, just like relations more
generally, can be treated as just sets of pairs. The definitions for
$f$ and $g$ specify the same set by means of different equations, and
as we know, sets are independent of how they are specified.
\end{explain}

\begin{ex}
We can also define functions by cases. For instance, we could define
$f \colon \Nat \to \Nat$  by
\[
f(x) =
\begin{cases}
  \frac{x}{2} & \text{if $x$ is even} \\
  \frac{x+1}{2} & \text{if $x$ is odd.}
\end{cases}
\]
This is fine, since every natural number is either even or odd, and
the output of this function will always be a natural number. Just
remember that if you define a function by cases, every possible input
must fall into exactly one case.
\end{ex}

\end{document}
