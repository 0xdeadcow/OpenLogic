% Part: sets-functions-relations
% Chapter: functions
% Section: operations

\documentclass[../../include/open-logic-section]{subfiles}


\begin{document}

\olfileid{sfr}{fun}{ops}
\olsection{Operations on functions}

\begin{explain}
We've already seen that the inverse~$f^{-1}$ of a !!{bijective}
function~$f$ is itself a function. It is also possible to compose
functions.
\end{explain}

\begin{defn}
 Let $f: X \rightarrow Y$ and $g: Y \rightarrow Z$. Then we
 \emph{compose} $g$ and $f$ to form the function $(g \circ f) \colon X
 \rightarrow Z$, where $(g \circ f)(x) = g(f(x))$.
\end{defn}

\begin{explain}
The function $(g \circ f) \colon X \rightarrow Z$ pairs each member of
$X$ with a member of~$Z$. We specify which member of $Z$ a member of
$X$ is paired with as follows---given an input $x \in X$, first apply
the function $f$ to $x$, which will output some $y \in Y$. Then apply
the function $g$ to $y$, which will output some $z \in Z$.
\end{explain}

\begin{ex}
Consider the functions $f(x) = x + 1$, and $g(x) = 2x$. What function
do you get when you compose these two? $(g \circ f)(x) = g(f(x))$. So
that means for every natural number you give this function, you first
add one, and then you multiply the result by two. So their composition
is $(g \circ f)(x) = 2(x+1)$.
\end{ex}

\begin{prob}
Show that if the functions $g \colon X \rightarrow Y$ and $f \colon Y
\rightarrow Z$ are both bijections then $(g \circ f) \colon X
\rightarrow Z$ is also a bijection.
\end{prob}

\end{document}
