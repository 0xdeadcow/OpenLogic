% Part: first-order-logic
% Chapter: model-theory
% Section: theory-of-m

\documentclass[../../include/open-logic-section]{subfiles}

\begin{document}

\olfileid{mod}{bas}{thm}
\section{The Theory of a \printtoken{S}{structure}}

\begin{defn}
  Given a !!{structure}~$\Struct M$, the \emph{theory} of
  $\Struct{M}$ is the set $\Theory{M}$ of !!{sentence}s
  that are true in $\Struct{M}$, i.e., $\Theory{M} =
  \Setabs{!A}{\Sat{M}{!A}}$.
\end{defn}

We also use the term ``theory'' informally to refer to sets
of !!{sentence}s having an intended interpretation, whether deductively
closed or not.

\begin{prop}\ollabel{prop:equiv}
  For any $\Struct{M}$,  $\Theory{M}$ is maximally
  consistent. Hence, if $\Struct{N} \models !A$ for every $!A
  \in \Theory{M}$, then $\Struct{M} \elemequiv
  \Struct{N}$.
\end{prop}

\begin{proof}
  $\Theory{M}$ is consistent because satisfiable (by definition). It
  is maximal since for any \emph{sentence} $!A$ either $!A$ is true in
  $\Struct{M}$ or its negation is. It immediately follows that
  $\Theory{M} \subseteq \Theory{N}$ and $\Theory{N} \subseteq
  \Theory{M}$, whence $\Struct{M} \elemequiv \Struct{N}$.
\end{proof}

\begin{rem}\ollabel{remark:R}
  Consider $\Struct{R} = \langle\Real, <\rangle$, the !!{structure}
  whose domain is the set $\Real$ of the real numbers, in the language
  comprising only a 2-place !!{predicate} interpreted as the $<$
  relation over the reals. Clearly $\Struct{R}$ is !!{nonenumerable};
  however, since $\Theory{R}$ is obviously consistent, by the
  L\"owenheim-Skolem theorem it has !!a{enumerable} model, say
  $\Struct{S}$, and by \olref{prop:equiv}, $\Struct{R}
  \equiv \Struct{S}$. Moreover, since $\Struct{R}$ and $\Struct{S}$
  are not isomorphic, this shows that the converse of
  \olref[iso]{thm:isom} fails in general.
\end{rem}

\end{document}
