% Part: first-order-logic
% Chapter: model-theory
% Section: reducts-and-expansions

\documentclass[../../include/open-logic-section]{subfiles}

\begin{document}

\olfileid{mod}{bas}{red}
\section{Reducts and Expansions}

Often it is useful or necessary to compare languages which have
symbols in common, as well as !!{structure}s for these languages.  The
most comon case is when all the symbols in !!a{language}~$\Lang{L}$
are also part of !!a{language}~$\Lang{L'}$, i.e., $\Lang{L} \subseteq
\Lang{L'}$. An $\Lang{L}$-!!{structure}~$\Struct{M}$ can then always
be expanded to an $\Lang{L'}$-!!{structure} by adding interpretations
of the additional symbols while leaving the interpretations of the
common symbols the same.  On the other hand, from an
$\Lang{L'}$-structure~$\Struct{M'}$ we can obtain an
$\Lang{L}$-structure simpy by ``forgetting'' the interpretations of
the symbols that do not occur in~$\Lang{L}$.

\begin{defn}
\ollabel{defn:reduct}
Suppose $\Lang L \subseteq \Lang L'$, $\Struct M$ is an
$\Lang L$-!!{structure} and $\Struct M'$ is an $\Lang L'$-!!{structure}.
$\Struct M$ is the \emph{reduct} of $\Struct M'$ to $\Lang L$, and
$\Struct M'$ is an \emph{expansion} of $\Struct M$ to $\Lang L'$ iff
\begin{enumerate}
\item $\Domain{M} = \Domain{M'}$
\item For every !!{constant}~$c \in \Lang L$, $\Assign{c}{M} =
  \Assign{c}{M'}$.
\item For every !!{function}~$f \in \Lang L$, $\Assign{f}{M} =
  \Assign{f}{M'}$.
\item For every !!{predicate}~$P \in \Lang L$, $\Assign{P}{M} =
  \Assign{P}{M'}$.
\end{enumerate}
\end{defn}

\begin{prop}
\ollabel{prop:reduct}
If an $\Lang{L}$-!!{structure}~$\Struct{M}$ is a reduct of an
$\Lang{L'}$-!!{structure} $\Struct{M'}$, then for all
$\Lang{L}$-!!{sentence}s~$!A$,
\[
\Sat{M}{!A} \text{ iff } \Sat{M'}{!A}.
\]
\end{prop}

\begin{proof}
  Exercise.
\end{proof}

\begin{prob}
Prove \olref[mod][bas][red]{prop:reduct}.
\end{prob}

\begin{defn}
When we have an $\Lang{L}$-structure $\Struct{M}$, and $\Lang{L'} =
\Lang{L} \cup \{P\}$ is the expansion of $\Lang{L}$ obtained by adding
a single $n$-place !!{predicate}~$P$, and $R \subseteq \Domain{M}^n$
is an $n$-place relation, then we write $\Expan{M}{R}$ for the
expansion~$\Struct{M'}$ of~$\Struct{M}$ with $\Assign{P}{M'} = R$.
\end{defn}

\end{document}
