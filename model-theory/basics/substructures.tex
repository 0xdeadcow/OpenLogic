% Part: first-order-logic
% Chapter: model-theory
% Section: substructures

\documentclass[../../include/open-logic-section]{subfiles}

\begin{document}

\olfileid{mod}{bas}{sub}
\olsection{Sub\printtoken{p}{structure}}

\begin{defn}
\ollabel{defn:substructure}
Given !!{structure}s $\Struct M$ and $\Struct M'$ for the same
language~$\Lang L$, we say that $\Struct M$ is a \emph{sub!!{structure}}
of $\Struct M'$, and $\Struct M'$ an \emph{extension} of $\Struct M$,
written $\Struct M \substruct \Struct M'$, iff
\begin{enumerate}
\item $\Domain{M} \subseteq \Domain{M'}$,
\item For each constant $c \in \Lang L$, $\Assign{c}{M} =
    \Assign{c}{M'}$;
\item For each $n$-place !!{predicate} $f \in \Lang L$
  $\Assign{f}{M}(a_1, \dots, a_n) = \Assign{f}{M'}(a_1, \dots, a_n)$
  for all $a_1$, \dots, $a_n \in \Domain{M}$.
\item For each $n$-place !!{predicate} $R \in \Lang L$, $\langle
  a_1, \dots, a_n\rangle \in \Assign{R}{M}$ iff $\langle a_1, \dots,
  a_n\rangle \in \Assign{R}{M'}$ for all $a_1$, \dots, $a_n \in
  \Domain{M}$.
\end{enumerate}
\end{defn}

\begin{rem}
\ollabel{rem:substructure}
If the language contains no constant or !!{function}s, then any $N
\subseteq \Domain{M}$ determines a sub!!{structure}~$\Struct N$ of
$\Struct M$ with !!{domain}~$\Domain{N} = N$ by putting $\Assign{R}{N} =
\Assign{R}{M} \cap N^n$.
\end{rem}

% prove something about this? Examples?

\end{document}
