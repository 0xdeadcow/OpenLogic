% Part: model-theory
% Chapter: interpolation
% Section: interpolation-proof

\documentclass[../../include/open-logic-section]{subfiles}

\begin{document}

\olfileid{mod}{int}{prf}

\olsection{Craig's Interpolation Theorem}

\begin{thm}[Craig's Interpolation Theorem]
\ollabel{thm:interpol} 
If $\Entails !A \lif !B$, then there is a !!{sentence} $!C$ such that
$\Entails !A \lif !C$ and $\Entails !C \lif !B$, and every
!!{constant}, !!{function}, and !!{predicate} (other than $\eq$) in
$!C$ occurs both in $!A$ and~$!B$. The !!{sentence} $!C$ is called an
\emph{interpolant} of $!A$ and~$!B$.
\end{thm}

\begin{proof}
Suppose $\Lang{L}_1$ is the language of $!A$ and $\Lang{L}_2$ is the
language of $!B$. Let $\Lang{L}_0 = \Lang{L}_1 \cap \Lang{L}_2$. For
each $i \in \{0, 1, 2 \}$, let $\Lang{L}'_i$ be obtained from
$\Lang{L}_i$ by adding the infinitely many new !!{constant}s $\Obj c_0,
\Obj c_1, \Obj c_2, \dots$. 

If $!A$ is unsatisfiable, $\lexists[x][\eq/[x][x]]$ is an
interpolant. If $\lnot !B$ is unsatisfiable (and hence $!B$ is valid),
$\lexists[x][\eq[x][x]]$ is an interpolant. So we may assume also that
both $!A$ and $\lnot !B$ are satisfiable.

In order to prove the contrapositive of the Interpolation Theorem,
assume that there is no interpolant for $!A$ and $!B$. In other words,
assume that $\{!A\}$ and $\{\lnot !B\}$ are inseparable in
$\Lang{L}_0$.

Our goal is to extend the pair $(\{ !A \}, \{\lnot!B\})$ to a
maximally inseparable pair $(\Gamma^*, \Delta^*)$.  Let $!A_0$,
$!A_1$, $!A_2$, \dots enumerate the !!{sentence}s of $\Lang{L}_1$, and
$!B_0$, $!B_1$, $!B_2$, \dots enumerate the !!{sentence}s
of~$\Lang{L}_2$. We define two increasing sequences of sets of
!!{sentence}s $(\Gamma_n, \Delta_n)$, for $n \ge 0$, as follows. Put
$\Gamma_0 = \{ !A\}$ and $\Delta_0 = \{\lnot !B \}$. Assuming
$(\Gamma_n, \Delta_n)$ are already defined, define $\Gamma_{n+1}$ and
$\Delta_{n+1}$ by:
\begin{enumerate}
\item If $\Gamma_n \cup \{!A_n \}$ and $\Delta_n$ are inseparable in
  $\Lang{L}'_0$, put $!A_n$ in $\Gamma_{n+1}$. Moreover, if $!A_n$ is
  an existential !!{formula} $\lexists[x][!S]$ then pick a new
  !!{constant} $c$ not occurring in $\Gamma_n$, $\Delta_n$, $!A_n$ or
  $!B_n$, and put $\Subst{!S}{c}{x}$ in $\Gamma_{n+1}$.
\item If $\Gamma_{n+1}$ and $\Delta_n \cup \{!B_n \}$ are inseparable
  in $\Lang{L}'_0$, put $!B_n$ in $\Delta_{n+1}$. Moreover, if $!B_n$
  is an existential !!{formula} $\lexists[x][!S]$, then pick a new
  !!{constant} $c$ not occurring in $\Gamma_{n+1}$, $\Delta_n$, $!A_n$
  or $!B_n$, and put $\Subst{!S}{c}{x}$ in $\Delta_{n+1}$.
\end{enumerate}
Finally, define:
\begin{align*}
  \Gamma^* & = \bigcup_{n\ge 0} \Gamma_n, & 
  \Delta^* & = \bigcup_{n\ge 0} \Delta_n.
\end{align*}
By simultaneous induction on $n$ we can now prove:
\begin{enumerate}
\item\ollabel{part-a} $\Gamma_n$ and $\Delta_n$ are inseparable in
  $\Lang{L}'_0$;
\item\ollabel{part-b} $\Gamma_{n+1}$ and $\Delta_n$ are inseparable in
    $\Lang{L}'_0$.
\end{enumerate}
The basis for \olref{part-a} is given by \olref[sep]{lem:sep1}. For
part \olref{part-b}, we need to distinguish three cases:
\begin{enumerate}
\item If $\Gamma_0 \cup \{!A_0 \}$ and $\Delta_0$ are separable, then
  $\Gamma_1 = \Gamma_0$ and \olref{part-b} is just \olref{part-a};
\item If $\Gamma_1 = \Gamma_0 \cup\{ !A_0\}$, then $\Gamma_1$ and
  $\Delta_0$ are inseparable by construction.
\item It remains to consider the case where $!A_0$ is existential, so
  that $\Gamma_1 = \Gamma_0 \cup \{ \lexists[x][!S], \Subst{!S}{c}{x}
  \}$. By construction, $\Gamma_0 \cup \{ \lexists[x][!S]\}$ and
  $\Delta_0$ are inseparable, so that by \olref[sep]{lem:sep2} also
  $\Gamma_0 \cup \{ \lexists[x][!S], \Subst{!S}{c}{x} \}$ and
  $\Delta_0$ are inseparable.
\end{enumerate}
This completes the basis of the induction for \olref{part-a} and
\olref{part-b} above. Now for the inductive step. For \olref{part-a}, if
$\Delta_{n+1} = \Delta_n \cup \{ !B_n \}$ then $\Gamma_{n+1}$ and
$\Delta_{n+1}$ are inseparable by construction (even when $!B_n$ is
existential, by \olref[sep]{lem:sep2}); if $\Delta_{n+1} = \Delta_n$
(because $\Gamma_{n+1}$ and $\Delta_n \cup \{!B_n\}$ are separable),
then we use the induction hypothesis on \olref{part-b}. For the
inductive step for \olref{part-b}, if $\Gamma_{n+2} = \Gamma_{n+1} \cup
\{!A_{n+1} \}$ then $\Gamma_{n+2}$ and $\Delta_{n+1}$ are
inseparable by construction (even when $!A_{n+1}$ is existential,
by \olref[sep]{lem:sep2}); and if  $\Gamma_{n+2} = \Gamma_{n+1}$ then
we use the inductive case for \olref{part-a} just proved. This
concludes the induction on \olref{part-a} and \olref{part-b}. 

It follows that $\Gamma^*$ and $\Delta^*$ are inseparable; if not, by
compactness, there is $n \ge 0$ that separates $\Gamma_n$ and
$\Delta_n$, against \olref{part-a}. In particular, $\Gamma^*$ and
$\Delta^*$ are consistent: for if the former or the latter is
inconsistent, then they are separated by $\lexists[x][\eq/[x][x]]$ or
  $\lforall[x][\eq[x][x]]$, respectively.

We now show that $\Gamma^*$ is maximally consistent in
$\Lang{L}'_1$ and likewise $\Delta^*$ in $\Lang{L}'_2$. For the
former, suppose that $!A_n \notin \Gamma^*$ and $\lnot !A_n
\notin \Gamma^*$, for some $n \ge 0$. If $!A_n \notin \Gamma^*$
then $\Gamma_n \cup \{!A_n \}$ is separable from $\Delta_n$, and
so there is $!C \in \Lang{L}'_0$ such that both:
\begin{align*}
  \Gamma^* & \Entails !A_n \lif !C, & 
  \Delta^* & \Entails \lnot !C.
\end{align*}
Likewise, if $\lnot !A_n \notin \Gamma^*$, there is $!C' \in
\Lang{L}'_0$ such that both:
\begin{align*}
  \Gamma^* & \Entails \lnot !A_n \lif !C', & 
  \Delta^* & \Entails \lnot !C'.
\end{align*}
By propositional logic, $\Gamma^* \Entails !C \lor !C'$ and
$\Delta^* \Entails \lnot (!C \lor !C')$, so $!C \lor
!C'$ separates $\Gamma^*$and $\Delta^*$. A similar argument
establishes that $\Delta^*$ is maximal. 

Finally, we show that $\Gamma^* \cap \Delta^*$ is maximally consistent
in $\Lang{L}'_0$. It is obviously consistent, since it is the
intersection of consistent sets. To show maximality, let $!S \in
\Lang{L}'_0$. Now, $\Gamma^*$ is maximal in $\Lang{L'_1}
\supseteq \Lang{L'_0}$, and similarly $\Delta^*$ is maximal in
$\Lang{L'_2} \supseteq \Lang{L'_0}$. It follows that either
$!S \in \Gamma^*$ or $\lnot !S \in \Gamma^*$, and either
$!S \in \Delta^*$ or $\lnot !S \in \Delta^*$. If $!S \in
\Gamma^*$ and $\lnot !S \in \Delta^*$ then $!S$ would
separate $\Gamma^*$ and $\Delta^*$; and if $\lnot !S \in
\Gamma^*$ and $!S \in \Delta^*$ then $\Gamma^*$ and $\Delta^*$
would be separated by $\lnot !S$. Hence, either $!S \in
\Gamma^* \cap \Delta^*$ or $\lnot !S \in \Gamma^* \cap \Delta^*$,
and $\Gamma^* \cap \Delta^*$ is maximal. 

Since $\Gamma^*$ is maximally consistent, it has a model
$\Struct{M}'_1$ whose !!{domain} $\Domain{M'_1}$ comprises all and
only the elements $\Assign{c}{M'_1}$ interpreting the
!!{constant}s---just like in the proof of the completeness theorem
(\olref[fol][com][cth]{thm:completeness}). Similarly, $\Delta^*$ has a
model $\Struct{M}'_2$ whose !!{domain} $\Domain{M'_2}$ is given by the
interpretations $\Assign{c}{M'_2}$ of the !!{constant}s.

Let $\Struct{M_1}$ be obtained from $\Struct{M'_1}$ by dropping
interpretations for !!{constant}s, !!{function}s, and !!{predicate}s in
$\Lang{L'_1} \setminus \Lang{L'_0}$, and similarly for
$\Struct{M_2}$. Then the map $h \colon M_1 \to M_2$ defined by
$h(\Assign{c}{M'_1}) = \Assign{c}{M'_2}$ is an
isomorphism in $\Lang{L}'_0$, because $\Gamma^* \cap \Delta^*$ is
maximally consistent in $\Lang{L}'_0$, as shown. This follows
because any $\Lang{L}'_0$-!!{sentence} either belongs to both
$\Gamma^*$ and $\Delta^*$, or to neither: so $\Assign{c}{M'_1} \in
\Assign{P}{M'_1}$ if and only if $\Atom{P}{c} \in \Gamma^*$ if and only if
$\Atom{P}{c} \in \Delta^*$ if and only if $\Assign{c}{M'_2} \in
\Assign{P}{M'_2}$. The other conditions satisfied by isomorphisms
can be established similarly.

Let us now define a model $\Struct{M}$ for the !!{language}
$\Lang{L_1} \cup \Lang{L_2}$ as follows:
\begin{enumerate}
\item The !!{domain} $\Domain{M}$ is just $\Domain{M_2}$, i.e., the
  set of all elements $\Assign{c}{M'_2}$; 
\item If !!a{predicate}~$P$ is in $\Lang{L_2} \setminus
  \Lang{L_1}$ then $\Assign{P}{M} = \Assign{P}{M'_2}$;
\item If a predicate $P$ is in $\Lang{L}_1\setminus \Lang{L}_2$ then
  $\Assign{P}{M} = h(\Assign{P}{M'_2})$, i.e.,
  $\tuple{\Assign{c_1}{M'_2}, \dots, \Assign{c_n}{M'_2}} \in
  \Assign{P}{M}$ if and only if $\tuple{\Assign{c_1}{M'_1}, \dots,
  \Assign{c_n}{M'_1}} \in \Assign{P}{M'_1}$.
\item If !!a{predicate} $P$ is in $\Lang{L}_0$ then $\Assign{P}{M} =
  \Assign{P}{M'_2} = h(\Assign{P}{M'_1})$. 
\item !!^{function}s of $\Lang{L}_1 \cup \Lang{L}_2$, including
  !!{constant}s, are handled similarly.
\end{enumerate}

Finally, one shows by induction on !!{formula}s that $\Struct{M}$ agrees
with $\Struct{M'_1}$ on all !!{formula}s of $\Lang{L'_1}$ and with
$\Struct{M'_2}$ on all !!{formula}s of $\Lang{L'_2}$. In particular,
$\Struct{M} \Entails \Gamma^* \cup \Delta^*$, whence $\Struct{M}
\Entails !A$ and $\Struct{M} \Entails \lnot!B$, and
$\not\Entails !A \lif !B$. This concludes the proof of
Craig's Interpolation Theorem.
\end{proof}

\end{document}
