% Part: turing-machines
% Chapter: machines-computations
% Section: unary-numbers

\documentclass[../../include/open-logic-section]{subfiles}

\begin{document}

\olfileid{tur}{mac}{una}
\olsection{Unary Representation of Numbers}

\begin{explain}
Turing machines work on sequences of symbols written on their tape.
Depending on the alphabet a Turing machine uses, these sequences of
symbols can represent various inputs and outputs.  Of particular
interest, of course, are Turing machines which compute
\emph{arithmetical} functions, i.e., functions of natural numbers.
A simple way to represent positive integers is by coding them
as sequences of a single symbol~$\TMstroke$.
\end{explain}

\begin{defn}
A Turing machine~$M$ \emph{computes} the function $f\colon \Nat^n \to \Nat$ iff
$M$ halts on input
\[
\TMstroke^{k_1} \TMblank \TMstroke^{k_2} \TMblank \dots \TMblank \TMstroke^{k_n}
\]
with output $\TMstroke^{f(k_1, \dots, k_n)}$.
\end{defn}

\begin{ex} \emph{Addition:}
Build a machine that when given an input of two non-empty strings of strokes~$n$ and
$m$, computes the function $f(n,m) = n + m$.

We want to come up with a machine that starts with two blocks of strokes on the tape,
and halts with one block of strokes. We first need a method to carry out. We know that 
the input strokes are separated by a blank, so one method would be to write a stroke on
the square containing the blank, and erase the first (or last) stroke. This would result in
a block of~$n + M$ strokes. Alternatively, we could proceed in a similar way to the
doubler, by erasing a stroke from the first block, and adding one to the second block of 
strokes until the first block has been removed completely. We will proceed with the
former example.
\[
\begin{tikzpicture}[->,>=stealth',shorten >=1pt,auto,node distance=2.8cm,
                    semithick]
  \tikzstyle{every state}=[fill=none,draw=black,text=black]

  \node[state]         (A)                     {$q_1$};
  \node[state]         (B) [right of=A] {$q_2$};
  \node[state]         (C) [right of=B] {$q_3$};

  \path (A) edge                      node {0,1,R} (B)
                  edge [loop above] node {1,1,R} (B)
            (B) edge [loop above] node {1,1,R} (B)
                  edge                      node {0,0,L} (C)
            (C) edge [loop above] node {1,0,N} (C);
\end{tikzpicture}
\]
\end{ex}

\begin{prob}
Show all the configurations of the machine for input $\tuple{3,5}$.
\end{prob}

\begin{prob}
\emph{Subtraction:} Design a Turing machine that when given an input of two non-empty 
strings of strokes of length \emph{n} and \emph{m}, where $n > m$, outputs a string of 
strokes of length \emph{n-m}. 
\end{prob}

\begin{prob}
\emph{Equality:} Design a Turing machine to compute the following function:
\[
equality(x,y) = 
\begin{cases}
  \text{1} & \text{if~$x = y$} \\
  \text{0} & \text{if~$x \neq y$}
\end{cases}
\]
where~$x$ and~$y$ are integers greater than~$0$.
\end{prob}

\end{document}
