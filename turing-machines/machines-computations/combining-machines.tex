% Part: turing-machines
% Chapter: machines-computations
% Section: combining-machines

\documentclass[../../include/open-logic-section]{subfiles}

\begin{document}

\olfileid{tur}{mac}{cmb}
\olsection{Combining Turing Machines}

\begin{explain}
As mentioned before, Turing machines have the same computing
power as a modern-day computer. The examples we have shown
so far have been fairly simple in nature. Sometimes, when computing
a complex function, it is beneficial to break the problem down into 
separate parts.

One of the key features of Turing machines is the ability for them to
be combined or hooked together. If we can find a natural way to break
a complex problem down into its constituent parts, we can tackle the
problem in several stages, creating several simple Turing machines and
combining then into one machine that can solve the problem. This point
is especially important when tackling the Halting Problem.
\end{explain}

\begin{ex}
\emph{Combining Machines:} Design a machine that computes the function
$f(m,n) = 2(m+n)$.

In order to build this machine, we can combine two machines we are already
familiar with: the addition machine, and the doubler. We begin by drawing 
up a state diagram for the addition machine.
\[
\begin{tikzpicture}[->,>=stealth',shorten >=1pt,auto,node distance=2.8cm,
                    semithick]
  \tikzstyle{every state}=[fill=none,draw=black,text=black]

  \node[state]         (A)                     {$q_0$};
  \node[state]         (B) [right of=A] {$q_1$};
  \node[state]         (C) [right of=B] {$q_2$};

  \path (A) edge                      node {0,1,R} (B)
                  edge [loop above] node {1,1,R} (B)
            (B) edge [loop above] node {1,1,R} (B)
                  edge                      node {0,0,L} (C)
            (C) edge [loop above] node {1,0,N} (C);
\end{tikzpicture}
\]
Instead of halting at state $q_2$, we want to continue operation in
order to double the output. Recall that the doubler machine erases
the first stroke in the input and writes two strokes in a separate
output. Let's add an instruction to made sure the tape head is reading 
the first stroke of the output of the addition machine.
\[
\begin{tikzpicture}[->,>=stealth',shorten >=1pt,auto,node distance=2.8cm,
                    semithick]
  \tikzstyle{every state}=[fill=none,draw=black,text=black]

  \node[state]         (A)                     {$q_0$};
  \node[state]         (B) [right of=A] {$q_1$};
  \node[state]         (C) [right of=B] {$q_2$};
  \node[state]         (D) [right of=C] {$q_3$};
  \node[state]         (E) [right of=D] {$q_4$};

  \path (A) edge                      node {0,1,R} (B)
                  edge [loop above] node {1,1,R} (B)
            (B) edge [loop above] node {1,1,R} (B)
                  edge                      node {0,0,L} (C)
            (C) edge [loop above] node {1,0,L} (C)
                  edge                      node {1,1,L} (D)
            (D) edge [loop above] node {1,1,L} (D)
                  edge                      node {0,0,R} (E);
\end{tikzpicture}
\]
It is now easy to double the input - all we have to do is connect the
doubler machine onto state $q_4$! This means renaming the states 
of the doubler machine so that they start at $q_4$ instead of $q_0$
- this way we don't end up with two starting states. The final diagram
should look like:
\[
\begin{tikzpicture}[->,>=stealth',shorten >=1pt,auto,node distance=2.8cm,
                    semithick]
  \tikzstyle{every state}=[fill=none,draw=black,text=black]
  \node[state]         (A)                     {$q_0$};
  \node[state]         (B) [right of=A] {$q_1$};
  \node[state]         (C) [right of=B] {$q_2$};
  \node[state]         (D) [right of=C] {$q_3$};
  \node[state]         (E) [right of=D] {$q_4$};
  \node[state]         (F) [below of=E] {$q_5$};
  \node[state]         (G) [left of=F]    {$q_6$};
  \node[state]         (H) [left of=G]   {$q_7$};
  \node[state]         (I) [left of=H]       {$q_8$};
  \node[state]         (J) [left of=I]       {$q_9$};

    \path (A) edge                      node {0,1,R} (B)
                  edge [loop above] node {1,1,R} (B)
            (B) edge [loop above] node {1,1,R} (B)
                  edge                      node {0,0,L} (C)
            (C) edge [loop above] node {1,0,L} (C)
                  edge                      node {1,1,L} (D)
            (D) edge [loop above] node {1,1,L} (D)
                  edge                      node {0,0,R} (E)
 	 (E) edge              node {1,0,R} (F)
            (F) edge [loop below] node {1,1,R} (F)
                 edge              node {0,0,L} (G)
            (G) edge [loop below] node {1,1,R} (G)
                 edge  node {0,1,R} (H)
            (H) edge [loop below] node {0,1,L} (H)
                 edge              node {1,1,L} (I)
            (I) edge [loop below]  node {1,1,L} (I)
                 edge              node {0,0,L} (J)
            (J) edge [loop below] node {1,1,L} (J)
                 edge              node {0,0,R} (A);
\end{tikzpicture}
\]

\end{ex}

\end{document}