% Part: turing-machines
% Chapter: undecidability
% Section: state-diagrams

\documentclass[../../include/open-logic-section]{subfiles}

\begin{document}

\olfileid{tms}{tms}{dia}
\olsection{Drawing State Diagrams}




How do we know the machine works? One way is to check it by tracking
its behaviour through several different inputs. We can do this by defining
the \emph{configuration} of the macine at a specific time.



\begin{tikzpicture}[->,>=stealth',shorten >=1pt,auto,node distance=2.8cm,
                    semithick]
  \tikzstyle{every state}=[fill=none,draw=black,text=black]

  \node[initial,state] (A)                    {$q_a$};
  \node[state]         (B) [above right of=A] {$q_b$};
  \node[state]         (D) [below right of=A] {$q_d$};
  \node[state]         (C) [below right of=B] {$q_c$};
  \node[state]         (E) [below of=D]       {$q_e$};

  \path (A) edge              node {0,1,L} (B)
            edge              node {1,1,R} (C)
        (B) edge [loop above] node {1,1,L} (B)
            edge              node {0,1,L} (C)
        (C) edge              node {0,1,L} (D)
            edge [bend left]  node {1,0,R} (E)
        (D) edge [loop below] node {1,1,R} (D)
            edge              node {0,1,R} (A)
        (E) edge [bend left]  node {1,0,R} (A);
\end{tikzpicture}


\begin{explain}
Although it is possible to represent Turing machines using only the
language we have mentioned, it may be useful to have a visual
representation of the machines available. Such visual, or
diagrammatic, representations of Turing machines are called
\emph{state diagrams}. The diagrams are composed of state cells
connected by arrows. The state cells represent the various states of
the machine. Each arrow represents a possible instruction to be
carried out from that state. Above each arrow is written the
instruction. For example, the instruction $\TMblank :\TMstroke
:\TMright$ on an arrow leading from state $q_0$ can be read as
\emph{if reading $\TMblank$ in state $q_0$, write $\TMstroke$, move
  right and go to state $q_1$}. This is equivalent to the instruction
$\tuple{q_0, \TMblank, \TMstroke, q_1}$.

It is easy to convert between state diagrams and instruction sets. The
following instruction set is for a Turing machine $T_Z$ that takes all
of the $\TMstroke$s on the tape and converts them to $\TMblank$s
(i.e., it computes the function $f(n) = 0$). The diagram.
\end{explain}

\end{document}
