% Part: turing-machines
% Chapter: undecidability
% Section: undecidability-proof

\documentclass[../../include/open-logic-section]{subfiles}

\begin{document}

\olfileid{tms}{und}{unp}
\olsection{The Undecidability Proof}

\begin{proof}
\begin{enumerate}
\item We know that given input $w$ and a Turing machine $M$ we can
effectively describe a set of !!{sentence}s $\Atom{T}{M,w}$ and
!!a{sentence} $\Atom{H}{M,w}$ (in the
language of first order-logic with identity and terms) such that:

$\Sat{T(M,w)}{\Atom{H}{M,w}}$ if, and only if, $M$ halts of input $w$.

So if $\Lang{L_T}$ is decidable (i.e., if I can decide whether
$\Sat{}{\Atom{H}{M,w}}$ or not) then the halting problem is
solvable. We need to
show that if there is a decision procedure for consequence in $\Lang{L_=}$
then the halting problem is solvable.

Since $\Atom{T}{M,w}$ is finite, $\Sat{T(M,w)}{ \Atom{H}{M,w} }$
iff $\Sat{}{\Atom{T}{M,w}_\land \lif
\Atom{H}{M,w}}$, where $\Atom{T}{M,w}_\land$ is the !!{sentence} that
results from conjoining all
the members of $\Atom{T}{M,w}$ together.

Next, there is a !!{sentence} $!B$ of $\Lang{L_=}$ that is valid iff
$\Atom{T}{M,w}_\land\lif \Atom{H}{M,w}$ is. In general, for any
!!{sentence} $!A$ of $\Lang{L_T}$ ,there is
!!a{sentence} of $\Lang{L_=}$ with the form
$\lforall[x]\lexists[\fac y][\Atom{R}{x,y}\lif !A']$ which is equally valid.
So,$\Sat{}{!A}$ if, and only if,
$\Sat{}{\lforall[x]\lexists[\fac y][\Atom{R}{x,y}\lif !A']}$.

We construct $\Sat{}{\lforall[x]\lexists[\fac y][\Atom{R}{x,y}\lif !A']}$ as
follows:

Given !!a{sentence} $!A$ of $\Lang{L_T}$, first "normalize" $!A$ to form
$!A*$.
We normalize !!a{sentence} by converting it to an equivalent !!{sentence}
in
which all names and !!{function}s appear only as arguments for the
logical constant identity. This means that all other predicates have
only !!{variable}s as their arguments.

Take $\Atom{P}{f,a}$ for example. We normalize $\Atom{P}{f,a}$ in two
steps. First, we take the name $a$ out of the scope of the !!{predicate}
$P$ as follows:

$\lexists[x][(\eq[x][a] \land \Atom{P}{f,x})]$

Second, we take the !!{function} out of the scope of the
!!{predicate} $P$ as follows:

$\lexists[x][(\eq[x][a] \land \lexists[y][(\eq[fx][y] \land
\Atom{P}{y}]))]$

Notice that $ !A \simeq !A*$.

Then from the normalized $!A*$, form $!A'$ by replacing all
!!{subformula}s of the form $\eq[ft][\delta]$ with the 
!!{subformula} $\Atom{R}{t,\delta}$, where $R$
is a 2-place !!{predicate} which does not appear in $!A*$.

Since $f$ is a 1-place !!{function}, whereas $R$ is a binary relation, and
there is no guarantee that $R$ is functional, i.e. that there is only one
object d such that $\Sat{M}{\Subst{\Atom{R}{t,\delta}}{\delta}{d}}$.
 What we need to do is force $R$ to be a functional relation, and this 
is what $\lforall[x]\lexists[\fac y][\Atom{R}{x,y}]$ does.

To continue the example, $\lexists[x][(\eq[x][a] \land
\lexists[\fac y][(\eq[fx][y] \land \Atom{P}{y}))]]$ becomes
$\lexists[x][(\eq[x][a] \land \lexists[y][(\Atom{R}{fx,y} 
\land \Atom{P}{y}))]]$.

Now form the !!{sentence} $\lforall[x]\lexists[\fac y][\Atom{R}{x,y}]\lif !A'$.


\item We now want to show that for any !!{sentence} $A$ of $L_T$,
$\Sat{}{\lforall[x]\lexists[\fac y][\Atom{R}{x,y}]\lif !A'}$ if, and only if,
$\Sat{}{!A}$. We will prove this in its contrapositive form.

Right to left: Suppose that
$\Sat/{}{\lforall[x]\lexists[\fac y][\Atom{R}{x,y}]\lif !A'}$. So there is some
!!{structure} $\Struct M$ such that
$\Sat{M}{\lforall[x]\lexists[\fac y][\Atom{R}{x,y}]}$ and $\Sat/{M}{!A'}$.

Based on the !!{structure} $\Struct M$, create a new !!{structure}
$\Struct M^*$ which is identical to $\Struct M$ except that we turn the
!!{predicate} $R$ back into $\fn{f}$. That is, whatever functional
relationis
assigned to $R$ in $\Struct M$ should be assigned to $\fn{f}$ in
$\Struct M^*$. Clearly, $\Sat/{M^*}{!A}$. We can give an inductive
argumenton the complexity of $!A$ to this effect. So, $\Sat/{}{!A}$.

Left to Right: Suppose $\Sat/{}{!A}$. So there is some !!{structure}
$\Struct M$ such that $\Sat/{M}{!A}$.

Based on the !!{structure} $\Struct M$, create a new !!{structure}
$\Struct M'$ which is identical to $\Struct M$ except that the functional
relation assigned to $\fn{f}$ in $\Struct M$ is assigned to $R$ in
$\Struct M'$. Since $R$ is assigned a functional relationship in
$\Struct M$, $\Sat{M}{\lforall[x]\lexists[\fac y][\Atom{R}{x,y}]}$. But,
clearly, $\Sat/{M}{!A'}$ (again, we can give an inductive argument to this
effect). So, $\Sat/{M}{\lforall[x]\lexists[\fac y][\Atom{R}{x,y}]\lif !A'}$,
and$\Sat/{}{\lforall[x]\lexists[\fac y][\Atom{R}{x,y}]\lif !A'}$.

\item In general, $\Sat/{}{\lforall[x]\lexists[\fac y][\Atom{R}{x,y}]\lif !A'}$
if, and only if, $\Sat/{\lforall[x]\lexists[\fac y][\Atom{R}{x,y}]}{!A'}$. So,
$\Sat{}{\Atom{T}{M,w}_\land \lif \Atom{H}{M,w}}$ iff
$\Sat/{\lforall[x]\lexists[\fac y][\Atom{R}{x,y}]}{(\Atom{T}{M,w}_\land \lif
\Atom{H}{M,w})'}$
(there is just one !!{function} in $\Atom{T}{M,w}_\land,
\Atom{H}{M,w}$).

Thus we can immediately conclude that if FOL with !!{identity} is
decidable,
then the halting problem is solvable. We know that
$\Sat{\Atom{T}{M,w}_\land}{\Atom{H}{M,w}}$
iff $\Sat{}{\Atom{T}{M,w}_\land \lif \Atom{H}{M,w}}$, and that
$\Sat{}{\Atom{T}{M,w}_\land \lif \Atom{H}{M,w}}$
iff $\Sat/{\lforall[x]\lexists![y][\Atom{R}{x,y}]}{(\Atom{T}{M,w}_\land
\lif\Atom{H}{M,w})'}$. So, if we have a decision procedure for consequence
in
$\Lang{L_=}$
then we can decide whether $\Atom{T}{M,w}_\land \lif \Atom{H}{M,w}$ is
valid or not, and thus
whether $\Atom{H}{M,w}$ is a consequence of $\Atom{T}{M,w}$ or not. It is
then possible to
determine whether a randomly chosen machine halts for a randomly chosen
input or not.

\item There is !!a{sentence} $!C$ of $\Lang{L}$ which is valid just in
case$\lforall[x]\lexists[\fac y][\Atom{R}{x,y}] \lif (\Atom{T}{M,w}_\land \lif
\Atom{H}{M,w})'$ is.
In general, for any !!{sentence} $!A$ of $\Lang{L_=}$ there is
!!a{sentence}
$\Lang{L_T}$ with the form $(!E \land !D) \lif !A(=/!E)$ which is equally
valid- i.e., $\Sat{}{!A}$ iff $(!E \land !D) \lif !A(=/!E)$.

Construct $(!E \land !D) \lif !A(=/!E)$ as follows:
\begin{itemize}
\item Let the !!{sentence} $!E$ be: $\lforall[x][\Atom{!E}{x, x}] \land
\lforall[x]\lforall[y][(\Atom{!E}{x, y} \lif \Atom{!E}{y, x})] \land
\lforall[x]\lforall[y]\lforall[z][((\Atom{!E}{x, y} \land \Atom{!E}{y,
z})
\lif \Atom{!E}{x, z})]$. $!E$ guarantees the equivalence relation.
\item Given !!a{sentence} $!A$ of $\Lang{L_=}$, form $!A(=/!E)$ (i.e.,
replace
the two-place logical !!{constant}~$\eq$ with a two-place !!{predicate}
$!E$ that does
not already appear in $!A$).
	
\item For each !!{predicate} $R$ in $!A$, form $!D_R$.
$!D_R$ is $\lforall[x_1...x_n y_1...y_n][((\Atom{!E}{x_1, y_1} \land ...
\land \Atom{!E}{x_n, y_n}) \lif (\Atom{R}{x_1, x_n} \liff \Atom{R}{y_1,
y_n}))]$. Each $!D_R$ guarantees substitution holds when
$\Atom{!E}{t,\delta}$ holds.

\item Form $!D$ by conjoining all of the $!D_R$ !!{sentence}s.

\item Then, for any !!{sentence} $!A$ of $\Lang{L_=}$, $\Sat{}{!A}$ iff
$(!E
\land !D) \lif !A(=/!E)$
\end{itemize}

We prove this in its contrapositive form.

Left to right: Suppose $\Sat/{}{!A}$.
So there is some !!{structure} $\Struct M$ where $\Sat/{M}{!A}$.
Construct $\Struct M^*$ which is just like $\Struct M$ except that
$\Struct{M^*} (!E) = \Setabs{x,y}{x \eq y}$.

$!E \land !D$ is true in $\Struct M^*$, since !!{identity} is an
equivalence relation and substitution holds for it.
But $!A( \eq / !E)$ will be false in $\Struct M^*$ (this can be shown
through induction).
So, $\Sat/{M^*}{(!E \land !D) \lif !A(=/!E)}$.
So, $\Sat/{}{(!E \land !D) \lif !A(=/!E)}$.

Right to left: Suppose $\Sat/{}{(!E \land !D) \lif !A(=/!E)}$.
So there is some !!{structure} $\Struct M$ where $\Sat/{M}{(!E \land !D)
\lif !A(=/!D)}$.
So, $\Sat{M}{!E \land !D}$ and $\Sat/{M}{!A(= / !E}$.
Since $\Sat{M}{!E}$, $\Struct{M}(!E)$ is an equivalence relation. Call
$\Struct{M} (E)
\sim$.
Construct $\Struct M^*$ as follows
\begin{enumerate}
\item The !!{domain} of $\Struct M^*$ is the set of equivalence classes
defined by $\sim$. In other words, where d and o are objects in $\Domain
M, d \in [0]$ iff $d \sim 0$, and if $\Domain{M}$ is $\{o_1, o_2, o_3,
...\}$ then $\Domain{M^*}$ is $\{[o_1], [o_2], [o_3], ...\}$.
\item $\Struct M^* (a_n) = [d]$ iff $\Struct M (a_n) \eq d$.
\item For all !!{predicate}s $R^n$ if $\tuple{d_1,...,d_n} \in
\Struct{M}(R_n)$ then $\tuple{[d_1],...,[d_n]} \in \Struct{M^*}
(R_n)$.
\end{enumerate}
Thus, $\tuple{[d_1],[d_2]} \in \Struct{M^*}(!E)$ iff $\tuple{d_1,d_2} \in
\Struct{M}(!E)$.
$\tuple{d_1,d_2} \in \Struct{M}(!E)$ iff $d_1 \sim d_2$.
$d_1 \sim d_2$ iff $ \eq [[d_1]][[d_2]]$
So, $\Struct{M^*}(!E) = \Setabs{x,y}{x \eq y}$.

So $\Sat/{M^*}{!A(=/!E)}$

So $\Sat/{M^*}{!A}$

So $\Sat/{}{!A}$.

\item So for any !!{sentence} $!A$ of $\Lang L_=$ there is a set $\{!E, !D
\}$
and a !!{sentence} $!A(=/!E)$ such that $\Sat{}{!A}$ iff $\Sat{\{ !E, !D
\}}{!A(=/!E)}$.

So in particular, there is a set $\{ !E, !D \}$ such that $\Sat{\{!E, !D\}
}{\lforall[x]\lexists[\fac y][\Atom{R}{x,y}]\lif (\Atom{T}{M,w}_\land \lif
\Atom{H}{M,w})'(=/!E)}$
iff $\Sat{}{\lforall[x]\lexists[\fac y][\Atom{R}{x,y}]\lif (\Atom{T}{M,w}_\land
\lif \Atom{H}{M,w})'}$.

$\Sat{}{\lforall[x]\lexists[\fac y][\Atom{R}{x,y}]\lif (\Atom{T}{M,w}_\land
\lif\Atom{H}{M,w})'}$
iff $\Sat{}{\Atom{T}{M,w}_\land \lif \Atom{H}{M,w}}$

$\Sat{}{\Atom{T}{M,w}_\land \lif \Atom{H}{M,w}}$ iff
$\Sat{\Atom{T}{M,w}}{\Atom{H}{M,w}}$

$\Sat{\Atom{T}{M,w}}{\Atom{H}{M,w}}$ iff $M$ halts for input $w$.

So if there is a decision procedure for consequence in $\Lang L$ then the
halting problem is solvable.

A quick look at this outline of the proof of the undecidability of FOL
should reveal that we have now reached the final step of the proof.

\begin{enumerate}
\item First-order logic is decidable iff there is an effective mechanical
method for determining whether a !!{sentence} is valid or not.
\item The Church-Turing thesis: every function which is computable is
Turing-computable.
\item The !!{function} $h$ such that $h(m,n) = 1$ if the Turing machine
described by $m$ halts for an input of $n$ and $0$ otherwise is not
Turing-computable.
\item Via the Church-Turing thesis and 3, $h$ is not computable.
\item If first-order logic is decidable then $h$ is computable.
\item Therefore, first-order logic is not decidable.
\end{enumerate}

In addition, as the proof above has made clear, we can also conclude two
corollaries:

Corollary one: First-order logic without identity is not decidable.
Corollary two: First-order logic without identity and terms is not
decidable.

Since all languages have a PET for consequence (the deductive system) we
can also conclude that none of these language have a NET for consequence.

In other words, while the deductive system will tell us that
$\Sat{\Gamma}{!A}$ in a finite amount of time, if $\Sat/{\Gamma}{A}$,
thereis no effective mechanical test that will tell us that
$\Sat/{\Gamma}{A}$.
\end{enumerate}
\end{proof}

\end{document}