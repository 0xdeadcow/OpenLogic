% Part: turing-machines
% Chapter: undecidability
% Section: undecidability-proof

\documentclass[../../include/open-logic-section]{subfiles}

\begin{document}

\olfileid{tms}{und}{unp}
\olsection{The Undecidability Proof}

\begin{proof}
We show that if first order logic without $\eq$ and terms is decidable
thenthe halting problem is solvable.
\begin{enumerate}
\item We know that given input $n$ and a Turing machine $\Struct M$ we can
effectively describe a set of sentences $\Delta$ and a sentence H (in the
language of first order-logic with identity and terms) such that:

$\Sat{\Delta}{H}$ if, and only if, $\Struct M$ halts of input $n$.

So if $\Lang{L_T}$ is decidable (i.e., if I can decide whether
$\Sat{\Delta}{H}$ or not) then the halting problem is solvable. We need to
show that if there is a decision procedure for consequence in $\Lang{L_=}$
then the halting problem is solvable.

Since $\Delta$ is finite, $\Sat{\Delta}{H}$ iff $\Sat{}{\Delta_\land \lif
H}$, where $\Delta_\land$ is the sentence that results from conjoining all
the members of $\Delta$ together.

Next, there is a sentence B of $\Lang{L_=}$ that is valid iff
$\Delta_\land\lif H$ is. In general, for any sentence $A$ of $\Lang{L_T}$ ,there is a
sentence of $\Lang{L_=}$ with the form
$\lforall{x}\lexists{y}{\Atom{R}{x}{y}}\lif A'$ which is equally valid.
So,$\Sat{}{A}$ if, and only if,
$\Sat{}{\lforall{x}\lexists{y}{\Atom{R}{x}{y}}\lif A'}$.

We construct $\Sat{}{\lforall{x}\lexists{y}{\Atom{R}{x}{y}}\lif A'}$ as
follows:

Given a sentence $A$ of $\Lang{L_T}$, first "normalize" $A$ to form $A*$.
We normalize a sentence by converting it to an equivalent sentence in
whichall names and function symbols appear only as arguments for the
logical
constant identity. This means that all other predicates have only
variablesas their arguments.

Take $\Atom{P}{f}{a}$ for example. We normalize $\Atom{P}{f}{a}$ in two
steps. First, we take the name $a$ out of the scope of the predicate P as
follows:

$\lexists{x}{\eq{x}{a} \land \Atom{P}{f}{x}}$

Second, we take the function symbol out of the scope of the predicate $P$
as follows:

$\lexists{x}{\eq{x}{a} \land \lexists{y}{\eq{fx}{y} \land \Atom{P}{y}}}$

Notice that $ A \simeq A*$.

Then from the normalized $A*$, form $A'$ by replacing all subformulas of
the form $\eq{ft}{\delta}$ with the subformula $R{t}{\delta}$, where $R$
isa 2-place predicate which does not appear in $A*$.

Since $f$ is a 1-place function, whereas R is a binary relation, and there
is no guarantee that R is functional, i.e. that there is only one object d
such that

To continue the example, $\lexists{x}{\eq{x}{a} \land
\lexists{y}{\eq{fx}{y} \land \Atom{P}{y}}}$ becomes $\lexists{x}{\eq{x}{a}
\land \lexists{y}{\Atom{R}{fx}{y} \land \Atom{P}{y}}}$.

Now form the sentence $\lforall{x}\lexists{y}{\Atom{R}{x}{y}}\lif A'$.
Quantifier thing.

\item We now want to prove that for any sentence $A$ of $L_T$,
$\Sat{}{\lforall{x}\lexists{y}{\Atom{R}{x}{y}}\lif A'}$ if, and only if,
$\Sat{}{A}$. We will prove this in its contrapositive form.

Right to left: Suppose that
$\Sat/{}{\lforall{x}\lexists{y}{\Atom{R}{x}{y}}\lif A'}$. So there is some
!!{structure} $\Struct M$ such that
$\Sat{M}{\lforall{x}\lexists{y}{\Atom{R}{x}{y}}}$ and $\Sat/{M}{A'}$.

Based on the !!{structure} $\Struct M$, create a new !!{structure}
$\Struct M*$ which is identical to $\Struct M$ except that we turn the
predicate $R$ back into $\fn{f}$. That is, whatever functional relation is
assigned to $R$ in $\Struct M$ should be assigned to $\fn{f}$ in
$\Struct M*$. Clearly, $\Sat/{M*}{A}$. We can give an inductive argument
on the complexity of A to this effect. So, $\Sat/{}{A}$.

Left to Right: Suppose $\Sat/{}{A}$. So there is some !!{structure}
$\Struct M$ such that $\Sat/{M}{A}$.

Based on the !!{structure} $\Struct M$, create a new !!{structure}
$\Struct M'$ which is identical to $\Struct M$ except that the functional
relation assigned to $\fn{f}$ in $\Struct M$ is assigned to $R$ in
$\Struct M'$. Since $R$ is assigned a functional relationship in
$\Struct M$, $\Sat{M}{\lforall{x}\lexists{y}{\Atom{R}{x}{y}}}$. But,
clearly, $\Sat/{M}{A'}$ (again, we can give an inductive argument to this
effect). So, $\Sat/{M}{\lforall{x}\lexists{y}{\Atom{R}{x}{y}}\lif A'}$,
and$\Sat/{}{\lforall{x}\lexists{y}{\Atom{R}{x}{y}}\lif A'}$.

\item In general, $\Sat/{}{\lforall{x}\lexists{y}{\Atom{R}{x}{y}}\lif A'}$
if, and only if, $\Sat/{\lforall{x}\lexists{y}{\Atom{R}{x}{y}}}{A'}$. So,
$\Sat{}{\Delta_\land \lif H}$ iff
$\Sat/{\lforall{x}\lexists{y}{\Atom{R}{x}{y}}}{(\Delta_\land \lif H)'}$
(there is just one function symbol in $\Delta_\land$, H).

Thus we can immediately conclude that if FOL with identity is decidable,
then the halting problem is solvable. We know that $\Sat{\Delta_\land}{H}$
iff $\Sat{}{\Delta_\land \lif H}$, and that $\Sat{}{\Delta_\land \lif H}$
iff $\Sat/{\lforall{x}\lexists{y}{\Atom{R}{x}{y}}}{(\Delta_\land \lif
H)'}$. So, if we have a decision procedure for consequence in $\Lang{L_=}$
then we can decide whether $\Delta_\land \lif H$ is valid or not, and thus
whether $H$ is a consequence of $\Delta$ or not. It is then possible to
determine whether a randomly chosen machine halts for a randomly chosen
input or not.

\item There is a sentence $C$ of $\Lang{L}$ which is valid just in case
$\lforall{x}\lexists{y}{\Atom{R}{x}{y}} \lif (\Delta_\land \lif H)'$ is.
Ingeneral, for any sentence $A$ of $\Lang{L_=}$ there is a sentence
$\Lang{L_T}$ with the form $(E \land S) \lif A(=/E)$ which is equally
valid- i.e., $\Sat{}{A}$ iff $(E \land S) \lif A(=/E)$.

Construct $(E \land S) \lif A(=/E)$ as follows:
\begin{itemize}
\item Let the sentence $E$ be: $\lforall{x}{\Atom{E}{x, x}} \land
\lforall{x} \lforall{y}{(\Atom{E}{x, y} \lif \Atom{E}{y, x})} \land
\lforall{x} \lforall{y} \lforall{z}{((\Atom{E}{x, y} \land \Atom{E}{y, z})
\lif \Atom{E}{x, z})}$. $E$ guarantees the equivalence relation.
\item Given a sentence $A$ of $\Lang{L_=}$, form $A(=/E)$ (i.e., replace
the two-place logical constant = with a two-place predicate $E$ that does
not already appear in $A$).
	\item For each predicate $R$ in $A$, form $S_R$.
$S_R$ is $\lforall{x_1...x_n y_1...y_n}{((\Atom{E}{x_1, y_1} \land ...
\land \Atom{E}{x_n, y_n}) \lif (\Atom{R}{x_1, x_n} \liff \Atom{R}{y_1,
y_n}))}$. Each $S_R$ guarantees substitution holds when
$\Atom{E}{t,\delta}$ holds.
	\item Form $S$ by conjoining all of the $S_R$ sentences.
\item Then, for any !!{sentence} $A$ of $\Lang{L_=}$, $\Sat{}{A}$ iff $(E
\land S) \lif A(=/E)$
\end{itemize}

We prove this in its contrapositive form.

Left to right: Suppose $\Sat/{}{A}$.
So there is some !!{structure} $\Struct M$ where $\Sat/{M}{A}$.
Construct $\Struct M*$ which is just like $\Struct M$ except that $\Struct
M* (E) = \Setabs{x,y}{x \eq y}$
$E \land S$ is true in $\Struct M*$, since identity is an equivalence
relation and substitution holds for it.
But $A( \eq / E)$ will be false in $\Struct M*$ (this can be shown through
induction).
So, $\Sat/{M*}{(E \land S) \lif A(=/E)}$.
So, $\Sat/{}{(E \land S) \lif A(=/E)}$.

Right to left: Suppose $\Sat/{}{(E \land S) \lif A(=/E)}$.
So there is some !!{structure} $\Struct M$ where $\Sat/{M}{(E \land S)
\lifA(=/E)}$.
So, $\Sat{M}{E \land S}$ and $\Sat/{M}{A( \eq / E}$.
Since $\Sat{M}{E}$, M(E) is an equivalence relation. Call $\Struct{M} (E)
~$
Construct $\Struct M*$ as follows
\begin{enumerate}
\item The !!{domain} of $\Struct M*$ is the set of equivalence classes
defined by $\sim$. In other words, where d and o are objects in $\Domain
M,d \in [0] iff d \sim 0$, and if $\Domain{M}$ is $\Setabs{o_1, o_2, o_3,
...}{}$ then $\Domain{M*}$ is $\Setabs{[o_1], [o_2], [o_3], ...}{}$.
\item $\Struct M* (a_n) \eq [d]$ iff $\Struct M (a_n) \eq d$.
\item For all !!{predicate}s $\Atom{R^n}$ if $\tuple{d_1,...,d_n} \in
\Struct{M} (\Atom{R_n})$ then $\tuple{[d_1],...,[d_n]} \in \Struct{M*}
(\Atom{R_n})$.
\end{enumerate}
Thus, $\tuple{[d_1],[d_2]} \in \Struct{M*} (E)$ iff $\tuple{d_1,d_2} \in
\Struct{M}(E)$.
$\tuple{d_1,d_2} \in \Struct{M}(E)$ iff $d_1 \sim d_2$.
$d_1 \sim d_2$ iff $[d_1] \eq [d_2]$

So, $\Struct{M*} (E) = \Setabs{x,y}{x \eq y}$

So $\Sat/{M*}{A(=/E)}$

So $\Sat/{M*}{A}$

So $\Sat/{}{A}$.

\item So for any sentence $A$ of $\Lang L_=$ there is a set $\{ E, S \}$
and a sentence $A(=/E)$ such that $\Sat{}{A}$ iff $\Sat{\{ E, S \}
}{A(=/E)}$.

So in particular, there is a set $\{ E, S \}$ such that $\Sat{\{ E, S\}
}{\lforall{x}\lexists{y}{\Atom{R}{x,y}}\lif (\Delta_\land \lif H)'(=/E)}
iff \Sat{}{\lforall{x}\lexists{y}{\Atom{R}{x,y}}\lif (\Delta_\land \lif
H)'}$.

$\Sat{}{\lforall{x}\lexists{y}{\Atom{R}{x,y}}\lif (\Delta_\land \lif H)'}
iff \Sat{}{\Delta_\land \lif H}$

$\Sat{}{\Delta_\land \lif H} iff \Sat{\Delta}{H}$

$\Sat{\Delta}{H} iff M$ halts for input $n$.

So if there is a decision procedure for consequence in $\Lang L$ then the
halting problem is solvable 9Them 5).

A quick look at this outline of the proof of the undecidability of FOL
should reveal that we have now reached the final step of the proof.

\begin{enumerate}
\item First-order logic is decidable iff there is an effective mechanical
method for determining whether a sentence is valid or not.
\item The Church-Turing thesis: every function which is computable is
Turing-computable.
\item The function $h$ such that $h{m,m} = 1$ if the Turing machine
described by $m$ halts for an input of $n$ and $0$ otherwise is not
Turing-computable.
\item Via the Church-Turing thesis and 3, $h$ is not computable.
\item If first-order logic is decidable then $h$ is computable.
\item Therefore, first-order logic is not decidable.
\end{enumerate}
In addition, as the proof above has made clear, we can also conclude two
corollaries:

Corollary one: First-order logic without identity is not decidable.
Corollary two: First-order logic without identity and terms is not
decidable.

Since all languages have a PET for consequence (the deductive system) we
can also conclude that none of these language have a NET for consequence.

In other words, while the deductive system will tell us that
$\Sat{\Gamma}{A}$ in a finite amount of time, if $\Sat/{\Gamma}{A}$, there
is no effective mechanical test that will tell us that $\Sat/{\Gamma}{A}$.

\end{enumerate}
\end{proof}

\end{document}