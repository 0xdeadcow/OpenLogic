\documentclass[modal-logic]{subfiles}

\begin{document}

\section{Relational Models for Normal Modal Logics}

\begin{defn}
A \emph{modal model}~$\Mod M = \langle W, R, V\rangle$ consists of
\begin{enumerate}
\item a nonempty set $ W\neq \emptyset$,
\item a relation $R \subseteq W \times W$, and
\item a valuation $V: \Var \times W \to \{\True, \False\}$
\end{enumerate}
\end{defn}

The set $W$ is often called the set of ``worlds,'' to reflect the
intuitive idea that the points of evaluation correspond to ``possible
worlds.'' Depending on the application, a different terminology may be
more appropriate, e.g., ``states.''  The relation~$R$ is usually
called the accessibility relation, and we say that if $wRv$ holds for
worlds $w, v \in W$ that ``$v$ is accessible from (or relatie to)
$w$.''  Again, this reflects the intutive interpretation that
world~$v$ is possible relative to $w$.  The valuation~$s$
represents the truth value of atomic propositions: $s(p, w) = \True$
if $p$ is ``true at''~$w$.

\begin{defn}\label{defn:mmodels}
\emph{Truth} of a formula~$!A$  at~$w$ in a~$\Mod M$, $\Mod M, w
\Mmodels !A$, is defined inductively as follows:
\begin{enumerate}
\item $\Mod M, w \Mmodels p$ iff $V(w, p) = \True$
\item $\Mod M, w \nMmodels \bot$ 
\item $\Mod M, w \Mmodels \lnot !B$ iff $\Mod s M, w \nMmodels !B$
\item $\Mod M, w \Mmodels !A \land !B$ iff $\Mod M, w \Mmodels !A$ and
  $\Mod M, w \Mmodels !B$
\item $\Mod M, w \Mmodels !A \lor !B$ iff $\Mod M, w \Mmodels !A$ or
  $\Mod M, w \Mmodels !B$ (or both)
\item $\Mod M, w \Mmodels !A \lif !B$ iff $\Mod M, w \nMmodels !A$ or
  $\Mod M, w \Mmodels !B$
\item\label{defn:sub:mmodels-box} $\Mod M, w \Mmodels \Box !A$ iff $\Mod M, w'
  \Mmodels !A$ for all $w' \in W$ with $wRw'$
\item $\Mod M, w \Mmodels \Diamond !A$ iff $\Mod M, w' \Mmodels !A$ for
  at least one $w' \in W$ with $wRw'$
\end{enumerate} 
\end{defn}

Note that by clause~\ref{mmodels-box}, a formula $\Box!B$ is satisfied at
$w$ whenever there are no~$w'$ with $wRw'$. In such a case $\Box !B$
is vacuously satisfied at~$w$. Also, $\Box !B$ may be satisifed at~$w$
even if $!B$ is not, and the truth of~$!B$ at~$w$ does not guarantee
the truth of~$\Diamond !B$ there---this only holds if $wRw$, e.g., if
$R$ is reflexive.

\begin{defn}
The \emph{proposition expressed} by a formula~$!A$ in $\Mod M$,
$\Prop{!A}{M}$, is the set of worlds at which $!A$ is true in~$\Mod
M$, i.e.,
\[
\Prop{!A}{M} = \{w \in W : \Mod M, w \Mmodels !A\}
\]
\end{defn}

\begin{thm}
\begin{enumerate}
\item $\Prop{\lnot !A}{M} = W \setminus \Prop{!A}{M}$
\item $\Prop{!A \land !B}{M} = \Prop{!A}{M} \cap \Prop{!B}{M}$
\item $\Prop{!A \lor !B}{M} = \Prop{!A}{M} \cup \Prop{!B}{M}$
\item $\Prop{\Diamond!A}{M} = R^{-1}[\Prop{!A}{M}]$
\end{enumerate}
\end{thm}

\begin{defn}
A \emph{frame}~$\Mod F = \langle W, R\rangle$ consists of a nonempty set~$W$ and a relation~$R \subseteq W \times W$ on it.  If $\Mod M = \langle W, R, V\rangle$ for some valuation~$V$, we say that $\Mod M$ is \emph{based on} the frame~$\Mod F$, and that $\Mod F$ is \emph{the frame of}~$\Mod M$.
\end{defn}

\end{document}
