% Part: first-order-logic
% Chapter: sequent-calculus
% Section: rules-and-proofs

\documentclass[../../include/open-logic-section]{subfiles}

\begin{document}

\olfileid{fol}{seq}{rul}
\olsection{Rules and Proofs}

%explain bit

Let $\Lang L$ be a first-order language with the usual constants, !p{variable}, and logical symbols and auxiliary symbols (parentheses and the comma). 

\begin{defn}[sequent]
A \emph{sequent} is an expression of the form
\[ \Gamma \Sequent \Delta \]
where $\Gamma$ and $\Delta$ are finite (possibly empty) sets of !p{formula} of the language $\Lang L$. The !!{formula}e in $\Gamma$ are the \emph{antecedent formulae}, while the formulae in $\Delta$ are the \emph{succedent formulae}. 

The intuitive idea behind a sequent is: if all of the antecedent !p{formula} hold, then some of the succedent !!{formula}e hold. That is, if $\Gamma = \{ \Gamma_1,\ldots,\Gamma_m\}$ and $\Delta = \{ \Delta_1,\ldots, \Delta_n\}$, then $\Gamma \Sequent \Delta$ abbreviates
\[ \Gamma_1 \land \cdots \land \Gamma_m \lif \Delta_1 \lor \cdots \lor \Delta_n \]

When $m=0$, $\hspace{1em} \Sequent \Delta$ means that $\Delta_1 \lor \cdots \Delta_n$ holds. When $n=0$, $\Gamma \Sequent \hspace{1em}$ means that $\Gamma_1 \land \cdots \land \Gamma_m$ gives a contradiction. An empty succedent is sometimes filled with the $\lfalse$ symbol. The empty sequent $\hspace{1em} \Sequent \hspace{1em}$ canonically represents a contradiction.
\end{defn}

\begin{defn}[Inference]
An \emph{inference} is an expression of the form

\[
\AxiomC{$S_1$}
\UnaryInfC{$S$}
\DisplayProof
\quad
\textrm{  or  }
\quad
\AxiomC{$S_1$}
\AxiomC{$S_2$}
\BinaryInfC{$S$}
\DisplayProof
\]
where $S, S_1$, and $S_2$ are sequents. $S_1$ and $S_2$ are called \emph{upper sequents} and $S$ is a \emph{lower sequent}.

Inferences represent the idea that whenever the upper sequent(s) is (are) asserted, from it, we may logically infer the lower sequent.
\end{defn}

For the following, let $\Gamma, \Delta, \Pi, \Lambda$ represent finite sets of !p{formula}, and let $!A, !B, !C, !D, \ldots$ represent individual !!{formula}e.

The rules for $\Log{LK}$ are divided into two main types: \emph{structural} rules and \emph{logical} rules. The logical rules are further divided into \emph{propositional} rules (quantifier-free) and \emph{quantifier} rules.\\ 

\underline{Structural rules:}

Weakening:
\[
\Axiom$ \Gamma \fCenter \Delta $
\UnaryInf$ !A, \Gamma \fCenter \Delta$
\DisplayProof
\quad
\textrm{  and  }
\quad
\Axiom$ \Gamma \fCenter \Delta$
\UnaryInf$ \Gamma \fCenter \Delta, !A$
\DisplayProof
\]
where $!A$ is called the \emph{weakening !!{formula}}.

Contraction:
\[
\Axiom$ !A, !A, \Gamma \fCenter \Delta $
\UnaryInf$ !A, \Gamma \fCenter \Delta $
\DisplayProof
\quad
\textrm{  and  }
\quad
\Axiom$ \Gamma \fCenter \Delta, !A, !A $
\UnaryInf$ \Gamma \fCenter \Delta, !A$
\DisplayProof
\]

Interchange:
\[
\Axiom$ \Gamma, !A, !B, \Pi \fCenter \Delta $
\UnaryInf$ \Gamma, !B, !A, \Pi \fCenter \Delta$
\DisplayProof
\quad
\textrm{  and  }
\quad
\Axiom$ \Gamma \fCenter \Delta, !A, !B, \Lambda $
\UnaryInf$ \Gamma \fCenter \Delta, !B, !A, \Lambda$
\DisplayProof
\]
The above three inferences can be trivial and numerous, and are  often denoted simply by a double line between the upper lower sequents.\\

Cut:
\[
\Axiom$ \Gamma \fCenter \Delta, !A$
\Axiom$ !A, \Pi \fCenter \Lambda $
\BinaryInf$ \Gamma, \Pi \fCenter \Delta, \Lambda$
\DisplayProof
\]

\underline{Logical rules:}
The rules are named by the logical symbol of the \emph{principal !!{formula}} of the inference (the formula containing $!A$ and/or $!B$ in the lower sequent). The designations ``left'' and ``right'' indicate whether the logical symbol has been introduced in an antecedent formula or a succedent formula (to the left or to the right of the sequent symbol).\\

\emph{Propositional Rules:}
\[
\Axiom$ \Gamma \fCenter \Delta, !A $
\RightLabel{$\lnot$ left}
\UnaryInf$ \lnot !A, \Gamma \fCenter \Delta$
\DisplayProof
\quad
\Axiom$!A, \Gamma \fCenter \Delta$
\RightLabel{$\lnot$ right}
\UnaryInf$ \Gamma \fCenter \Delta, \lnot !A $
\DisplayProof
\]

\[
\Axiom$ !A, \Gamma \fCenter \Delta$
\RightLabel{$\land$ left}
\UnaryInf$ !A \land !B, \Gamma \fCenter \Delta$
\DisplayProof
\quad
\Axiom$!B, \Gamma \fCenter \Delta$
\RightLabel{$\land$ left}
\UnaryInf$!A \land !B, \Gamma \fCenter \Delta$
\DisplayProof
\quad
\Axiom$\Gamma \fCenter \Delta, !A$
\Axiom$ \Gamma \fCenter \Delta, !B$
\RightLabel{$\land$ right}
\BinaryInf$ \Gamma \fCenter \Delta, !A \land !B $
\DisplayProof
\]

\[
\Axiom$!A,\Gamma\fCenter \Delta$
\Axiom$ !B, \Gamma \fCenter \Delta$
\RightLabel{$\lor$ left}
\BinaryInf$ !A \lor !B, \Gamma \fCenter \Delta$
\DisplayProof
\quad
\Axiom$\Gamma \fCenter \Delta, !A$
\RightLabel{$\lor$ right}
\UnaryInf$ \Gamma \fCenter \Delta, !A \lor !B$
\DisplayProof
\quad
\Axiom$ \Gamma \fCenter \Delta, !B$
\RightLabel{$\lor$ right}
\UnaryInf$ \Gamma \fCenter \Delta, !A \lor !B$
\DisplayProof
\]

\[
\Axiom$ \Gamma \fCenter \Delta, !A$
\Axiom$ !B, \Pi \fCenter \Lambda$
\RightLabel{$\lif$ left}
\BinaryInf$ !A \lif !B, \Gamma, \Pi \fCenter \Delta, \Lambda$
\DisplayProof
\quad
\Axiom$ !A, \Gamma \fCenter \Delta, !B$
\RightLabel{$\lif$ right}
\UnaryInf$ \Gamma \fCenter \Delta, !A \lif !B $
\DisplayProof
\]

\emph{Quantifier Rules:}

\[
\Axiom$ F(t), \Gamma \fCenter \Delta$
\RightLabel{$\lforall$ left}
\UnaryInf$ \lforall[x] F(x),\Gamma \fCenter \Delta$
\DisplayProof
\quad
\Axiom$ \Gamma \fCenter \Delta, F(a) $
\RightLabel{$\lforall$ right}
\UnaryInf$ \Gamma \fCenter \Delta, \lforall[x] F(x)$
\DisplayProof
\]

Where $t$ is a term, and $a$ is a constant which does not occur anywhere in the lower sequent of the $\lforall[]$ right rule. We call $a$ the \emph{eigenvariable}.

\[
\Axiom$ F(a), \Gamma \fCenter \Delta $
\RightLabel{$\lexists$ left}
\UnaryInf$ \lexists[x] F(x), \Gamma \fCenter \Delta$
\DisplayProof
\quad
\Axiom$ \Gamma \fCenter \Delta, F(t) $
\RightLabel{$\lexists$ right}
\UnaryInf$ \Gamma \fCenter \Delta, \lexists[x] F(x)$
\DisplayProof
\]
Where $t$ is a term, and $a$ is a constant which does not occur in the lower sequent of the $\lexists$ left rule. Again, we call $a$ the \emph{eigenvariable}. 

The condition that an eigenvariable not occur in the upper sequent of the $\lforall$ right/$\lexists$ left rule is called the \emph{eigenvariable condition}.

\begin{defn}[initial sequent]
An \emph{initial sequent} is a sequent of the form $!A \Sequent !A$ for any !!{formula} $!A$ in the language.
\end{defn}

\begin{defn}[LK derivation]
An $\Log{LK}$-derivation of a sequent $S$ is a tree of sequents satisfying the following conditions:
\begin{enumerate}
\item The topmost sequents of the tree are initial sequents
\item Every sequent in the tree (except $S$) is an upper sequent of an inference whose lower sequent stands directly below that sequent in the tree.
\end{enumerate}
We then say that $S$ is the \emph{end-sequent} of the derivation and that $S$ is \emph{derivable in $\Log{LK}$} (or $\Log{LK}$-derivable).
\end{defn}

\begin{defn}[LK theorem]
A !!{formula} $!A$ is a \emph{theorem} of $\Log{LK}$ if the sequent $\hspace{1em} \Sequent !A$ is $\Log{LK}$-derivable.
\end{defn}




















\end{document}
