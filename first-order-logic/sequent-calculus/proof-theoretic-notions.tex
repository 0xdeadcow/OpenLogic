% Part: first-order-logic
% Chapter: sequent-calculus
% Section: proof-theoretic-notions

\documentclass[../../include/open-logic-section]{subfiles}

\begin{document}

\olfileid{fol}{seq}{ptn}
\olsection{Proof-Theoretic Notions}

\begin{wordy}
Just as we've defined a number of important semantic notions
(validity, entailment, satisfiabilty), we now define corresponding
\emph{proof-theoretic notions}. These are not defined by appeal to
satisfaction of sentneces in structures, but by appeal to the
provability or unprovability of certain sequents.  It was an important
discovery, due to G\"odel, that these notions coincide.  That they do
is the content of the \emph{completeness theorem}.
\end{wordy}

\begin{defn}[Theorems]
A sentence~$!A$ is a \emph{theorem} if there is a proof in LK of the
sequent $\Sequent !A$. We write $\Proves !A$ if $!A$ is a theorem and
$\Proves/ !A$ if it is not.
\end{defn}

\begin{defn}[Provability]
A sentence $!A$ is \emph{provable from} a set of sentences~$\Gamma$,
$\Gamma \Proves !A$, if there is a finite subset~$\Gamma_0 \subseteq
\Gamma$ such that LK derives $\Gamma_0 \Sequent !A$.  If $!A$ is not
provable from $\Gamma$ we write $\Gamma \Proves/ !A$.
\end{defn}

\begin{defn}[Consistency]
A set of sentences~$\Gamma$ is \emph{consistent} iff $\Gamma \Proves!
\lfalse$.  If $\Gamma$ is not consistent we say it is \emph{inconsistent}.
\end{defn}

\begin{prop}
$\Gamma \Proves !A$ iff $\Gamma \cup \{\lnot A\}$ is inconsistent.
\end{prop}

\begin{prop}
$\Gamma$ is inconsistent iff $\Gamma \Proves {!A}$ for every sentence~$!A$.
\end{prop}



\end{document}
