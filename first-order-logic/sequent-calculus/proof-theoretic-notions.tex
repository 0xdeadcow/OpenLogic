% Part: first-order-logic
% Chapter: sequent-calculus
% Section: proof-theoretic-notions

\documentclass[../../include/open-logic-section]{subfiles}

\begin{document}

\olfileid{fol}{seq}{ptn}
\olsection{Proof-Theoretic Notions}

\begin{explain}
Just as we've defined a number of important semantic notions
(validity, entailment, satisfiabilty), we now define corresponding
\emph{proof-theoretic notions}. These are not defined by appeal to
satisfaction of sentences in !!{structure}s, but by appeal to the
!!{derivability} or !!{nonderivability} of certain sequents.  It was
an important discovery, due to G\"odel, that these notions coincide.
That they do is the content of the \emph{completeness theorem}.
\end{explain}

\begin{defn}[Theorems]
A sentence~$!A$ is a \emph{theorem} if there is a !!{derivation}
in~$\Log{LK}$ of the sequent $\quad \Sequent !A$. We write
$\Proves[\Log{LK}] !A$ if $!A$ is a theorem and $\Proves/[\Log{LK}]
!A$ if it is not.
\end{defn}

\begin{defn}[!!^{derivability}]
A sentence $!A$ is \emph{!!{derivable} from} a set of
sentences~$\Gamma$, $\Gamma \Proves[\Log{LK}] !A$, if there is a
finite subset~$\Gamma_0 \subseteq \Gamma$ such that $\Log{LK}$
!!{derive}s $\Gamma_0 \Sequent !A$.  If $!A$ is not !!{derivable} from
$\Gamma$ we write $\Gamma \Proves/[\Log{LK}] !A$.
\end{defn}

\begin{defn}[Consistency]
A set of sentences~$\Gamma$ is \emph{consistent} iff $\Gamma
\Proves/[\Log{LK}] \lfalse$.  If $\Gamma$ is not consistent, i.e., if
$\Gamma \Proves[\Log{LK}] \lfalse$, we say it is \emph{inconsistent}.
\end{defn}

\begin{prop}
\ollabel{prop:prov-incons}
$\Gamma \Proves[\Log{LK}] !A$ iff $\Gamma \cup \{\lnot !A\}$ is inconsistent.
\end{prop}

\begin{proof}
Exercise.
\end{proof}

\begin{prob}
Prove \olref[fol][seq][ptn]{prop:prov-incons}
\end{prob}

\begin{prop}
\ollabel{prop:incons}
$\Gamma$ is inconsistent iff $\Gamma \Proves[\Log{LK}] {!A}$ for every
  sentence~$!A$.
\end{prop}

\begin{proof}
Exercise.
\end{proof}

\begin{prob}
Prove \olref[fol][seq][ptn]{prop:incons}
\end{prob}

\begin{prop}
\ollabel{prop:proves-compact}
If $\Gamma \Proves !A$ iff for some finite $\Gamma_0 \subseteq
\Gamma$, $\Gamma_0 \Proves !A$.
\end{prop}

\begin{proof}
Follows immediately from the definion of~$\Proves$.
\end{proof}

\end{document}
