% Part: first-order-logic
% Chapter: syntax-and-semantics
% Section: expressing-relations

\documentclass[../../include/open-logic-section]{subfiles}

\begin{document}

\olfileid{fol}{syn}{exr} 

\olsection{Expressing relations in \article{structure}
  \printtoken{s}{structure}}

\begin{explain}
One main use !!{formula}s can be put to is to express properties and
relations in !!a{structure}~$\Struct M$ in terms of the primitives of
the language~$\Lang L$ of~$\Struct M$.  By this we mean the following:
the !!{domain} of $\Struct M$ is a set of objects.  The !!{constant}s,
!!{function}s, and !!{predicate}s are interpreted in~$\Struct M$ by
some objects in$\Domain M$, functions on~$\Domain M$, and relations
on~$\Domain M$.  For instance, if $\Obj A^2_0$ is in $\Lang L$, then
$\Struct M$ assigns to it a relation~$R = \Assign{{\Obj
  A^2_0}}{M}$. Then the formula $\Atom{\Obj A^2_0}{\Obj x_1, \Obj x_2}$
\emph{expresses} that very relation, in the following sense: if a
variable assignment~$s$ maps $\Obj x_1$ to $a \in \Domain{M}$ and
$\Obj x_2$ to $b \in \Domain M$, then
\[
Rab \text{\quad iff\quad} \Sat{M}{\Atom{\Obj A^2_0}{\Obj x_1, \Obj x_2}}[s].
\]
Since we don't just have atomic !!{formula}s, but can combine them
using the logical connectives and the quantifiers, more complex
!!{formula}s can define other relations which aren't directly built
into~$\Struct M$.  We're interested in how to do that, and
specifically, which relations we can define in !!a{structure}.
\end{explain}

\begin{defn}
Let $!A(\Obj x_1,\dots, \Obj x_n)$ be a !!{formula} of $\Lang L$ in
which only $\Obj x_1$,\dots, $\Obj x_n$ occur free, and let $\Struct
M$ be !!a{structure} for~$\Lang L$. $!A(\Obj x_1,\dots, \Obj x_n)$
\emph{expresses the relation}~$R \subseteq \Domain M^n$ iff
\[
Ra_1\dots a_n \text{\quad iff\quad} \Sat{M}{\Atom{!A}{\Obj
    x_1,\dots, \Obj x_n}}[s]
\]
for any variable assignment~$s$ with $s(\Obj x_i) = a_i$ ($i = 1,
\dots, n$).
\end{defn}

\begin{ex}
In the standard model of arithmetic~$\Struct N$, the !!{formula} $\Obj
x_1 < \Obj x_2 \lor \Obj x_1 = \Obj x_2$ expresses the $\le$ relation
on~$\Nat$. The !!{formula} $\Obj x_2 = \Obj x_1'$ expresses the
successor relation, i.e., the relation $R \subseteq \Nat^2$ where
$Rnm$ holds if $m$ is the successor of~$n$. The formula $\Obj x_1 =
\Obj x_2'$ expresses the predecessor relation.  The !!{formula}s
$\lexists[\Obj x_3][(\eq/[\Obj x_3][\Obj 0] \land \eq[\Obj x_2][(\Obj
    x_1 + \Obj x_3)])]$ and $\lexists[\Obj x_3][\eq[\Obj x_1 + {\Obj
      x_3}'][x_2]]$ both express the $\Obj <$ relation.  This means
that the predicate symbol~$<$ is actually superfluous in the language
of arithmetic; it can be defined.
\end{ex}

\begin{prob}
Find !!{formula}s in $\Lang L_A$ which express the following relations:
\begin{enumerate}
\item $n$ is between $i$ and $j$;
\item $n$ evenly divides $m$ (i.e., $m$ is a multiple of $n$);
\item $n$ is a prime (i.e., no number other than $1$ and $n$ evenly
  divides~$n$).
\end{enumerate}
\end{prob}

\begin{prob}
Suppose the formula $!A(\Obj x_1, \Obj x_2)$ expresses the relation $R
\subseteq \Domain M^2$ in a !!{structure}~$\Struct M$. Find formulas
that express the following relations:
\begin{enumerate}
\item the inverse $R^{-1}$ of $R$;
\item the relative product $R \mid R$;
\end{enumerate}
Can you find a way to express $R^+$, the transitive closure of~$R$?
\end{prob}

\end{document}
