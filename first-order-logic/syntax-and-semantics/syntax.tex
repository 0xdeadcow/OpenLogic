% Part: first-order-logic
% Chapter: syntax-and-semantics
% Section: syntax

\documentclass[syntax-and-semantics]{subfiles}

\begin{document}

\section{Syntax}

%This still needs to be macro'd.

The basic language of first-order logic contains \emph{variables}, \emph{constants}, \emph{predicates} and sometimes \emph{functions}.  From them, together with the logical connectives and the quantifiers, \emph{terms} and \emph{formulas} are formed.  In the general case, we make use of the following symbols in first-order logic:

\begin{itemize}
\item Logical connectives: $\lnot$ (negation), $\land$ (conjunction), $\lor$ (disjunction), $\lif$ (conditional), $\liff$ (biconditional), $\lforall$ (universal quantifier), $\lexists$ (existential quantifier).
\item The propositional constant for falsity~$\lfalse$.
\item The two-place equality predicate~$=$.
\item A denumerable set of $n$-place predicates for each $n>0$: $\Obj A^n_0$, $\Obj A^n_1$, $\Obj A^n_2$, \dots
\item A denumerable set of constants: $\Obj c_0$, $\Obj c_1$, $\Obj c_2$, \dots.
\item A denumerable set of $n$-place function symbols for each $n>0$: $\Obj f^n_0$, $\Obj f^n_1$, $\Obj f^n_2$, \dots
\item A denumerable set of variables: $\Obj v_0$, $\Obj v_1$, $\Obj v_2$, \dots
\end{itemize}

% Explanation of using ~ vs. $\lnot$, ampersand versus wedge, arrow
% vs. horseshoe, quantifiers, etc. compared to LPL and Logic Book
% Perhaps an explanation for "wordy" of constants as 0-place function
% symbols

% Include discussion of language restricted to a certain vocabulary,
% extensions of a language, examples (eg language of arithmetic, set
% theory, etc.)

\begin{defn}[Term]
The set of \emph{terms}~$\Trm$ of the language of first-order logic is
defined inductively:
\begin{enumerate}
\item Variables and constants are terms.
\item If $f$ is an $n$-place function symbol and $t_1$, \dots, $t_n$
  are terms, then $\Atom{f}{t_1, \ldots, t_n}$ is a term.
\item Nothing else is a term.
\end{enumerate}
A term containing no variables is a \emph{closed term}.
\end{defn}

\begin{defn}[Formula]
The set of \emph{formulae}~$\Frm$ of the language of first-order logic
is defined inductively:
\begin{enumerate}
\item $\bot$ is an atomic formula.
\item If $R$ is an $n$-place predicate and $t_1$, \dots, $t_n$ are
  terms, then $\Atom{R}{t_1,\ldots, t_n}$ is an (atomic) formula.
\item If $t_1$ and $t_2$ are terms, then $=(t_1, t_2)$ is an atomic formula.
\item If $!A$ is a formula, then $\lnot !A$ is a formula.
\item If $!A$ and $!B$ are formulae, then $(!A \land !B)$ is a formula.
\item If $!A$ and $!B$ are formulae, then $(!A \lor !B)$ is a formula.
\item If $!A$ and $!B$ are formulae, then $(!A \lif !B)$ is a formula.
\item If $!A$ and $!B$ are formulae, then $(!A \liff !B)$ is a formula.
\item If $!A$ is a formula and $x$ is a variable, then $\lforall[x]\,
  !A$ is a formula.
\item If $!A$ is a formula and $x$ is a variable, then $\lexists[x]\,
  !A$ is a formula.
\item Nothing else is a formula.
\end{enumerate}
\end{defn}

%Vacuous quantification

Some logic texts require that the variable $x$ must occur in $!A$ in
order for $\lexists x\, !A$ and $\lexists x\, !A$ to count as
formulas.  Nothing bad happens if you don't require this, and it makes
things easier.

%option 1 for subformula
\begin{defn}[Subformula]
Given a formula $!A$, the set of \emph{subformulae} of $!A$,
$\SubFrm{!A}$, is defined inductively as follows:
\begin{enumerate}
\item If $!A$ is atomic, then $\SubFrm{!A} = \{!A\}$.
\item $\SubFrm{\lnot !A} = \{\lnot !A\} \cup \SubFrm{!A}$
\item If $!B$ is also a formula, then $\SubFrm{!A \land !B} = \{!A \land !B\} \cup \SubFrm{!A} \cup \SubFrm{!B}$
\item If $!B$ is also a formula, then $\SubFrm{!A \lor !B} = \{!A \lor !B\} \cup \SubFrm{!A} \cup \SubFrm{!B}$
\item If $!B$ is also a formula, then $\SubFrm{!A \lif !B} = \{!A \lif !B\} \cup \SubFrm{!A} \cup \SubFrm{!B}$
\item If $!B$ is also a formula, then $\SubFrm{!A \liff !B} = \{!A \liff !B\} \cup \SubFrm{!A} \cup \SubFrm{!B}$
\item If $x$ is a variable, then $\SubFrm{\forall[x][!A]} = \{\lforall[x][!A]\} \cup \SubFrm{!A}$
\item If $x$ is a variable, then $\SubFrm{\lexists[x][!A]} = \{\lexists[x][!A]\} \cup \SubFrm{!A}$
\end{enumerate}
The set of all \emph{proper} subformulae of $!A$ is $\SubFrm{!A} -
\{!A\}$. That is, a formula $!B$ is a proper subformula of $!A$ if and
only if $!B$ is a subformula of $!A$ but is distinct from $!A$.
\end{defn}

\begin{defn}[Immediate subformula]
Given formulae $!A$ and $!B$, define \emph{immediate subformula} inductively as follows:
\begin{enumerate}
\item Atomic formulae have no immediate subformulae.
\item $!A$ is the immediate subformula of $\lnot !A$.
\item $!A$ and $!B$ are the immediate subformulae of $(!A \land !B)$.
\item $!A$ and $!B$ are the immediate subformulae of $(!A \lor !B)$.
\item $!A$ and $!B$ are the immediate subformulae of $(!A \lif !B)$.
\item $!A$ and $!B$ are the immediate subformulae of $(!A \liff !B)$.
\item $!A$ is the immediate subformula of $\lforall[x][!A]$.
\item $!A$ is the immediate subformula of $\lexists[x][!A]$.
%I listed these all separately rather than combining them because eventually we will have to be able to let people pick which logical symbols they use. if we can make it work without looking so "listy" (so we can condense the clauses), that would probably look better.
\end{enumerate}
\end{defn}
%Wordy explanation of subformula in here

%option 2: Alternate definition of subformula via the definition of immediate subformula? This one avoids the SBF notation.

\begin{defn}[Subformula]
Define \emph{subformula} as follows:
\begin{enumerate}
\item If $!B$ is equal to $!A$, then $!B$ is a subformula of $!A$.
\item If $!B$ is an immediate subformula of $!A$, then $!B$ is a subformula of $!A$.
\item If $!C$ is a subformula of $!B$ and $!B$ is a subformula of $!A$, then $!C$ is also a subformula of $!A$.
\end{enumerate}
A formula $!B$ is a \emph{proper} subformula of $!A$ if and only if it is a subformula of $!A$ and is not equal to $!A$.
\end{defn}

% option 3: Or we could do it this way, combining immediate, proper, (regular) subformula definitions, so this would go after the definition of immediate subformula

\begin{defn}[Proper subformula]
Define \emph{proper subformula} as follows:
\begin{enumerate}
\item If $!B$ is an immediate subformula of $!A$, then $!B$ is a proper subformula of $!A$.
\item If $!C$ is a proper subformula of $!B$ and $!B$ is a proper subformula of $!A$, then $!C$ is also a proper subformula of $!A$.
\end{enumerate}
\end{defn}

\begin{defn}[Subformula]
Define \emph{subformula} as follows:
\begin{enumerate}
\item If $!B$ is equal to $!A$, then $!B$ is a subformula of $!A$.
\item If $!B$ is a proper subformula of $!A$, then $!B$ is a subformula of $!A$.
\end{enumerate}
\end{defn}

%This is what I figured RZ meant for an inductive definition of free occurrences. I didn't combine it all at once with the definition of a formula because I think it would be too long...

\begin{defn}[Free occurrences of a variable]
Define a \emph{free} occurrence of a variable in a formula inductively:
\begin{enumerate}
\item If $!A$ is atomic, all variables in $!A$ occur free in $!A$.
\item If $!A$ is a formula, then the free variable occurrences of $\lnot !A$ are exactly those in $!A$.
\item If $!A$ and $!B$ are formulae, then the free variable occurrences in $(!A \land !B)$ are those in $!A$ together with those in $!B$.
\item If $!A$ and $!B$ are formulae, then the free variable occurrences in $(!A \lor !B)$ are those in $!A$ together with those in $!B$.
\item If $!A$ and $!B$ are formulae, then the free variable occurrences in $(!A \lif !B)$ are those in $!A$ together with those in $!B$.
\item If $!A$ and $!B$ are formulae, then the free variable occurrences in $(!A \liff !B)$ are those in $!A$ together with those in $!B$.
\item If $!A$ is a formula, then the free variable occurrences in $\lforall[x][!A]$ are all of those in $!A$ except for occurrences of $x$.
\item If $!A$ is a formula, then the free variable occurrences in $\lexists[x][!A]$ are all of those in $!A$ except for occurrences of $x$.
\end{enumerate}
%Again, if we could figure out a way to condense those clauses and still be able to allow people to "pick and choose" which logical connectives to use, that would look better
\end{defn}

\begin{defn}[Sentence]
A formula $!A$ is a \emph{sentence} if and only if it is a (well-formed) formula and contains no free occurrences of variables.\end{defn}

%"Wordy" explanation of bound variable in here

\begin{defn}[Substitution of a term for a variable]
If $!A$ is a formula, $x$ is a variable, and $t$ is a term, $A(x/t)$ is the result of replacing all free occurrences of $x$ in $!A$ with $t$. Substitution may be vacuous ($x$ need not occur in $!A$). \end{defn}





















\end{document}
