% Part: first-order-logic
% Chapter: syntax-and-semantics
% Section: syntax

\documentclass[open-logic-section]{subfiles}

\begin{document}

\section{Syntax}

\begin{wordy}
Expressions of first-order logic are built up from a basic vocabulary
containing \emph{variables}, \emph{constants}, \emph{predicates} and
sometimes \emph{functions}.  From them, together with logical
connectives, quantifiers, and punctuation symbols such as parenteses
and commas, \emph{terms} and \emph{formulas} are formed.  

Informally, predicates are names for properties and relations,
constants are names for individual objects, and functions are names
for mappings.  These, except for the identity predicate~$\leq$, are the
\emph{non-logical symbols} and together make up a language.  Any
first-order language~$\Lang L$ is determined by its non-logical
symbols.  In the most general case, $\Lang L$ contains infinitely
many symbols of each kind.
\end{wordy}

In the general case, we make use of the following symbols in
first-order logic:

\begin{enumerate}
\item Logical symbols
\begin{enumerate}
\item Logical connectives: $\lnot$ (negation), $\land$ (conjunction),
  $\lor$ (disjunction), $\lif$ (conditional), $\liff$ (biconditional),
  $\lforall$ (universal quantifier), $\lexists$ (existential
  quantifier).
\item The propositional constant for falsity~$\lfalse$.
\item The two-place equality predicate~$\leq$.
\item A denumerable set of variables: $\Obj v_0$, $\Obj v_1$, $\Obj
  v_2$, \dots
\end{enumerate}
\item Non-logical symbols, making up the \emph{standard
  language}~$\Lang L$ of first-order logic
\begin{enumerate}
\item A denumerable set of $n$-place predicates for each $n>0$: $\Obj
  A^n_0$, $\Obj A^n_1$, $\Obj A^n_2$, \dots
\item A denumerable set of constants: $\Obj c_0$, $\Obj c_1$, $\Obj
  c_2$, \dots.
\item A denumerable set of $n$-place function symbols for each $n>0$:
  $\Obj f^n_0$, $\Obj f^n_1$, $\Obj f^n_2$, \dots
\end{enumerate}
\item Punctuation marks: (, ), and the comma.
\end{enumerate}

% Alternate symbols

\begin{intro}
You may be familiar with different symbols than the ones we use
above. Logic texts (and teachers) commonly use either $\sim$ or $\neg$
for negation, $\wedge$ or $\&$ for conjunction, and $\rightarrow$ or
$\supset$ for the conditional. Other alternatives include $\cdot$ for
conjunction, $\equiv$ for biconditional and ! for negation. It is very
common to use lower case letters (e.g. $a$, $b$, $c$) from the
beginning of the Latin alphabet for constants (sometimes called
names), and lower case letters from the end (e.g. $x$, $y$, $z$)
for variables. Quantifier variations include $\forall$ and $(x)$
(where $x$ is a variable) for the universal quantifier and
$\exists$ and $(Ex)$ for the existential quantifier.
\end{intro}

%Polish notation?

\begin{wordy}
Once a first-order language~$\Lang L$ is given, we can define
expressions built up from the basic vocabulary of~$\Lang L$.  These
include in particular \emph{terms} and \emph{formulas}.
\end{wordy}

\begin{defn}[Term]
The set of \emph{terms}~$\Trm$ of~$\Lang L$ is
defined inductively:
\begin{enumerate}
\item Every variable is a term.
\item Every constant of~$\Lang L$ is a term.
\item If $f$ is an $n$-place function symbol and $t_1$, \dots, $t_n$
  are terms, then $\Atom{f}{t_1, \ldots, t_n}$ is a term.
\item Nothing else is a term.
\end{enumerate}
A term containing no variables is a \emph{closed term}.
\end{defn}

\begin{wordy}
The constants appear in our specification of the language and the
terms as a separate category of symbols, but they could instead have
been included as zero-place function symbols.  We could then do
without the second clause in the definition of terms. We just have to
understand $\Atom{f}{t_1, \ldots, t_n}$ as just $f$ by itself if $n =
0$.
\end{wordy}

\begin{defn}[Formula]
The set of \emph{formulae}~$\Frm[L]$ of the language~$Lang L$
is defined inductively as follows:
\begin{enumerate}
\item $\bot$ is an atomic formula.
\item If $R$ is an $n$-place predicate of~$\Lang L$ and $t_1$, \dots,
  $t_n$ are terms of~$\Lang L$, then $\Atom{R}{t_1,\ldots, t_n}$ is an
  (atomic) formula.
\item If $t_1$ and $t_2$ are terms of~$\Lang L$, then $\leq[t_1][t_2]$
  is an atomic formula.
\item If $!A$ is a formula, then $\lnot !A$ is a formula.
\item If $!A$ and $!B$ are formulae, then $(!A \land !B)$ is a formula.
\item If $!A$ and $!B$ are formulae, then $(!A \lor !B)$ is a formula.
\item If $!A$ and $!B$ are formulae, then $(!A \lif !B)$ is a formula.
\item If $!A$ and $!B$ are formulae, then $(!A \liff !B)$ is a formula.
\item If $!A$ is a formula and $x$ is a variable, then $\lforall[x]\,
  !A$ is a formula.
\item If $!A$ is a formula and $x$ is a variable, then $\lexists[x]\,
  !A$ is a formula.
\item Nothing else is a formula.
\end{enumerate}
\end{defn}

\begin{wordy}
By convention, $\lnot \leq[t_1][t_2]$ is written as $\leq/[t_1][t_2]$.
\end{wordy}

\begin{intro}
Some logic texts require that the variable~$x$ must occur in~$!A$ in
order for $\lexists[x][!A]$ and $\lforall[x][!A$] to count as
formulas.  Nothing bad happens if you don't require this, and it makes
things easier.
\end{intro}

\begin{defn}[Main connective]
The \emph{main connective} (or \emph{principal} connective) of a
formula is the outermost connective of the formula (so atomic formulae
have no main connective). For example, the main connective of $(\lnot
!A)$ is $\lnot$, the main connective of $(!A \lor !B)$ is $\lor$, etc.

Furthermore, we give names to formulae with each type of main connective:

\begin{figure}[!h]
\centering
\begin{tabular}{| c | c | c |}
\hline
Main connective & Type of formula & Example\\
\hline
$\lnot$ & negation & $\lnot !A$ \\
$\land$ & conjunction & $!A \land !B$ \\
$\lor$ & disjunction & $!A \lor !B$ \\
$\lif$ & implication & $!A \lif !B$ \\
$\liff$ & double implication & $!A \liff !B$ \\
$\lforall[][]$ & universal & $\lforall[x][!A]$ \\
$\lexists[][]$ & existential & $\lexists[x][!A]$\\ \hline
\end{tabular}
\end{figure}

\end{defn}

%option 1 for subformula
\begin{defn}[Subformula]
Given a formula $!A$, the set of \emph{subformulae} of $!A$,
$\SubFrm{!A}$, is defined inductively as follows:
\begin{enumerate}
\item If $!A$ is atomic, then $\SubFrm{!A} = \{!A\}$.
\item $\SubFrm{\lnot !A} = \{\lnot !A\} \cup \SubFrm{!A}$
\item If $!B$ is also a formula, then $\SubFrm{!A \land !B} = \{!A \land !B\} \cup \SubFrm{!A} \cup \SubFrm{!B}$
\item If $!B$ is also a formula, then $\SubFrm{!A \lor !B} = \{!A \lor !B\} \cup \SubFrm{!A} \cup \SubFrm{!B}$
\item If $!B$ is also a formula, then $\SubFrm{!A \lif !B} = \{!A \lif !B\} \cup \SubFrm{!A} \cup \SubFrm{!B}$
\item If $!B$ is also a formula, then $\SubFrm{!A \liff !B} = \{!A \liff !B\} \cup \SubFrm{!A} \cup \SubFrm{!B}$
\item If $x$ is a variable, then $\SubFrm{\forall[x][!A]} = \{\lforall[x][!A]\} \cup \SubFrm{!A}$
\item If $x$ is a variable, then $\SubFrm{\lexists[x][!A]} = \{\lexists[x][!A]\} \cup \SubFrm{!A}$
\end{enumerate}
The set of all \emph{proper} subformulae of $!A$ is $\SubFrm{!A} -
\{!A\}$. That is, a formula $!B$ is a proper subformula of $!A$ if and
only if $!B$ is a subformula of $!A$ but is distinct from $!A$.
\end{defn}

\begin{defn}[Immediate subformula]
Given formulae $!A$ and $!B$, define \emph{immediate subformula} inductively as follows:
\begin{enumerate}
\item Atomic formulae have no immediate subformulae.
\item $!A$ is the immediate subformula of $\lnot !A$.
\item $!A$ and $!B$ are the immediate subformulae of $(!A \land !B)$.
\item $!A$ and $!B$ are the immediate subformulae of $(!A \lor !B)$.
\item $!A$ and $!B$ are the immediate subformulae of $(!A \lif !B)$.
\item $!A$ and $!B$ are the immediate subformulae of $(!A \liff !B)$.
\item $!A$ is the immediate subformula of $\lforall[x][!A]$.
\item $!A$ is the immediate subformula of $\lexists[x][!A]$.
% I listed these all separately rather than combining them because
% eventually we will have to be able to let people pick which logical
% symbols they use. if we can make it work without looking so "listy"
% (so we can condense the clauses), that would probably look better.
\end{enumerate}
\end{defn}

% option 2: Alternate definition of subformula via the definition of
% immediate subformula? This one avoids the SBF notation.

\begin{defn}[Subformula]
Define \emph{subformula} as follows:
\begin{enumerate}
\item If $!B$ is equal to $!A$, then $!B$ is a subformula of $!A$.
\item If $!B$ is an immediate subformula of $!A$, then $!B$ is a
  subformula of $!A$.
\item If $!C$ is a subformula of $!B$ and $!B$ is a subformula of
  $!A$, then $!C$ is also a subformula of $!A$.
\end{enumerate}
A formula $!B$ is a \emph{proper} subformula of $!A$ if and only if it
is a subformula of $!A$ and is not equal to $!A$.
\end{defn}

% option 3: Or we could do it this way, combining immediate, proper,
% (regular) subformula definitions, so this would go after the
% definition of immediate subformula

\begin{defn}[Proper subformula]
Define \emph{proper subformula} as follows:
\begin{enumerate}
\item If $!B$ is an immediate subformula of $!A$, then $!B$ is a
  proper subformula of $!A$.
\item If $!C$ is a proper subformula of $!B$ and $!B$ is a proper
  subformula of $!A$, then $!C$ is also a proper subformula of $!A$.
\end{enumerate}
\end{defn}

\begin{defn}[Subformula]
Define \emph{subformula} as follows:
\begin{enumerate}
\item If $!B$ is equal to $!A$, then $!B$ is a subformula of $!A$.
\item If $!B$ is a proper subformula of $!A$, then $!B$ is a
  subformula of~$!A$.
\end{enumerate}
\end{defn}

%This is what I figured RZ meant for an inductive definition of free
%occurrences. I didn't combine it all at once with the definition of a
%formula because I think it would be too long...

\begin{defn}[Free occurrences of a variable]
Define a \emph{free} occurrence of a variable in a formula inductively:
\begin{enumerate}
\item If $!A$ is atomic, all variables in $!A$ occur free in $!A$.
\item If $!A$ is a formula, then the free variable occurrences of
  $\lnot !A$ are exactly those in $!A$.
\item If $!A$ and $!B$ are formulae, then the free variable
  occurrences in $(!A \land !B)$ are those in $!A$ together with those
  in $!B$.
\item If $!A$ and $!B$ are formulae, then the free variable
  occurrences in $(!A \lor !B)$ are those in $!A$ together with those
  in $!B$.
\item If $!A$ and $!B$ are formulae, then the free variable
  occurrences in $(!A \lif !B)$ are those in $!A$ together with those
  in $!B$.
\item If $!A$ and $!B$ are formulae, then the free variable
  occurrences in $(!A \liff !B)$ are those in $!A$ together with those
  in $!B$.
\item If $!A$ is a formula, then the free variable occurrences in
  $\lforall[x][!A]$ are all of those in $!A$ except for occurrences of
  $x$.
\item If $!A$ is a formula, then the free variable occurrences in
  $\lexists[x][!A]$ are all of those in $!A$ except for occurrences of
  $x$.
\end{enumerate}
% Again, if we could figure out a way to condense those clauses and
% still be able to allow people to "pick and choose" which logical
% connectives to use, that would look better
\end{defn}

\begin{defn}[Sentence]
A formula $!A$ is a \emph{sentence} if and only if it is a
(well-formed) formula and contains no free occurrences of variables.
\end{defn}

% "Wordy" explanation of bound variable in here

\begin{defn}[Substitution of a term for a variable]
If $!A$ is a formula, $x$ is a variable, and $t$ is a term, $A(x/t)$
is the result of replacing all free occurrences of $x$ in $!A$ with
$t$. Substitution may be vacuous ($x$ need not occur in $!A$).
\end{defn}


\end{document}
