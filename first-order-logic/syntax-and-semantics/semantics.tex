% Part: first-order-logic
% Chapter: syntax-and-semantics
% Section: semantics

\documentclass[syntax-and-semantics]{subfiles}

\begin{document}

% I will macro this once any major changes have been suggested/implemented
\section{Semantics}

%Change then name of this module to Structures instead of Semantics? and have semantics be a larger heading containing Structures and Semantic notions?

%I like Aldo's notation of |M| for domain of M. Can we figure out some way to switch out D^M with |M|, if desired? I find the D^M notation unappealing, especially when using other superscripts 

%I replaced what I had originally from Nicole's notes to definitions from Aldo's

\begin{defn}[Structure]
A \emph{structure}, $\Struct{M}$, for a language $\Lang{L}$ of first-order logic consists of the following elements:
\begin{quote}
\begin{description}
\item[Domain:] a non-empty set, $D^\Struct{M}$ 
\item[Name assignment:] for each constant symbol $\Obj c$ of $\Lang{L}$, an element $\Obj c^\Struct{M} \in D^\Struct{M}$
\item[Relations:] for each $n$-place predicate symbol $\Obj R$ of $\Lang{L}$ (other than $=$), an $n$-ary relation $\Obj R^\Struct{M} \subseteq (D^\Struct{M})^n$
\item[Functions:] for each $n$-place function symbol $\Obj f$ of $\Lang{L}$, an $n$-ary function $\Obj f^\Struct{M} : (D^\Struct{M})^n \rightarrow D^\Struct{M}$

\end{description}
\end{quote}
\end{defn}

\begin{wordy}
Recall that a term is \emph{closed} if it is a term or if it is a function of terms only (without variables, whether free or bound).
\end{wordy}

\begin{defn}[Denotation of closed terms]
In a structure $\Struct{M}$, the \emph{denotation} of the closed term $\Obj f(\Obj t_1 \ldots \Obj t_n)^\Struct{M}$ is identical to $\Obj f^\Struct{M}(\Obj t_1^\Struct{M} \ldots \Obj t_n^\Struct{M})$. 
\end{defn}

\begin{defn}[Covered structure]
A structure is \emph{covered} if every element of the domain is the denotation of some closed term. 
\end{defn}

%Should I make d an \Obj in the following definition?
\begin{defn}[Structure variants]
The variant $\Struct{M}^\alpha_d$ of the structure $\Struct{M}$ is a structure just like $\Struct{M}$ except that it assigns the name $\alpha$ to the element $d$ in the domain of $\Struct{M}$.
\end{defn}

%Possibly we could have the following be another "boxed" section, or a wordy section, or a boxed wordy section?

The stipulations we make as to what counts as a structure impact our logic. For example, the choice to prevent empty domains ensures (given the usual account of truth for quantified sentences), that $\lexists[x][(Fx \lor \lnot Fx)]$ is a logical truth. And the stipulation that all names must refer to an object in the domain ensures that the existential generalization is a sound pattern of inference: \emph{Fa}, therefore $\lexists[x][Fx]$. If we allowed names to refer outside the domain, or to not refer, then we would be on our way to a \emph{free logic}, in which existential generalization requires an additional premise: \emph{Fa} and $\lexists[x][(x=a)]$, therefore $\lexists[x][Fx]$.

\begin{defn}[Truth in a structure]

Let $!B$ and $!C$ be sentences, $!A$ be a formula with at most $x$ free, and $t_n$ a closed term throughout. Read $\Struct{M} \Sat !A$ as ``the structure $\Struct{M}$ makes $!A$ true''.  Then truth in the structure $\Struct{M}$ is defined as follows:
\begin{enumerate}
\item $\Struct{M} \not\Sat \lfalse$
\item $\Struct{M} \Sat \ltrue$
\item $\Struct{M} \Sat \Obj R(\Obj t_1 \ldots \Obj t_n)$ iff $\langle \Obj t_1^\Struct{M} \ldots \Obj t_n^\Struct{M} \rangle \in \Obj R^\Struct{M}$
\item $\Struct{M} \Sat (\Obj t_1 = \Obj t_2)$ iff $\Obj t_1^\Struct{M}=\Obj t_2^\Struct{M}$
\item $\Struct{M} \Sat \lnot !B$ iff $\Struct{M} \not\Sat !B$
\item $\Struct{M} \Sat !B \land !C$ iff $\Struct{M} \Sat !B$ and $\Struct{M} \Sat !C$
\item $\Struct{M} \Sat !B \lor !C$ iff $\Struct{M} \Sat !B$ or $\Struct{M} \Sat !C$
\item $\Struct{M} \Sat !B \lif !C$ iff $\Struct{M} \not\Sat !B$ or $\Struct{M} \Sat !C$
\item $\Struct{M} \Sat \lexists[x][!A]$ iff there is a $d$ in the domain of $\Struct{M}$ such that $\Struct{M}^\alpha_d \Sat A!(x/\alpha)$, where $\alpha$ is new to $!A$
\item $\Struct{M} \Sat \lforall[x][!A]$ iff for every $d$ in the domain of $m$, $\Struct{M}^\alpha_d \Sat !A(x/\alpha)$, where $\alpha$ is new to $!A$
\end{enumerate}
\end{defn}

%Moved metatheoretical concepts ("semantic notions") to a new section, semantic-notions

\end{document}