% Part: first-order-logic
% Chapter: syntax-and-semantics
% Section: terms-formulas

\documentclass[../../include/open-logic-section]{subfiles}

\begin{document}

\olfileid{fol}{syn}{frm}

\olsection{Terms and Formulas}

Once a first-order language~$\Lang L$ is given, we can define
expressions built up from the basic vocabulary of~$\Lang L$.  These
include in particular \emph{terms} and \emph{!p{formulas}}.

\begin{defn}[Term]
The set of \emph{terms}~$\Trm$ of~$\Lang L$ is
defined inductively:
\begin{enumerate}
\item Every variable is a term.
\item Every !!{constant} of~$\Lang L$ is a term.
\item If $f$ is an $n$-place !!{function} and $t_1$, \dots, $t_n$
  are terms, then $\Atom{f}{t_1, \ldots, t_n}$ is a term.
\item Nothing else is a term.
\end{enumerate}
A term containing no variables is a \emph{closed term}.
\end{defn}

\begin{explain}
The constants appear in our specification of the language and the
terms as a separate category of symbols, but they could instead have
been included as zero-place function symbols.  We could then do
without the second clause in the definition of terms. We just have to
understand $\Atom{f}{t_1, \ldots, t_n}$ as just $f$ by itself if $n =
0$.
\end{explain}

\begin{defn}[Formula]
The set of \emph{!p{formulae}}~$\Frm[L]$ of the language~$\Lang L$
is defined inductively as follows:
\begin{enumerate}
\item $\lfalse$ is an atomic !!{formula}.
\item $\ltrue$ is an atomic !!{formula}.
\item If $R$ is an $n$-place !!{predicate} of~$\Lang L$ and $t_1$, \dots,
  $t_n$ are terms of~$\Lang L$, then $\Atom{R}{t_1,\ldots, t_n}$ is an
  (atomic) !!{formula}.
\item If $t_1$ and $t_2$ are terms of~$\Lang L$, then $\eq[t_1][t_2]$
  is an atomic !!{formula}.
\item If $!A$ is a !!{formula}, then $\lnot !A$ is a !!{formula}.
\item If $!A$ and $!B$ are !p{formula}, then $(!A \land !B)$ is a !!{formula}.
\item If $!A$ and $!B$ are !p{formula}, then $(!A \lor !B)$ is a !!{formula}.
\item If $!A$ and $!B$ are !p{formula}, then $(!A \lif !B)$ is a !!{formula}.
\item If $!A$ and $!B$ are !p{formula}, then $(!A \liff !B)$ is a !!{formula}.
\item If $!A$ is a !!{formula} and $x$ is a variable, then $\lforall[x]\,
  !A$ is a !!{formula}.
\item If $!A$ is a !!{formula} and $x$ is a variable, then $\lexists[x]\,
  !A$ is a !!{formula}.
\item Nothing else is a !!{formula}.
\end{enumerate}
\end{defn}

\begin{explain}
By convention, $\lnot \eq[t_1][t_2]$ is written as $\eq/[t_1][t_2]$.
\end{explain}

\begin{intro}
Some logic texts require that the variable~$x$ must occur in~$!A$ in
order for $\lexists[x][!A]$ and $\lforall[x][!A$] to count as
formulas.  Nothing bad happens if you don't require this, and it makes
things easier.
\end{intro}


\end{document}
