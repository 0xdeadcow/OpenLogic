% Part: first-order-logic
% Chapter: syntax-and-semantics
% Section: size-of-structures

\documentclass[../../include/open-logic-section]{subfiles}

\begin{document}

\olfileid{fol}{syn}{siz} 

\olsection{Expressing the size of \printtoken{p}{structure}}

\begin{explain}
There are some properties of structures we can express even without
using the non-logical symbols of a language.  For instance, there are
!!{sentence}s which are true in a !!{structure} iff the !!{domain} of
the !!{structure} has at least, at most, or exactly a certain
number~$n$ of !!{element}s.
\end{explain}

\begin{prop}
The !!{sentence}
\begin{align*}
!A_{\ge n} & \ident \lexists[x_1][\lexists[x_2][\dots\lexists[x_n][{}]]] &
  (\eq/[x_1][x_2] \land {} 
  \eq/[x_1][x_3] \land \eq/[x_1][x_4] \land \dots \land \eq/[x_1][x_n] \land {}\\
& & \eq/[x_2][x_3] \land \eq/[x_2][x_4] \land \dots \land {} \eq/[x_2][x_n] \land {} \\
& & \vdots\\
& & \eq/[x_{n-1}][x_n])
\end{align*}
is true in a !!{structure}~$\Struct M$ iff $\Domain M$ contains at
least $n$ !!{element}s. Consequently, $\Sat{M}{\lnot !A_{\ge n}}$ iff
$\Domain M$ contains at most~$n$ !!{element}s.

\end{prop}

\begin{prop}
The !!{sentence}
\begin{align*}
!A_{= n} & \ident \lexists[x_1][\lexists[x_2][\dots\lexists[x_n][{}]]] &
  (\eq/[x_1][x_2] \land {} 
  \eq/[x_1][x_3] \land \eq/[x_1][x_4] \land \dots \land \eq/[x_1][x_n] \land {}\\
& & \eq/[x_2][x_3] \land \eq/[x_2][x_4] \land \dots \land {} \eq/[x_2][x_n] \land {} \\
& & \vdots\\
& & \eq/[x_{n-1}][x_n] \land {} \\
& &\lforall[y][(\eq[y][x_1] \lor \dots \eq[y][x_n]])\dots))
\end{align*}
is true in a !!{structure}~$\Struct M$ iff $\Domain M$ contains
exactly $n$ !!{element}s.
\end{prop}

\begin{prop}
A !!{structure} is infinite iff it is a model of
\[
\{!A_{\ge 1}, !A_{\ge 2}, !A_{\ge 3}, \dots \}
\]
\end{prop}

There is no single purely logical sentence which is true in~$\Struct
M$ iff $\Domain M$ is infinite.  However, one can give !!{sentence}s with
non-logical !!{predicate}s which only have infinite models (although
not every infinite !!{structure} is a model of them).  The property of
being a finite structure, and the property of being a
!!{nonenumerable} structure cannot even be expressed with an infinite
set of !!{sentence}s.  These facts follow from the compactness and
L\"owenheim-Skolem theorems.

\end{document}
