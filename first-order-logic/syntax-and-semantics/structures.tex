% Part: first-order-logic
% Chapter: syntax-and-semantics
% Section: Structures

\documentclass[open-logic-section]{subfiles}

\begin{document}

\section{Structures for First-order Languages}

\begin{wordy}
First-order languages are, by themselves, \emph{uninterpreted:} the
constants, functions, and predicates have nospecific meaning attached
to them.  Meanings are given by specifying a \emph{structure}
(somtimes called an \emph{interpretation}).  It specifies the
\emph{domain}, i.e., the objects which the constats pick out, the
functions operate on, and the quantifiers range over. In addition, it
specifies which constants pick out which objects, how a function maps
objects to objects, and which objects the predicates apply to.
Structures are the basis for \emph{semantic} notions in logic, e.g.,
the notion of consequence, validity, satisfiablity.
\end{wordy}

\begin{defn}[Structure]
A \emph{structure}, $\Struct{M}$, for a language $\Lang{L}$ of
first-order logic consists of the following elements:
\begin{enumerate}
\item \emph{Domain:} a non-empty set, $\Domain M$ 
\item \emph{Name assignment:} for each constant symbol $c$ of
  $\Lang{L}$, an element $\Assign{c}{M} \in \Domain M$
\item \emph{Relations:} for each $n$-place predicate symbol $R$ of
  $\Lang{L}$ (other than $\leq$), an $n$-ary relation $\Assign{R}{M}
  \subseteq \Domain{M}^n$
\item \emph{Functions:} for each $n$-place function symbol $\Obj f$ of
  $\Lang{L}$, an $n$-ary function $\Assign{f}{M} \colon
  \Domain{M})^n \to \Domain{M}$
\end{enumerate}
\end{defn}

\begin{wordy}
Recall that a term is \emph{closed} if it is a term or if it is a
function of terms only (without variables, whether free or bound).
\end{wordy}

\begin{defn}[Denotation of closed terms]
If $t$ is a closed term of the langage~$\Lang L$ and $\Struct M$ i a structure for~$\Lang L$, the \emph{denotation}~$\Value{t}{M}$ is defied as follows:
\begin{enumerate}
\item If $t$ is just the constant $c$, then $\Value{c}{M} = \Assign{c}{M}$.
\item If $t$ is of the form $\Atom{f}{t_1, \ldots, t_n}$, then
  $\Value{t}{M}$ is $\Assign{f}{M}(\Value{t_1}{M}, \ldots,
  \Value{t_n}{M})$.
\end{enumerate}
\end{defn}

\begin{defn}[Covered structure]
A structure is \emph{covered} if every element of the domain is the
denotation of some closed term.
\end{defn}

The stipulations we make as to what counts as a structure impact our
logic. For example, the choice to prevent empty domains ensures (given
the usual account of truth for quantified sentences), that
$\lexists[x][(!A(x) \lor \lnot !A(x))]$ is a logical truth. And the
stipulation that all names must refer to an object in the domain
ensures that the existential generalization is a sound pattern of
inference: $!A(a)$, therefore $\lexists[x][!A(x)]$. If we allowed names
to refer outside the domain, or to not refer, then we would be on our
way to a \emph{free logic}, in which existential generalization
requires an additional premise: $!A(a)$ and $\lexists[x][\leq[x][a]]$,
therefore $\lexists[x][!A(x)]$.


\end{document}
