% Part: first-order-logic
% Chapter: completeness
% Section: maximally-consistent-sets

% Definition of maximally consistent sets. Properties of mcs required
% for completeness proved are in maximally-consistsnt-sets.tex in the
% chapter on the proof system used.

\documentclass[../../include/open-logic-section]{subfiles}

\begin{document}

\olfileid{fol}{com}{mcs}
\olsection{Maximally Consistent Sets of \usetoken{P}{sentence}}

\begin{defn}
\ollabel{mcs}
A set~$\Gamma$ of !!{sentence}s is \emph{maximally consistent} iff
\begin{enumerate}
\item $\Gamma$ is consistent, and
\item if $\Gamma \subsetneq \Gamma'$, then $\Gamma'$ is inconsistent.
\end{enumerate}
\end{defn}

\begin{explain}
An alternate definition equivalent to the above is: a set $\Gamma$ of
sentences is \emph{maximally consistent} iff
\begin{enumerate}
\item $\Gamma$ is consistent, and
\item If $\Gamma \cup \{ !A \}$ is consistent, then $!A \in \Gamma$.
\end{enumerate}
In other words, one cannot add !!{sentence}s not already in~$\Gamma$
to a maximally consistent set~$\Gamma$ without making the resulting
larger set inconsistent.
\end{explain}

\begin{explain}
Maximally consistent sets are important in the completeness proof
since we can guarantee that every consistent set of !!{sentence}s~$\Gamma$
is contained in a maximally consistent set~$\Gamma^*$, and a maximally
consistent set contains, for each !!{sentence}~$!A$, either $!A$ or its
negation $\lnot !A$. This is true in particular for atomic !!{sentence}s,
so from a maximally consistent set in a language suitably expanded by
!!{constant}s, we can construct a !!{structure} where the
interpretation of !!{predicate}s is defined according to which atomic
!!{sentence}s are in~$\Gamma^*$. This !!{structure} can then be shown to
make all !!{sentence}s in~$\Gamma^*$ (and hence also in $\Gamma$)
true. The proof of this latter fact requires that $\lnot !A \in
\Gamma^*$ iff $!A \notin \Gamma^*$, $(!A \lor !B) \in \Gamma^*$ iff $!A
\in \Gamma^*$ or $!B \in \Gamma^*$, etc.
\end{explain}

\begin{prop}
\ollabel{prop:mcs}
Suppose $\Gamma$ is maximally consistent. Then:
\begin{enumerate}
\item \ollabel{prop:mcs-prov-in} If $\Gamma \Proves !A$, then $!A \in
  \Gamma$.

\item \ollabel{prop:mcs-either-or} For any $!A$, either $!A \in
  \Gamma$ or $\lnot !A \in \Gamma$.

\tagitem{prvAnd}{\ollabel{prop:mcs-and} $(!A \land !B) \in \Gamma$
  iff both $!A \in \Gamma$ and $!B \in \Gamma$.}{}

\tagitem{prvOr}{\ollabel{prop:mcs-or} $(!A \lor !B) \in \Gamma$ iff
  either $!A \in \Gamma$ or $!B \in \Gamma$.}{}

\tagitem{prvIf}{\ollabel{prop:mcs-if} $(!A \lif !B) \in \Gamma$ iff
  either $!A \notin \Gamma$ or $!B \in \Gamma$.}{}
\end{enumerate}
\end{prop}

\begin{proof}
Let us suppose for all of the following that $\Gamma$ is maximally
consistent.
\begin{enumerate}
\item If $\Gamma \Proves !A$, then $!A \in \Gamma$.

Suppose that $\Gamma \Proves !A$. Suppose to the contrary that $!A
\notin \Gamma$: then since $\Gamma$ is maximally consistent, $\Gamma
\cup \{!A\}$ is inconsistent, hence $\Gamma \cup \{!A\} \Proves
\lfalse$. By
\tagrefs{prfSC/{fol:seq:prv:prop:provability-contr},prfND/{fol:ntd:prv:prop:provability-contr}},
$\Gamma$ is inconsistent. This contradicts the assumption that
$\Gamma$ is consistent. Hence, it cannot be the case that $!A \notin
\Gamma$, so $!A \in \Gamma$.

\item For any $!A$, either $!A \in \Gamma$ or $\lnot !A \in \Gamma$.

Suppose to the contrary that for some~$!A$ both $!A \notin \Gamma$ and
$\lnot !A \notin \Gamma$. Since $\Gamma$ is maximally consistent,
$\Gamma \cup \{!A\}$ and $\Gamma \cup \{\lnot !A\}$ are both
inconsistent, so $\Gamma \cup \{!A\} \Proves \lfalse$ and $\Gamma \cup
\{\lnot !A\} \Proves \lfalse$. By
\tagrefs{prfSC/{fol:seq:prv:prop:provability-exhaustive},prfND/{fol:ntd:prv:prop:provability-exhaustive}},
$\Gamma$ is inconsistent, a contradiction. Hence there cannot be such
a !!{sentence}~$!A$ and, for every $!A$, $!A \in \Gamma$ or $\lnot !A
\in \Gamma$.

\tagitem{defAnd}{}{%
\iftag{probAnd}{Exercise.}{%
$(!A \land !B) \in \Gamma$ iff both $!A \in \Gamma$ and $!B \in \Gamma$:

For the forward direction, suppose $(!A \land !B) \in \Gamma$. Then
$\Gamma \Proves !A \land !B$. By
\tagrefs{prfSC/{fol:seq:prv:prop:provability-land-left},prfND/{fol:ntd:prv:prop:provability-land-left}},
$\Gamma \Proves !A$ and $\Gamma \Proves !B$. By
\olref{prop:mcs-prov-in}, $!A \in \Gamma$ and $!B \in \Gamma$, as
required.

For the reverse direction, let $!A \in \Gamma$ and $!B \in
\Gamma$. Then $\Gamma \Proves !A$ and $\Gamma \Proves !B$. By
\tagrefs{prfSC/{fol:seq:prv:prop:provability-land-right},prfND/{fol:ntd:prv:prop:provability-land-right}},
$\Gamma \Proves !A \land !B$. By \olref{prop:mcs-prov-in}, $(!A \land
!B) \in \Gamma$.}}

\tagitem{defOr}{}{%
\iftag{probOr}{Exercise.}{%
$(!A \lor !B) \in \Gamma$ iff either $!A \in \Gamma$ or $!B \in \Gamma$.

For the contrapositive of the forward direction, suppose that $!A
\notin \Gamma$ and $!B \notin \Gamma$. We want to show that $(!A \lor
!B) \notin \Gamma$. Since $\Gamma$ is maximally consistent, $\Gamma
\cup \{!A\} \Proves \lfalse$ and $\Gamma \cup \{!B\} \Proves \lfalse$.
By
\tagrefs{prfSC/{fol:seq:prv:prop:provability-lor-left},prfND/{fol:ntd:prv:prop:provability-lor-left}},
$\Gamma \cup \{(!A \lor !B)\}$ is inconsistent. Hence, $(!A \lor !B)
\notin \Gamma$, as required.

For the reverse direction, suppose that $!A \in \Gamma$ or $!B \in
\Gamma$. Then $\Gamma \Proves !A$ or $\Gamma \Proves !B$. By
\tagrefs{prfSC/{fol:seq:prv:prop:provability-lor-right},prfND/{fol:ntd:prv:prop:provability-lor-right}},
$\Gamma \Proves !A \lor !B$. By \olref{prop:mcs-prov-in}, $(!A \lor
!B) \in \Gamma$, as required.}}

\tagitem{defIf}{}{%
\iftag{probIf}{Exercise.}{%
$(!A \lif !B) \in \Gamma$ iff either $!A \notin \Gamma$ or $!B \in \Gamma$:

For the forward direction, let $(!A \lif !B) \in \Gamma$, and suppose
to the contrary that $!A \in \Gamma$ and $!B \notin \Gamma$. On these
assumptions, $\Gamma \Proves !A \lif !B$ and $\Gamma \Proves !A$. By
\tagrefs{prfSC/{fol:seq:prv:prop:provability-mp},prfND/{fol:ntd:prv:prop:provability-mp}},
$\Gamma \Proves !B$. But then by \olref{prop:mcs-prov-in}, $!B \in
\Gamma$, contradicting the assumption that $!B \notin \Gamma$.

For the reverse direction, first consider the case where $!A \notin
\Gamma$. By \olref{prop:mcs-either-or}, $\lnot !A \in \Gamma$ and
hence $\Gamma \Proves \lnot !A$. By
\tagrefs{prfSC/{fol:seq:prv:prop:provability-lif},prfND/{fol:ntd:prv:prop:provability-lif}},
$\Gamma \Proves !A \lif !B$. Again by \olref{prop:mcs-prov-in}, we get
that $(!A \lif !B) \in \Gamma$, as required.

Now consider the case where $!B \in \Gamma$. Then $\Gamma \Proves !B$
and by
\tagrefs{prfSC/{fol:seq:prv:prop:provability-lif},prfND/{fol:ntd:prv:prop:provability-lif}},
$\Gamma \Proves !A \lif !B$. By \olref{prop:mcs-prov-in}, $(!A \lif
!B) \in \Gamma$.}}
\end{enumerate}
\end{proof}

\begin{prob}
Complete the proof of \olref[fol][com][mcs]{prop:mcs}.
\end{prob}

\end{document}
