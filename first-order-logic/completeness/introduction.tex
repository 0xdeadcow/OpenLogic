% Part: first-order-logic
% Chapter: completeness
% Section: introduction

\documentclass[../../include/open-logic-section]{subfiles}

\begin{document}

\olfileid{fol}{com}{int}
\olsection{Introduction}

The completeness theorem is one of the most fundamental results about
logic.  It comes in two formulations, the equivalence of which we'll
prove.  In its first formulation it says something fundamental about
the relationship between semantic consequence and our proof system: if
a !!{sentence}~$!A$ follows from some !!{sentence}s $\Gamma$, then
there is also a !!{derivation} that establishes $\Gamma \Proves !A$.
Thus, the proof system is as strong as it can possibly be without
proving things that don't actually follow.  In its second formulation,
it can be stated as a model existence result: every consistent set of
!!{sentence}s is satisfiable.

These aren't the only reasons the completeness theorem---or rather,
its proof---is important.  It has a number of important consequences,
some of which we'll discuss separately.  For instance, since any
!!{derivation} that shows $\Gamma \Proves !A$ is finite and so can
only use finitely many of the !!{sentence}s in~$\Gamma$, it follows by
the completeness theorem that if $!A$ is a consequence of $\Gamma$, it
is already a consequence of a finite subset of~$\Gamma$.  This is
called \emph{compactness}.  Equivalently, if every finite subset of
$\Gamma$ is consistent, then $\Gamma$ itself must be consistent.  It
also follows from \emph{the proof of} the completeness theorem that
any satisfiable set of !!{sentence}s has a finite or !!{denumerable}s
model.  This result is called the L\"owenheim-Skolem theorem.

\end{document}
