% Part: first-order-logic
% Chapter: completeness
% Section: henkin-expansion

\documentclass[../../include/open-logic-section]{subfiles}

\begin{document}

\olfileid{fol}{com}{hen}

\olsection{Henkin Expansion}

\begin{explain}
Part of the challenge in proving the completeness theorem is that the
model we construct from a maximally consistent set~$\Gamma$ must make
all the quantified !!{formula}s in~$\Gamma$ true.  In order to
guarantee this, we use a trick due to Leon Henkin.  In essence, the
trick consists in expanding the language by infinitely many constants
and adding, for each !!{formula} with one free !!{variable} $!A(x)$ a
formula of the form 
\iftag{prvEx}
      {$\lexists[x][!A] \lif !A(c)$}
      {$\lnot\lforall[x][!A] \lif \lnot !A(c)$},
where $c$ is one of the new !!{constant}s.  When we construct the
!!{structure} satisfying~$\Gamma$, this will guarantee that each
\iftag{prvEx}
{true existential sentence has a witness}
{false universal sentence has a counterexample}
among the new constants.
\end{explain}

\begin{lem}
If $\Gamma$ is consistent in $\Lang L$ and $\Lang L'$ is obtained from
$\Lang L$ by adding !!a{denumerable} set of new !!{constant}s $c_1$,
$c_2$, \dots, then $\Gamma$ is consistent in~$\Lang L'$.
\end{lem}

\begin{defn}
A set $\Gamma$ of !!{formula}s of a language $\Lang {L}$ is
\emph{saturated} if and only if for each !!{formula} $!A \in \Frm[L]$
and !!{variable}~$x$ there is a !!{constant} $c$ such that
\iftag{prvEx}
      {$\lexists[x][!A] \lif !A(c) \in \Gamma$}
      {$\lnot\lforall[x][!A] \lif \lnot !A(c) \in \Gamma$}.
\end{defn}

The following definition will be used in the proof of the next theorem.

\begin{defn}
Fix an enumeration $\tuple{!A_1, x_1}$, $\tuple{!A_2, x_2}$, \dots of
all !!{formula}-!!{variable} pairs of~$\Lang L'$, where $x_i$ is the
only free !!{variable} in~$!A_i$.  We define the !!{sentence}s~$!D_n$
by recursion on~$n$. Assuming that $!D_1$, \dots,~$!D_n$ have already
been defined, let $c_{n+1}$ the first new !!{constant} not occurring
in $!D_1$, \dots,~$!D_n$, and let $!D_{n+1}$ be the !!{formula}
\iftag{prvEx} {$\lexists[x_{n+1}][!A_{n+1}(x_{x+1})] \lif
  !A_{n+1}(c_{n+1})$} {$\lnot\lforall[x_{n+1}][!A_{n+1}(x_{n+1})] \lif
  \lnot !A_{n+1}(c_{n+1})$}.  This includes the case where $n = 0$ and
the list of previous $!D_i$'s is empty, i.e., $!D_1$ is 
\iftag{prvEx}
{$\lexists[x_1][!A_1] \lif !A_1(c_1)$} 
{$\lnot\lforall[x_1][!A_1] \lif \lnot !A_1(c_1)$}.
\end{defn}

\begin{thm}
Every consistent set~$\Gamma$ can be extended to a saturated
consistent set~$\Gamma'$.
\end{thm}

\begin{proof}
Given a consistent set of sentences~$\Gamma$ in a language~$\Lang{L}$,
expand the language by adding !!a{denumerable} set of new
!!{constant}s to~$\Lang{L'}$. By the previous Lemma, $\Gamma$ is still
consistent in the richer language. Further, let $!D_i$ be as in the
previous definition: then $\Gamma \cup \{!D_1, !D_2, \dots\}$ is
saturated by construction. Let
\begin{align*}
\Gamma_0 & = \Gamma \\
\Gamma_{n+1} & = \Gamma_n \cup \{!D_{n+1} \}
\end{align*}
i.e., $\Gamma_n = \Gamma \cup \{ !D_1, \dots, !D_n \}$, and let
$\Gamma' = \bigcup_{n} \Gamma_n$.  To show that $\Gamma'$ is
consistent it suffices to show, by induction on~$n$, that each
set~$\Gamma_n$ is consistent.

The induction basis is simply the claim that $\Gamma_0 = \Gamma$ is
consistent, which is the hypothesis of the theorem.  For the induction
step, suppose that $\Gamma_{n-1}$ is consistent but $\Gamma_n =
\Gamma_{n-1} \cup \{!D_n\}$ is inconsistent. Recall that $!D_n$~is
\iftag{prvEx}
{$\lexists[x_{n}][!A_{n}(xn)] \lif !A_{n}(c_{n})$}
{$\lnot\lforall[x_{n}][!A_{n}(x_n)] \lif \lnot !A_{n}(c_{n})$}. 
where $!A(x)$ is a !!{formula} of $\Lang{L'}$ with only the
variable~$x_n$ free and not containing any !!{constant}s~$c_i$ where
$i \ge n$.

If $\Gamma_{n-1} \cup \{!D_n\}$ is inconsistent, then $\Gamma_{n-1}
\Proves \lnot !D_n$, and hence both of the following hold:
\iftag{prvEx}{
\[
\Gamma_{n-1} \Proves \lexists[x_n][!A_n(x_n)]
\qquad
\Gamma_{n-1} \Proves \lnot !A_n(c_n)
\]}{
\[
\Gamma_{n-1} \Proves \lnot\lforall[x_n][!A_n(x_n)]
\qquad
\Gamma_{n-1} \Proves !A_n(c_n)
\]}
Here $c_n$ does not occur in $\Gamma_{n-1}$ or $!A_n(x_n)$ (remember,
it was added only with~$!D_n$). By
\olref[seq][prv]{thm:strong-generalization}, from
\iftag{prvEx}
{$\Gamma \Proves \lnot !A_n(c_n)$} 
{$\Gamma \Proves !A_n(c_n)$}, 
we obtain 
\iftag{prvEx}
{$\Gamma \Proves \lforall[x_n][\lnot !A_n(x_n)]$}
{$\Gamma \Proves \lforall[x_n][!A_n(x_n)]$}.
Thus we have that both 
\iftag{prvEx}
{$\Gamma_{n-1} \Proves \lexists[x_n][!A_n]$ and 
$\Gamma_{n-1} \Proves \lforall[x_n][\lnot !A_n(x_n)]$}
{$\Gamma_{n-1} \Proves \lnot\lforall[x_n][!A_n(x_n)]$ and 
$\Gamma_{n-1} \Proves \lforall[x_n][!A_n(x_n)]$},
so $\Gamma$ itself is inconsistent. 
\iftag{prvEx}
{(Note that 
\iftag{prvAll}
{$\lforall[x_n][\lnot !A_n] \Proves
  \lnot\lexists[x_n][!A_n(x_n)]$}
{$\lforall[x_n][\lnot !A_n]$ is defined as
  $\lnot\lexists[x_n][\lnot\lnot !A_n(x_n)]$ and
  $\lnot\lexists[x_n][\lnot\lnot !A_n(x_n)] \Proves
  \lnot\lexists[x_n][!A_n(x_n)]$}.)}{}
Contradiction: $\Gamma_{n-1}$ was supposed to be consistent. Hence
$\Gamma_n \cup \{ !D_n\}$ is consistent.
\end{proof}

\end{document}
