% Part: first-order-logic
% Chapter: completeness
% Section: henkin-expansion

\documentclass[open-logic-section]{subfiles}

\begin{document}

\olfileid{fol}{com}{hen}
\olsection{Henkin Expansion}

\begin{wordy}
Part of the challene in proving the completeness theorem is that the
model we construct from a maximally consistent set~$\Gamma$ must make
all the quantified formulas in~$\Gamma$ true.  In order to guarantee
this, we use a trick due to Leon Henkin.  In essence, the trick
consists in expanding the language by infinitely many constants and
adding, for each formula with one free variable $[!A(x)]$ a formula of
the form $\lexists[x][!A(x)] \lif !A(c)$, where $c$ is one of the new
constant symbols.  
\end{wordy}

\begin{defn}
  A set $\Gamma$ of formulas of a language $\Lang {L}$ is
  \emph{saturated} if and only if for each formula $!A \in
  \Frm {L}$ and variable $x$ there is a constant $c$ such that
  $\lexists[x][!A] \lif !A(c) \in \Gamma$.
\end{defn}

% do we need to also add conditionals for the universally quantified 
% formulas or does max. consistency take care of those?

\begin{prop}
  Every consistent set $\Gamma$ can be extended to a saturated
  consistent set.
\end{prop}

% proof in Aldo's notes, prop:saturation

\end{document}
