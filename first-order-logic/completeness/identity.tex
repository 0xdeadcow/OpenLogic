% Part: first-order-logic
% Chapter: completeness
% Section: identity

\documentclass[open-logic-section]{subfiles}

\begin{document}

\olfileid{fol}{com}{ide}
\olsection{Identity}

\begin{wordy}
The construction of the term model given in the preceding section is
enough to establish completeness for first-order logic without
identity.  We could define satisfaction and provability for a language
without~$leq$, and construct a maximally consistent saturated set
$\Gamma^*$ for a given se $\Gamma$.  The term model would then be a
model of $\Gamma^*$ (and hence of $\Gamma$).  It does not work,
however, if identity is present.  The reason is that $\Gamma$ might
contain a sentence $\eq[t][t']$, but in the term model the value of
any term is that term itself. Hence, if $t$ and $t'$ are different
terms, their values in the term model, i.e., $t$ and $t'$
respectively, are differnt and so $\eq[t][t']$ is false.  We can fix
this, however, using a construction known as ``factoring.''  This
construction works generally for so-called congruence relations.
\end{wordy}

\begin{defn}
Let $\Gamma^*$ be a maximally consistent set of sentences in~$\Lang
L$. We define the relation $\approx$ on the set of closed terms
of~$\Lang L$ by
\[
t \approx t' \text{\quad iff \quad} \eq[t][t'] \in \Gamma^*
\]
\end{defn}

\begin{prop}
The relation $\approx$ has the following properties:
\begin{enumerate}
\item $\approx$ is reflexive
\item $\approx$ is symmetric
\item  $\approx$ is transitive
\item If $t \approx t'$, $f$ is a function, and $t_1$, \dots, $t_n$ are
  terms, then $f(t_1,\dots, t, \dots t_n) \approx f(t_1, \dots, t',
  \dots, t_n)$.
\end{enumerate}
\end{prop}

% Prove

\begin{defn}
Define $\Trm/\approx$.
\end{defn}

\begin{defn}
Define term model mod $\approx$
\end{defn}

\begin{prop}
$\Sat{M(\Gamma^*)/\approx}{!A}$ iff $!A \in \Gamma^*$.
\end{prop}

\end{document}
