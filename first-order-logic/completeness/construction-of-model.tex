% Part: first-order-logic
% Chapter: completeness
% Section: construction-of-model

\documentclass[../../include/open-logic-section]{subfiles}

\begin{document}

\olfileid{fol}{com}{mod}
\olsection{Construction of Model}

\begin{defn}
\ollabel{termmodel}
Let $\Gamma^*$ be a maximally consistent, saturated set of sentences
in a language~$\Lang L$.  The \emph{term model}~$\Struct M(\Gamma^*)$
of $\Gamma^*$ is the first-order !!{structure} defined as follows:
\begin{enumerate}
\item The domain~$\Domain{M(\Gamma^*)}$ is the set of all closed terms
  of~$\Lang L$.
\item The interpretation of a constant $c$ is $c$ itself:
  $\Assign{c}{M(\Gamma^*)} = c$.
\item The function~$f$ is assigned the function
\[
\Assign{f}{M(\Gamma^*)}(t_1, \dots, t_n) = f(\Value{t_1}{M(\Gamma^*)},
\dots, \Value{t_1}{M(\Gamma^*)})
\]
\item If $R$ is an $n$-place !!{predicate}, then $\langle t_1, \dots,
  t_n\rangle \in \Assign{R}{M(\Gamma^*)}$ iff $\Atom{R}{t_1, \dots,
    t_n} \in \Gamma^*$.
\end{enumerate}
\end{defn}

\begin{lem}[Truth Lemma]
\ollabel{truth}
$\Sat{M(\Gamma^*)}{!A}$ iff $!A \in \Gamma^*$
\end{lem}

\begin{proof}
For the forward direction, let $\Sat{M(\Gamma^*)}{!A}$. We want to show that $!A \in \Gamma^*$. \\

Base Case: If $!A$ is atomic, then $!A = R(t_1, \dots, t_n)$ for $R$ an $n$-place !!{predicate}. Hence 
$\langle t_1, \dots, t_n \rangle \in R^{\Struct M(\Gamma^*)}$, and by the construction of $\Struct 
M(\Gamma^*)$, $R(t_1, \dots, t_n) \in \Gamma^*$.\\

Induction Hypothesis: Fix $k$ and let $!A \in \Gamma^*$ for any $!A$ with degree $<k$. \\

Induction Step: Proceed by cases determined by the main connective of $!A$.

\begin{enumerate}

\item \tagitem{prvAnd} If $!A= !B \land !C$, then since $\Sat{M(\Gamma^*)}{!B \land !C}$, we have both 
$\Sat{M(\Gamma^*)}{!B}$ and $\Sat{M(\Gamma^*)}{!C}$. By the induction hypothesis, we get both 
that $!B \in \Gamma^*$ and that $!C \in \Gamma^*$. By the maximal consistency of $\Gamma^*$, 
$!B \land !C \in \Gamma^*$.

\item \tagitem{prvOr} If $!A = !B \lor !C$, then since $\Sat{M(\Gamma^*)}{!B \lor !C}$, we have at least one of 
$\Sat{M(\Gamma^*)}{!B}$ or $\Sat{M(\Gamma^*)}{!C}$. If $\Sat{M(\Gamma^*)}{!B}$, then by the 
induction hypothesis, $!B \in \Gamma^*$, so by the maximal consistency of $\Gamma^*$, $!B \lor !C 
\in \Gamma^*$ (similar if $\Sat{M(\Gamma^*)}{!C}$).

\item \tagitem{prvNot} If $!A=\lnot !B$, consider the following cases according to the main connective of $!B$:
\begin{enumerate}

\item \tagitem{prvNot} If $!B = \lnot !C$, then $\Sat{M(\Gamma^*)}{\lnot\lnot !C}$, so $\Sat{M(\Gamma^*)}{!C}$. 
Thus by the induction hypothesis, $!C \in \Gamma^*$, so by the maximal consistency of 
$\Gamma^*$, $\lnot !C \notin \Gamma^*$, so $!A = \lnot \lnot !C \in \Gamma^*$. 

\item \tagitem{prvAnd} If $!B = !C \land !D$, then $\Sat/{M(\Gamma^*)}{!C}$ or $\Sat/{M(\Gamma^*)}{!D}$. If 
$\Sat/{M(\Gamma^*)}{!C}$, then by the induction hypothesis, $\lnot !C \in \Gamma^*$, so $!C \land !D 
\notin \Gamma^*$, hence $\lnot (!C \land !D) \in \Gamma^*$ (similar if $\Sat/{M(\Gamma^*)}{!D}$).

\item \tagitem{prvOr} If $!B = !C \lor !D$, then $\Sat/{M(\Gamma^*)}{!C}$ and $\Sat/{M(\Gamma^*)}{!D}$. Hence, $!C \notin 
\Gamma^*$ and $!D \notin \Gamma^*$, so $!C \lor !D \notin \Gamma^*$. Therefore, $\lnot (!C \lor !D) \notin 
\Gamma^*$.

\item \tagitem{prvIf} If $!B = !C \lif !D$, then $\Sat{M(\Gamma^*)}{\lnot(!C \lif !D)}$, hence $\Sat{M(\Gamma^*)}{!C}$ and 
$\Sat{M(\Gamma^*)}{\lnot !D}$. By the induction hypothesis, $!C \in \Gamma^*$ and $\lnot !D \in 
\Gamma^*$, so by the maximal consistency of $\Gamma^*$, $!C \lif !D \notin \Gamma^*$. Thus $\lnot 
(!C \lif !D) \in \Gamma^*$.

\item \tagitem{prvAll} If $!B=\lforall[x][!C]$, then $\Sat{M(\Gamma^*)}{\lnot \lforall[x][!C]}$. Hence, for some variable 
assignment $s$, $\Sat/{M(\Gamma^*)}{!C}[s]$. If $s(x)=c$, then that means $\Sat/{M(\Gamma^*)}{!C(x/c)}$, 
so by the induction hypothesis, $\lnot !C(c) \in \Gamma^*$. By \olref[seq]{thm:provability-quantifiers}, since 
$\Gamma^* \Proves \lnot !C (c)$, we have $\Gamma^* \Proves \lnot \lforall[x][!C]$. By the maximal 
consistency of $\Gamma^*$, $\lnot \lforall[x][!C]\in \Gamma^*$. 

\item \tagitem{prvEx} If $!B = \lexists[x][!C]$, then $\Sat{M(\Gamma^*)}{\lnot \lexists[x][!C]}$. Hence, for any variable assignment 
$s$, $\Sat/{M(\Gamma^*)}{!C}[s]$. By saturation, $\lnot \lforall[x](\lnot !C) \lif \lnot (\lnot !C(c)) \in \Gamma^*$, 
hence $\lforall[x][\lnot !C] \in \Gamma^*$ or $!C (c) \in \Gamma^*$. The latter cannot be, since if $s'$ is the 
variable assignment such that $s'(x)=c$, then $\Sat/{M(\Gamma^*)}{!C}[s']$ and the induction hypothesis imply 
that $!C (c) \notin \Gamma^*$. Hence, we must have $\lforall[x][\lnot!C] \in \Gamma^*$. Since $\{ \lforall[x][\lnot !C] \} 
\Proves \lnot \lexists[x][!C]$, by the maximal consistency of $\Gamma^*$, we conclude that $\lnot 
\lexists[x][!C] \in \Gamma^*$. 

\end{enumerate}

\tagitem{prvNot} Hence, if $!A$ is a negation, the induction step goes through.

\item \tagitem{prvIf} If $!A = !B \lif !C$, then $\Sat{M(\Gamma^*)}{!B \lif !C}$, so $\Sat/{M(\Gamma^*}{!B}$ or $\Sat{M 
(\Gamma^*)}{!C}$. If we have the latter, then by the induction hypothesis, $!C \in \Gamma^*$, so $!B \lif !C \in 
\Gamma^*$. If the former, by the same argument as in the previous case (case 4), $\lnot !B \in \Gamma^*$, hence 
by the maximal consistency of $\Gamma^*$, $!B \notin \Gamma^*$, and $!B \lif !C \in \Gamma^*$.

\item \tagitem{prvAll} If $!A= \lforall[x][!B]$. Then $\Sat{M(\Gamma^*)}{\lforall[x][!B]}$, so for every variable assignment $s$, 
$\Sat{M(\Gamma^*)}{!B}[s]$. Suppose to the contrary that $\lforall[x][!B]\notin \Gamma^*$: then by saturation, 
$\lnot !B(c) \in \Gamma^*$ for some $c$. However, if $s'$ is the variable assignment such that $s'(x)=c$, then 
$s'$ would satisfy $\Sat{M(\Gamma^*)}{!B}[s']$, contradicting the earlier result that $\Sat/{M(\Gamma^*)}{!B}[s']$. 
Therefore, we must have $\lforall[x][!B] \in \Gamma^*$.

\item \tagitem{prvEx} Lastly, consider when $!A = \lexists[x][!B]$. Then for some variable assignment $s$ with $s(x)=c$, 
$\Sat{M(\Gamma^*)}{!B}[s]$. Thus, $\Sat{M(\Gamma^*)}{!B(c)}$, and by our induction hypothesis, $!B(c) \in \Gamma^*$. 
By \olref[seq]{thm:provability-quantifiers}, since $\Gamma^* \Proves !B(c)$, we have $\Gamma^* \Proves \lexists[x][!B]$. 
Then, by the maximal consistency of $\Gamma^*$, we conclude that $\lexists[x][!B] \in \Gamma^*$.

\end{enumerate}

Therefore, the induction step for the forward direction goes through.\\

For the reverse direction, let $!A \in \Gamma^*$. Suppose to the contrary that $\Sat/{M(\Gamma^*)}{!A}$. Then 
$\Sat{M(\Gamma^*)}{\lnot !A}$, so by the forward direction of the proof, $\lnot !A \in \Gamma^*$. Contradiction: $\Gamma^*$ 
is consistent. Hence, we must have that $\Sat{M(\Gamma^*)}{!A}$.

\end{proof}

\end{document}
