% Part: first-order-logic
% Chapter: completeness
% Section: construction-of-model

\documentclass[../../include/open-logic-section]{subfiles}

\begin{document}

\olfileid{fol}{com}{mod}

\olsection{Construction of a Model}

We will begin by showing how to construct !!a{structure} which
satisfies a maximally consistent, saturated set of !!{sentence}s in a
language~$\Lang L$ without~$\eq$.

\begin{defn}
\ollabel{termmodel}
Let $\Gamma^*$ be a maximally consistent, saturated set of !!{sentence}s
in a language~$\Lang L$.  The \emph{term model}~$\Struct M(\Gamma^*)$
of $\Gamma^*$ is the !!{structure} defined as follows:
\begin{enumerate}
\item The !!{domain}~$\Domain{M(\Gamma^*)}$ is the set of all closed terms
  of~$\Lang L$.
\item The interpretation of !!a{constant} $c$ is $c$ itself:
  $\Assign{c}{M(\Gamma^*)} = c$.
\item The !!{function}~$f$ is assigned the function
\[
\Assign{f}{M(\Gamma^*)}(t_1, \dots, t_n) = f(\Value{t_1}{M(\Gamma^*)},
\dots, \Value{t_1}{M(\Gamma^*)})
\]
\item If $R$ is an $n$-place !!{predicate}, then $\tuple{t_1, \dots,
  t_n} \in \Assign{R}{M(\Gamma^*)}$ iff $\Atom{R}{t_1, \dots,
    t_n} \in \Gamma^*$.
\end{enumerate}
\end{defn}

\begin{lem}[Truth Lemma]
\ollabel{lem:truth} Suppose $!A$ does not contain~$\eq$. Then
$\Sat{M(\Gamma^*)}{!A}$ iff $!A \in \Gamma^*$.
\end{lem}

\begin{proof}
We prove both directions simultaneously, and by induction on $!A$.
\begin{enumerate}
\item \indcase{!A}{R(t_1, \dots,
  t_n)}{$\Sat{M(\Gamma^*)}{\Atom{R}{t_1, \dots, t_n}}$ iff
  $\tuple{t_1, \dots, t_n} \in \Assign{R}{M(\Gamma^*)}$ (by the
  definition of satisfaction) iff $R(t_1, \dots, t_n) \in \Gamma^*$
  (the construction of $\Struct M(\Gamma^*)$.}

\tagitem{prvNot}{% 
\indcase{!A}{\lnot !B}{$\Sat{M(\Gamma^*)}{\indfrm}$ iff
  $\Sat/{M(\Gamma^*)}{!B}$ (by definition of satisfaction).  By
  induction hypothesis, $\Sat/{M(\Gamma^*)}{!B}$ iff $!B \notin
  \Gamma^*$. By \olref[mcs]{prop:mcs}\olref[mcs]{prop:mcs-either-or},
  $\lnot !B \in \Gamma^*$ if $!B \notin \Gamma^*$; and $\lnot !B
  \notin \Gamma^*$ if $!B \in \Gamma^*$ since $\Gamma^*$ is
  consistent.}}{}

\tagitem{prvAnd}{% 
\iftag{probAnd}{%
\indcase!{!A}{!B \land !C}{}}{%
\indcase{!A}{!B \land !C}{$\Sat{M(\Gamma^*)}{\indfrm}$ iff we have
  both $\Sat{M(\Gamma^*)}{!B}$ and $\Sat{M(\Gamma^*)}{!C}$ (by
  definition of satisfaction) iff both $!B \in \Gamma^*$ and $!C \in
  \Gamma^*$ (by the induction hypothesis). By
  \olref[mcs]{prop:mcs}\olref[mcs]{prop:mcs-and}, this is the case
  iff $(!B \land !C) \in \Gamma^*$.}}}{}

\tagitem{prvOr}{% 
\iftag{probOr}{%
\indcase!{!A}{!B \lor !C}{}}{%
\indcase{!A}{!B \lor !C}{$\Sat{M(\Gamma^*)}{\indfrm}$ iff at
  $\Sat{M(\Gamma^*)}{!B}$ or $\Sat{M(\Gamma^*)}{!C}$ (by definition of
  satisfaction) iff $!B \in \Gamma^*$ or $!C \in \Gamma^*$ (by
  induction hypothesis). This is the case iff $(!B \lor !C) \in
  \Gamma^*$ (by \olref[mcs]{prop:mcs}\olref[mcs]{prop:mcs-or}).}}}{}

\tagitem{prvIf}{%
\iftag{probIf}{%
\indcase!{!A}{!B \lif !C}{}}{%
\indcase{!A}{!B \lif !C}{$\Sat{M(\Gamma^*)}{\indfrm}$ iff
  $\Sat/{M(\Gamma^*}{!B}$ or $\Sat{M (\Gamma^*)}{!C}$ (by definition
  of satisfaction) iff $!B \notin \Gamma^*$ or $!C \in \Gamma^*$ (by
  induction hypothesis). This is the case iff $(!B \lif !C) \in
  \Gamma^*$ (by \olref[mcs]{prop:mcs}\olref[mcs]{prop:mcs-if}).}}}{}

\tagitem{prvAll}{%
\iftag{probAll}{%
\indcase!{!A}{\lforall[x][!B(x)]}{}}{%
\indcase{!A}{\lforall[x][!B(x)]}{Suppose that
  $\Sat{M(\Gamma^*)}{\indfrm}$, then for every variable assignment $s$,
  $\Sat{M(\Gamma^*)}{!B(x)}[s]$. Suppose to the contrary that
  $\lforall[x][!B(x)]\notin \Gamma^*$: Then by
  \olref[mcs]{prop:mcs}\olref[mcs]{prop:mcs-either-or},
  $\lnot\lforall[x][!B(x)] \in \Gamma^*$. By saturation,
\iftag{prvEx}
      {$(\lexists[x][\lnot !B(x)] \lif \lnot !B(c)) \in \Gamma^*$}
      {$(\lnot\lforall[x][!B(x)] \lif \lnot !B(c)) \in \Gamma^*$}
  for some $c$, so by
  \olref[mcs]{prop:mcs}\olref[mcs]{prop:mcs-prov-in}, $\lnot !B(c) \in
  \Gamma^*$.  Since $\Gamma^*$ is consistent, $!B(c) \notin
  \Gamma^*$. By induction hypothesis, $\Sat/{M(\Gamma^*)}{!B(c)}$.
  Therefore, if $s'$ is the variable assignment such that $s(x)=c$,
  then $\Sat/{M(\Gamma^*)}{!B(x)}[s']$, contradicting the earlier
  result that $\Sat{M(\Gamma^*)}{!B(x)}[s]$ for all~$s$.  Thus, we
  have $\indfrm \in \Gamma^*$.

Conversely, suppose that $\lforall[x][!B(x)] \in \Gamma^*$. By
\olref[seq][prv]{thm:provability-quantifiers} and
\olref[mcs]{prop:mcs}\olref[mcs]{prop:mcs-prov-in}, $!B(t) \in
\Gamma^*$ for every term~$t \in \Domain{M(\Gamma^*)}$. By inductive
hypothesis, $\Sat{M(\Gamma^*)}{!B(t)}$ for every term~$t \in
\Domain{M(\Gamma^*)}$. Let $s$ be the variable assigment with $s(x) =
t$. Then $\Sat{M(\Gamma^*)}{!B(x)}[s]$ for any such~$s$, hence
$\Sat{M(\Gamma^*)}{\indfrm}$.}}}{}

\tagitem{prvEx}{%
\iftag{probEx}{%
\indcase!{!A}{\lexists[x][!B(x)]}{}}{%
\indcase{!A}{\lexists[x][!B(x)]}{First suppose that $\Sat{M(\Gamma^*)}{\indfrm}$. By the definition of satisfaction, for
  some variable assignment $s$, $\Sat{M(\Gamma^*)}{!B(x)}[s]$. The
  value $s(x)$ is some term~$t \in \Domain{M(\Gamma^*)}$. Thus,
    $\Sat{M(\Gamma^*)}{!B(t)}$, and by our induction hypothesis,
    $!B(t) \in \Gamma^*$.  By \olref[seq][prv]{thm:provability-quantifiers}
    we have $\Gamma^* \Proves \lexists[x][!B(x)]$.  Then, by
    \olref[mcs]{prop:mcs}\olref[mcs]{prop:mcs-prov-in}, we can
    conclude that $\indfrm \in \Gamma^*$.

Conversely, suppose that $\lexists[x][!B(x)] \in \Gamma^*$. Because
$\Gamma^*$ is saturated, $(\lexists[x][!B(x)] \lif !B(c)) \in
\Gamma^*$. By
\olref[seq][prv]{prop:provability}\olref[seq][prv]{prop:provability-mp}
and \olref[mcs]{prop:mcs}\olref[mcs]{prop:mcs-prov-in}, $!B(c) \in
\Gamma^*$. By inductive hypothesis, $\Sat{M(\Gamma^*)}{!B(c)}$. Now
consider the variable assignment with $s(x) =
\Assign{c}{M(\Gamma^*)}$. Then $\Sat{M(\Gamma^*)}{!B(x)}[s]$. By
definition of satisfaction,
$\Sat{M(\Gamma^*)}{\lexists[x][!B(x)]}$.}}}{}
\end{enumerate}
\end{proof}

\begin{probtag}{probOr,probAnd,probIf,probEx,probAll}
Complete the proof of \olref[fol][com][mod]{lem:truth}.
\end{probtag}


\end{document}
