% Part: first-order-logic
% Chapter: completeness
% Section: construction-of-model

\documentclass[../../include/open-logic-section]{subfiles}

\begin{document}

\olfileid{fol}{com}{mod}
\olsection{Construction of Model}

We will begin by showing how to construct a model which satisfies a maximally consistent, saturated set of sentences in a language~$\Lang L$ without the identity symbol.

\begin{defn}
\ollabel{termmodel}
Let $\Gamma^*$ be a maximally consistent, saturated set of sentences
in a language~$\Lang L$.  The \emph{term model}~$\Struct M(\Gamma^*)$
of $\Gamma^*$ is the first-order !!{structure} defined as follows:
\begin{enumerate}
\item The domain~$\Domain{M(\Gamma^*)}$ is the set of all closed terms
  of~$\Lang L$.
\item The interpretation of a constant $c$ is $c$ itself:
  $\Assign{c}{M(\Gamma^*)} = c$.
\item The function~$f$ is assigned the function
\[
\Assign{f}{M(\Gamma^*)}(t_1, \dots, t_n) = f(\Value{t_1}{M(\Gamma^*)},
\dots, \Value{t_1}{M(\Gamma^*)})
\]
\item If $R$ is an $n$-place !!{predicate}, then $\langle t_1, \dots,
  t_n\rangle \in \Assign{R}{M(\Gamma^*)}$ iff $\Atom{R}{t_1, \dots,
    t_n} \in \Gamma^*$.
\end{enumerate}
\end{defn}

\begin{lem}[Truth Lemma]
\ollabel{truth}
$\Sat{M(\Gamma^*)}{!A}$ iff $!A \in \Gamma^*$
\end{lem}

\begin{proof}
For the forward direction, suppose $\Sat{M(\Gamma^*)}{!A}$. We want to
show that $!A \in \Gamma^*$.
\begin{enumerate}
\item \indcase{!A}{R(t_1, \dots, t_n)}{$\langle t_1, \dots, t_n
  \rangle \in \Assign{R}{M(\Gamma^*)}$ by the definition of
  satisfaction, and by the construction of $\Struct M(\Gamma^*)$,
  $R(t_1, \dots, t_n) \in \Gamma^*$.}

\tagitem{prvNot}{%
\indcase{!A}{\lnot !B}{since $\Sat{M(\Gamma^*)}{\indfrm}$,
  $\Sat/{M(\Gamma^*)}{!B}$.  By induction hypothesis, $!B \notin
  \Gamma^*$. By \olref[mcs]{prop:mcs}\olref[mcs]{prop:mcs-either-or},
  $\lnot !B \in \Gamma$.}}{}

\tagitem{prvAnd}{% 
\indcase{!A}{!B \land !C}{since $\Sat{M(\Gamma^*)}{\indfrm}$, we have
  both $\Sat{M(\Gamma^*)}{!B}$ and $\Sat{M(\Gamma^*)}{!C}$. By the
  induction hypothesis, we get both that $!B \in \Gamma^*$ and that
  $!C \in \Gamma^*$. By
  \olref[mcs]{prop:mcs}\olref[mcs]{prop:mcs-and}, $!B \land !C \in
  \Gamma^*$.}}{}

\tagitem{prvOr}{% 
\indcase{!A}{!B \lor !C}{since $\Sat{M(\Gamma^*)}{\indfrm}$, we have
  at least one of $\Sat{M(\Gamma^*)}{!B}$ or
  $\Sat{M(\Gamma^*)}{!C}$. If $\Sat{M(\Gamma^*)}{!B}$, then by the
  induction hypothesis, $!B \in \Gamma^*$, so by
  \olref[mcs]{prop:mcs}\olref[mcs]{prop:mcs-or}, $!B \lor !C \in
  \Gamma^*$ (and similar if $\Sat{M(\Gamma^*)}{!C}$).}}{}

\tagitem{prvIf}{%
\indcase{!A}{!B \lif !C}{since $\Sat{M(\Gamma^*)}{\indfrm}$,
  $\Sat/{M(\Gamma^*}{!B}$ or $\Sat{M (\Gamma^*)}{!C}$. In the former
  case, by induction hypothesis, $!B \notin \Gamma^*$; in the latter
  case, $!C \in \Gamma^*$. Since either $!B \notin \Gamma^*$ or $!C
  \in \Gamma^*$, by \olref[mcs]{prop:mcs}\olref[mcs]{prop:mcs-if}, $!B
  \lif !C \in \Gamma^*$.}}{}

\tagitem{prvAll}{%
\indcase{!A}{\lforall[x][!B(x)]}{since $\Sat{M(\Gamma^*)}{\indfrm}$,
  for every variable assignment $s$,
  $\Sat{M(\Gamma^*)}{!B(x)}[s]$. Suppose to the contrary that
  $\lforall[x][!B(x)]\notin \Gamma^*$: Then by
  \olref[mcs]{prop:mcs}\olref[mcs]{prop:mcs-either-or},
  $\lnot\lforall[x][!B(x)] \in \Gamma*$. By saturation,
  $\lnot\lforall[x][!B(x)] \lif \lnot !B(c) \in \Gamma^*$ for some
  $c$, so by \olref[mcs]{prop:mcs}\olref[mcs]{prop:mcs-prov-in},
  $\lnot !B(c) \in \Gamma^*$.  By
  \olref[mcs]{prop:mcs}\olref[mcs]{prop:mcs-either-or}, $!B(c) \notin
  \Gamma^*$. By induction hypothesis, $\Sat/{M(\Gamma^*)}{!B(c)}$.  By
  Therefore, if $s'$ is the variable assignment such that $s'(x)=c$,
  then $\Sat/{M(\Gamma^*)}{!B(x)}[s']$, contradicting the earlier
  result that $\Sat{M(\Gamma^*)}{!B(x)}[s]$ for all~$s$.  Therefore,
  we must have $\lforall[x][!B(x)] \in \Gamma^*$.}}{}

\tagitem{prvEx}{%
\indcase{!A}{\lexists[x][!B(x)]}{since $\Sat{M(\Gamma^*)}{\indfrm}$, for
  some variable assignment $s$, $\Sat{M(\Gamma^*)}{!B(x)}[s]$. The
  value $s(x)$ is some !!{constant}~$c \in \Domain{M(\Gamma^*)}$. Thus,
    $\Sat{M(\Gamma^*)}{!B(c)}$, and by our induction hypothesis,
    $!B(c) \in \Gamma^*$.  By \olref[seq]{thm:provability-quantifiers}
    we have $\Gamma^* \Proves \lexists[x][!B(x)]$.  Then, by
    \olref[mcs]{prop:mcs}\olref[mcs]{prop:mcs-prov-in}, we can
    conclude that $\lexists[x][!B(x)] \in \Gamma^*$.}}{}
\end{enumerate}

For the reverse direction, let $!A \in \Gamma^*$. Suppose to the
contrary that $\Sat/{M(\Gamma^*)}{!A}$. Then $\Sat{M(\Gamma^*)}{\lnot
  !A}$, so by the forward direction of the proof, $\lnot !A \in
\Gamma^*$. Contradiction: $\Gamma^*$ is consistent. Hence, we must
have that $\Sat{M(\Gamma^*)}{!A}$.
\end{proof}

\end{document}
