% Part: first-order-logic
% Chapter: completeness
% Section: downward-ls

\documentclass[../../include/open-logic-section]{subfiles}

\begin{document}

\olfileid{fol}{com}{dls}
\olsection{The L\"owenheim-Skolem Theorems}

\begin{thm} 
\ollabel{downward-ls}
If $\Gamma$ is consistent then it has a countable model, i.e., it is
satisfiable in a !!{structure} whose domain is either finite or
denumerably infinite.
\end{thm}

\begin{proof}
  If $\Gamma$ is consistent, the !!{structure} $\Struct M$ delivered by
  the proof of the completeness theorem has a domain $\Domain{M}$ whose
  cardinality is bounded by that of the set of the terms of the
  language $\Lang L$. So $\Struct M$ is at most denumerably infinite.
\end{proof}

\begin{thm} 
\ollabel{noidentity-ls}
If $\Gamma$ is consistent set of sentences in the language of first-order logic without identity, then it has a denumerable model, i.e., it is
satisfiable in a !!{structure} whose domain is denumerably infinite.
\end{thm}

\begin{proof}
  If $\Gamma$ is consistent and contains no sentences in which identity appears, then the !!{structure} $\Struct M$ delivered by
  the proof of the completness theorem has a domain $\Domain{M}$ whose
  cardinality is identical to that of the set of the terms of the
  language $\Lang L$. So $\Struct M$ is denumerably infinite.
\end{proof}

\end{document}
