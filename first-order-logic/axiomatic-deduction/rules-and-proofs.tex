% Part: first-order-logic 
% Chapter: axiomatic-deduction 
% Section: rules-and-proofs

\documentclass[../../include/open-logic-section]{subfiles}

\begin{document}

\olfileid{fol}{axd}{rul}

\olsection{Rules and \usetoken{P}{derivation}}

Let $\Lang L$ be a first-order language with the usual constants,
!!{variable}s, logical symbols, and auxiliary symbols (parentheses and the
comma).

\begin{defn}[Tautological Instance] 
A !!{formula} $!A$ is a \emph{tautological instance} if and only if
there is a tautology $!B(p_1, \dots, p_n)$ of propositional logic and
first-order !!{formula}s $!C_1$,~\dots,~$!C_n$ such that $!A \ident
\Subst{!B}{!C_1}{p_1},\dots,\Subst{}{!C_n}{p_n}$.
\end{defn}

\begin{defn}[Axioms] 

The set $\Ax$ of \emph{axioms} of first-order logic
comprises all !!{formula}s obtained by prefixing any number of
universal quantifiers to the following:
\begin{description}
\item[{[\textbf{Ax0}]}] $!A$, where $!A$ is a tautological instance;
\item[{[\textbf{Ax1}]}] $\lforall[x][!B] \lif \Subst{!B}{t}{x}$, if $t$ is
!!{free for} $x$ in $!B$; 
\item[{[\textbf{Ax2}]}] $\lforall[x][(!A \lif
!B)] \lif (\lforall[x][!A] \lif \lforall[x][!B])$; 
\item[{[\textbf{Ax3}]}]
$!B \lif \lforall[x][!B]$, if $x$ is \emph{not} free in $!B$;
\item[{[\textbf{Ax4}]}] $\eq[x][x]$; \item[{[\textbf{Ax5}]}] $\eq[x][y]
\lif (\Subst{!B}{x}{z} \lif \Subst{!B}{y}{z})$, if both $x$ and $y$ are
!!{free for} $z$ in $!B$. 
\end{description} 
\end{defn}

\begin{defn}[!!^{derivation}] 
  \emph{!!^a{derivation}} from $\Gamma$ is a finite sequence of !!{formula}s, each
  one of which is either an axiom, or a member of $\Gamma$, or
  obtained by previous !!{formula}s by \emph{modus ponens}. A !!{formula}
  $\varphi$ is \emph{!!{derivable}} from $\Gamma$, written $\Gamma \vdash
  \varphi$, if there is a !!{derivation} from $\Gamma$ ending in $\varphi$. 
\end{defn}

\begin{explain}
  Since the axioms for predicate logic comprise all the propositional
  axioms, and the only rule (\emph{viz.}, MP) is the same, all the
  propositional proof-theoretic properties such the Deduction Theorem,
  Cut, Monotony, etc., carry over from the propositional case.
\end{explain}

\begin{explain}
Since tautological instances are all axioms, the following proposition
follows immediately by $n$ applications of \emph{modus
  ponens}. Accordingly, from now on we freely employ purely
propositional steps in proofs and justify them by reference to
``Proposition T.''
\end{explain}

\begin{prop}\ollabel{prop:T}
  If $\Gamma \vdash \varphi_1 ,\ldots \Gamma \vdash \varphi_n$ and
  $\varphi_1 \supset (\varphi_2 \supset \cdots \supset(\varphi_n
  \supset \psi) \cdots )$ is a tautological instance, then $\Gamma
  \vdash \psi$.
\end{prop}

\begin{proof}
  The formula   $\varphi_1 \supset (\varphi_2 \supset \cdots \supset(\varphi_n
  \supset \psi) \cdots )$ is a tautological instance and hence an
  axiom; $n$ applications of MP give the desired result.
\end{proof}

\end{document}
