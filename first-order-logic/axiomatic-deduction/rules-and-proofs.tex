% Part: first-order-logic
% Chapter: axiomatic-proofs
% Section: rules-and-proofs

\documentclass[../../include/open-logic-section]{subfiles}

\begin{document}

\olfileid{fol}{axp}{rul}

\olsection{Rules and \usetoken{P}{derivation}}

Let $\Lang L$ be a first-order language with the usual constants,
!!{variable}s, logical symbols, and auxiliary symbols (parentheses
and the comma).

\begin{defn}[Tautological Instance]
  A !!{formula} $!A$ of predicate logic is a \emph{tautological
    instance} if and only if there is a tautology $!B(p_1,\ldots,p_n)$ of
  propositional logic and first-order !!{formula}s
  $!C_1,\ldots,!C_n$ such that $!A =
  \Subst{!B}{!C_1}{p_1},\ldots,\Subst{}{!C_n}{p_n}$. 
\end{defn}

\begin{defn}[Axioms]
  The set \textbf{Ax} of the \emph{Axioms} of first-order logic
  comprises all !!{formula}s obtained by prefixing any number of universal
  quantifiers to the following:
  \begin{description}
  \item[{[\textbf{Ax0}]}] $!A$, where $!A$ is a tautological
    instance;
  \item[{[\textbf{Ax1}]}] $\lforall[x][!B] \lif \Subst{!B}{t}{x}$,
    if $t$ is !!{free for} $x$ in $!B$;
  \item[{[\textbf{Ax2}]}] $\lforall[x][(!A \lif !B)] \lif
    (\lforall[x][!A] \lif \lforall[x][!B])$;
  \item[{[\textbf{Ax3}]}] $!B \lif \lforall[x][!B]$, if $x$ is
    \emph{not} free in $!B$;
  \item[{[\textbf{Ax4}]}] $\eq[x][x]$;
  \item[{[\textbf{Ax5}]}] $\eq[x][y] \lif (\Subst{!B}{x}{z}
    \lif \Subst{!B}{y}{z})$, if both $x$ and $y$ are !!{free for}
    $z$ in $!B$.
  \end{description}
\end{defn}

\begin{defn}[!!^{derivation}]
  If $\Gamma$ is a set of !!{formula}s of $\Lang L$ and $!A$ a
!!{formula}, then a \emph{!!{derivation}} of $!A$ from $\Gamma$ is a finite
sequence $!A_1,\ldots, !A_n$ of !!{formula}s such that
  $!A_n = !A$ and for each $i\le n$ one of the following
  holds:
  \begin{itemize}
  \item $!A_i \in \Gamma$; or
  \item $!A_i$ is an axiom; or
  \item there are $j,k<i$ that $!A_j$ is $!A_k \lif
    !A_i$.
  \end{itemize}
\end{defn}

\begin{explain}
The last clause just says that $!A_i$ can be obtained from
previously occurring $!A_k$ and $!A_k \lif !A_i$ by
\emph{modus ponens} (MP). 
\end{explain}

\begin{prop}\ollabel{prop:easyproofs}
 The following are !!{derivable}:
  \begin{enumerate}
  \item \emph{Transitivity}: $\{!A\lif !B, !B\lif !C\} \Proves
    !A\lif!C$\ollabel{prop:easyproofs:trans}
  \item $\Proves !A \lif !A$. \ollabel{prop:easyproofs:refl}
  \end{enumerate}
\end{prop}

\begin{proof}
  For part \olref{prop:easyproofs:trans}, the following instances
  of \textbf{Ax1.} and \textbf{Ax2.} are needed:
  \begin{eqnarray*}
    &&(!B \lif !C) \lif (!A \lif (!B \lif !C)); \\
    && (!A \lif (!B \lif !C)) \lif ((!A \lif !B) \lif
    (!A \lif !C)).
  \end{eqnarray*}
For part \olref{prop:easyproofs:refl}, consider the instance of
\textbf{Ax1.}: $!A \lif ((!A \lif !A) \lif
!A)$ and distribute the outermost implication by \textbf{Ax2.}
\end{proof}

\begin{prop}\ollabel{prop:contain}
  For any set $\Gamma$ of !!{formula}s:
  \begin{enumerate}
  \item If $!A$ is an axiom then $\Gamma \Proves !A$;
  \item If $!A$ is in $\Gamma$ then $\Gamma \Proves
    !A$;\ollabel{prop:contain:reflex}
  \item \emph{Monotony}: if $\Gamma \subseteq \Delta$ and $\Gamma
    \Proves !A$ then also $\Delta \Proves
    !A$;\ollabel{prop:contain:monotony} 
  \item $\Gamma \Proves !A$ if and only if there is a
    \emph{finite} !!{subset} $\Gamma_0$ of $\Gamma$ such that $\Gamma_0
    \Proves !A$.
  \end{enumerate}
\end{prop}

\begin{thm}\ollabel{thm:deduction-thm}
  \emph{Deduction Theorem}: $\Gamma \cup \{!A\} \Proves !B$ if
  and only if $\Gamma \Proves !A \lif !B$.
\end{thm}
\begin{proof}
For this proof we rely on the following instances of \textbf{Ax0}:
\begin{enumerate}
  \item $!A \lif (!B \lif !A)$;
  \item $(!A \lif (!B \lif !C)) \lif
    ((!A \lif !B) \lif (!A \lif !C))$;
\end{enumerate}

  The ``if'' direction is immediate; if $\Gamma \Proves !A \lif
  !B$ then also $\Gamma \cup \{!A\}\Proves !A \lif
  !B$, so there is !!a{derivation} of $!A\lif !B$ from $\Gamma \cup
  \{!A\}$, and one more application of \emph{modus ponens} gives
  $\Gamma \cup \{!A\} \Proves !B$.

  For the converse, proceed by induction on theorems. If $!B \in
  \Gamma$ or $!B$ is an axiom then also $\Gamma \Proves !B \lif
  (!A \lif !B)$ by 1, and \emph{modus ponens}
  gives $\Gamma\cup \Proves !A \lif !B$; and if $!B \in \{
  !A\}$ then $\Gamma \Proves !A \lif !B$ because the
  last !!{sentence} is the same as $!A \lif !A$.

  For the inductive step, suppose $!B$ is obtained by \emph{modus
    ponens} from $!C \lif !B$ and $!C$. Then $\Gamma \cup
  \{ !A\} \Proves !C \lif !B$ and $\Gamma \cup \{
  !A\} \Proves !C$. By inductive hypothesis, both 
  \begin{align*}
   & \Gamma \Proves !A \lif (!C \lif !B); \\
   & \Gamma \Proves !A \lif !C.
  \end{align*}
But also
\[
\Gamma \Proves (!A \lif (!C \lif !B)) \lif
((!A\lif !C)  \lif (!A \lif !B)),
\]
by 2, and two applications of \emph{modus ponens} give
$\Gamma \Proves !A \lif !B$, as required.
\end{proof}

\begin{prop}\ollabel{prop:derivfacts}
  The following are all !!{derivable}:
  \begin{enumerate}
  \item $\Proves (!A \lif !B) \lif ((!B \lif !C)
    \lif (!A \lif !C)$; \ollabel{derivfacts:a}
  \item \emph{Contraposition}: if $\Gamma \cup \{ \lnot !A\}
    \Proves \lnot !B$ then $\Gamma \cup \{ !B\} \Proves
    !A$; \ollabel{derivfacts:b}
  \item \emph{Ex Falso Quodlibet}: $\{ !A, \lnot!A\} \Proves
    !B$; \ollabel{derivfacts:c}
  \item \emph{Double Negation}: $\{ \lnot\lnot!A\} \Proves
    !A$;\ollabel{derivfacts:d}
  \item if $\Gamma \Proves \lnot\lnot!A$ then $\Gamma \Proves
    !A$;\ollabel{derivfacts:e}
\end{enumerate}
\end{prop}


\begin{proof}
Part \olref{derivfacts:a} follows from Prop. \olref{prop:easyproofs}, part
\olref{prop:easyproofs:trans} by two applications of the Deduction
Theorem. For part \olref{derivfacts:b}:

\begin{tabular}{rll}
  1. & $\Gamma \cup \{ \lnot !A\} \Proves \lnot !B$ & hyp. \\
  2. & $\Gamma \cup \Proves \lnot !A \lif \lnot !B$ & Ded. Thm.\\
  3. & $\Gamma \Proves (\lnot !A \lif \lnot !B) \lif (!B
  \lif !A)$ & \textbf{Ax3.} and monotony \\
  4. & $\Gamma \Proves !B \lif !A$ & MP \\
  5. & $\Gamma \cup \{ !B\} \Proves !A$ & Ded. Thm.
\end{tabular}
  
\noindent For part \olref{derivfacts:c} we have $\{\lnot !A,
\lnot !B \} \Proves \lnot !A$ by Prop \olref{prop:contain}, part
\olref{prop:contain:reflex}, and $\{ !A, \lnot!A\} \Proves
!B$ follows by \olref{derivfacts:b}. Part  \olref{derivfacts:d}: since
$\{\lnot\lnot !A, \lnot !A \} \Proves \lnot \lnot\lnot
!A$ by \emph{ex falso quodlibet}, apply  \olref{derivfacts:b}. Now
for part  \olref{derivfacts:e}, $\Gamma \Proves \lnot\lnot !A
\supset !A$ by the previous part, the deduction theorem, and
monotony, so if $\Gamma \Proves \lnot\lnot!A$ also $\Gamma \Proves
!A$ by MP. 
\end{proof}

\begin{thm}
  \emph{Cut:} if $\Gamma \Proves !A$ and $\Delta, !A \Proves
  !C$ then $\Gamma \cup \Delta \Proves !C$. 
\end{thm}

\begin{proof}
The following shows that the conclusion is !!{derivable}:

\begin{tabular}{rll}
1. & $\Gamma \Proves !A$ & hyp. \\
2. & $\Gamma\cup\Delta \Proves !A$ & monotony, 1.\\
3. & $\Delta, !A \Proves !C$ & hyp. \\
4. & $\Gamma \cup \Delta, !A \Proves !C$ & monotony, 3.\\
5. & $\Gamma \cup \Delta \Proves !A \lif !C$ & Ded. Thm., 4.\\
6. &  $\Gamma \cup \Delta \Proves !C$ & MP 2, 5.
\end{tabular}

\smallskip\noindent
``Monotony'' refers to Prop. \olref{prop:contain}, part
\olref{prop:contain:monotony}.
\end{proof}

\begin{def}\ollabel{def:trivial}
  $!A \lif !B, !A \Proves !B$.
\end{def}

\begin{proof}
  Clearly $!A \lif !B \Proves !A\lif !B$. Apply the
  Deduction Theorem.
\end{proof}

\begin{def}\ollabel{def:notphi}
  $!A \lif \lnot !A \Proves \lnot !A$.
\end{def}

\begin{proof} The following shows the conclusion is !!{derivable}.
  
  \begin{tabular}{rll}
    1. & $\lnot\lnot!A \Proves !A$ & Double negation \\
    2. & $\lnot\lnot \lnot\lnot!A \Proves \lnot\lnot!A$
    & Double negation \\ 
    3. & $\lnot\lnot(!A \lif \lnot !A) \Proves !A
    \lif \lnot !A$ & Double negation\\
    4. & $\lnot!A \Proves \lnot\lnot\lnot !A$ &
    contraposition, 2\\ 
    5. & $\lnot\lnot(!A \lif \lnot !A), !A
    \Proves \lnot !A$ & Deduction theorem, 3 \\
    6. & $\lnot\lnot!A, \lnot\lnot(!A \lif \lnot
    !A) \Proves  \lnot !A$ & Cut, 1, 5 \\
    7. &  $\lnot\lnot!A, \lnot\lnot(!A \lif \lnot
    !A) \Proves  \lnot\lnot\lnot !A$ & Cut 4, 6 \\
    8. & $\lnot \lnot !A \Proves \lnot (!A \lif \lnot
    !A)$ & Contraposition, 7 \\
    9. & $(!A \lif \lnot !A) \Proves \lnot !A$ &
    contraposition, 8. 
 \end{tabular}

\begin{prop} \ollabel{prop:consistent} 
The following hold:
  \begin{enumerate}
  \item $!B \Proves \lnot \lnot !B$\ollabel{prop:consistent:0}
  \item $!A \lif \lnot !B, !B \Proves \lnot!A$
     \ollabel{prop:consistent:1}
  \item $!A\lif !B,!A\lif\lnot !B  \Proves
    \lnot!A$.    \ollabel{prop:consistent:2}
  \item if $\Gamma \cup \{ !A \}\Proves !B$ and $\Gamma
    \cup \{ !A\} \Proves \lnot !B$ then $\Gamma \Proves
    \lnot!A$.    \ollabel{prop:consistent:3}
  \end{enumerate}
\end{prop}

\begin{proof}
For part \olref{prop:consistent:0}, we have $\lnot\lnot\lnot!B
\Proves \lnot !B$ by double negation, whence  $!B \Proves \lnot
\lnot !B$ by Prop. \olref{prop:derivfacts}, part
\olref{derivfacts:b}.  For part \olref{prop:consistent:1} we have:

\begin{tabular}{rll}
 1. & $\Proves (\lnot\lnot !A \lif \lnot\lnot\lnot !B)
 \lif (\lnot\lnot !B \lif \lnot !A)$ & \textbf{Ax3} \\
 2. & $\lnot\lnot !A \lif \lnot\lnot\lnot !B,
 \lnot\lnot !B \Proves \lnot!A$ & 1, Ded. Thm., twice \\ 
 3. & $!B \Proves \lnot\lnot !B$ & Double Neg. \\
 4. & $\lnot\lnot !A \lif \lnot\lnot\lnot !B,
  !B \Proves \lnot!A$ & 2, 3, Cut \\
 5. & $\lnot !B \lif \lnot\lnot\lnot !B, !A\lif
 \lnot !B \Proves !A \lif \lnot\lnot\lnot!B$ &
 Transitivity (Prop. \olref{prop:easyproofs}, part
 \olref{prop:easyproofs:trans}) \\
 6. & $\Proves \lnot!B \lif \lnot\lnot\lnot!B$ & Double
 Neg. + Ded. Thm. \\
 7. & $!A\lif \lnot !B \Proves !A \lif
 \lnot\lnot\lnot !B$ & 5, 6, Cut \\
 8. & $\Proves \lnot\lnot !A \lif !A$ & Double. Neg +
 Ded. Thm. \\
 9. & $\lnot\lnot!A \lif !A, !A \lif
 \lnot\lnot\lnot !B \Proves \lnot\lnot !A \lif
 \lnot\lnot\lnot !B$ & Transitivity \\
 10. & $!A \lif \lnot\lnot\lnot !B \Proves
 \lnot\lnot!A \lif \lnot\lnot\lnot!B$ & 8, 9, Cut \\
 11. & $!A \lif \lnot!B \Proves \lnot\lnot !A\lif
 \lnot\lnot\lnot!B$ & 7, 10, Cut \\
 12. & $!A \lif \lnot !B, !B \Proves \lnot !A$ & 4,
 11, Cut.
\end{tabular}

\smallskip\noindent 
Part \olref{prop:consistent:2} follows because:

\begin{tabular}{rll}
  1. & $!A \lif !B, !A \Proves !B $ &lemma \olref{lemma:trivial} \\
  2. & $ !A \lif \lnot !B, !B \Proves \lnot!A$ & part
  \olref{prop:consistent:1} \\ 
  3. & $!A \lif !B, !A \lif \lnot !B, !A
  \Proves \lnot !A$ & 1, 2, Cut. \\
  4. &  $!A \lif !B, !A \lif \lnot !B
  \Proves  !A\lif \lnot !A$ & deduction theorem \\
  5. &  $!A \lif \lnot !A \Proves \lnot !A$   &
  lemma \olref{lemma:notphi} \\
  6. & $!A \lif !B, !A \lif \lnot !B
  \Proves \lnot !A$ & 4, 5, Cut.
\end{tabular}

\smallskip\noindent
Finally, for part \olref{prop:consistent:3}: from the hypotheses by the
deduction theorem, $\Gamma \Proves !A \lif !B$ and $\Gamma
\Proves !A \lif \lnot !B$; from part \olref{prop:consistent:2}
and monotony, $\Gamma, !A \lif !B, !A \lif \lnot !B
  \Proves \lnot !A$; two applications of Cut give the desired result. 
\end{proof}


\end{proof}




\end{document}