% Part: first-order-logic
% Chapter: undecidability
% Section: representing-tms

\documentclass[open-logic-section]{subfiles}

\begin{document}

\olfileid{fol}{und}{rep}
\olsection{Representing Turing Machines}

\begin{wordy}
In order to represent Turing machines and their behavior by a sentence
of first-order logic, we have to define a suitable language. The
language consists of two parts: predicates for describing
configurations of the machine, and expressions for counting execution
steps (``moments'') and positions on the tape. The latter require an
initial moment, $\Obj 0$, a ``successor'' function which is
traditionally written as a postfix $\Obj{'}$, and an ordering $x
\Obj{<} y$ of ``before.''
\end{wordy}

\begin{defn}
Given a Turing machine $M = \langle Q, \Sigma, \delta, s\rangle$, the
language~$\Lang L_M$ consists of:
\begin{enumerate}
\item A two-place predicate $\Obj Q_q(x, y)$ for every state~$q \in Q$.
\item A two-place predicate $\Obj S_\sigma(x, y)$ for every
  symbol~$\sigma\in \Sigma$
\item A constant $\Obj 0$
\item A one-place function $\Obj{'}$
\item A two-place predicate $\Obj <$
\end{enumerate}
\end{defn}

For each number $n$ there is a canonical term $\overline n$, the
\emph{numeral} for $n$, which represents it in~$\Lang L_M$. $\overline
0$ is $\Obj 0$, $\overline 1$ is $\Obj 0'$, $\overline 2$ is $\Obj
0''$, and so on. More formally:
\begin{align*}
\overline 0 & = \Obj 0 \\
\overline{n+1} &= \overline{n}'
\end{align*}

The sentences describing the operation of the Turing machine~$M$ on
input $w$ are the following:
\begin{enumerate}
\item[I] Axioms describing numbers:
\begin{enumerate}
\item A sentence that says that the successor function is injective:
\[
\forall \Obj x \forall \Obj y(\eq[\Obj x'][\Obj y'] \lif \eq[\Obj x][\Obj y])
\]
\item A sentence that says that every number is less than its successor:
\[
\forall \Obj x(\Obj x < \Obj x')
\]
\item A sentence that ensures that $<$ is transitive:
\[
\forall \Obj x\forall \Obj y\forall \Obj z(
(\Obj x < \Obj y \land \Obj y < \Obj z) \lif \Obj x < \Obj z)
\]
\end{enumerate}
\item[II.] Axioms describing the input configuration:
\begin{enumerate}
\item $M$ is in the inital state~$s$ at time~0, scanning square 0:
\[
\Obj Q_s(\Obj 0, \Obj 0)
\]
\item The first $n$ squares contain the symbols $\triangleright$,
  $\sigma_{i_1}$, \dots, $\sigma_{i_n}$:
\[
\Obj S_\triangleright(\Obj 0, \Obj 0) \land 
\Obj S_{i_1}(\overline 1, \Obj 0) \land 
\dots \land
\Obj S_{i_n}(\overline n, \Obj 0) 
\]
\item Otherwise, the tape is empty:
\[
\forall \Obj x(\overline{} < x \lif \Obj
S_\text{\textvisiblespace}(\Obj x, \Obj 0)
\]
\end{enumerate}
\item[III.] Axioms describing the transition from one configuration to
  the next:

For the following, let $A(x, y)$ be the conjunction of all sentences
of the form
\[
\forall \Obj z(
(\Obj z < \Obj x \lor \Obj x < \Obj z \land \Obj S_\sigma(\Obj z, \Obj y)) 
\lif \Obj S_\sigma(\Obj z, \Obj y'))
\]
where $\sigma \in \Sigma$.
\item For every instruction $\langle q_i, \sigma, q_j, \sigma',
  L\rangle$, the sentence:
\[
\lforall[\Obj x] \lforall[\Obj y](
   (\Obj Q_{q_i}(\Obj x', \Obj y) \land \Obj S_{\sigma}(\Obj x, \Obj y)) \lif
   (\Obj Q_{q_j}(\Obj x, \Obj y) \land \Obj S_{\sigma'}(\Obj x, \Obj y) \land 
A(\Obj x, \Obj y)))
\]
\item For every instruction $\langle q_i, \sigma, q_j, \sigma',
  R\rangle$, the sentence:
\[
\lforall[\Obj x]\lforall[\Obj y](
   (\Obj Q_{q_i}(\Obj x, \Obj y) \land \Obj S_{\sigma}(\Obj x, \Obj y)) \lif
   (\Obj Q_{q_j}(\Obj x', \Obj y) \land \Obj S_{\sigma'}(\Obj x, \Obj y) \land 
A(\Obj x, \Obj y)))
\]
\end{enumerate}
(Probably also need axioms saying every square as exactly one symbol on
it at all times, machine always in exactly one state.)


Let $T(M, w)$ be the conjunction of all the above sentences for Turing
machine~$M$ and input~$w$

The sentence $H(M, w)$:
\[
\lexists[\Obj x]\lexists[\Obj y][\Obj Q_h(\Obj x, \Obj y)]
\]
expresses that the Turing machine~$M$ halts on input~$w$.

\end{document}
