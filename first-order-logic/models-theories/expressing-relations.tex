% Part: first-order-logic
% Chapter: models-theories
% Section: expressing-relations

\documentclass[../../include/open-logic-section]{subfiles}

\begin{document}

\olfileid{fol}{mat}{exr} 

\olsection{Expressing Relations in \article{structure}
  \printtoken{S}{structure}}

\begin{explain}
One main use !!{formula}s can be put to is to express properties and
relations in !!a{structure}~$\Struct M$ in terms of the primitives of
the language~$\Lang L$ of~$\Struct M$.  By this we mean the following:
the !!{domain} of $\Struct M$ is a set of objects.  The !!{constant}s,
!!{function}s, and !!{predicate}s are interpreted in~$\Struct M$ by
some objects in$\Domain M$, functions on~$\Domain M$, and relations
on~$\Domain M$.  For instance, if $\Obj A^2_0$ is in $\Lang L$, then
$\Struct M$ assigns to it a relation~$R = \Assign{{\Obj
  A^2_0}}{M}$. Then the formula $\Atom{\Obj A^2_0}{\Obj x_1, \Obj x_2}$
\emph{expresses} that very relation, in the following sense: if a
variable assignment~$s$ maps $\Obj x_1$ to $a \in \Domain{M}$ and
$\Obj x_2$ to $b \in \Domain M$, then
\[
Rab \text{\quad iff\quad} \Sat{M}{\Atom{\Obj A^2_0}{\Obj x_1, \Obj x_2}}[s].
\]
Note that we have to involve variable assignments here: we can't just
say ``$Rab$ iff $\Sat{M}{\Atom{\Obj A^2_0}{a, b}}$'' because $a$ and
  $b$ are not symbols of our language: they are !!{element}s
  of~$\Domain{M}$.

Since we don't just have atomic !!{formula}s, but can combine them
using the logical connectives and the quantifiers, more complex
!!{formula}s can define other relations which aren't directly built
into~$\Struct M$.  We're interested in how to do that, and
specifically, which relations we can define in !!a{structure}.
\end{explain}

\begin{defn}
Let $!A(\Obj x_1,\dots, \Obj x_n)$ be a !!{formula} of $\Lang L$ in
which only $\Obj x_1$,\dots, $\Obj x_n$ occur free, and let $\Struct
M$ be !!a{structure} for~$\Lang L$. $!A(\Obj x_1,\dots, \Obj x_n)$
\emph{expresses the relation}~$R \subseteq \Domain M^n$ iff
\[
Ra_1\dots a_n \text{\quad iff\quad} \Sat{M}{\Atom{!A}{\Obj
    x_1,\dots, \Obj x_n}}[s]
\]
for any variable assignment~$s$ with $s(\Obj x_i) = a_i$ ($i = 1,
\dots, n$).
\end{defn}

\begin{ex}
In the standard model of arithmetic~$\Struct N$, the !!{formula} $\Obj
x_1 < \Obj x_2 \lor \Obj x_1 = \Obj x_2$ expresses the $\le$ relation
on~$\Nat$. The !!{formula} $\Obj x_2 = \Obj x_1'$ expresses the
successor relation, i.e., the relation $R \subseteq \Nat^2$ where
$Rnm$ holds if $m$ is the successor of~$n$. The formula $\Obj x_1 =
\Obj x_2'$ expresses the predecessor relation.  The !!{formula}s
$\lexists[\Obj x_3][(\eq/[\Obj x_3][\Obj 0] \land \eq[\Obj x_2][(\Obj
    x_1 + \Obj x_3)])]$ and $\lexists[\Obj x_3][\eq[(\Obj x_1 + {\Obj
      x_3}')][x_2]]$ both express the $\Obj <$ relation.  This means
that the predicate symbol~$<$ is actually superfluous in the language
of arithmetic; it can be defined.
\end{ex}

This idea is not just interesting in specific !!{structure}s, but
generally whenever we use a language to describe an intended model or
models, i.e., when we consider theories. These theories often only
contain a few !!{predicate}s as basic symbols, but in the domain they
are used to describe often many other relations play an important
role.  If these other relations can be systematically expressed by the
relations that interpret the basic !!{predicate}s of the language, we
say we can \emph{define} them in the language.

\begin{prob}
Find !!{formula}s in $\Lang L_A$ which define the following relations:
\begin{enumerate}
\item $n$ is between $i$ and $j$;
\item $n$ evenly divides $m$ (i.e., $m$ is a multiple of $n$);
\item $n$ is a prime number (i.e., no number other than $1$ and $n$ evenly
  divides~$n$).
\end{enumerate}
\end{prob}

\begin{prob}
Suppose the formula $!A(\Obj x_1, \Obj x_2)$ expresses the relation $R
\subseteq \Domain M^2$ in a !!{structure}~$\Struct M$. Find formulas
that express the following relations:
\begin{enumerate}
\item the inverse $R^{-1}$ of $R$;
\item the relative product $R \mid R$;
\end{enumerate}
Can you find a way to express $R^+$, the transitive closure of~$R$?
\end{prob}

\begin{prob}
Let $\Lang{L}$ be the language containing a 2-place predicate symbol
$<$ only (no other !!{constant}s, !!{function}s or !!{predicate}s---
except of course~$\eq$). Let $\Struct{N}$ be the structure such that
$\Domain{N} = \Nat$, and $\Assign{<}{N} = \Setabs{\tuple{n,m}}{n <
  m}$. Prove the following:
\begin{enumerate}
\item $\{ 0 \}$ is definable in $\Struct{N}$;
\item $\{ 1 \}$ is definable in $\Struct{N}$;
\item $\{ 2 \}$ is definable in $\Struct{N}$;
\item for each $n \in \Nat$, the set $\{ n \}$ is definable in
  $\Struct{N}$;
\item every finite subset of $\Domain{N}$ is definable in
  $\Struct{N}$;
\item every co-finite subset of $\Domain{N}$ is definable in
  $\Struct{N}$ (where $X \subseteq \Nat$ is co-finite iff
  $\Nat \setminus X$ is finite).
\end{enumerate}
\end{prob}


\end{document}
