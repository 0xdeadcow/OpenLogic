% Part: first-order-logic
% Chapter: natural-deduction
% Section: provability

% verification of properties of provability needed for maximally
% consistent sets in the completeness chapter.

\documentclass[../../include/open-logic-section]{subfiles}

\begin{document}

\olfileid{fol}{ntd}{prv}
\olsection{Properties of \usetoken{S}{derivability}}


\begin{prop}[Monotony]
\ollabel{prop:monotony}
If $\Gamma \subseteq \Delta$ and $\Gamma \Proves !A$, then $\Delta
\Proves !A$.
\end{prop}

\begin{proof}
Any !!{derivation} of $!A$ from $\Gamma$ is also a !!{derivation} of
$!A$ from~$\Delta$.
\end{proof}

\begin{prop}\ollabel{prop:provability}
\begin{enumerate}
\item \ollabel{prop:provability-contr} If $\Gamma \Proves
  !A$ and $\Gamma \cup \{ !A\} \Proves \lfalse$, then
  $\Gamma$ is inconsistent.

\item \ollabel{prop:provability-lnot} If $\Gamma \cup \{!A\}
  \Proves \lfalse$, then $\Gamma \Proves \lnot !A$.

\item \ollabel{prop:provability-exhaustive} If $\Gamma \cup \{!A\}
  \Proves \lfalse$ and $\Gamma \cup \{\lnot !A\}
  \Proves \lfalse$, then $\Gamma \Proves \lfalse$.

\tagitem{prvOr}{\ollabel{prop:provability-lor-left} If $\Gamma \cup
  \{!A\} \Proves \lfalse$ and $\Gamma \cup \{!B\}
  \Proves \lfalse$, then $\Gamma \cup \{!A \lor !B\}
  \Proves \lfalse$.}{}

\tagitem{prvOr}{\ollabel{prop:provability-lor-right} If $\Gamma
  \Proves !A$ or $\Gamma \Proves !B$,
      then $\Gamma \Proves !A \lor !B$.}{}

\tagitem{prvAnd}{\ollabel{prop:provability-land-left} If $\Gamma
  \Proves !A \land !B$ then $\Gamma \Proves !A$
  and $\Gamma \Proves !B$.}{}

\tagitem{prvAnd}{\ollabel{prop:provability-land-right} If $\Gamma
  \Proves !A$ and $\Gamma \Proves !B$,
 then $\Gamma \Proves !A \land !B$.}{}

\tagitem{prvIf}{\ollabel{prop:provability-mp} If $\Gamma \Proves !A$
  and $\Gamma \Proves !A \lif !B$, then $\Gamma
  \Proves !B$.}{}

\tagitem{prvIf}{\ollabel{prop:provability-lif} If $\Gamma \Proves
  \lnot !A$ or $\Gamma \Proves !B$, then $\Gamma
  \Proves !A \lif !B$.}{}
\end{enumerate}
\end{prop}

\begin{proof}
\begin{enumerate}
\item Let the !!{derivation} of $!A$ from $\Gamma$ be
  $\Pi_1$ and the !!{derivation} of $\lfalse$ from $\Gamma \cup \{!A\}$
  be $\Pi_2$. We can then !!{derive}:
\[
\AxiomC{}
\RightLabel{$\Pi_1$}
\DeduceC{$!A$}
\AxiomC{[$!A$]$^1$}
\RightLabel{$\Pi_2$}
\DeduceC{$\lfalse$}
\RightLabel{$\lnot$ Intro$_1$}
\UnaryInfC{$\lnot !A$}
\RightLabel{$\lnot$ Elim}
\BinaryInfC{$\lfalse$}
\DisplayProof
\]
In the new !!{derivation}, the assumption $!A$ is discharged, so it is
!!a{derivation} from $\Gamma$.

\item Suppose that $\Gamma \cup \{!A\} \Proves \lfalse$. Then there is a
  derivation of $\lfalse$ from $\Gamma \cup \{!A\}$.  Let $\Pi$ be
  the !!{derivation} of $\lfalse$, and consider
\[
\AxiomC{$[!A]^1$}
\RightLabel{$\Pi$}
\DeduceC{$\lfalse$}
\RightLabel{$\lnot$ Intro$_1$}
\UnaryInfC{$\lnot !A$}
\DisplayProof
\]

\item There are !!{derivation}s $\Pi_1$ and $\Pi_2$ of $\lfalse$ from
  $\Gamma \cup,\{ !A \}$ and $\lfalse$ from $\Gamma \cup \{ \lnot !A
  \}$, respectively. We can then !!{derive}
\[
\AxiomC{$[!A]^1$}
\RightLabel{$\Pi_1$}
\DeduceC{$\lfalse$}
\RightLabel{$\lnot$ Intro$_1$}
\UnaryInfC{$\lnot !A$}
\AxiomC{$[\lnot !A]^2$}
\RightLabel{$\Pi_2$}
\DeduceC{$\lfalse$}
\RightLabel{$\lnot$ Intro$_2$}
\UnaryInfC{$\lnot \lnot !A$}
\RightLabel{$\lnot$ Elim}
\BinaryInfC{$\lfalse$}
\DisplayProof
\]
Since the assumptions $!A$ and $\lnot !A$ are discharged, this is
!!a{derivation} from~$\Gamma$ alone. Hence $\Gamma \Proves \lfalse$.

% prop:provability-lor-left
\tagitem{defOr}{}{
\iftag{probOr}{Exercise.}{Exercise.}}

% prop:provability-lor-right
\tagitem{defOr}{}{
\iftag{probOr}{Exercise.}{%
Suppose $\Gamma \Proves !A$. There is a !!{derivation} $\Pi$ of $!A$
from~$\Gamma$. We can !!{derive}
\[
\AxiomC{}
\RightLabel{$\Pi$}
\DeduceC{$!A$}
\RightLabel{$\lor$ Intro}
\UnaryInfC{$!A \lor !B$}
\DisplayProof
\]
Therefore $\Gamma \Proves !A \lor !B$. The proof for when $\Gamma
\Proves !B$ is similar.}}

% prop:provability-land-left
\tagitem{defAnd}{}{\iftag{probAnd}{Exercise}{If $\Gamma \Proves !A
    \land !B$, there is !!a{derivation} $\Pi$ of $!A \land !B$ from
    $\Gamma$. Consider
\[
\AxiomC{}
\RightLabel{$\Pi$}
\DeduceC{$!A \land !B$}
\RightLabel{$\land$ Elim}
\UnaryInfC{$!A$}
\DisplayProof
\]
Hence, $\Gamma \Proves !A$.  A similar !!{derivation} 
shows that $\Gamma \Proves !B$.}}

% prop:provability-land-right
\tagitem{defAnd}{}{\iftag{probAnd}{Exercise.}{If $\Gamma \Proves !A$
    as well as $\Gamma \Proves !B$, there are !!{derivation}s $\Pi_1$
    of $!A$ and $\Pi_2$ of $!B$ from $\Gamma$.  Consider
\[
\AxiomC{}
\RightLabel{$\Pi_1$}
\DeduceC{$!A$}
\AxiomC{}
\RightLabel{$\Pi_2$}
\DeduceC{$!B$}
\RightLabel{$\land$ Intro}
\BinaryInfC{$!A \land !B$}
\DisplayProof
\]
The undischarged assumptions of the new !!{derivation} are all
in~$\Gamma$, so we have $\Gamma \Proves !A \land !B$.}}

% prop:provability-mp
\tagitem{defIf}{}{\iftag{probIf}{Exercise.}{Suppose that $\Gamma
    \Proves !A$ and $\Gamma \Proves !A \lif !B$.  There are
    !!{derivation}s $\Pi_1$ of $!A$ from $\Gamma$ and $\Pi_2$ of $!A
    \lif !B$ from $\Gamma$. Consider:
\[
\AxiomC{}
\RightLabel{$\Pi_1$}
\DeduceC{$!A$}
\AxiomC{}
\RightLabel{$\Pi_2$}
\DeduceC{$!A \lif !B$}
\RightLabel{$\lif$ Elim}
\BinaryInfC{$!B$}
\DisplayProof
\]
This means that $\Gamma \Proves !B$.}}

% prop:provability-lif
\tagitem{defIf}{}{\iftag{probIf}{Exercise.}{First suppose $\Gamma
    \Proves \lnot !A$.  Then there is a !!{derivation} of $\lnot !A$
    from~$\gamma$.  The following !!{derivation} shows that $\Gamma
    \Proves !A \lif !B$:
\[
\AxiomC{}
\RightLabel{$\Pi_0$}
\DeduceC{$\lnot !A$}
\AxiomC{$[!A]^1$}
\RightLabel{$\lnot$ Elim}
\UnaryInfC{$\lfalse$}
\RightLabel{$\lfalse$ Elim}
\UnaryInfC{$!B$}
\RightLabel{$\lif$ Intro$_1$}
\BinaryInfC{$!A \lif !B$}
\DisplayProof
\]

Now suppose $\Gamma \Proves !B$.  Then there is a !!{derivation}~$\Pi$ of
$!B$ from~$\Gamma$. The following !!{derivation} shows that $\Gamma
\Proves !A \lif !B$:
\[
\AxiomC{}
\RightLabel{$\Pi$}
\DeduceC{$!B$}
\AxiomC{$[!A]^1$}
\RightLabel{$\land$ Intro}
\BinaryInfC{$!A \land !B$}
\RightLabel{$\land$ Elim}
\UnaryInfC{$!B$}
\RightLabel{$\lif$ Intro$_1$}
\UnaryInfC{$!A \lif !B$}
\DisplayProof
\] }}
\end{enumerate}
\end{proof}

\begin{probtag}{probOr,probAnd,probIf}
Complete the proof of \olref[fol][ntd][prv]{prop:provability}.
\end{probtag}

\begin{thm}
\ollabel{thm:strong-generalization} If $c$ is a constant not occurring
in $\Gamma$ or $!A(x)$ and $\Gamma \Proves !A(c)$, then $\Gamma
\Proves \lforall[x][!A(c)]$.
\end{thm}

\begin{proof}
Let $\Pi$ be an !!{derivation} of $!A(c)$ from $\Gamma$.  By adding a
$\lforall$ Intro inference, we obtain a proof of
$\lforall[x][!A(x)]$. Since $c$ does not occur in $\Gamma$ or $!A(x)$,
the eigenvariable condition is satisfied.
\end{proof}

\begin{thm}
\ollabel{thm:provability-quantifiers}
\begin{tagenumerate}{prvEx,prvAll}
\tagitem{prvEx}{If $\Gamma \Proves !A(t)$ then $\Gamma \Proves
  \lexists[x][!A(x)]$.}{}

\tagitem{prvAll}{If $\Gamma \Proves \lforall[x][!A(x)]$ then $\Gamma
  \Proves !A(t)$.}{}
\end{tagenumerate}
\end{thm}

\begin{proof}
\begin{tagenumerate}{prvEx,prvAll}
\tagitem{prvEx}{Suppose $\Gamma \Proves !A(t)$. Then there is a
  !!{derivation}~$\Pi$ of $!A(t)$ from~$\Gamma$. The !!{derivation}
\[
\AxiomC{}
\RightLabel{$\Pi$}
\DeduceC{$!A(t)$}
\RightLabel{$\lexists$ Intro}
\UnaryInfC{$\lexists[x][!A(x)]$}
\DisplayProof
\]
shows that $\Gamma \Proves \lexists[x][!A(x)]$.}{}

\tagitem{prvAll}{Suppose $\Gamma \Proves \lforall[x][!A(x)]$. Then there is
  !!a{derivation} $\Pi$ of $\lforall[x][!A(x)]$ from~$\Gamma$.  The
  !!{derivation}
\[
\AxiomC{}
\RightLabel{$\Pi$}
\DeduceC{$\lforall[x][\Atom{!A}{x}]$}
\RightLabel{$\lforall$ Elim}
\UnaryInfC{$\Atom{!A}{t}$}
\DisplayProof
\]
shows that $\Gamma \Proves !A(t)$.}{}

\end{tagenumerate}
\end{proof}
\end{document}
