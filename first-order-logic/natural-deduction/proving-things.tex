% Part: first-order-logic
% Chapter: natural-deduction
% Section: proving-things

\documentclass[../../include/open-logic-section]{subfiles}

\begin{document}

\olfileid{fol}{ntd}{pro}

\olsection{Examples of \usetoken{P}{derivation}}

\begin{ex}
Give a derivation for the !!{formula} $(!A \land !B) \lif !A$.

We begin by writing the desired end-!!{formula} at the bottom of the derivation.
\[
\AxiomC{}
\UnaryInfC{$(!A\land !B) \lif !A$}
\DisplayProof
\]

Next, we need to figure out what kind of inference could 
result in !!a{formula} of this form. The !!{main operator}
of the end-!!{formula} is $\lif$, so we're looking for a 
$\lif$ rule. Since we are introducing a $\lif$ 
into the end-!!{formula}, we will set ourselves up to
use the $\lif$ intro rule. It is best to write down the assumptions
involved and label the inference rules as you progress, so it is easy to see
whether all assumptions have been discharged at the end of
the proof.


\[
\AxiomC{[$!A \land !B$]$^1$}
\noLine
\UnaryInfC{.}
\noLine
\UnaryInfC{.}
\noLine
\UnaryInfC{.}
\noLine
\UnaryInfC{$!A$}
\RightLabel{$\lif$ Intro$_1$} 
\UnaryInfC{$(!A\land !B) \lif !A$}
\DisplayProof
\]

We now need to fill in the steps from the assumption $!A \land !B$ to $!A$.
Since we only have one connective to deal with, $\land$, we must
use the $\land$ elim rule. This gives us the following proof:

\[
\AxiomC{[$!A \land !B$]$^1$}
\RightLabel{$\land$ Elim}
\UnaryInfC{$!A$}
\RightLabel{$\lif$ Intro$_1$} 
\UnaryInfC{$(!A\land !B) \lif !A$}
\DisplayProof
\]

We now have a correct !!{derivation} of the formula $(!A \land
!B) \lif !A$.
\end{ex}

\begin{ex}
Give a !!{derivation} of the !!{formula} $(\lnot !A \lor !B)
\lif (!A \lif !B)$.

Begin by writing the desired end-!!{formula} at the bottom of the 
derivation.
\[
\AxiomC{}
\UnaryInfC{$(\lnot !A \lor !B) \lif (!A \lif !B)$}
\DisplayProof
\]
To find a logical rule that could give us this end-!!{formula}, we look at
the logical connectives in the end-!!{formula}: $\lnot$, $\lor$, and
$\lif$. We only care at the moment about  the first occurence
of $\lif$ because it is the !!{main operator} of the !!{sentence}
 in the end-sequent, while $\lnot$, $\lor$ and the second occurence of 
 $\lif$ are inside the scope of another connective, so we will
take care of those later. We therefore start with the $\lif$ intro rule. 
This must look like:
\[
\AxiomC{[$\lnot !A \lor !B$]$^1$}
\noLine
\UnaryInfC{.}
\noLine
\UnaryInfC{.}
\noLine
\UnaryInfC{.}
\noLine
\UnaryInfC{$!A \lif !B$}
\RightLabel{$\lif$ Intro$_1$} 
\UnaryInfC{$(\lnot !A \lor !B) \lif (!A \lif !B)$}
\DisplayProof
\]

This leaves us with two possibilities to continue. Either we can
continue working from the bottom up and do another application
of the $\lif$ intro rule, or we can work from the top down and apply
a $\lor$ elim rule. Let us apply the latter.

\[
\AxiomC{[$\lnot !A \lor !B$]$^1$}
\AxiomC{[$\lnot !A$]$^2$}
\noLine
\UnaryInfC{.}
\noLine
\UnaryInfC{.}
\noLine
\UnaryInfC{.}
\noLine
\UnaryInfC{$!A \lif !B$}
\AxiomC{[$!B$]$^2$}
\noLine
\UnaryInfC{.}
\noLine
\UnaryInfC{.}
\noLine
\UnaryInfC{.}
\noLine
\UnaryInfC{$!A \lif !B$}
\RightLabel{$\lor$ Elim$_2$}
\TrinaryInfC{$!A \lif !B$}
\RightLabel{$\lif$ Intro$_1$} 
\UnaryInfC{$(\lnot !A \lor !B) \lif (!A \lif !B)$}
\DisplayProof
\]

Next, we will apply $\lif$ Intro to the two rightmost branches.

\[
\AxiomC{[$\lnot !A \lor !B$]$^{(1)}$}
\AxiomC{[$\lnot !A$]$^{(2)}$}
\noLine
\UnaryInfC{.}
\noLine
\UnaryInfC{.}
\noLine
\UnaryInfC{.}
\noLine
\AxiomC{[$!A$]$^3$}
\noLine
\UnaryInfC{.}
\noLine
\UnaryInfC{.}
\noLine
\UnaryInfC{.}
\RightLabel{$\lif$ Intro$_3$}
\BinaryInfC{$!A \lif !B$}
\AxiomC{[$!B$]$^{(2)}$}
\noLine
\UnaryInfC{.}
\noLine
\UnaryInfC{.}
\noLine
\UnaryInfC{.}
\AxiomC{[$!A$]$^4$}
\noLine
\UnaryInfC{.}
\noLine
\UnaryInfC{.}
\noLine
\UnaryInfC{.}
\RightLabel{$\lif$ Intro$_4$}
\BinaryInfC{$!A \lif !B$}
\RightLabel{$\lor$ Elim$_2$}
\TrinaryInfC{$!A \lif !B$}
\RightLabel{$\lif$ Intro$_1$} 
\UnaryInfC{$(\lnot !A \lor !B) \lif (!A \lif !B)$}
\DisplayProof
\]

The middle branch of the tree presents a contradiction, so
we should use one of the $\lfalse$ rules. 

\[
\AxiomC{[$\lnot !A \lor !B$]$^{(1)}$}
\AxiomC{[$\lnot !A$]$^{(2)}$}
\AxiomC{[$!A$]$^3$}
\RightLabel{$\lfalse$ Intro}
\BinaryInfC{$\lfalse$}
\noLine
\UnaryInfC{.}
\noLine
\UnaryInfC{.}
\noLine
\UnaryInfC{.}
\RightLabel{$\lif$ Intro$_3$}
\UnaryInfC{$!A \lif !B$}
\AxiomC{[$!B$]$^{(2)}$}
\noLine
\UnaryInfC{.}
\noLine
\UnaryInfC{.}
\noLine
\UnaryInfC{.}
\AxiomC{[$!A$]$^4$}
\noLine
\UnaryInfC{.}
\noLine
\UnaryInfC{.}
\noLine
\UnaryInfC{.}
\RightLabel{$\lif$ Intro$_4$}
\BinaryInfC{$!A \lif !B$}
\RightLabel{$\lor$ Elim$_2$}
\TrinaryInfC{$!A \lif !B$}
\RightLabel{$\lif$ Intro$_1$} 
\UnaryInfC{$(\lnot !A \lor !B) \lif (!A \lif !B)$}
\DisplayProof
\]

By using $\lfalse$ Elim, we can introduce $!B$ and
complete the branch.

\[
\AxiomC{[$\lnot !A \lor !B$]$^{(1)}$}
\AxiomC{[$\lnot !A$]$^{(2)}$}
\AxiomC{[$!A$]$^3$}
\RightLabel{$\lfalse$ Intro}
\BinaryInfC{$\lfalse$}
\RightLabel{$\lfalse$ Elim}
\UnaryInfC{$!B$}
\RightLabel{$\lif$ Intro$_3$}
\UnaryInfC{$!A \lif !B$}
\AxiomC{[$!B$]$^{(2)}$}
\noLine
\UnaryInfC{.}
\noLine
\UnaryInfC{.}
\noLine
\UnaryInfC{.}
\AxiomC{[$!A$]$^4$}
\noLine
\UnaryInfC{.}
\noLine
\UnaryInfC{.}
\noLine
\UnaryInfC{.}
\RightLabel{$\lif$ Intro$_4$}
\BinaryInfC{$!A \lif !B$}
\RightLabel{$\lor$ Elim$_2$}
\TrinaryInfC{$!A \lif !B$}
\RightLabel{$\lif$ Intro$_1$} 
\UnaryInfC{$(\lnot !A \lor !B) \lif (!A \lif !B)$}
\DisplayProof
\]

Now looking at the rightmost branch, we have no connectives
to apply Elim rules to, so we must use an Intro rule. Since we have
both $!B$ and $!A$ as assumptions, we can apply $\land$ Intro.
\[
\AxiomC{[$\lnot !A \lor !B$]$^{(1)}$}
\AxiomC{[$\lnot !A$]$^{(2)}$}
\AxiomC{[$!A$]$^3$}
\RightLabel{$\lfalse$ Intro}
\BinaryInfC{$\lfalse$}
\RightLabel{$\lfalse$ Elim}
\UnaryInfC{$!B$}
\RightLabel{$\lif$ Intro$_3$}
\UnaryInfC{$!A \lif !B$}
\AxiomC{[$!B$]$^{(2)}$}
\AxiomC{[$!A$]$^4$}
\RightLabel{$\land$ Intro}
\BinaryInfC{$!A \land !B$}
\noLine
\UnaryInfC{.}
\noLine
\UnaryInfC{.}
\noLine
\UnaryInfC{.}
\RightLabel{$\lif$ Intro$_4$}
\UnaryInfC{$!A \lif !B$}
\RightLabel{$\lor$ Elim$_2$}
\TrinaryInfC{$!A \lif !B$}
\RightLabel{$\lif$ Intro$_1$} 
\UnaryInfC{$(\lnot !A \lor !B) \lif (!A \lif !B)$}
\DisplayProof
\]
Since our goal of the rightmost branch is to get $!B$,
we can apply $\land$ elim and finish the proof.

\[
\AxiomC{[$\lnot !A \lor !B$]$^{(1)}$}
\AxiomC{[$\lnot !A$]$^{(2)}$}
\AxiomC{[$!A$]$^3$}
\RightLabel{$\lfalse$ Intro}
\BinaryInfC{$\lfalse$}
\RightLabel{$\lfalse$ Elim}
\UnaryInfC{$!B$}
\RightLabel{$\lif$ Intro$_3$}
\UnaryInfC{$!A \lif !B$}
\AxiomC{[$!B$]$^{(2)}$}
\AxiomC{[$!A$]$^4$}
\RightLabel{$\land$ Intro}
\BinaryInfC{$!A \land !B$}
\RightLabel{$\land$ Elim}
\UnaryInfC{$!B$}
\RightLabel{$\lif$ Intro$_4$}
\UnaryInfC{$!A \lif !B$}
\RightLabel{$\lor$ Elim$_2$}
\TrinaryInfC{$!A \lif !B$}
\RightLabel{$\lif$ Intro$_1$} 
\UnaryInfC{$(\lnot !A \lor !B) \lif (!A \lif !B)$}
\DisplayProof
\]
\end{ex}



\begin{ex}
Give a !!{derivation} of the !!{formula} $\lexists[x][\lnot !A(x)]
\lif \lnot \lforall[x][!A(x)]$.

When dealing with quantifiers, we have to make sure not to violate the
eigenvariable condition, and sometimes this requires us to play around
with the order of carrying out certain inferences. In general, it
helps to try and take care of rules subject to the eigenvariable
condition first (they will be lower down in the finished proof).

Starting as usual, we write

\[
\AxiomC{}
\UnaryInfC{$\lexists[x][\lnot !A(x)]\lif \lnot \lforall[x][!A(x)]$}
\DisplayProof
\]

We start by applying the $\lif$ intro rule.

\[
\AxiomC{[$\lexists[x][\lnot !A(x)]$]$^1$}
\noLine
\UnaryInfC{.}
\noLine
\UnaryInfC{.}
\noLine
\UnaryInfC{.}
\noLine
\UnaryInfC{$\lnot \lforall[x][!A(x)]$}
\RightLabel{$\lif$ Intro}
\UnaryInfC{$\lexists[x][\lnot !A(x)]\lif \lnot \lforall[x][!A(x)]$}
\DisplayProof
\]

Since there is no obvious rule to apply to $\lnot \lforall[x][!A(x)]$,
we will proceed by setting up the $\lexists$ Elim rule. We must be aware
of the eigenvariable condition, and choose a constant that does not appear
in $\lexists[x][\Atom{!A}{x}]$ or any assumptions that it depends on.

\[
\AxiomC{[$\lexists[x][\lnot !A(x)]$]$^1$}
\AxiomC{[$\lnot !A(a)$]$^2$}
\noLine
\UnaryInfC{.}
\noLine
\UnaryInfC{.}
\noLine
\UnaryInfC{.}
\RightLabel{$\lexists$ Elim$_2$}
\BinaryInfC{$\lnot \lforall[x][!A(x)]$}
\RightLabel{$\lif$ Intro}
\UnaryInfC{$\lexists[x][\lnot !A(x)]\lif \lnot \lforall[x][!A(x)]$}
\DisplayProof
\]

At this point, our only option is to carry out a $\lnot$ elim rule on
$\lnot \lforall[x][!A(x)]$, which requires that we derive a contradiction with
$\lforall[x][!A(x)]$ as the assumption. We can set it up as follows:

\[
\AxiomC{[$\lexists[x][\lnot !A(x)]$]$^1$}
\AxiomC{[$\lnot !A(a)$]$^2$}
\noLine
\UnaryInfC{.}
\noLine
\UnaryInfC{.}
\noLine
\UnaryInfC{.}
\AxiomC{[$\lforall[x][!A(x)]$]$^3$}
\noLine
\UnaryInfC{.}
\noLine
\UnaryInfC{.}
\noLine
\UnaryInfC{.}
\RightLabel{$\lfalse$ Intro}
\BinaryInfC{$\lfalse$}
\RightLabel{$\lnot$ Intro$_3$}
\UnaryInfC{$\lnot \lforall[x][!A(x)]$}
\RightLabel{$\lexists$ Elim$_2$}
\BinaryInfC{$\lnot \lforall[x][!A(x)]$}
\RightLabel{$\lif$ Intro}
\UnaryInfC{$\lexists[x][\lnot !A(x)]\lif \lnot \lforall[x][!A(x)]$}
\DisplayProof
\]

It looks like we are close to getting a contradiction. The easiest
rule to apply is the $\lforall$ Elim, which has no eigenvariable
conditions. Since we can use any eigenvariable we want, it makes
the most sense to continue using~$a$ so we can reach a contradiction.

\[
\AxiomC{[$\lexists[x][\lnot !A(x)]$]$^1$}
\AxiomC{[$\lnot !A(a)$]$^2$}
\AxiomC{[$\lforall[x][!A(x)]$]$^3$}
\RightLabel{$\lforall$ Elim}
\UnaryInfC{$\Atom{!A}{a}$}
\RightLabel{$\lfalse$ Intro}
\BinaryInfC{$\lfalse$}
\RightLabel{$\lnot$ Intro$_3$}
\UnaryInfC{$\lnot \lforall[x][!A(x)]$}
\RightLabel{$\lexists$ Elim$_2$}
\BinaryInfC{$\lnot \lforall[x][!A(x)]$}
\RightLabel{$\lif$ Intro}
\UnaryInfC{$\lexists[x][\lnot !A(x)]\lif \lnot \lforall[x][!A(x)]$}
\DisplayProof
\]

It is important, especially when dealing with quantifiers, to double
check at this point that the eigenvariable condition has not been
violated. Since the only rule we applied that is subject to the
eigenvariable condition was $\exists$ Elim, and the eigenvariable~$a$
does not occur in any assumptions it depends on, this is a
correct derivation.
\end{ex}

\begin{prob}
Give !!{derivation}s of the following !!{formula}s:
\begin{enumerate}
\item $\lnot(!A \lif !B) \lif (!A \land \lnot !B)$
\item $\lforall[x][(!A(x) \lif !B)] \lif (\lexists[y][!A(y)] \lif !B)$
\end{enumerate}
\end{prob}

\end{document}
