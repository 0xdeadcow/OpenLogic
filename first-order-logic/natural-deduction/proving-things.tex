% Part: first-order-logic
% Chapter: natural-deduction
% Section: proving-things

\documentclass[../../include/open-logic-section]{subfiles}

\begin{document}

\olfileid{fol}{ntd}{pro}

\olsection{Examples of \usetoken{P}{derivation}}

\begin{ex}
Let's give a !!{derivation} of the !!{formula} $(!A \land !B) \lif !A$.

We begin by writing the desired end-!!{formula} at the bottom of the derivation.
\[
\AxiomC{}
\UnaryInfC{$(!A\land !B) \lif !A$}
\DisplayProof
\]

Next, we need to figure out what kind of inference could result in
!!a{formula} of this form. The !!{main operator} of the
end-!!{formula} is $\lif$, so we'll try to arrive at the
end-!!{formula} using the $\lif$ Intro rule. It is best to write down
the assumptions involved and label the inference rules as you
progress, so it is easy to see whether all assumptions have been
discharged at the end of the proof.
\[
\AxiomC{$[!A \land !B]^1$}
\DeduceC{$!A$}
\RightLabel{$\lif$ Intro$_1$} 
\UnaryInfC{$(!A\land !B) \lif !A$}
\DisplayProof
\]

We now need to fill in the steps from the assumption $!A \land !B$ to $!A$.
Since we only have one connective to deal with, $\land$, we must
use the $\land$ elim rule. This gives us the following proof:
\[
\AxiomC{$[!A \land !B]^1$}
\RightLabel{$\land$ Elim}
\UnaryInfC{$!A$}
\RightLabel{$\lif$ Intro$_1$} 
\UnaryInfC{$(!A\land !B) \lif !A$}
\DisplayProof
\]

We now have a correct !!{derivation} of the formula $(!A \land
!B) \lif !A$.
\end{ex}

\begin{ex}
Now let's give a !!{derivation} of the !!{formula} $(\lnot !A \lor !B)
\lif (!A \lif !B)$.

We begin by writing the desired end-!!{formula} at the bottom of the 
derivation.
\[
\AxiomC{}
\UnaryInfC{$(\lnot !A \lor !B) \lif (!A \lif !B)$}
\DisplayProof
\]
To find a logical rule that could give us this end-!!{formula}, we look at
the logical connectives in the end-!!{formula}: $\lnot$, $\lor$, and
$\lif$. We only care at the moment about  the first occurence
of $\lif$ because it is the !!{main operator} of the !!{sentence}
 in the end-sequent, while $\lnot$, $\lor$ and the second occurence of 
 $\lif$ are inside the scope of another connective, so we will
take care of those later. We therefore start with the $\lif$ Intro rule. 
A correct application must look as follows:
\[
\AxiomC{$[\lnot !A \lor !B]^1$}
\DeduceC{$!A \lif !B$}
\RightLabel{$\lif$ Intro$_1$} 
\UnaryInfC{$(\lnot !A \lor !B) \lif (!A \lif !B)$}
\DisplayProof
\]

This leaves us with two possibilities to continue. Either we can
continue working from the bottom up and look for another application
of the $\lif$ Intro rule, or we can work from the top down and apply a
$\lor$ Elim rule. Let us apply the latter. We will use the assumption
$\lnot !A \lor B$ as the leftmost premise of $\lor$ Elim.  For a valid
application of $\lor$ Elim, the other two premises must be identical
to the conclusion $!A \lif !B$, but each may be derived in turn from
another assumption, namely the two disjuncts of $\lnot !A \lor !B$.
So our !!{derivation} will look like this:
\[
\AxiomC{$[\lnot !A \lor !B]^1$}
\AxiomC{$[\lnot !A]^2$}
\DeduceC{$!A \lif !B$}
\AxiomC{$[!B]^2$}
\DeduceC{$!A \lif !B$}
\RightLabel{$\lor$ Elim$_2$}
\TrinaryInfC{$!A \lif !B$}
\RightLabel{$\lif$ Intro$_1$} 
\UnaryInfC{$(\lnot !A \lor !B) \lif (!A \lif !B)$}
\DisplayProof
\]

In each of the two branches on the right, we want to !!{derive} $!A
\lif !B$, which is best done using $\lif$ Intro.
\[
\AxiomC{$[\lnot !A \lor !B]^1$}
\AxiomC{$[\lnot !A]^2, [!A]^3$}
\DeduceC{$!B$}
\RightLabel{$\lif$ Intro$_3$}
\UnaryInfC{$!A \lif !B$}
\AxiomC{$[!B]^2, [!A]^4$}
\DeduceC{$!B$}
\RightLabel{$\lif$ Intro$_4$}
\UnaryInfC{$!A \lif !B$}
\RightLabel{$\lor$ Elim$_2$}
\TrinaryInfC{$!A \lif !B$}
\RightLabel{$\lif$ Intro$_1$} 
\UnaryInfC{$(\lnot !A \lor !B) \lif (!A \lif !B)$}
\DisplayProof
\]

For the two missing parts of the !!{derivation}, we need
!!{derivation}s of $!B$ from $\lnot !A$ and $!A$ in the middle, and
from $!A$ and $!B$ on the left.  Let's take the former first. $\lnot
!A$ and $!A$ are the two premises of $\lfalse$ Intro:
\[
\AxiomC{$[\lnot !A]^2$}
\AxiomC{$[!A]^3$}
\RightLabel{$\lfalse$ Intro}
\BinaryInfC{$\lfalse$}
\DeduceC{$!B$}
\DisplayProof
\]
By using $\lfalse$ Elim, we can obtain $!B$ as a conclusion and
complete the branch.
\[
\AxiomC{$[\lnot !A \lor !B]^1$}
\AxiomC{$[\lnot !A]^2$}
\AxiomC{$[!A]^3$}
\RightLabel{$\lfalse$ Intro}
\BinaryInfC{$\lfalse$}
\RightLabel{$\lfalse$ Elim}
\UnaryInfC{$!B$}
\RightLabel{$\lif$ Intro$_3$}
\UnaryInfC{$!A \lif !B$}
\AxiomC{$[!B]^2, [!A]^4$}
\DeduceC{$!B$}
\RightLabel{$\lif$ Intro$_4$}
\UnaryInfC{$!A \lif !B$}
\RightLabel{$\lor$ Elim$_2$}
\TrinaryInfC{$!A \lif !B$}
\RightLabel{$\lif$ Intro$_1$} 
\UnaryInfC{$(\lnot !A \lor !B) \lif (!A \lif !B)$}
\DisplayProof
\]

Let's now look at the rightmost branch.  Here it's important to
realize that the definition of !!{derivation} \emph{allows assumptions
  to be discharged} but \emph{does not require} them to be.  In other
words, if we can derive $!B$ from one of the assumptions $!A$ and $!B$
without using the other, that's ok.  And to !!{derive} $!B$ from $!B$
is trivial: $!B$ by itself is such a !!{derivation}, and no inferences
are needed.  So we can simply delete the assumtion $!A$.
\[
\AxiomC{$[\lnot !A \lor !B]^1$}
\AxiomC{$[\lnot !A]^2$}
\AxiomC{$[!A]^3$}
\RightLabel{$\lfalse$ Intro}
\BinaryInfC{$\lfalse$}
\RightLabel{$\lfalse$ Elim}
\UnaryInfC{$!B$}
\RightLabel{$\lif$ Intro$_3$}
\UnaryInfC{$!A \lif !B$}
\AxiomC{$[!B]^2$}
\RightLabel{$\lif$ Intro}
\UnaryInfC{$!A \lif !B$}
\RightLabel{$\lor$ Elim$_2$}
\TrinaryInfC{$!A \lif !B$}
\RightLabel{$\lif$ Intro$_1$} 
\UnaryInfC{$(\lnot !A \lor !B) \lif (!A \lif !B)$}
\DisplayProof
\]
Note that in the finished !!{derivation}, the rightmost $\lif$ Intro
inference does not actually discharge any assumptions.
\end{ex}

\begin{ex}
When dealing with quantifiers, we have to make sure not to violate the
eigenvariable condition, and sometimes this requires us to play around
with the order of carrying out certain inferences. In general, it
helps to try and take care of rules subject to the eigenvariable
condition first (they will be lower down in the finished proof).

Let's see how we'd give a !!{derivation} of the !!{formula}
$\lexists[x][\lnot !A(x)] \lif \lnot \lforall[x][!A(x)]$.
Starting as usual, we write
\[
\AxiomC{}
\UnaryInfC{$\lexists[x][\lnot !A(x)]\lif \lnot \lforall[x][!A(x)]$}
\DisplayProof
\]
We start by writing down what it would take to justify that last step
using the $\lif$ Intro rule.
\[
\AxiomC{$[\lexists[x][\lnot !A(x)]]^1$}
\DeduceC{$\lnot \lforall[x][!A(x)]$}
\RightLabel{$\lif$ Intro}
\UnaryInfC{$\lexists[x][\lnot !A(x)]\lif \lnot \lforall[x][!A(x)]$}
\DisplayProof
\]
Since there is no obvious rule to apply to $\lnot \lforall[x][!A(x)]$,
we will proceed by setting up the !!{derivation} so we can use the
$\lexists$ Elim rule. Here we must pay attention to the eigenvariable
condition, and choose a constant that does not appear in
$\lexists[x][\Atom{!A}{x}]$ or any assumptions that it depends on.
(Since no !!{constant}s appear, however, any choice will do fine.)
\[
\AxiomC{$[\lexists[x][\lnot !A(x)]]^1$}
\AxiomC{$[\lnot !A(a)]^2$}
\DeduceC{$\lnot \lforall[x][!A(x)]$}
\RightLabel{$\lexists$ Elim$_2$}
\BinaryInfC{$\lnot \lforall[x][!A(x)]$}
\RightLabel{$\lif$ Intro}
\UnaryInfC{$\lexists[x][\lnot !A(x)]\lif \lnot \lforall[x][!A(x)]$}
\DisplayProof
\]
In order to derive $\lnot \lforall[x][!A(x)]$, we will attempt to use
the $\lnot$ Intro rule: this requires that we derive a contradiction,
possibly using $\lforall[x][!A(x)]$ as an additional assumption. Of
coursem, this contradiction may involve the assumption $\lnot !A(a)$
which will be discharged by the $\lif$ Intro inference. We can set it
up as follows:
\[
\AxiomC{$[\lexists[x][\lnot !A(x)]]^1$}
\AxiomC{$[\lnot !A(a)]^2, [\lforall[x][!A(x)]]^3$}
\DeduceC{$\lfalse$}
\RightLabel{$\lnot$ Intro$_3$}
\UnaryInfC{$\lnot \lforall[x][!A(x)]$}
\RightLabel{$\lexists$ Elim$_2$}
\BinaryInfC{$\lnot \lforall[x][!A(x)]$}
\RightLabel{$\lif$ Intro}
\UnaryInfC{$\lexists[x][\lnot !A(x)]\lif \lnot \lforall[x][!A(x)]$}
\DisplayProof
\]
It looks like we are close to getting a contradiction. The easiest
rule to apply is the $\lforall$ Elim, which has no eigenvariable
conditions. Since we can use any term we want to replace the
universally qauntified~$x$, it makes the most sense to continue
using~$a$ so we can reach a contradiction.
\[
\AxiomC{$[\lexists[x][\lnot !A(x)]^1$}
\AxiomC{$[\lnot !A(a)]^2$}
\AxiomC{$[\lforall[x][!A(x)]]^3$}
\RightLabel{$\lforall$ Elim}
\UnaryInfC{$!A(a)$}
\RightLabel{$\lfalse$ Intro}
\BinaryInfC{$\lfalse$}
\RightLabel{$\lnot$ Intro$_3$}
\UnaryInfC{$\lnot \lforall[x][!A(x)]$}
\RightLabel{$\lexists$ Elim$_2$}
\BinaryInfC{$\lnot \lforall[x][!A(x)]$}
\RightLabel{$\lif$ Intro}
\UnaryInfC{$\lexists[x][\lnot !A(x)]\lif \lnot \lforall[x][!A(x)]$}
\DisplayProof
\]

It is important, especially when dealing with quantifiers, to double
check at this point that the eigenvariable condition has not been
violated. Since the only rule we applied that is subject to the
eigenvariable condition was $\exists$ Elim, and the eigenvariable~$a$
does not occur in any assumptions it depends on, this is a
correct derivation.
\end{ex}

\begin{prob}
Give !!{derivation}s of the following !!{formula}s:
\begin{enumerate}
\item $\lnot(!A \lif !B) \lif (!A \land \lnot !B)$
\item $\lforall[x][(!A(x) \lif !B)] \lif (\lexists[y][!A(y)] \lif !B)$
\end{enumerate}
\end{prob}

\end{document}
