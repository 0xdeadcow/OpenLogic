% Part: first-order-logic
% Chapter: model-theory
% Section: overspill

\documentclass[../../include/open-logic-section]{subfiles}

\begin{document}

\olfileid{fol}{mod}{ove}
\olsection{Overspill}

\begin{thm}
\ollabel{overspill}
  If a set $\Gamma$ of sentences has arbitrarily large finite models,
  then it has an infinite model.
\end{thm}

\begin{proof}
  Expand the language of $\Gamma$ by adding countably many new
  constants $c_0$, $c_1$, \dots and consider the set $\Gamma \cup
  \{c_i \neq c_j : i \neq j\}$. To say that $\Gamma$ has arbitrarily
  large finite models means that for every $m >0$ there is $n\ge m$
  such that $\Gamma$ has a model of cardinality~$n$. This implies that
  $\Gamma \cup \{c_i \neq c_j : i \neq j\}$ is finitely
  satisfiable. By compactness, $\Gamma \cup \{c_i \neq c_j : i \neq
  j\}$ has a model $\Struct M$ whose domain must be infinite, since it
  satisfies all inequalities $c_i \neq c_j$.
\end{proof}

\begin{prop}
\ollabel{inf-not-fo}
  There is no sentence $!A$ of any first-order language
  that is true in a structure $\Struct M$ if and only
  if the domain $\Domain{M}$ of the structure is infinite.
\end{prop}

\begin{proof}
  If there were such a $!A$, its negation $\lnot !A$ would
  be true in all and only the finite structures, and it would
  therefore have arbitrarily large finite models but it would lack an
  infinite model, contradicting \olref{overspill}.
\end{proof}

\end{document}
