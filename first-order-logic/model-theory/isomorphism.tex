% Part: first-order-logic
% Chapter: model-theory
% Section: isomorphism

\documentclass[../../include/open-logic-section]{subfiles}

\begin{document}

\olfileid{fol}{mod}{iso}
\olsection{Isomorphic Structures}

\begin{defn}
\ollabel{elem-equiv}
  Given two structures $\Struct{M}$ and $\Struct M'$ for the same
  language $\mathcal{L}$, we say that $\Struct{M}$ is
  \emph{elementarily equivalent to} $\Struct M'$, written $\Struct{M}
  \equiv \Struct M'$, if and only if for every \emph{sentence} $!A$ of
  $\Lang L$, $\Sat{M}{!A}$ iff $\Sat{M'}{!A}$.
\end{defn}

\begin{defn}
\ollabel{iso}
  Given two structures $\Struct{M}$ and $\Struct M'$ for the same
  language $\Lang L$, we say that    $\Struct{M}$ is
  \emph{isomorphic to} $\Struct M'$, written
  $\Struct{M} \simeq \Struct M'$, if and only if there is a
  function $h\colon \Domain{M} \to \Domain{M'}$ such that:
  \begin{enumerate}
  \item $h$ is one-one: if $h(x) =
    h(y)$ then $x = y$; 
  \item $\mathsf{h}$ is onto $\Domain{M'}$: for every $y \in
    \Domain{M'}$ there is $x \in \Domain{M}$ such that $h(x) = y$;
  \item for every constant $c$: $h(\Assign{c}{M}) =
    \Assign{c}{M'}$; 
  \item for every $n$-place predicate symbol $P$: $\langle
    a_1,\dots,a_n\rangle \in \Assign{P}{M}$ if and only if  $\langle
    h(a_1),\dots,h(a_n)\rangle \in \Assign{P}{M'}$;
  \item for every $n$-place function symbol $f$:
    $h(\Assign{f}{M}(a_1,\dots,a_n)) =
    \Assign{f}{M'}(h(a_1),\ldots,h(a_n))$.
 \end{enumerate}
\end{defn}

\begin{thm}
\ollabel{iso-eleq}
  If $\Struct{M} \simeq \Struct M'$ then $\Struct{M} \equiv
  \Struct M'$.
\end{thm}

\begin{proof}
  Let $h$ be an isomorphism of $\Struct{M}$ onto
  $\Struct M'$; for any assignment $s$, $h \circ s$ is the
  composition of $h$ and $s$, i.e., the assignment in
  $\Struct M'$ such that  $(h \circ s)(x) = h(s(x))$.
  We proceed by induction on $t$ $!A$ and prove the stronger claims: 
  \begin{align*}
  &  h(\Value{t}{M}[s]) = \Value{t}{M'}[h\circ s];\\
  &  \Sat{M}{!A}[s] \text{ if and only if }
    \Sat{M'}{!A}[h \circ s].
  \end{align*}
Make sure to take note at each step of how each of the five properties
characterizing isomorphisms is used.    
\end{proof}


\end{document}
