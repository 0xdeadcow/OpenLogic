\documentclass[propositional-logic]{subfiles}

\begin{document}

\section{Completeness of Propositional Logic}


\begin{thm}\label{thm:propositional-logic:completeness-1}
If $\Gamma \models !A$, then $\Gamma \vdash !A$.
\end{thm}

The completeness theorem follows from the following version, which is
the version we'll actually prove.

\begin{thm}\label{thm:propositional-logic:completeness-2}
If $\Sigma$ is consistent, then $\Gamma$ is satisfiable.
\end{thm}

\begin{proof}[Proof 
of \ref{thm:propositional-logic:completeness-1} from
\ref{thm:propositional-logic:completeness-2}]

Suppose that $\Gamma \not\vdash !A$.  Then $\Gamma \cup \{\lnot !A\}$
is consistent.  By \ref{thm:propositional-logic:completeness-2},
$\Gamma \cup \{\lnot !A\}$ is satisfiable.  Hence, $\Gamma \not\models
!A$.
\end{proof}

To establish \ref{thm:propositional-logic:completeness-2}, we
construct a satisfying truth-value assignment for a given consistent
set~$\Sigma$ of formulas.  We do this by extending $\Sigma$ to a
maximally consistent set of formulas.  From a maximally consistent
set, a satisfying assignment can be read off easily.

\begin{defn} 
A set $\Sigma$ is \emph{maximally consistent} if it is
\begin{enumerate}
\item consistent and
\item if $\Sigma \subseteq \Sigma'$ and $\Sigma'$ is consistent, then
  $\Sigma = \Sigma'$.
\end{enumerate}
\end{defn}

\begin{lem}
If $\Sigma$ is maximally consistent and $\Sigma \vdash !B$, then $!B
\in \Sigma$.
\end{lem}

\begin{proof} 
$\Sigma$ is consistent and $\Sigma \vdash !B$, then $\Sigma \cup
  \{!B$\} is consistent.  Since $\Sigma$ is maximally consistent, we
  have $\Sigma = \Sigma \cup \{ !B \}$, hence $!B \in \Sigma$.
\end{proof}

\begin{lem}\label{lem:max-con-complete}
If $\Sigma$ is maximally consistent, then for all~$!B$, either $!B \in
\Sigma$ or $\lnot !B \in \Sigma$.
\end{lem}

\begin{proof}
If neither $!B \in \Sigma$ nor $\lnot !B \in \Sigma$, then $\Sigma
\cup \{!B\}$ would be consistent and not equal to $\Sigma$
\end{proof}

\begin{lem}\ref{lem:max-con-disjunct}
If $\Sigma$ is maximally consistent, then $!B \lor !C \in \Sigma$ iff
$!B \in \Sigma$ or $!C \in Sigma$.
\end{lem}

\begin{proof}
Only if: Suppose not.  Then for some $!B$ and~$!C$, $!B \lor !C\in
\Sigma$ but neither $!B \in \Sigma$ nor $!C \in \Sigma$.

By the preceding Lemma, $\lnot !B \in \Sigma$ and $\lnot !C \in
\Sigma$.  But then $\Sigma \vdash \lnot(!B \lor !C)$ (exercise: derive
$\not !B \lif (\lnot !C \lif \lnot(!B \lor !C)$.  Thus, $\Sigma$ would
be inconsistent.

If: Follows from \label{lem:max-con-complete} and the fact that $!B
\lif (!B \lor !C)$ and $!C \lif (!B \lor !C)$are derivable.
\end{proof}

\begin{lem}
If $\Sigma$ is maximally consistent, then $!B \land !C \in \Sigma$ iff
$!B \in \Sigma$ and $!C \in Sigma$.
\end{lem}

\begin{proof}
Exercise.
\end{proof}

\begin{defn}
If $\Sigma$ is maximally consistent, then let $v_\Sigma: \Var \to \{0,
1\}$ be defined by
\[
v_\Sigma(p) =
\begin{cases}
1 & \text{if $p \in \Sigma$} \\ 
0 & \text{otherwise.}
\end{cases}
\]
Note that by \ref{lem:max-con-complete} and since $\Sigma$ is
consistent, $v_\Sigma$ is well-defined for all~$p$.
\end{defn}

\begin{lem}\label{lem:truth}
If $\Sigma$ is maximally consistent, then $v_\Sigma \models !B$ iff
$!B \in \Sigma$.
\end{lem}

\begin{proof}
By induction on the complexity of ~$!B$.  If $!B$ is atomic, the claim
holds by the definition of~$v_\Sigma$.

If $!B \ident \lnot !C$, then $v_\Sigma \models lnot !C$ iff $v_\Sigma
\not\models !C$, iff (by induction hypothesis) $!C \notin \Sigma$ iff
(by \ref{lem:max-con-complete}) $\lnot !C \in \Sigma$.

Suppose $!C \ident !C \lor !D$.  Then $v_\Sigma \models !C \lor !D$
iff $v_\Sigma \models !C$ or $v_\Sigma \models !D$.

By induction hypothesis, this is the case iff $!C \in \Sigma$ or $!D
\in \Sigma$.  This in turn holds iff $!C \lor !D \in \Sigma$ by
\ref{lem:max-con-disjunct}.

Cases for $!C \land !D$ and $!C \lif !D$: exercises.
\end{proof}

\begin{lem}\label{lem:lindenbaum}
If $\Sigma$ is consistent, then there is a $\Sigma^* \supseteq \Sigma$
which is maximally consistent.
\end{lem}

\begin{proof}
Suppose $!A_0$, $!A_1$, $!A_2$, \dots{} is an enumeration of all
formulas.  Then let $\Sigma_0 = \Sigma$ and for $i = 0$, $1$, $2$,
\dots,
\[
\Sigma_{i+1} = 
\begin{cases}
\Sigma_i \cup \{!A_i\} & \text{if $\Sigma_i \cup \{!A_i\}$ is
  consistent}\\ \Sigma_i \cup \{\lnot !A_i\} & \text{otherwise.}
\end{cases}
\] 

Each $\Sigma_i$ is consistent (exercise), hence so is $Sigma^* =
\bigcup_{i=0}^\infty \Sigma_i$.

Exercise: why is $\Sigma^*$ so defined maximally consistent?
\end{proof}

\begin{proof}{Proof of \ref{thm:propositional-logic:completeness-2}} 
Suppose $\Sigma$ is consistent.  Then by \ref{lem:lindenbaum}, there
is a $\Sigma^* \supseteq \Sigma$ which is maximally consistent.  By
\ref{lem:truth}, $v_{\Sigma^*} \models !B$ for all $!B \in \Sigma^*$.
Consequently, $\Sigma^*$ and its subset $\Sigma$ are satisfiable.
\end{proof}

\end{document}
