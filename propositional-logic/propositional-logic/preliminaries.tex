% Part: first-order-logic 
% Chapter: axiomatic-proofs 
% Section: propositional-logic

\documentclass[../../include/open-logic-section]{subfiles}

\begin{document}

\olfileid{fol}{axp}{pre}

\olsection{Preliminaries}

The language $\Lang L_0$ of classical propositional logic
comprises of basic symbols, countably many propositional !!{variable}s
$p_0, p_1,\ldots$ as well as symbols for the !!{operator}s $\lnot$
(not), $\lif$ (if \dots then) and the two parentheses $($ and
$)$. We assume that these symbols are all distinct and no one occurs as
a part of another one. We refer to the set of the propositional
!!{variable}s as the set $\mathsf{At}_0$ of the atomic !!{sentence}s.

\begin{defn} [Formulas]
The set $F_0$ of the !!{formula}s of the language
  $\Lang L_0$ is inductively defined as the smallest set of strings
  over the alphabet containing $\mathsf{At}_0$ and such that
  if $!A$ and $!B$ are in $F_0$, then so are:
\begin{enumerate}
\tagitem{prvNot} {$(\lnot !A)$;}{}
\tagitem{prvIf} {$(!A \lif !B)$.}{}
\end{enumerate}
\end{defn}     
                        
\begin{thm} \ollabel{thm:induction}
\emph{Principle of induction on !!{formula}s}: If some
  property $P$ holds of all the propositional !!{variable}s and is such
  that it holds for $(\lnot !A)$ and $(!A \lif !B)$
  whenever it holds for $!A$ and $!B$, then $P$ holds of all
 !!{formula}s in $F_0$ .
\end{thm}

\begin{proof}
  Let $S$ be the collection of all !!{formula}s with property $P$, so
  that, in particular, $S\subseteq F_0$. Then $S$ contains
  the propositional !!{variable}s and is closed under the !!{operator}s;
  since $F_0$ is the smallest such class, also
  $F_0\subseteq S$. So $F_0 = S$, and every
  formula has propery $P$.
\end{proof}

\begin{prob} Prove that any !!{formula} in $F_0$ is \emph{balanced}, in
that
  it has as many left parentheses as right ones.
\end{prob}

\begin{prob} Prove that no
  !!{formula} begins with $\lnot$ and that no proper initial segment of
  !!a{formula} is !!a{formula}.
\end{prob}

\begin{explain}
 The !!{formula}s $(!A \lor !B)$ and $(!A \land !B)$
 abbreviate $((\lnot !A) \lif !B)$ and $\lnot(!A
 \lif (\lnot !B))$, respectively. Similarly, $!A \equiv
 !B$ abbreviates $(!A \lif !B) \land (!B \lif
 !A)$. Parentheses around $\lnot
 !$ are usually dropped, with the understanding that $\lnot$
 binds the shortest !!{formula} that follows it; outermost parentheses are
 likewise usually dropped.
 \end{explain}
 
\begin{prop}[Unique Readability]
\emph{Unique Readability} Any !!{formula}
  $!A$ in $ F_0$ has exactly one  parsing as one
  of the following
  \begin{enumerate}
  \item $p_n$ for some $p_n \in  \mathsf{At}_0$
  \tagitem{prvNot}{$(\lnot !B)$ for some $!B$ in $F_0$;}{}
  \tagitem{prvIf}{$(!B \lif !D)$ for some $!B, !D$ in
    $F_0$.}{}
  \end{enumerate}
  Moreover, such parsing is \emph{unique}, in that, e.g., $!A$
  cannot have the form $(\lnot !B)$ \emph{ in two different ways}.
\end{prop}

\begin{proof}
By induction on $!A$. For instance, suppose that $!A$ has
two distinct readings as $(!B \lif !D)$ and $(!B' \lif
!D')$. Then $!B$ and $!B'$ must be the same (or else one would
be a proper initial segment of the other); so is the two readings of
$!A$ are distinct it must be because $!D$ and $!D'$ are
distinct readings of the same sequence of symbols, which is impossible
by the inductive hypothesis. 
\end{proof}

 \begin{thm} \ollabel{thm:rec} 
 \emph{Principle of definition by
     recursion}: for any set $V$ and !!{function}s $\mathbf{i} :
   \mathsf{At}_0 \to V$ and $h_1, h_2$ from $V$ and $V \times V$,
   respectively, into $V$, there exists exactly one function $f :
   F_0 \to V$ satisfying the following equations:
  \begin{align*}
    f(p_n) &= \mathbf{i}(p_n) \\
    f(\lnot !A) &= h_1(f(!A))\\
    f(!A \lif !B) &= h_2(f(!A),f(!B))
  \end{align*}
\end{thm}

\begin{proof}
  Let $\mathcal{F}$ be the class of all functions $g$ such that:
  \begin{itemize}
  \item $\Domain{g}$ is finite and closed under !!{subformula}s;
  \item whenever $!A$ is in $\Domain{g}$ then $g$ satisfies the
    equation corresponding to $!A$.
  \end{itemize}
  Put $f = \bigcup \mathcal{F}$. It is easy to see that: 
  \begin{enumerate}
  \item No two functions $g$ and $g'$ in $\mathcal{F}$ disagree on any
    of the arguments on which they are both defined. Hence, $f$ is
    well-defined as a function. (This requires unique readability.)
  \item $f$ satisfies the equations.
  \item $f$ is unique.
  \item $f$ is total, i.e., $\Domain{f} = F_0$.
  \end{enumerate}
These are established by induction on !!{formula}s.
\end{proof}

\begin{defn} [Uniform Substitution]
  \emph{Uniform substitution}.  If $!A$ and $!B$ are !!{formula}s,
  and $p_i$ is a propositional !!{variable}, then
  $\Subst{!A}{!B}{p_i}$ denotes the result of replacing each
  occurrence of $p_i$ by an occurrence of $!B$ in $!A$;
  similarly, the simultaneous substitution of $p_1,\ldots,p_n$ by
  !!{formula}s $\psi_1,\ldots,\psi_n$ is denoted by
  $\Subst{!A}{!B_1}{p_1},\ldots ,\Subst{}{!B_n}{p_n}$.
\end{defn}

\begin{prob}
  Give a mathematically rigorous definition of
  $\Subst{!A}{!B}{p_i}$ using Theorem \olref[fol][axp][pro]{thm:rec}.
\end{prob}

\end{document}

